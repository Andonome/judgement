\chapter{Beasts}
\label{bestiary}

\epigraph{Look round this universe. What an immense profusion of beings, animated and organised, sensible and active! You admire this prodigious variety and fecundity. But inspect a little more narrowly these living existences, the only beings worth regarding. How hostile and destructive to each other! How insufficient all of them for their own happiness! How contemptible or odious to the spectator! The whole presents nothing but the idea of a blind Nature, impregnated by a great vivifying principle, and pouring forth from her lap, without discernment or parental care, her maimed and abortive children!}{Hume}

\settoggle{genExamples}{true}

\noindent
The world swarms with monsters, above and below.
This land belongs to them.
Sentient little creatures inhabit little pockets, like rabbits living inside a fox's hunting ground.
They travel together and tramp down roads, just as herds of sheep leave paths through dense forests.

The forest's predators mostly eat deer, aurochs, boar, and other prey animals.
They don't need humanity to survive, they simply don't care if their food can think or sing, they only care that it's small, and meaty.
Once a village hurts them enough, with swords, arrows, or just throwing rocks, predators usually leave that spot alone, at least for a little while.
But a new light, or loud noises, within the predator's territory, will excite them like nothing else.

\section[Forest Creatures]{Forest Creatures \A}

\begin{multicols}{2}

\noindent
Predators love the dense forests for setting an ambush.
They also love the rare open patches of marsh or swamp they can find, as they can see prey from a long way off.
The forest loves her predators, and provides everything they need to feed.

Large travelling groups sometimes set roaring bonfires, and prepare a perimeter of ready warriors.
Smaller groups generally dig their fires into the ground and curse every crackle, knowing that any noise will bring curious eyes.
Some prefer to sleep without a fire, and simply huddle together in the darkness, and hope.

Each creature presented -- whether predator or prey -- fits into a larger world, so each has notes on its \textbf{ecology}.
Many can be used as \glspl{ingredient} for magical \glspl{boon}, which are listed next to the creature.

\newBeast{Aurochs}% Name
  {auroch}% label
  {wild cows that create terrifying stampedes}% description

The wilder regions of \gls{fenestra} still see herds of the primitive cow: the auroch.
They have massive horns and tend to be more aggressive then the farm animals.

\auroch[\npc{\A}{Auroch}]

\paragraph{Ecology:} Across the wild planes stampede droves of wild cows, more primitive and larger than the cows we know today.
Muscular and tall, if ever they stampede nearby there is little to do except climb a tree or hope to run into a house before they arrive.
They can travel in herds of twenty to over one hundred.

\begin{boxtext}

  The storms are bad, and thunder has been going for the last few hours.  Then the ground starts to rumble.  Trees crash in the distance, and a hundred great aurochs stampede towards you.

\end{boxtext}

\index[mana]{Earth!Auroch hooves}
\paragraph{\Glspl{ingredient}:}
auroch hooves, crushed into a fine powder, provide Earth Sphere \glspl{boon}.

\paragraph{Encounters:} Aurochs are peaceful, their their stampedes can still be quite deadly.

\begin{itemize}

  \item
  The troupe see aurochs peacefully grazing in the distance.
  \item
  Nearby aurochs stampede.
  Have the troupe roll \roll{Wits}{Wyldcrafting} (\tn[8]) to notice the tremors, and their meaning in time.
  Anyone failing must roll \roll{Dexterity}{Wyldcrafting}, \tn[9], to dodge the incoming stampede.
  Failure indicates $2D6$ Damage as the stampede mauls the character with horns and hooves.
  Several wolves chase the aurochs, visible only once they have passed.

\end{itemize}

\newBeast{Basilisks}% Name
  {basilisk}% label
  {many-limbed lizards, the size of a \glsentrytext{warden}'s entourage, with the stench of rotten eggs}% description

Basilisks are massive, terrifying lizards.
They have a deathly odour which can paralyse anyone who smells it.
They grow to the length of four horses long, with eight powerful legs propelling their huge bodies forward like a centipede.
Their scaly bodies are particularly tough and so are prized as top-quality leather armour.
Often, the best way to deal with a basilisk is siege weapons.

\begin{boxtext}

  The wind brings a nasty stench with it.

  ``Time to go!'', shouts the captain.
  ``There's a basilisk about''.
  The troop draw up their tents extra quickly, and move away from the wind as snapping branches can be heard in the far distance.

  ``We can't outrun the beast, so try not to make too much noise.''

\end{boxtext}

\basilisk

\paragraph{Natural Abilities:} A basilisk's odour forces anyone smelling it to make a Strength check, each round, at a \gls{tn} of 10 minus the creature's Speed, or gain 3 \glspl{fatigue} as the character wretches or simply stares into space with disgust.

\paragraph{Ecology:}
Basilisks always move into the wind, so they appear downwind of any potential prey.
The more excited they become, they stronger their stench grows, and once the wind changes, anything nearby starts to cough and retch.

These massive lizards prefer hunting live prey, but will eat a corpse or eggs if they happen upon the meal.
Any time a basilisk finds chitincrawler eggs low enough to the ground, they eat them, possibly bashing at a tree to dislodge them, and the chitincrawlers can do very little about it.
This helps keeps crawler numbers low.
As a result, if the local basilisk population ever diminishes, the forest will soon see an explosion in chitincrawlers.%
\footnote{If this ever happens in your campaign, replace any basilisk encounters with $1D2$ chitincrawlers.}

Before each warm season, basilisks mate in a massive, writing, indiscriminate mass.
When basilisks mate, the plants that grow on them multiply too, such as \nameref{molted_basilisk}.%
\footnote{See \autopageref{molted_basilisk}.}
Once the warmth arrives, they dig their eggs into the earth, which then hatch in a swarm.

\basiliskSwarm

\noindent
Basilisks hibernate throughout the Winter, often in a pile in caverns.

\index[mana]{Air!Basilisk gullets}
\index[mana]{Earth!Basilisk hide}
\paragraph{\Glspl{ingredient}:}
squeezing a basilisk's fresh gullet-juices into the air boosts all casters' Air magic, wherever the juices fly (\roll{Strength}{Medicine}, \tn[10] to use properly).
And their hide, if dried and powdered, can produce Earth \glspl{boon}.
The herbalist rolls \roll{Dexterity}{Medicine} (\tn[10]) to mix the potions right, and each Margin on the roll grants a useable \gls{boon}

\paragraph{Encounters:}

\begin{itemize}
  \item{The basilisk's stench approaches while the troupe see an old man in the distance.
  He waves to them in greeting, and if they approach he tells them that he wants to die a warrior's death, so he has taken his sword and decided to fight a basilisk.
  If the \glspl{pc} do not intervene, the basilisk comes, and eats the old man before he can lift his heavy broadsword.}
  \item
  In a village, the archers start to gather at the walls as someone can smell a basilisk.
  It comes fast, and the stench penetrates so strongly that one of the archers feints and falls from the wall.
  If the \glspl{pc} do not save him, the basilisk drags him back into the forest as more arrows stick into his rough hide.
  \item
  A lone trader comes up behind the \glspl{pc}, and soon after everyone starts to smell the stench of the basilisk.
  The trader begs the \glspl{pc} to stay and fight while he runs - if the clothes infuse with the basilisk's smell, they will be ruined forever.
\end{itemize}

\newBeast{Bears}% Name
  {bear}% label
  {known for stalking, and theft, and sometimes aggression}% description

Bears can be found over all of Fenestra, including populated areas.
However, they are typically not dangerous if left alone.

\bear

\paragraph{Encounters:} Are curious but typically peaceful creatures in the mild or warm seasons.
However, when the world lies under snow, bears grow hungry, then reckless.

\begin{itemize}
  \item
  In dead of Winter, the troupe find an attractive cavern to rest in.
  However, the moment torch light hits the cavern's interior, they see a great bear inside, lying so still it could be dead.
  Have them roll \roll{Dexterity}{Stealth} to escape without waking the bear.
  If they fail, the bear immediately attacks the largest troupe member until it has secured a meal.
  \item
  One mild season, just before the cold hits, the last watchman in the night sees a bear watching the troupe.
  It does not approach -- it merely watches them, and stalks them over the next three nights.
  If the troupe ever show some weakness, or if one simply leaves to take a piss, the bear attacks, but backs off at the first real sign of danger.
  \item
  A single bear stands in a river, washing off webbing from a chitincrawler's web.
\end{itemize}

\newBeast{Boars}% Name
  {boar}% label
  {famously bad-tempered and highly defensive, but also good eating}% description


Boars tend to inhabit the deeper forests, and pick up a number of scars from fighting with the terrifying creatures who live there.
They have to be active all year round to survive, and like most forest creatures get braver as Winter approaches.

\boar

\index[mana]{Fire!Bear's heart}
\paragraph{\Glspl{ingredient}:}
a bear's heart, when pulled out in the heat of battle, serves as a potent \gls{boon}.
The blood, when squeezed out into the surrounding air, boosts Fire magic.
Pulling the heart out in time requires a \roll{Dexterity}{Medicine} roll at \tn[10].

\paragraph{Encounters:} Boars generally don't want to fight, so most encounters will be passive.

\begin{itemize}

  \item
  The troupe see a boar in the distant road, drinking from a puddle.
  If they wait a while, he leaves, but if they approach suddenly, he attacks once then flees.
  \item
  The troupe stumble upon two cute little wild piglets playing at the side of the road.
  The next moment, the mother boar screams and rushes towards them.
  If they run, she stays with her children and stops attacking.

\end{itemize}

\widePic[t]{Studio_DA/chitincrawler}

\newBeast{Chitincrawlers}% Name
  {chitincrawler}% label
  {arachnid the size of a horse, with wolf-like jaws and knife-like hands}% description

Chitincrawlers are large, armoured, spider-like creatures.
They have eight appendages, each ending with four-pronged knife-like `hands'.
Their jaws are more wolf-like than arachnid.

When someone sees a chitincrawler, it means death has come, and will be watching.
Their stride and hunger mean they can outrun a dog.
No human or gnoll can escape, except through superior numbers, and intimidation.
Therefore, seeing one in the distance means death has come, and watches carefully.

Chitincrawlers do not have the brains to feel fear.
When they see a group of twenty armed warriors, they really just see food.
They understand enough to not wander too close, but often approach anyway, excited by all the food they see, and unaware that swords and shields represent pain.
At this point, if those `warriors' loose their cool and run away, the chitincrawler will eat the slowest.

\chitincrawler

\begin{boxtext}

You awake sticky and heavy.
It's hard to get up.
Looking around, you see everyone in the circle has been stuck in a massive web while they slept.
Looking up at the treetops, you find a great, black creature descending with eight outstretched arms.

\end{boxtext}

\paragraph{Natural Abilities:} Chitincrawlers can slowly lay a sticky web-like substance over large areas -- often over bushes or tree branches so as to maximize contact points once something enters.
Once some creature has been caught the chitincrawler quickly descends.
If the prey is weak, it pulls the prey up the tree to feed on later.
If the prey might be strong enough to break free, it quickly bites through limbs to start bleeding it dry.

Noticing the web before stepping into it requires a \roll{Wits}{Wyldcrafting} action, \tn[7] during the daylight and higher at night or in the twilight of a forest.
Breaking free of the web requires a \roll{Strength}{Athletics} action at \tn[7] plus the chitincrawler's Strength.

\paragraph{Ecology:}
For the most part, chitincrawlers live a solitary life in their large territories, and enjoy watching places from afar with their excellent eyes -- they can spend days in a new area just watching to see how many animals and of what type pass through it.

They primarily hunt deer, boar and griffins -- their webs can capture all of them with equal ease.
On occasion, humans have successfully destroyed enough chitincrawler nests that the local population dwindled.
However, this always allows the local griffin population to flourish, which then eat all of the local deer herds, then become desperate enough to begin constant and desperate attacks on human and elven settlements of all sizes.%
\footnote{If this happens in your campaign, replace any chitincrawler encounters with griffin encounters until after the next warm season, double the number encountered, and give each a +2 bonus to Morale checks.}

During Cantea and Toldea, chitincrawlers find a potential mate, and the two travel together while the female decides whether or not to accept the mate, eat him, or both.

During the winter, all chitincrawlers hibernate, becoming part of the surrounding snow.
Any chitincrawler encounters should be missed during this time, or replaced with a very small chance of catching one hibernating under the snow.

\index[mana]{Fate!Chitincrawler spinnerets}
\paragraph{\Glspl{ingredient}:}
chitincrawler spinnerets (the web-spinning organs), at the back of the thorax, provide Fate Sphere \glspl{boon}.
The sticky substance must be spun on a loom (with \roll{Dexterity}{Crafts}, \tn[10]), until it collapses into dust, at which point it provides $1D6$ uses.

\paragraph{Encounters:} Chitincrawlers, despite no apparent intelligence, show incredible planning ability when laying their traps.

\begin{itemize}
  \item
  While the group sleeps, anyone on watch notes quiet movement in the dense trees above, then they see a thin, translucent mucus gently descend upon a sleeping companion.
  Every Failure Margin on a \roll{Strength}{Vigilance} check indicates one companion has already been trapped.
  \item
  The troupe look suddenly off the path to see a chitincrawler waving its arms frantically.
  This male chitincrawler knows that a female watches them from the other side, and hopes to gain her favour; he blocks the troupe, herding them back, while she grabs one from behind and runs.

  This behaviour may seem like intelligence, but in fact it is pure instinct.
  \item
  The troupe reach a muddy slope in the road which they must descend carefully if they don't want to fall over (\roll{Dexterity}{Wyldcrafting}, \tn[7]).
  They roll \roll{Wits}{Vigilance} (\tn[8]) to notice the web laid at the bottom of the slope.
  A chitincrawler immediately attacks from the top.
  It requires no roll to move, but every time one of the troupe move, they must make a new roll not to tumble downwards.
  The chitincrawler rams them to push them downhill.
  \item
  In the dead of Winter, the troupe find a snowy mound, and spot a hibernating chitincrawler's leg poking out from under the snow.
  A chitincrawler waking from hibernation typically kills everything in its path in order to get enough energy to hibernate again, so they must be careful.
\end{itemize}

\newBeast{Griffins}% Name
  {griffin}% label
  {beak-wielding mammal, with bat-like wings. They hunt in packs}% description

The favoured mount of elves, griffins nest in treetops using large branches and leaves.
They hunt deer, badgers, aurochs and anything else they can get their claws into.
Most hunt in packs.

\begin{boxtext}

  You've found nothing these past two days.
  You're still sure you're going in the right direction, but you have at least another day's march till you reach the town.
  The smaller members of the group don't look like they can make it.

  Looking up to the trees you can see a great nest packed with eggs, as big as a man's head\ldots

\end{boxtext}

Griffins' long arms extend into massive leathery wings.
When sprinting on all-fours, the wings mostly disappear, but once unfurled, they can take off.
Older, heavier, griffins often cannot fly simply by sprinting, and have to find a tree to climb, in order to take off.

Griffin beaks have a wicked point, used to pierce their prey's jugular.
Once the bleeding starts, they back away to safety, and wait for the creature to come to a natural stop before returning to feed.

Griffins can mimic sounds they hear with great accuracy, and often mimic their prey in order to attract it.
When scared, they often mimic larger animals, such as bears or basilisks.
This habit earned them the group-noun `theatre'.

\griffin[\npc{\A}{Griffin}]

\paragraph{Ecology:} Griffins live in areas of tall trees where they can safely build their nests.
They are excellent climbers, learning first to scurry along trees by digging their sharp claws in and then later to glide in order to swoop down on prey from above.
By the time a griffin has learnt to fly, it can have a wingspan of up to 15 feet.
Griffins like to nest at the edges of forests, allowing them to fly out and catch grazing creatures such as deer on the open planes.
Griffins will often hunt alone, but if a prey is spotted which is too large to catch  they will often hunt together.

\index[mana]{Air!Griffin feathers}
\paragraph{\Glspl{ingredient}:}
enough feathers, dried and ground up into a powder, grant a single \gls{boon} for Air magic (no roll required).

\paragraph{Encounters:}

\begin{itemize}

  \item
  A lone griffin circles overhead a few times, and then leaves.
  The troupe may think it has taken no interest in them, but soon after, half a dozen griffins descend to make one attack, and test their mettle.
  The griffins continue swooping down until the troupe injure one of their number.
  \item
  Walking across a mountainous range, a theatre of griffins decides to watch the troupe from afar.
  Once the troupe pass a difficult section, they swoop down and attempt to grab a troupe member, pulling them off the side of a ledge, and letting them fall.
  The griffins have no intention of fighting fairly -- they know they just need to pull someone over the edge to kill them.\footnote{You should allow players to spend 5 \glspl{fp} to avoid sudden death from cliff edges.}
  \ifnum\value{temperature}>1
    \item
    Two griffins circle above the \glspl{pc} then descend and attack.
    If any of the \glspl{pc} pass a \roll{Wits}{Wildcrafting} check, \gls{tn} 8, they notice the griffins are defending a nest above.
    If the \glspl{pc} back off, the griffins leave them alone (they only want to defend their nest).
  \else
    \item
    Entering the forest, the characters hear a man say `those look tasty'.
    Anything they say repeats back to them, interspersed with `those look tasty'.
    Getting closer, they find a griffin watching from the treetops, mimicking whatever they say.
  \fi
  \item
  \ifodd\value{r4}
    A griffin nest lies in the treetops.
    A single egg could make a meal for three men.
    The parent has left, so a \roll{Speed}{Athletics} roll allows a troupe member to get the eggs before a parent returns (\tn[9]).
  \else
    Late at night, a voice calls out from the forest.

    \textit{``Hoo-rah, up-she-rises!''}

    It continues to sing this single verse, repeating it with exactly the same tone and pitch, then goes quiet, then starts up again.

    If any response repeats twice, the griffin mimics that response, and walks closer to hear better\ldots
  \fi
  \item
  A large griffin circles above, eyeing up the troupe's smallest member.
  It attempts to pick them up, and fly off, in a single attack.
  \item
  A griffin swoops down and snatches a baby in a village.
  If the adventurers run after it, they soon hear baby-cries in the forest, but when they approach, they find a griffin with blood covering its beak, repeating the sounds of the crying baby.

\end{itemize}

\newBeast{Mouthdiggers}% Name
  {mouthdigger}% label
  {ambush predator which digs like a mole.  The size of a dog, but jaws expand to three times a reasonable size}% description

Imagine a very large mole, opening its mouth impossibly wide, showing more rows of teeth than a shark.
These furry creatures could almost appear as a large dog if seen outside of the ground, but most people (and deer) usually only see its tonsils exploding from underneath.
Once it has its prey clenched, it immediately retreats down its narrow hole, to be eaten from the feet to the head.

\begin{boxtext}

  The ground beside you explodes in a storm of teeth, all snapping for your leg.

\end{boxtext}

\paragraph{Ecology:} Mouthdiggers' favourite food is the auroch -- they can snip the leg-tendons from one, watch it bleed out and then crawl over and feast on it for some days to come.
Their greatest strength is that, while lying mostly underground, they are almost completely scentless, so creatures which coordinate by their nose have a hard time spotting the lethal trap in waiting.

Much like badgers they create long, winding Paths of underground homes.  Anyone with a Strength greater than 1 is too wide to fit into such narrow passages.

\index[mana]{Earth!Young mouthdigger liver}
\paragraph{\Glspl{ingredient}:}
the liver of a young mouthdigger serves as an Earth \gls{boon}, with minimal preparation.
Any which can open their eyes have already reached adulthood.
Of course they only birth young in the warm seasons, and the young remain below-ground until adulthood, so accessing the litter always presents a challenge.

A litter consists of $2D6$ sprogs.

\paragraph{Encounters:} There is only one way a mouthdigger attacks -- by surprise.  They dig into a bush, build a warren, clean up the surrounding area so nobody can see the entrance, then jump out of the bush and bite.  If unsuccessful, the mouthdigger typically crawls back into its hole.

\begin{itemize}

  \item
  While approaching a settlement, the troupe hear screaming.
  Two children were playing, and a mouthdigger exploded from the ground and dragged one child back underground.
  The child has already died, but the troupe may wish to find a way to kill the creature anyway, so the village can rest easy.
  \item
  As night falls, the first watchman notes a mouthdigger out in the open, and exposed, scuttling into one of its holes.
  By morning, the entrance to the cavern looks like a normal bush.
  The watchman can do whatever they want with this information.
  \item
  A mouthdigger jumps up, and bites a heavily armoured troupe member, then drags them mostly underground.
  The character will probably still be alive, but the troupe still have to get them out while the creature pulls its victim further downward.

\end{itemize}

\mouthdigger

\newBeast{Stirges}% Name
  {stirge}% label
  {insects with size of a fist, with a blood-draining stinger}% description

These blood-sucking insects, the size of a fist, can prove a real nuisance once they become angry.
Their wicked-sharp sucker can nip and slice into any exposed skin they can find, draining life slowly.
And while most build their hives from leaves and mucus, they will use skin and bone if they find a corpse.
This creates an ugly mound of femurs, fibulae, and ribs, all buzzing with activity.

While queens don't leave the nest, their eggs do.
Hives will often reproduce by male stirges, who carry eggs on their stingers in order to inject them into the softest spot they can find on a mammal.
The eggs will lay dormant for about a week, quietly feeding off their host's blood-stream, until they have enough material to develop a miniature nest inside the creature.
They then use the host as a mobile nest, as a miniature swarm develops inside the creature, growing larger, until the creature finally dies, and the swarm feeds on its corpse to grow more drones, and uses the bones and skin to form a new nest.

\label{stirgeEggs}
Any sting by a swarm of stirges has a 1 in 6 chance of injecting an egg.
However, not many stirges carry eggs in their stingers, so once they successfully inject the eggs, no other targets will suffer the same fate from the same swarm.

Checking a stirge-sting for eggs requires rooting around inside the wound.
Anyone checking for eggs makes a \roll{Dexterity}{Medicine} roll (\tn[12]).
Success inflicts a point of Damage, but will identify whether or not any eggs lie inside the wound.
Failure inflicts 3 Damage, and always indicates that the wound has no eggs (whether or not it does).

\paragraph{Natural Abilities:}
Stirges fly, and do so only to press their stingers into animals, and drink blood.
Those attacked by stirge-stingers gain \pgls{fatigue} instead of Damage, and may become the vessel for their eggs (see \vpageref[above]{stirgeEggs}).

\paragraph{Ecology:}
Over the cold months, the queen hibernates while the hive dies.
She eats some eggs, and leaves others to hatch, and begin hunting for blood.
Stirges will suck the blood from anything which moves.

When the weather becomes really hot, stirges multiply, creating new colonies.
Throughout this time, they attack anything which moves.

\index[mana]{Fire!Stirge queen}
\paragraph{\Glspl{ingredient}:}
A freshly caught stirge queen makes for an excellent, and immediate \gls{boon} for Fire spells.
The \gls{boon} requires a little drying (\tn[5] to prepare properly), but only provides enough for one \gls{boon}.

\stirgeSwarm

\newBeast{Wolves}% Name
  {wolf}% label
  {opportunistic food-thieves, who also drain an area of deer}% description

\paragraph{Ecology:} Wolves live everywhere except islands.  In the Winter, many die, and they become rarer, but in Summer they breed and run quickly.  While farmers know them as thieves, the majority of the time, wolves hunt wild animals, like boar or deer.

\begin{boxtext}

  A tiny scratching sound is heard -- only a little louder than the crackling fire.
  Rucksage is wandering away into the distance, and as the fire flares up you see the face of a wolf in the shadows, dragging his baggage away.
  A dozen wolves gather and tear the bag open, pulling the food out.

\end{boxtext}

\wolf[\npc{\A\T}{Wolves}]

\paragraph{Encounters:} During warmer seasons, wolves never attack humans.
Encounters may involve seeing wolves running in the distance, but nothing more.
When food becomes leaner, they become braver.

\begin{itemize}

  \item
  One night on watch, a pack of wolves stalks the troupe.
  If the watchman fails the \roll{Strength}{Vigilance} check to stay alert, one wolf grabs the smallest troupe member's backpack and flees.
  \item
  Yelps echo around the forest.
  Two wolves have been caught in a chitincrawler's web, and the rest of the pack cannot free them, so they just cry and bay loudly.
\end{itemize}

\newBeast[\E]{Woodspies}% Name
  {woodspy}% label
  {land-octopus, able to camouflage itself by changing texture and colour, and change size}% description

\widePic{Studio_DA/woodspy}

Woodspies are like a type of camouflaging octopus, adapted to exist solely on land.
They have three to eight tentacles and transform their shape in the manner of a hand making a shadow puppet and then adjust their skin-texture to match.
They grapple with opponents and then start gnawing into prey with a powerful beak, located underneath the main body.

\begin{boxtext}

  The tree to your side shifts, and tentacles reach out of it.
  Half of the trunk was really a translucent creature, waiting to grab you and pull you up the tree, far away from the rest of the troupe.

\end{boxtext}

\paragraph{Natural Abilities:} Woodspies are too soft to hurt creatures through punching -- they must first grapple and then start sticking their beak, held underneath the main body, into the target.
However, their many limbs allow them to grapple creatures without becoming vulnerable.%
\exRef{core}{the Core rules}{grappling}

Noticing a woodspy while camouflaged requires a \roll{Wits}{Vigilance} action, \tn[14]; bonuses can be assigned for sunny or particularly open areas.

\woodspy

\paragraph{Ecology:} Woodspies inhabit all manner of areas but prefer open plains and forests where there is plenty of cover and plenty of food.  They always hunt alone.  Despite being animals, they are in fact rather intelligent, although this intelligence only knows how to watch and calculate -- they do not communicate much.

Woodspies love fish, deer, gnomes, badgers, humans, and aurochs.
Animals with too sharp a bite (such as wolves and bears) typically put up too much of a fight for woodspies to bother.

\index{Rivers}
Despite being octopods, fast-running water presents a problem for woodspies.
It throws them about, and removes their control.
So despite their ability to breathe underwater, and their ease of moving in the sea, they rarely attack boats on a river, and never venture to the ocean's surface during a storm.

During the Winter, many venture underground to hibernate (others remain active).
Their initial excursions bring some much-welcome excrement to caving systems, feeding slimes, lichen, fungi, and myriad insects.

\index[mana]{Water!Woodspy beak}
\paragraph{\Glspl{ingredient}:}
the beak of a woodspy, once ground into a fine powder, provides a number of Water \glspl{boon} equal to the creature's Speed Bonus.

\paragraph{Encounters:} Woodspies' keen intelligence allows them to plan attacks like no other creature.
They will use every part of their environment and the troupe's exact situation to launch the perfect attack.

They know their prey well, and tend not to attack when they feel endangered.
If something looks to dangerous to subdue, woodspies often follow their prey for some time.

Every woodspy which has seen an archer will recognize the bow, and understand how to hide behind cover.

\begin{itemize}
  \item
  A single tentacle reaches down, grabs the smallest troupe member, and lifts them up into a tall tree.
  The woodspy then focusses only on climbing the massive tree.
  A \roll{Speed}{Athletics} check is required to follow the creature upwards, and only two troupe members will be able to approach at a time.
  \item
  In the distance, the troupe notice a massive woodspy slowly descending a tree.
  It waves its tentacle around a bush, slowly.
  A mouthdigger rushes up to bite the tentacle, but the woodspy quickly grabs it, then pulls it up into the trees.
  \ifnum\value{temperature}=1
    \item
    The troupe approach a river with a strong current.
    Have them roll, \roll{Wits}{Vigilance}, \tn[9].
    Success means they have spotted a woodspy upstream, plopping into the water in order to lay an ambush.
    Even knowing that the ambush exists will not guarantee safety, since the troupe cannot move easily in the water.
    If they wander downstream, the woodspy follows.
    \item
    A woodspy picks up a bush, then waves it around, provoking a fight with the \glspl{pc}.
    If they approach the bush, it uses the diversion to grab any bags they leave on the ground.
    The creature then climbs a tree, and tears open the bag, rummaging around it for any food, while throwing the rest on the ground.
  \else
    \item
    As the troupe approach a village, a nearby bush reveals itself to be a smaller woodspy, and flees.
    If they let the creature go, the next day it returns and takes a less well-armoured victim by surprise.
    \item
    Deep in the forest, a woodspy has drifted into a deep sleep, and as it dreams unknowable dreams, it flickers through rainbow colours, and changes its skin-texture.
    \item
    The troupe discover a deer-corpse, writhing with little woodspy babies, each the size of a hand.
  \fi
\end{itemize}

\end{multicols}

\section[Labyrinth Creatures]{Labyrinth Creatures \A}

\begin{multicols}{2}

\noindent
The underground world is a sandwich.

The top layers pull in a little trickle of food from above, and nutrient-rich streams flow through, producing rare fungal gardens, or little cave-dwellers leave droppings a little inside, producing food for the wandering acidic oozes.
Any passage leading downwards eventually sucks down any organic matter, and usually produces a slimy slope -- a natural trap, which pulls people into the labyrinth below.

Below, a cold empty desert, full of crooked, empty veins, yawns.
A few creatures crawl down to these depths to hibernate in peace.
Occasional dwarves leave their life's savings down a long, dark passage.
And rare undead wander into the darkness to stand, and stare, in meditation.

This freezing, frostless, labyrinth, creates a game with strict rules for all its inhabitants.
Any loud noise can travel for miles underground, so predators often stay still, and listen well.
But if they wait too long in an empty patch, they will starve.
Prey-creatures play the same game on the other side, waiting, breathing silently, and listening for any small sounds of movement in the distance to indicate that something moves either away from them, or towards them.
And once a chase begins, both predator and prey can only run if they know the area well; the home-team commands a striking advantage, since they don't have to feel for the road ahead as they run.

Of course, fire is an option, but it produces its own complications.

Far below the frigid labyrinth, the air becomes warm again.
People don't go beyond this point and return.

\newBeast{Acidic Oozes}% Name
  {ooze}% label
  {uniform gastropod, with just enough mind to move towards movement and digest it}% description

\widePic{Studio_DA/jelly}

Oozes are mobile, gelatinous, blobs with about the same level of intelligence as a quick-thinking bush -- sometimes they ooze along the underground caverns, sometimes they crawl into rooms to digest their latest victim and appear as a little pool of water hidden away in a small pit.

\paragraph{Natural Abilities:} Young oozes are generally partially transparent, especially when hungry; spotting them requires a \roll{Wits}{Vigilance} action, against their \roll{Wits}{Stealth}.
Older oozes are hardier, and larger, but less good at hiding.

Some ooze also have the Projectiles Skill, indicating they can shoot acid up to 5 squares away.

In combat oozes always slither blindly at people, shifting about randomly so as not to be hit, and wrestling with their targets.
Once a target is grappled, the ooze inflicts damage equal to its Strength immediately, and again at the start of each turn.

Larger oozes also heal 1 \gls{hp} per round by melding back together, but can only regenerate half their \glspl{hp} this way before they need to feed.
Their forms can still be bruised, and sufficient separation will kill them forever.

An ooze's greatest weakness is fire.
They suffer $1D6$ Damage from any torches, and receive no \gls{dr} when attacked with fire.

\paragraph{Ecology:} There are a large variety of ever-evolving gelatinous creatures which inhabit the underground realms.
Most eat the underground mushrooms and the scum from underground lakes.
Many skulk along ceilings and drop on anything which passes beneath them.
They possess less sentience than the simplest insect and operate by a very simple set of rules involving moving towards anything which smells like an edible, and (if the ooze is especially clever) moving away from anything that hurts too much.

Oozes can also be found occasionally out in the ocean, slowly digesting any fish (or fishermen) who wander into them.
They may not be fast, but once they have someone it can be difficult to get out of their grasp.

Oozes, while highly acidic, are a dwarven delicacy once prepared properly.

Few people know this, but oozes are the only creatures in Fenestra with a longer natural life-span than elves.
They continue to grow anytime they can eat, occasionally amalgamating with other oozes, or spawning smaller oozes after enveloping another.
An ooze which cannot `mate' in this fashion will simply continue to grow until something stronger comes along to kill it.

\jelly

\index[mana]{Fate!Black and brown oozes bodies}
\paragraph{\Glspl{ingredient}:}
black and brown oozes, once collected and left without air for a year, produces a powerful \gls{boon} for Fate magic.
How much depends on how much of the body one can capture, but it could produce as many \glspl{boon} as it has \glspl{hp}.

\paragraph{Encounters:}

\begin{itemize}
  \item
  A nearly invisible ooze sleeps on the cave's ceiling which the troupe walk under.
  Have them roll \roll{Wits}{Vigilance} to notice before it drops on them.
  \item
  The troupe find a dark and lighter ooze intertwined.
  It's not clear if they're mating, fighting, or if the darker one is giving birth to the other, but if the troupe disturb them, they both attack.
  \item
  On the ground is a puddle with ten \glsentrylongpl{gp}, a silver ring, a dagger's blade, and helmet.
  This is an ooze, along with the remains of a dwarf.
  If the troupe approach, they may become its next meal.
\end{itemize}

\jelly

\newBeast{Umber Hulks}% Name
  {umber_hulk}% label
  {beetle-like creatures, the size of a wagon}% description

Umber hulks are massive insect-like creatures with gnashing mandibles.
They mostly eat mushrooms and acidic, underground oozes, but devour absolutely anything if it happens to be near.

\begin{boxtext}

  Small, sticky orbs hang from the ceiling.

\end{boxtext}

\paragraph{Natural Abilities:} When stressed or angry, an umber hulk's odour forces anyone smelling it to make a Strength check, each round, at a \gls{tn} of 10 minus the creature's Speed, or gain 3 \glspl{fatigue} as the character wretches.
This stench quickly fills tunnels, and can serve as an early warning sign, but not always.
The stench only occurs when the creature starts galloping.

\paragraph{Ecology:}
They move quickly, eat quickly, and worst of all, their eggs hibernate until food is near.
This makes them extremely difficult to get rid of once a tunnel system has been infested.

They naturally live underground, where their eggs can grow undisturbed, but many venture a few miles outside of their caverns when hungry.
They encourage their eggs to hatch by heating them through rubbing their massive, armoured, limbs together, to produce a vibration, and a strange kind of purr.

Due to heat encouraging the eggs to hatch, anyone wandering nearby with a torch often encourages the entire brood to exit their shells immediately, and cry for their parents.

\begin{boxtext}

  Once you crack one open, you find a tiny little insect with a soft, white shell.
  The smell on the inside fills the entire tunnel, and a second later a fast clatter comes from farther down the tunnel.
  It's getting closer.

  You turn to see an insectoid creature, bigger than any war horse, with a solid, black shell, racing towards you.

\end{boxtext}

\index[mana]{Fire!Umber hulk eggs}
\paragraph{\Glspl{ingredient}:}
umber hulk eggs can produce Fire \glspl{boon} once properly dried (but must remain unhatched).
The quantity depends on the brood-size and how close they are to hatching.
The result is $2D6-5$ Fire \glspl{boon} (if the number is less than 1, the brood was not mature enough to produce any \glspl{boon}).

\paragraph{Encounters:}

\begin{itemize}
  \item
  Ahead, an umber hulk lies in wait for the troupe.
  While they are not masters of stealth, simply sitting in the dark works well for them.
  Have the troupe roll \roll{Wits}{Vigilance}, \tn[9] due to the twilight.
  \item
  An umber hulk sees the troupe and gives chase.
  After they run for a little while, it stops, and refuses to chase them further.
  Observant characters will notice droppings on the cave's floor which indicates this is the edge of the umber hulk's territory.
  Whatever new creature has staked a claim to this land terrifies the umber hulk.
  \item
  An umber hulk is masticating a dead gnome's leg.
  He was clearly rich, as a bag of silver pieces lays by his side, together with a couple of magical scrolls which he apparently did not get a chance to use.
\end{itemize}

\umberhulk

\newBeast{Watchers}% Name
  {watcher}% label
  {half-animal, half-plant. They release hallucinogenic toxins when stepped on}% description

Somewhere between an very sentient plant and a very slow creature, watchers simply wait where they are, tentacles outstretched like scrawny roots across some tens or hundreds of feet.

\paragraph{Abilities:} Once a tentacle is stepped on, the creature slowly lets out a hallucinogenic gas to confuse their targets.
A \roll{Wits}{Vigilance} task, \tn[10], is required each round to not waste one's time running from, attacking or conversing with mental illusions of one's own making.
The hallucinations vary, but whenever a player mentions some danger -- whether a trap, monster, or enemy, they appear, typically in the distance, around the edge of their visions.
Meanwhile, everyone involved suffers 4 \glspl{fatigue} each round.

Watchers can be difficult to spot; they like to hide in darkened corners and pretend to be a shrub or a pile of rubbish -- finding them requires a \roll{Wits}{Vigilance} Action at \tn[10].
Their gases fill massive areas, so it can often be a time race to either escape or find and destroy the creature.
Destroying it of course just releases more gas, but at least potential victims have the joy of knowing their corpse will not be slowly eaten while paralysed with fatigue by an alien face with too many eyes.

\pic{Decky/watcher}

\watcher

\index[mana]{Water!Watcher gas-sack}
\paragraph{\Glspl{ingredient}:}
if someone grabs one and contains it, before the gas releases (\roll{Dexterity}{Caving}, \tn[10]), they can create $1D3$ Water \glspl{boon}.

\paragraph{Encounters:}

\begin{itemize}

  \item
  The troupe enter a cavern and rest for the night, noting a number of odd fungi.
  Once the group sleep, the member taking watch begins hallucinating all the late-night encounters they have ever had coming to feed.
  Once they wake other troupe members, each rolls to come to their senses.
  \item
  The troupe wander through a cavern and notice the little tendrils of watchmen lying everywhere.
  Have them roll \roll{Dexterity}{Stealth} not to step on any tendrils.
  The exit lies 40 squares away.

\end{itemize}

\end{multicols}


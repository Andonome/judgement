\chapter{Beasts}
\label{bestiary}

\epigraph{Look round this universe. What an immense profusion of beings, animated and organised, sensible and active! You admire this prodigious variety and fecundity. But inspect a little more narrowly these living existences, the only beings worth regarding. How hostile and destructive to each other! How insufficient all of them for their own happiness! How contemptible or odious to the spectator! The whole presents nothing but the idea of a blind Nature, impregnated by a great vivifying principle, and pouring forth from her lap, without discernment or parental care, her maimed and abortive children!}{David Hume}

\settoggle{genExamples}{true}

\begin{multicols}{2}
\subsection{\Glsfmttext{fenestra}'s Crawling Face}

\Gls{fenestra} looks like Earth from afar, but the proportions differ.
This celestial body has a little less mass, so people can jump a touch higher, and dragons find it a little easier to take flight.

\Gls{fenestra}'s air has more oxygen, which helps trees and animals grow tall.
Even when people cut down the forest to make way for the widest road they can possibly maintain, the trees cover the roads in shadow.

This world swarms with monsters, above and below.
This land belongs to them.
Sentient little creatures inhabit little pockets, like rabbits living inside a fox's hunting ground.
They travel together and tramp down roads, just as herds of sheep leave paths through dense forests.
But no matter how wide they make their roads, the tall trees leave them in shadow.

The forest's predators eat deer, aurochs, boar, and other prey animals.
They don't need humanity to survive, they simply don't care if their food can think or sing, they only care that it wriggles and squeals.
They wander the dark forests, bored and lonely, listening intently for anything interesting.
Any new loud noises, whether wandering feet, hooves, or fire, make them excited, and hungry.
They don't have any instincts about avoiding humans, or even \glspl{village}.
A tall wall, hiding noise and light, sounds like fun!

Once \pgls{village} hurts them enough, with arrows and rocks, a predator usually leaves that spot alone for a week or so.
Or perhaps next time, they will remember to approach from the other side.
No \gls{village} in \gls{fenestra} will ever clear all the predators in an area, any more than rabbits can kill all the foxes.
The land belongs to them.

Gnomes, dwarves, and badgers hide underground.
Birds, humans and beavers sneak across the land, building little houses to keep themselves safe at night, then emerge to forage in the Sunlight.

\end{multicols}

\section[Forest Creatures]{Forest Creatures \A}

\begin{multicols}{2}

\noindent
Predators love the dense forests for setting an ambush.
They also love the rare open patches of marsh or swamp they can find, as they can see prey from a long way off.
The forest loves her predators, and provides everything they need to feed.

Large travelling groups sometimes set roaring bonfires, and prepare a perimeter of ready warriors.
Smaller groups generally dig their fires into the ground and curse every crackle, knowing that any noise will bring curious eyes.
Some prefer to sleep without a fire, and simply huddle together in the darkness, and hope.

\rotRates
A number of creatures' body-parts can be used as \glspl{ingredient}.
Harvesting them like this requires processing the piece into something useful (e.g. \pgls{boon}, or \pgls{talisman}) before it rots, and the rate of rot depend on the season.

Anywhere in the \gls{deep} counts as a cold season, because the \gls{deep} remains cold forever.

\newBeast{Aurochs}% Name
  {auroch}% label
  {wild cows that create terrifying stampedes}% description
still roam \gls{fenestra}.
These primitive cows have massive horns and tend to be more aggressive then the farm animals.

\auroch[\npc{\A}{Auroch}]

\paragraph{Ecology:} Across the wild planes stampede droves of wild cows, more primitive and larger than the cows we know today.
Muscular and tall, if ever they stampede nearby there is little to do except climb a tree or hope to run into a house before they arrive.
They can travel in herds of twenty to over one hundred.

\begin{boxtext}
  Rain has drowned everything, and the thunder begins to grow louder.
  Then the ground starts to rumble.
  Trees crash in the distance, and a hundred great aurochs stampede towards you.
\end{boxtext}

\index[mana]{Earth!Auroch hooves~\glsentrytext{A}}
\paragraph{\Glspl{ingredient}:}
auroch hooves, crushed into a fine powder, provide Earth Sphere \glspl{ingredient}.

\showEnc

\begin{itemize}
  \item
  The troupe see aurochs peacefully grazing in the distance.
  \item
  Nearby aurochs stampede.
  Have the troupe roll \roll{Wits}{Survival} (\tn[8]) to notice the tremors, and their meaning in time.
  Anyone failing must roll \roll{Dexterity}{Survival}, \tn[9], to dodge the incoming stampede.
  Failure indicates $3D6$ Damage as the stampede mauls the character with horns and hooves.
  Wolves chase the aurochs, but come into view only once the aurochs have passed.

\end{itemize}

\newBeast{\Glsfmtplural{basilisk}}% Name
  {basilisk}% label
  {many-limbed lizards, the size of a \glsentrytext{warden}'s entourage, with the stench of rotten eggs}% description
are massive, terrifying lizards.
They have a deathly odour which can paralyse anyone who smells it.
They grow to the length of four horses long, with eight powerful legs propelling their huge bodies forward like a centipede.
Their scaly bodies are particularly tough and so are prized as top-quality leather armour.
Often, the best way to deal with \pgls{basilisk} is siege weapons.

\begin{boxtext}
  The wind brings a nasty stench with it.

  ``Time to go!'', shouts the captain.
  ``There's \pgls{basilisk} about''.
  The troop draw up their tents extra quickly, and move away from the wind as snapping branches can be heard in the far distance.

  ``We can't outrun the beast, so try not to make too much noise.''
\end{boxtext}

\basilisk

\paragraph{Natural Abilities:} \pgls{basilisk}'s odour forces anyone smelling it to make a Strength check, each round, at a \gls{tn} of 10 minus the creature's Speed, or gain 3 \glspl{ep} as the character wretches or simply stares into space with disgust.

\paragraph{Ecology:}
\Glspl{basilisk} always move into the wind, so they appear downwind of any potential prey.
The more excited they become, they stronger their stench grows, and once the wind changes, anything nearby starts to cough and retch.

These massive lizards prefer hunting live prey, but will eat a corpse or eggs if they happen upon the meal.
Any time \pgls{basilisk} finds \gls{crawler} eggs low enough to the ground, they eat them, possibly bashing at a tree to dislodge them, and the \glspl{crawler} can do nothing about it.
This helps keeps crawler numbers low.
As a result, if the local \gls{basilisk} population ever diminishes, the forest will soon see an explosion in \glspl{crawler}.%
\footnote{If this ever happens in your campaign, replace any \gls{basilisk} encounters with $1D2$ \glspl{crawler}.}

Over each \gls{cFour}, \glspl{basilisk} mate in a massive, writhing, undiscriminating mass.
When \glspl{basilisk} mate, the plants that grow on them multiply too, such as \nameref{liskShine}.%
\footnote{See \autopageref{liskShine}.}
Once the warmth of \gls{cFive} arrives, they dig their eggs into the earth before forgetting that they exist and wandering away.

\Glspl{basilisk} hibernate over \gls{cTwo}, often in a pile in caverns, where skein pick at any food-scraps in their teeth.

\index[mana]{Air!\Glsentrytext{basilisk} gullets~\glsentrytext{A}}
\index[mana]{Earth!\Glsentrytext{basilisk} hide~\glsentrytext{A}}
\paragraph{\Glspl{ingredient}:}
squeezing \pgls{basilisk}'s fresh gullet-juices out provides an Air \gls{ingredient}, and their hide, if dried and powdered, produces a number of Earth \glspl{ingredient} equal to half the creature's Strength Bonus.

\showEnc

\begin{itemize}
  \item{The \gls{basilisk}'s stench approaches while the troupe see an old man in the distance.
  He waves to them in greeting, and if they approach he tells them that he wants to die a warrior's death, so he has taken his sword and decided to fight \pgls{basilisk}.
  If the \glspl{pc} do not intervene, the \gls{basilisk} comes, and eats the old man before he can lift his heavy broadsword.}
  \item
  In \pgls{village}, the archers start to gather at the walls as someone can smell \pgls{basilisk}.
  It comes fast, and the stench penetrates so strongly that one of the archers feints and falls from the wall.
  If the \glspl{pc} do not save him, the \gls{basilisk} drags him back into the forest as more arrows stick into his rough hide.
  \item
  A massive crap with white specks, the size of a barrel, blocks the road.
  Any investigation reveals a human skeleton and \mkPrice[gp]{10} inside the crap, as \pgls{basilisk} ate a trader.
  \item
  A lone trader comes up behind the \glspl{pc}, and soon after everyone starts to smell the stench of the \gls{basilisk}.
  The trader begs the \glspl{pc} to stay and fight while he runs - if the clothes infuse with the \gls{basilisk}'s smell, they will be ruined forever.
\end{itemize}

\newBeast{Bears}% Name
  {bear}% label
  {known for stalking, and theft, and sometimes aggression}% description
can be found over all of \gls{fenestra}, including populated areas.
However, they are typically not dangerous if left alone.

\bear

\showEnc
Bears are curious but typically peaceful creatures over the mild and warm \glspl{cycle}.
But during \gls{cOne}, before \gls{cTwo} brings snow, bears become recklessly hungry as they prepare for hibernation by growing fat.

\index[mana]{Fire!Bear's heart~\glsentrytext{A}}
\paragraph{\Glspl{ingredient}:}
a bear's heart, when pulled out in the heat of battle, serves as a Fire \gls{ingredient}.
Pulling the heart out in time requires a \roll{Dexterity}{Medicine} roll at \tn[10].

\begin{itemize}
  \item
  In dead of \gls{cTwo}, the troupe find an attractive cavern to rest in.
  However, the moment torch light hits the cavern's interior, they see a great bear inside, lying so still it could be dead.
  Have them roll \roll{Dexterity}{Stealth} to escape without waking the bear (\tn[8]).
  If they fail, the bear immediately attacks the largest troupe member until it has secured a meal.
  \item
  One \gls{cOne}, just before \gls{cTwo}'s snow hits, the last watchman in the night sees a bear watching the troupe.
  It does not approach -- it merely watches them, and stalks them over the next three nights.
  If the troupe ever show some weakness, or if one simply leaves to take a piss, the bear attacks, but backs off at the first real sign of danger.
  \item
  A single bear stands in a river, washing off webbing from \pgls{crawler}'s web.
\end{itemize}

\newBeast[\Sw]{Bluepins}% Name
  {bluepin}% label
  {blue-winged butterflies, known to bring good luck}% description
are blue-winged butterflies, known for bringing good luck.
As a result, people like to push a sewing pin through them, and display them on walls.

If encountered, anyone nearby receives $1D6$~\glspl{fp}.
\index[mana]{Fate!A sack of bluepin butterflies~\glsentrytext{Sw}}

Laying a trap with sweet-smelling smoke can harvest enough bluepins to make a Fate \gls{ingredient}.
The trap requires an \roll{Intelligence}{Survival} roll (\tn[12]).
On a tie, the bluepin population dwindles, and this area will not see them again for three years.

\newBeast{Boars}% Name
  {boar}% label
  {famously bad-tempered and highly defensive, but also good eating}% description
tend to inhabit the deeper forests, and pick up a number of scars from fighting with the terrifying creatures who live there.
They have to be active all year round to survive, and like most forest creatures get braver as \gls{cTwo} approaches.

\boar

\showEnc
Boars generally don't want to fight, so most encounters will be passive.

\begin{itemize}

  \item
  The troupe see a boar in the distant road, drinking from a puddle.
  If they wait a while, he leaves, but if they approach suddenly, he attacks once then flees.
  \item
  The troupe stumble upon two cute little wild piglets playing at the side of the road.
  The next moment, the mother boar screams and rushes towards them.
  If they run, she stays with her children and stops attacking.

\end{itemize}

\widePic[t]{Studio_DA/chitincrawler}

\newBeast{\Glsfmtplural{crawler}}% Name
  {chitincrawler}% label
  {arachnid the size of a horse, with wolf-like jaws and knife-like hands}% description
are large, armoured, spider-like creatures.
They have eight appendages, each ending with four-pronged knife-like `hands'.
Their jaws are more wolf-like than arachnid.

When someone sees \pgls{crawler}, it means death has come, and will be watching.
Their stride and hunger mean they can outrun a dog.
No human or gnoll can escape, except through superior numbers, and intimidation.
Therefore, seeing one in the distance means death has come, and watches carefully.

\Glspl{crawler} do not have the brains to feel fear.
When they see a group of twenty armed warriors, they really just see food.
They understand enough to not wander too close, but often approach anyway, excited by all the food they see, and unaware that swords and shields represent pain.
At this point, if those `warriors' loose their cool and run away, the \gls{crawler} will eat the slowest.

\chitincrawler

\begin{boxtext}
  You awake sticky and heavy.
  It's hard to get up.
  Looking around, you see everyone in the circle has been stuck in a massive web while they slept.
  Looking up at the treetops, you find a great, black creature descending with eight outstretched arms.
\end{boxtext}

\paragraph{Natural Abilities:} \Glspl{crawler} can slowly lay a sticky web-like substance over large areas -- often over bushes or tree branches so as to maximize contact points once something enters.
Once some creature has been caught the \gls{crawler} quickly descends.
If the prey is weak, it pulls the prey up the tree to feed on later.
If the prey might be strong enough to break free, it quickly bites through limbs to start bleeding it dry.

Noticing the web before stepping into it requires a \roll{Wits}{Survival} action, \tn[7] during the daylight and higher at night or in the twilight of a forest.
Breaking free of the web requires a \roll{Strength}{Athletics} action at \tn[7] plus the \gls{crawler}'s Strength.

\paragraph{Ecology:}
\glspl{crawler} live a solitary life in their large territories, and enjoy watching places from afar with their long-sighted eyes -- they can spend days in a new area just watching to see how many animals and of what type pass through it.

They primarily hunt deer, boar and \glspl{griffin} -- their webs can capture all of them with equal ease.
On occasion, humans have successfully destroyed enough \gls{crawler} nests that the local population dwindled.
However, this always allows the local \gls{griffin} population to flourish, which then eat all of the local deer herds, then become desperate enough to begin constant and desperate attacks on human and elven settlements of all sizes.%
\footnote{If this happens, replace any \gls{crawler} encounters with \gls{griffin} encounters until after the next \gls{cThree}, double the number encountered, and give each a +2 bonus to Morale checks due to hunger.}
\label{crawlerBalance}

All \glspl{crawler} hibernate over \gls{cTwo}, becoming part of the surrounding snow.

During \gls{cFour}, \glspl{crawler} find a potential mate, and the two travel together while the female decides whether or not to accept the mate, or eat him, or both.
The female relies on ambush tactics, while the male attempts to herd food towards her.

By the end of \gls{cFour} or \gls{cFive}, the female discards some eggs, often with a carcase wrapped up next to them.
The eggs grow, and wait for a dry day, before emerging as a swarm, and hunting together once, then going their separate ways.

\crawlerSwarm
\label{crawlerSwarm}

\index[mana]{Fate!Chitincrawler spinnerets~\glsentrytext{A}}
\paragraph{\Glspl{ingredient}:}
\gls{crawler} spinnerets (the web-spinning organs), at the back of the thorax, provide  \pgls{ingredient} of the Fate~\gls{sphere}.

\showEnc\label{chitin:tactics}
\Glspl{crawler}, despite no apparent intelligence, show incredible planning ability when laying their traps.
Of course, this comes entirely from instinct, so they show no kind of planning once an attack has begun.

As far as their insect-mind understands, tasty-things move, and most moving-things are tasty.
To them, a group of two-dozen archers looks like a herd of deer, so they will simply move out, and attack, completely unaware of the danger, until something wounds them.

\begin{itemize}
  \item
  While the group sleeps, anyone on watch notes quiet movement in the dense trees above, then they see a thin, translucent mucus gently descend upon a sleeping companion.
  Every Failure Margin on a \roll{Strength}{Vigilance} check indicates one companion has already been trapped.
  \item
    The troupe look suddenly off the path to see \pgls{crawler} waving its arms frantically.
    This male \gls{crawler} knows that a female watches them from the other side, and hopes to gain her favour; he blocks the troupe, herding them back, while she grabs one from behind and runs.

  This behaviour may seem like intelligence, but in fact it is pure instinct.
  \item
  The troupe reach a muddy slope in the road which they must descend carefully if they don't want to fall over (\roll{Dexterity}{Survival}, \tn[7]).
  They roll \roll{Wits}{Vigilance} (\tn[8]) to notice the web laid at the bottom of the slope.
  A \gls{crawler} immediately attacks from the top.
  It requires no roll to move, but every time one of the troupe move, they must make a new roll not to tumble downwards.
  The \gls{crawler} rams them to push them downhill.
  \item
  Around the end of a cold season, the troupe find a snowy mound, and spot a hibernating \gls{crawler}'s leg poking out from under the snow.
  \Pgls{crawler} waking from hibernation typically kills everything in its path in order to get enough energy to hibernate again, so they must be careful.
\end{itemize}

\newBeast{Griffins}% Name
  {griffin}% label
  {beak-wielding mammal, with bat-like wings. They hunt in packs}% description
hunt deer, badgers, aurochs and anything else they can get their claws into.
Most hunt in packs.

\begin{boxtext}
  You've found nothing these past two days.
  You're still sure you're going in the right direction, but you have at least another day's march till you reach the town.
  The smaller members of the group don't look like they can make it.

  Looking up to the trees you can see a great nest packed with eggs, as big as a man's head\ldots
\end{boxtext}

\Glspl{griffin}' long arms extend into massive leathery wings.
When sprinting on all-fours, the wings fold into the body, but once unfurled, they can take off.
Older, heavier, \glspl{griffin} often cannot fly simply by sprinting, and have to find a tree to climb, in order to take off.

\Gls{griffin} beaks have a wicked point, used to pierce their prey's jugular.
Once the bleeding starts, they back away to safety, and wait for the creature to come to a natural stop before returning to feed.

\Glspl{griffin} can mimic sounds they hear with great accuracy, and often mimic their prey in order to attract it.
When scared, they often mimic larger animals, such as bears or \glspl{basilisk}.
This habit earned them the group-noun `theatre'.

\griffin[\npc{\A}{Griffin}]

\paragraph{Ecology:} \Glspl{griffin} live in areas of tall trees where they can safely build their nests.
They are excellent climbers, learning first to scurry along trees by digging their sharp claws in and then later to glide in order to swoop down on prey from above.
By the time \pgls{griffin} has learnt to fly, it can have a wingspan of up to 15 feet.
\Glspl{griffin} like to nest at the edges of forests, allowing them to fly out and catch grazing creatures such as deer on the open planes.
\Glspl{griffin} will often hunt alone, but if a prey is spotted which is too large to catch  they will often hunt together.

\index[mana]{Air!\Glsentrytext{griffin} wings~\glsentrytext{A}}
\paragraph{\Glspl{ingredient}:}
the wings of sprightly adults can provide an Air \gls{ingredient}.
Slower \glspl{griffin} -- who must run before take-off -- do not provide useable \glspl{ingredient}.

\showEnc

\begin{itemize}
  \item
  A lone \gls{griffin} circles overhead three times, and then leaves.
  The troupe may think it has taken no interest in them, but soon after, half a dozen \glspl{griffin} descend to make one attack, and test their mettle.
  The \glspl{griffin} continue swooping down until the troupe injure one of their number.
  \item
  Walking across a mountainous range, a theatre of \glspl{griffin} decides to watch the troupe from afar.
  Once the troupe pass a difficult section, they swoop down and attempt to grab a troupe member, pulling them off the side of a ledge, and letting them fall.
  The \glspl{griffin} have no intention of fighting fairly -- they know they just need to pull someone over the edge to kill them.\footnote{You should allow players to spend 5 \glspl{fp} to avoid sudden death from cliff edges.}
  \ifnum\value{temperature}>1
    \item
    Two \glspl{griffin} circle above the \glspl{pc} then descend and attack.
    If any of the \glspl{pc} pass a \roll{Wits}{Wildcrafting} check, \gls{tn} 8, they notice the \glspl{griffin} are defending a nest above.
    If the \glspl{pc} back off, the \glspl{griffin} leave them alone (they only want to defend their nest).
  \else
    \item
    Entering the forest, the characters hear a man say `those look tasty'.
    Anything they say repeats back to them, interspersed with `those look tasty'.
    Getting closer, they find \pgls{griffin} watching from the treetops, mimicking whatever they say.
  \fi
  \item
  \ifodd\value{r4}
    \Pgls{griffin} nest lies in the treetops.
    A single egg could make a meal for three men.
    The parent has left, so a \roll{Speed}{Athletics} roll allows a troupe member to get the eggs before a parent returns (\tn[9]).
  \else
    Late at night, a voice calls out from the forest.

    \textit{``Hoo-rah, up-she-rises!''}

    It continues to sing this single verse, repeating it with exactly the same tone and pitch, then goes quiet, then starts up again.

    If any response repeats twice, the griffin mimics that response, and walks closer to hear better\ldots
  \fi
  \item
  A large griffin circles above, eyeing up the troupe's smallest member.
  It attempts to pick them up, and fly off, in a single attack.
  \item
  \Pgls{griffin} swoops down and snatches a baby in \pgls{village}.
  If the adventurers run after it, they soon hear baby-cries in the forest, but when they approach, they find \pgls{griffin} with blood covering its beak, repeating the sounds of the crying baby.

\end{itemize}

\newBeast{\Glsfmtplural{digger}}% Name
  {mouthdigger}% label
  {ambush predator which digs like a mole.  The size of a dog, but jaws expand to three times a reasonable size}% description
dig.
Imagine a large mole, opening its mouth impossibly wide, showing more rows of teeth than a shark.
These furry creatures could almost appear as a large dog if seen outside of the ground, but most people (and deer) usually only see its tonsils exploding from underneath.
Once it has its prey clenched, it immediately retreats down its narrow hole, to be eaten from the feet to the head.

\begin{boxtext}
  The ground beside you explodes in a storm of teeth, all snapping for your leg.
\end{boxtext}

\paragraph{Ecology:} \Glspl{digger}' favourite food is the auroch -- they can snip the leg-tendons from one, watch it bleed out and then crawl over and feast on it for some days to come.
Their greatest strength is that, while lying mostly underground, they are almost scentless, so creatures which coordinate by their nose have a hard time spotting the lethal trap in waiting.

Much like badgers they create long, winding paths of underground homes.
When they dig under crops, these large tunnels can weaken the ground, causing people (and donkeys) to fall into a sudden pit.

Over \gls{cFive}, male \glspl{digger} pull their quarry half-way down the earth, but leave them alive to scream.
The unique half-buried death cries attract any nearby females (this saves the male the effort of yelling, while also displaying the size of their kill).

\index[mana]{Earth!Young \glsentrytext{digger} liver~\glsentrytext{A}}
\paragraph{\Glspl{ingredient}:}
the liver of a young \glsentrytext{digger} serves as an Earth \gls{ingredient}, with minimal preparation.
Any which can open their eyes have already reached adulthood.
Of course they only birth young in the warm seasons, and the young remain below-ground until adulthood, so accessing the litter always presents a challenge.

If found, a litter has $2D6$ sprogs, and provides half as many \glspl{ingredient}.

\showEnc
There is only one way \pgls{digger} attacks -- by surprise.  They dig into a bush, build a warren, clean up the surrounding area so nobody can see the entrance, then jump out of the bush and bite.  If unsuccessful, the \gls{digger} typically crawls back into its hole.

\begin{itemize}
  \item
  While approaching a settlement, the troupe hear screaming.
  Two children were playing, and \pgls{digger} exploded from the ground and dragged one child back underground.
  The child has already died, but the troupe may wish to find a way to kill the creature anyway, so the \gls{village} can rest easy.
  \item
  As night falls, the first watchman notes \pgls{digger} out in the open, and exposed, scuttling into one of its holes.
  By morning, the entrance to the cavern looks like a normal bush.
  The watchman can do whatever they want with this information.
  \item
  \Pgls{digger} jumps up, and bites a heavily armoured troupe member, then drags them underground.
  The character will probably still be alive, but the troupe still have to get them out while the creature pulls its victim further downward.

\end{itemize}

\mouthdigger

\newBeast{Stirges}% Name
  {stirge}% label
  {insects with size of a fist, with a blood-draining stinger}% description
grow to the size of a farmer's fist.
Their wicked-sharp sucker can nip and slice into any exposed skin they can find, draining life slowly.
And while most build their hives from leaves and mucus, they will use skin and bone if they find a corpse.
This creates an ugly mound of femurs, fibulae, and ribs, all buzzing with activity.

While queens don't leave the nest, their eggs do.
Hives will often reproduce by male stirges, who carry eggs on their stingers in order to inject them into the softest spot they can find on a mammal.
The eggs will lay dormant for about a week, quietly feeding off their host's blood-stream, until they have enough material to develop a miniature nest inside the creature.
They then use the host as a mobile nest, as a miniature swarm develops inside the creature, growing larger, until the creature finally dies, and the swarm feeds on its corpse to grow more drones, and uses the bones and skin to form a new nest.

\label{stirgeEggs}
Any sting by a swarm of stirges has a 1 in 6 chance of injecting an egg.
However, not many stirges carry eggs in their stingers, so once they successfully inject the eggs, no other targets will suffer the same fate from the same swarm.

Checking a stirge-sting for eggs requires rooting around inside the wound.
Anyone checking for eggs makes a \roll{Dexterity}{Medicine} roll (\tn[12]).
Success inflicts a point of Damage, but will identify whether or not any eggs lie inside the wound.
Failure inflicts 3 Damage, and always indicates that the wound has no eggs (whether or not it does).

\paragraph{Natural Abilities:}
Stirges fly, and do so only to press their stingers into animals, and drink blood.
Those attacked by stirge-stingers gain \pgls{ep} instead of Damage, and may become the vessel for their eggs (see \vpageref[above]{stirgeEggs}).

\paragraph{Ecology:}
Over the cold months, the queen hibernates while the hive dies.
She eats some eggs, and leaves others to hatch, and begin hunting for blood.
Stirges will suck the blood from anything which moves.

When the weather becomes really hot, stirges multiply, creating new colonies.
Throughout this time, they attack anything which moves.

\index[mana]{Fire!Stirge queen~\glsentrytext{Sw}}
\paragraph{\Glspl{ingredient}:}
freshly caught stirge queens makes for a potent Fire \gls{ingredient}, which provides the power of two.
Using this \gls{ingredient} to make \pgls{boon} will yield two, but using it to make \pgls{talisman} requires some care; when \pgls{witch} uses two \glspl{ingredient} instead of  one, the excess energy always produces an effect of some kind.%
\exRef{stories}{Stories}{randomSpellFailure}

\stirgeSwarm

\newBeast{Wolves}% Name
  {wolf}% label
  {opportunistic food-thieves, who also drain an area of deer}% description
spend most of their time following deer, but people mostly know them for thieving from farms.

\paragraph{Ecology:} Wolves live everywhere except islands.
In the cold seasons, many die, and they become rarer, but in warmer seasons they breed and run quickly.
While farmers know them as thieves, the majority of the time, wolves hunt wild animals, like boar or deer.

\begin{boxtext}
  A feint scratching sound is heard -- only a little louder than the crackling fire.
  Rucksage is wandering away into the distance, and as the fire flares up you see the face of a wolf in the shadows, dragging his baggage away.
  A dozen wolves gather and tear the bag open, pulling the food out.
\end{boxtext}

\wolf[\npc{\A\T}{Wolves}]

\showEnc
During warmer seasons, wolves never attack humans.
Encounters may involve seeing wolves running in the distance, but nothing more.
When food becomes leaner, they become braver, and try to steal food.

\begin{itemize}
  \item
  One night on watch, a pack of wolves stalks the troupe.
  If the watchman fails the \roll{Strength}{Vigilance} check to stay alert, one wolf grabs the smallest troupe member's backpack and flees.
  \item
  Yelps echo around the forest.
    Two wolves have been caught in \pgls{crawler}'s web, and the rest of the pack cannot free them, so they just cry and bay loudly.
\end{itemize}

\widePic[t]{Studio_DA/woodspy}
\newBeast[\E]{\Glsfmtplural{woodspy}}% Name
  {woodspy}% label
  {land-octopus, able to camouflage itself by changing texture, colour, and size}% description
are like a type of camouflaging octopus, adapted to exist solely on land.
They have three to eight tentacles and transform their shape in the manner of a hand making a shadow puppet and then adjust their skin-texture to match.
They grapple with opponents and then start gnawing into prey with a powerful beak, located underneath the main body.

\begin{boxtext}
  The tree to your side shifts, and tentacles reach out of it.
  Half of the trunk was really a translucent creature, waiting to grab you and pull you up the tree, far away from the rest of the troupe.
\end{boxtext}

\paragraph{Natural Abilities:} \glspl{woodspy} are too soft to hurt creatures through punching -- they must first grapple and then start sticking their beak, held underneath the main body, into the target.
However, their many limbs allow them to grapple creatures without becoming vulnerable.%
\exRef{core}{the Core rules}{grappling}

Noticing \pgls{woodspy} while camouflaged requires a \roll{Wits}{Vigilance} action, \tn[14]; bonuses can be assigned for sunny or particularly open areas.

\woodspy

\paragraph{Ecology:} \Glspl{woodspy} inhabit all manner of areas but prefer open plains and forests where there is plenty of cover and plenty of food.  They always hunt alone.  Despite being animals, they are in fact rather intelligent, although this intelligence only knows how to watch and calculate -- they do not communicate much.

\Glspl{woodspy} love fish, deer, gnomes, badgers, humans, and aurochs.
Animals with too sharp a bite (such as wolves and bears) typically put up too much of a fight for \glspl{woodspy} to bother.

\index{Rivers}
Despite being octopods, fast-running water presents a problem for \glspl{woodspy}.
It throws them about, and removes their control.
So despite their ability to breathe underwater, and their ease of moving in the sea, they rarely attack boats on a river, and never venture to the ocean's surface during a storm.

During the cold seasons, most venture underground to half-hibernate with slow movements, while others remain active.
Their initial excursions bring some much-welcome excrement to caving systems, feeding slimes, lichen, fungi, and myriad insects.

\index[mana]{Water!\Glsentrytext{woodspy} beak~\glsentrytext{E}}
\paragraph{\Glspl{ingredient}:}
the beak of \pgls{woodspy}, once ground into a fine powder, provides a number of Water \glspl{ingredient} equal to the creature's Speed~Bonus.

\showEnc[\E]
\noindent
\Glspl{woodspy}' keen intelligence allows them to plan attacks like no other creature.
They will use every part of their environment and the troupe's exact situation to launch the perfect attack.
They know their prey well, and tend not to attack when they feel endangered.
If something looks to dangerous to subdue, \glspl{woodspy} often follow their prey for some time.

Every \gls{woodspy} which has seen an archer will recognize the bow, and understand how to hide behind cover.
They also remember what creatures value, and may lure wolves by dangling bits of meat around them, or catch humans by dropping coins next to a road.

\begin{itemize}
  \item
  A single tentacle reaches down, grabs the smallest troupe member, and lifts them up into a tall tree.
  The \gls{woodspy} then focusses only on climbing the massive tree.
  A \roll{Speed}{Athletics} check is required to follow the creature upwards, and only two troupe members will be able to approach at a time.
  \item
  In the distance, the troupe notice a massive \gls{woodspy} slowly descending a tree.
  It waves its tentacle around a bush, slowly.
  \Pgls{digger} rushes up to bite the tentacle, but the \gls{woodspy} quickly grabs it, then pulls it up into the trees.
  \ifnum\value{temperature}=1
    \item
    The troupe approach a river with a strong current.
    Have them roll, \roll{Wits}{Vigilance}, \tn[9].
    Success means they have spotted \pgls{woodspy} upstream, plopping into the water in order to lay an ambush.
    Even knowing that the ambush exists will not guarantee safety, since the troupe cannot move easily in the water.
    If they wander downstream, the \gls{woodspy} follows.
    \item
    \Pgls{woodspy} picks up a bush, then waves it around, provoking a fight with the \glspl{pc}.
    If they approach the bush, it uses the diversion to grab any bags they leave on the ground.
    The creature then climbs a tree, and tears open the bag, rummaging around it for any food, while throwing the rest on the ground.
  \else
    \item
    As the troupe approach \pgls{village}, a nearby bush reveals itself to be a smaller \gls{woodspy}, and flees.
    If they let the creature go, the next day it returns and takes a less well-armoured victim by surprise.
    \item
    Deep in the forest, \pgls{woodspy} has drifted into a deep sleep, and as it dreams unknowable dreams, it flickers through rainbow colours, and changes its skin-texture.
    \item
    The troupe discover a deer-corpse, writhing with little \gls{woodspy} babies, each the size of a hand.
  \fi
\end{itemize}

\end{multicols}

\section[Labyrinth Creatures]{Labyrinth Creatures \A}

\begin{multicols}{2}

\noindent
The underground world is a sandwich.

The top layers pull in a little trickle of food from above, and nutrient-rich streams flow through, producing rare fungal gardens, or little cave-dwellers leave droppings a little inside, producing food for the wandering acidic oozes.
Any passage leading downwards eventually sucks down any organic matter, and usually produces a slimy slope -- a natural trap, which pulls people into the labyrinth below.

Below, a cold empty desert, full of crooked, empty veins, yawns.
A few creatures crawl down to these depths to hibernate in peace.
Occasional dwarves leave their life's savings down a long, dark passage.
And rare undead wander into the darkness to stand, and stare, in meditation.

This freezing, frostless, labyrinth, creates a game with strict rules for all its inhabitants.
Any loud noise can travel for miles underground, so predators often stay still, and listen well.
But if they wait too long in an empty patch, they will starve.
Prey-creatures play the same game on the other side, waiting, breathing silently, and listening for any small sounds of movement in the distance to indicate that something moves either away from them, or towards them.
And once a chase begins, both predator and prey can only run if they know the area well; the home-team commands a striking advantage, since they don't have to feel for the road ahead as they run.

Of course, fire is an option, but it produces its own complications.

Far below the frigid labyrinth, the air becomes warm again.
People don't go beyond this point and return.

\widePic{Studio_DA/jelly}
\newBeast{Acidic Oozes}% Name
  {ooze}% label
  {uniform gastropod, with just enough mind to move towards movement and digest it}% description
are mobile, gelatinous, blobs with about the same level of intelligence as a quick-thinking bush -- sometimes they ooze along the underground caverns, sometimes they crawl into rooms to digest their latest victim and appear as a little pool of water hidden away in a small pit.

\paragraph{Natural Abilities:} Young oozes are generally partially transparent, especially when hungry; spotting them requires a \roll{Wits}{Vigilance} action, against their \roll{Wits}{Stealth}.
Older oozes are hardier, and larger, but less good at hiding.

Some ooze also have the Projectiles Skill, indicating they can shoot acid up to 5 squares away.

In combat oozes always slither blindly at people, shifting about randomly so as not to be hit, and wrestling with their targets.
Once a target is grappled, the ooze inflicts damage equal to its Strength immediately, and again at the start of each turn.

Larger oozes also heal 1 \gls{hp} per round by melding back together, but can only regenerate half their \glspl{hp} this way before they need to feed.
Their forms can still be bruised, and sufficient separation will kill them forever.

An ooze's greatest weakness is fire.
They suffer $1D6$ Damage from any torches, and receive no \gls{dr} when attacked with fire.

\paragraph{Ecology:} There are a large variety of ever-evolving gelatinous creatures which inhabit the underground realms.
Most eat the underground mushrooms and the scum from underground lochs.
Many skulk along ceilings and drop on anything which passes beneath them.
They possess less sentience than the simplest insect and operate by a simple set of rules involving moving towards anything which smells like an edible, and (if the ooze is especially clever) moving away from anything that hurts too much.

Oozes can also be found occasionally out in the ocean, slowly digesting any fish (or fishermen) who wander into them.
They may not be fast, but once they have someone it can be difficult to get out of their grasp.

They can occasionally eat glowshrooms,%
\footnote{Glowing mushrooms, \vpageref{glowshroom}.}
which makes them far less able to take people by surprise, and can even provide a mobile light-source as they follow a troupe, trying to eat them while illuminating the path.
The ooze's slow digestion makes these mushrooms glow even brighter, which can be a great help when wandering in the dark, as long as the troupe can run faster than the ooze, and as long as they don't need to rest.

Oozes, while highly acidic, are a dwarven delicacy once prepared properly.

Few people know this, but oozes are the only creatures in \gls{fenestra} with a longer natural life-span than elves.
They continue to grow anytime they can eat, occasionally amalgamating with other oozes, or spawning smaller oozes after enveloping another.
An ooze which cannot `mate' in this fashion will simply continue to grow until something stronger comes along to kill it.

\jelly

\index[mana]{Fate!Black and brown oozes bodies~\glsentrytext{Sw}}
\paragraph{\Glspl{ingredient}:}
black and brown oozes, once collected and left without air for a season, provide excellent Fate \glspl{ingredient}.
How much depends on how much of the body one can capture, but it could produce half as many \glspl{ingredient} as it had \glspl{hp}.

\showEnc

\begin{itemize}
  \item
  A nearly invisible ooze sleeps on the cave's ceiling which the troupe walk under.
  Have them roll \roll{Wits}{Vigilance} to notice before it drops on them.
  \item
  The troupe find a dark and lighter ooze intertwined.
  It's not clear if they're mating, fighting, or if the darker one is giving birth to the other, but if the troupe disturb them, they both attack.
  \item
  On the ground is a puddle with ten \glsentrylongpl{gp}, a silver ring, a dagger's blade, and helmet.
  This is an ooze, along with the remains of a dwarf.
  If the troupe approach, they may become its next meal.
\end{itemize}

\jelly

\newBeast[\Sw]{Skein}% Name
  {skein}% label
  {blind cave-lizards, which attack all flames}% description
grow to the size of a hand.
They use delicate little tongues to pick up tiny insects and debris from cavern floors, ceilings, and streams; and delicate spines on their fingers let them climb anywhere.

\skeinSwarm

Skein have no eyes but still hate torches.
They sense the heat through their translucent-pink skin and instantly attack anything next to the fire.

\newBeast[\E\Pl]{Spore Folk}% Name
  {spore_folk}% label
  {self-made fungal masses, blind and hyper-dimorphic}% description
are self-replicating fungal \glspl{artefact} which absorb the feint mana within deep caverns.
They mix underground \glspl{ingredient} with mushroom growths to make \glspl{talisman}, which then become more spore folk.
Their bodies and minds have such variety that humanoids often fail to understand that they form a single colony.
The reverse holds too -- spore folk struggle to tell the difference between a gnome and a gnoll.

They communicate like any \gls{artefact} (see \vpageref{dead_communication}), which will make speaking with them difficult.

They run the range of intelligence, from insect to scholar, and develop social skills about the same rate as intelligence.
As a result, most communities conflate intelligence and ethics entirely.
They will treat `bad people' as if they were simply ignorant rather than malicious, but also take every mistake as a sign of evil intentions.
If a troupe communicate that they have become lost, the spore folk will take this as a sign of malicious intentions (because anyone who becomes lost does not know the proper way to act), while giving gifts to them will make them inclined to believe statements the troupe make (because people who know good manners should also know what they are talking about in general).

\paragraph{Ecology:}
Spore folk colonies are smart enough to remain untroubled by most underground creatures.
Their primary enemy is gnomes, who struggle to resist the temptation to study and pull apart these living \glspl{artefact}.

Once dead, spore folk release any \glsentrylongpl{mp} in their bodies.

\paragraph{Children}
are the \glspl{talisman} which have started to develop sentience.
The other \glspl{talisman} don't matter -- all exist to protect the colony.
When \pgls{talisman} starts to form sentience, then the colony will place it somewhere secluded and safe.

\wotWosFungus

\showTalisman

\showEnc[\glssymbol{plant}]

\begin{dlist}
  \item
  They give their children to the troupe in order to start a new colony somewhere else.
  They try to explain this (by thinking very loud), and if that fails, they push their children in the troupe's back-packs or arms.
  If they see the troupe harm these blobs of fungus, they will take the children back.
  \item
  They trap the troupe until they defecate, by using spells, blocking passages, et c.
  \item
  They give the troupe edible fungus in order to help them.
  The fungus may or may not be edible (spore folk don't know much about what flesh-people like to eat).
  \item
  They try to trade.
  They have some bones and metal left over from people who died in a cavern some time ago (weeks? centuries?), and will accept food, non-alcoholic water, \glspl{talisman}, \glspl{artefact}, or meat.
  \item
  They ignore the troupe.
  \item
  They try to communicate with the troupe, starting with Mathematics, and working their way up to full conversations.
  The troupe probably don't have time for this, but the spore folk will persevere as long as the troupe remain, and will not let them sleep.
\end{dlist}

\sporeFolk

\sporeFolk

\sporeFolk

\newBeast{Umber Hulks}% Name
  {umber_hulk}% label
  {beetle-like creatures, the size of a wagon}% description
are massive insect-like creatures with gnashing mandibles.
They eat mushrooms and acidic, underground oozes, but devour absolutely anything if it happens to be near.

\begin{boxtext}
  Small, sticky orbs hang from the ceiling.
\end{boxtext}

\paragraph{Natural Abilities:} When stressed or angry, an umber hulk's odour forces anyone smelling it to make a Strength check, each round, at a \gls{tn} of 10 minus the creature's Speed, or gain 3 \glspl{ep} as the character wretches.
This stench quickly fills tunnels, and can serve as an early warning sign, but not always.
The stench only occurs when the creature starts galloping.

\paragraph{Ecology:}
They move quickly, eat quickly, and worst of all, their eggs hibernate until food is near.
This makes them difficult to get rid of once a tunnel system has been infested.

They naturally live underground, where their eggs can grow undisturbed, but may venture miles beyond their caverns when hungry.
They encourage their eggs to hatch by heating them through rubbing their massive, armoured, limbs together, to produce a vibration, and a strange kind of purr.

Due to heat encouraging the eggs to hatch, anyone wandering nearby with a torch often encourages the entire brood to exit their shells immediately, and cry for their parents.

\begin{boxtext}
  Once you crack one open, you find a tiny little insect with a soft, white shell.
  The smell on the inside fills the entire tunnel, and a second later a fast clatter comes from farther down the tunnel.
  It's getting closer.

  You turn to see an insectoid creature, bigger than any war horse, with a solid, black shell, racing towards you.
\end{boxtext}

\index[mana]{Fire!Umber hulk eggs~\glsentrytext{A}}
\paragraph{\Glspl{ingredient}:}
unhatched umber hulk eggs can produce Fire \glspl{ingredient}.
The quantity depends on the brood-size and how close they are to hatching.
The result is $2D6-5$ Fire \glspl{ingredient} (minimum of~1).

\showEnc

\begin{itemize}
  \item
  Ahead, an umber hulk lies in wait for the troupe.
  While they are not masters of stealth, simply sitting in the dark works well for them.
  Have the troupe roll \roll{Wits}{Vigilance}, \tn[9] due to the twilight.
  \item
  An umber hulk sees the troupe and gives chase.
  After they run for a little while, it stops, and refuses to chase them further.
  Observant characters will notice droppings on the cave's floor which indicates this is the edge of the umber hulk's territory.
  Whatever new creature has staked a claim to this land terrifies the umber hulk.
  \item
  An umber hulk is masticating a dead gnome's leg.
  He was clearly rich, as the cloth bag by his body lies open, showing a little pile of \glsentrylongpl{sp} and two scrolls.
\end{itemize}

\umberhulk

\newBeast{Watchers}% Name
  {watcher}% label
  {half-animal, half-plant. They release hallucinogenic toxins when stepped on}% description
seem something between a nearly-sentient plant and a morbidly slow creature.
They spend most of their lives still, tentacles outstretched like scrawny roots across some tens or hundreds of feet.

\paragraph{Natural Abilities:}
Once a tentacle is stepped on, the creature slowly lets out a hallucinogenic gas to confuse their targets.
A \roll{Wits}{Vigilance} task, \tn[10], is required each round to not waste one's time running from, attacking or conversing with mental illusions of one's own making.
The hallucinations vary, but whenever a player mentions some danger -- whether a trap, monster, or enemy, they appear, typically in the distance, around the edge of their visions.
Meanwhile, everyone involved suffers 4 \glspl{ep} each round.

Watchers can be difficult to spot; they like to hide in darkened corners and pretend to be a shrub or a pile of rubbish -- finding them requires a \roll{Wits}{Vigilance} Action at \tn[10].
Their gases fill massive areas, so it can often be a time race to either escape or find and destroy the creature.
Destroying it of course just releases more gas, but at least potential victims have the joy of knowing their corpse will not be slowly eaten while paralysed with fatigue by an alien face with too many eyes.

\paragraph{Ecology:}
All the deadly gas which watchers release comes from the air.
Paradoxically, they do an excellent job of \emph{purifying} underground air, as they collect all the toxic particles.

Savvy underground wanderers know to keep a sharp eye out when the air smells sweet despite walking two miles below the surface.

Despite their lack of size, strength, mobility, or any kind of sentience, watchers are the only natural predators umberhulks have.

\pic{Decky/watcher}

\watcher

\index[mana]{Water!Watcher gas-sack~\glsentrytext{A}}
\paragraph{\Glspl{ingredient}:}
if someone grabs one and contains it, before the gas releases (\roll{Dexterity}{Caving}, \tn[10]), the gas sack serves as $1D3$ Water~\glspl{ingredient}.

\showEnc[\glssymbol{plant}]

\begin{itemize}
  \item
  The troupe enter a cavern and rest for the night, noting a number of odd fungi.
  Once the group sleep, the member taking watch begins hallucinating all the late-night encounters they have ever had coming to feed.
  Once they wake other troupe members, each rolls to come to their senses.
  \item
  The troupe wander through a cavern and notice the little tendrils of watchmen lying everywhere.
  Have them roll \roll{Dexterity}{Stealth} not to step on any tendrils.
  The exit lies 40 squares away.

\end{itemize}

\end{multicols}


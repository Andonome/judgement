\chapter[\Glsfmtplural{fiend} in the Forest]{\Glsfmtplural{fiend}}
\label{fiends}

\toggletrue{genExamples}

In a world of wandering monsters, sensible habits let people outsmart most of them.
One can predict beasts, but not calculating, intelligent enemies.

These sentient creatures live outside of civilizations, and feed on it.
Unlike beasts, they do not usually stop to feed after killing one person, but can kill entire caravans, or sometimes whole \glspl{village}.

\subsubsection{Fiendish Plans}
provide suggested events and rumours.
If the troupe approach too close, they may become part of the plans; otherwise, let them hear about the results of \gls{fiend}'s plans through local gossip.

The plans have no hard rules or duration.
Once a plan reaches fruition or failure, you can roll another, and note down the current events on your \gls{gm} sheet.

\section{\Glsfmtplural{bandit}}
\label{bandit}

\begin{multicols}{2}

\noindent
\Glspl{bandit} \glsdesc{bandit}.

Many bandits begin life the same way:

\begin{enumerate}
  \item
  One freezing-cold \gls{cTwo} leaves \pgls{village} without much in the way of supplies.
  \item
  People with alternative places to stay, leave.
  \item
  A trader arrives, and sees a desolate \gls{village}, with nothing to trade.

  At this point, the remaining farmers have two options, and they know it:
  \begin{itemize}
    \item
    Let the trader return to town, and inform all the other traders not to bother with that desolate little \gls{village} on the outskirts ever again.
    \item
    Take the trader's items and survive.
  \end{itemize}
  \item
  The \gls{guard} soon come to destroy the newly formed `bandit \gls{village}'.
  \item
  Assuming the farmers are sensible, they leave, set up camp somewhere in the forest, and set up some new homes there.
\end{enumerate}

These groups usually don't live long, but some manage to create a long-term place of residence, and plague all the \glspl{village} around for years, or even decades.
Once they've set themselves up with a long-term base, they form the same relationship to civilization as any other \glspl{fiend} with one important difference -- they can walk into any town and trade, just as long as nobody recognizes their faces, or their goods.

\index{Brigands}
People sometimes make a distinction between bandits and brigands, where the former begin as farmers who take to robbery, and the latter start as members of the \gls{guard}.

\humanfarmer[\npc{\E\Hu}{Bandit}]

\humansoldier[\npc{\E\Hu}{Seasoned Bandit}]

\subsection{Plans}

\sidebox[17]{
  \begin{boxtable}
    Frigid & +4 \\
    Mild & +1 \\
    Warm & +0 \\
    \hline
    Decimated & +1 \\
  \end{boxtable}
}

Roll $1D6$ to find out what the bandits want this week, with Bonuses for the temperature, because people show their teeth once hungry.
And if the bandits ever suffer serious losses during a fight, add another~+1 to the roll.

So right now, the bandits want\ldots

\begin{dlist}
  \item
  To split the community in two.
  The roads cannot sustain them all, so they will build a second encampment, as far away as possible, and half the group will move.

  Place another bandit \gls{village} on the map.
  \item
  To just relax for a day.

  They enter the largest town, posing as foreigners, and enjoy the bars,markets, and gossip for a day.
  \item
  To sell vegetables.

  They've had a good harvest this year, and want to sell the excess.
  They lie about where they've come from.
  \item
  The troupe reach \pgls{bothy}, and find it already full of `traders', going to sell rare fabrics.

  Before morning, they pull weapons out, and attempt to rob the most valuable items of anyone staying there.
  \item
  To rob \glspl{bothy} at night.

  They lay quietly, and simply wait for any traders to leave their gear outside.
  Then they wedge the door shut from the outside with a specially-shaped piece of wood which pushes against the frame.
  People inside only hear the cries of horses.
  \item
  To get better weapons.

  The bandits surround the \glspl{pc} on the road, but act like traders.
  They will in fact pay a good price for the weapons if the \glspl{pc} allow them.
  If the \glspl{pc} refuse politely, the bandits quietly request to purchase weapons at a more convenient location for the \glspl{pc}.
  \item
  To break their friends out of \pgls{broch}.

  The \glspl{guard} have $2D6$ bandits imprisoned at the top of \pgls{broch}, waiting for the \gls{sunGuard} to come and take them to the nearest town's \gls{court}.
  \item
  To gain a fierce reputation.

  They begin by changing extortion tactics.
  Only one bandit emerges to demand that traders hand over their goods.
  The rest hide nearby, but will not emerge.
  If the traders refuse, the bandits create a storm of arrows, then pillage.

  Soon, everyone in the area will know to hand over their goods to even a single bandit, which lets a single bandit carry out a robbery.

  \item
  to lay siege to \pgls{village}.

  The siege lasts $1D6$ days before the \gls{village} gives in, and hands over their remaining food.
  \begin{enumerate}
    \item
    The \glspl{pc} hear about how traders do not return from that region.
    \item
    Someone escapes to inform everyone of the siege.
    \item
    If the siege was successful, \pgls{jotter} asks the \glspl{pc} to deliver food from the nearest town.
    But if the \glspl{pc} take more than 3 days, then the \gls{village} falls -- some flee to a nearby town, some join the \gls{guard}, and some become bandits themselves.
  \end{enumerate}
  \item
  To kidnap a seamstress.

  The bandits' \gls{village} has sheep and wool, and even a loom, but the few women there don't know much about making clothes.

  \begin{enumerate}
    \item
    The \glspl{pc} hear about the kidnapping, and that the \gls{templeOfFrost} will pay well for the return of the kidnapped woman.

    \item
    If they find the bandits' \gls{village}, they find the woman irritated, but otherwise fine; and find the bandits with a lot of fancy,warm clothes.
  \end{enumerate}
  \item
  To kill \pgls{jotter}.

  The bandits have spies in the \gls{guard} who told them of the \gls{jotter} who continuously requests that \gls{guard} rangers hunt for them.
  \begin{enumerate}
    \item
    The \gls{jotter} in question requests the \glspl{pc} hunt for bandits in a particular region (the rough area is correct, but hunting will be difficult - \tn[14]).
    \item
    The \gls{guard} spy meets the troupe at a cross-road, and asks them which \gls{jotter} was at their last \gls{broch}.
    An honest answer means he heads straight towards her, and the \glspl{pc} hear of her death soon after.
  \end{enumerate}
\end{dlist}

\end{multicols}

\section{\Glsfmtplural{lich}}
\label{lich}
\index{Liches|textbf}

\begin{multicols}{2}

\noindent
Corpses which stave off decay continue to rot any time the soul exercises the body, so once a lich creates itself, it begins with a fresh countdown to its own complete and natural death.
A lich's activities revolve around staying cold and motionless.
They typically find a lair underground, in the cold depths, and attempt to gather servants who can act on their behalf, and help them gain more power.

Their every night is spent completely aware of their own unmoving body, bored, irritated, but so scared of their own death that they continue lying motionless, with nothing to do but plan, and cast whatever spells require no motion.
Luckily for them, the dead can cast spells without the need to articulate or gesticulate in the usual way.%
\footnote{See \autopageref{dead_communication}.}

Despite the stale mana in caves, liches can gather enough to cast a \textit{Witness} spells each night, and discover a lot about their surroundings.
Many start to amass their power by acting as information brokers.
\Glspl{seeker} who pull at a source of information too much can find themselves face-to-face with a corpse which speaks directly into their minds.

\lich

Once they have a cool, dry place to store their bodies, they begin to amass short and long term armies.
The short-term armies consist of any powerful creatures in the area (gnomish alchemists, \gls{guard} leaders, dwarven merchants).
The long-term always consists of armies of ghouls `on-ice', and never placed far from the lich's body.

\showStdSpells

\subsection{Plans}

Roll $1D6$ to find the lich's plans.
Liches like subtle plans, which can fester to fruition without anyone noticing, but they never entirely succeed; so once a plan has begun, drop as many clues as you can for the players, but make each one less remarkable than sheep-shit in a farm.

If the lich's plan will take too long for the \glspl{pc} to get involved, just roll another; liches love multitasking.

\begin{dlist}
  \item
  To create an untraceable identity.
  \begin{enumerate}
    \item
    Servile ghasts take letters to a shaft made by gnomes.
    Three gnomish warrens collect their letters in a nexus, then send them up the shaft with pulleys, and the ghast drops the letter into the pile half way up the dark passage, where nobody can see the added mail.
    \item
    The mail goes to a city which no longer exists, but the dwarves at the top know where it should go, and send it along.
    \item
    Once the letter arrives at the wrong place, it must be returned.
    The return address on the back takes it to the intended location.
    \item
    \Glspl{jotter} and \glspl{seeker} have uncovered three conspirators, working for someone with different aliases, all with the same strange handwriting, but cannot understand how they all receive their coinage and instructions through letters sent from themselves, to a non-existent location.
  \end{enumerate}
  \item
  To create decoy copies of itself.
  \begin{enumerate}
    \item
    A \gls{guard} group (and now, local heros) managed to track down the lich, scarred its face, and cut its hand off.
    One of them still wears the lich's hand as a war-trophy.
    The lich escaped, using magic and minions.
    \item
    The lich begins capturing people with a similar sex, height, and build to what it had in life, to create duplicates.
    It turns them into ghasts, scars them across the face, and cuts their hands off.

    One escaped, and when someone killed it, rumours circulated about the man who `killed the one-handed lich'.
    \item
    The lich creates multiple fake bases, each of them with a crypt, a few traps, and undead guards.
    Each one receives a few \glspl{talisman}, each ready to target the first living creature inside the room.
    And in the centre of each crypt, another ghast with one hand, about the same height and build as the lich.
  \end{enumerate}
  \item
  To destroy a town, utterly, drink every soul, raise them as ghouls, and add to his undead army.
  \begin{enumerate}
    \item
    It begins with a secret road, from its secret lair, to a quiet place, three miles upstream from the town, where weeping willows hang over the river.
    He marches a dozen ghouls back and forth until a road forms, although a couple forget what they were meant to be doing, and end up floating down river.

    Everyone gossips, but nobody checks.
    `Some dead thing is up to something, I'll bet'.
    \item
    Ghasts dig underneath the local town's river, under the grid that bars entrance from the water.

    When \glspl{woodspy} slip in and grab people, they complain to the \gls{warden}, who responds by raising taxes.
  \end{enumerate}
  \item
  To start an honest business.
  The lich simply orders the creation of a well, with a large bucket, not far from a crossroads.
  A wooden sign displays prices on the side.

  \begin{multicols}{2}
    \begin{nametable}[cX]{The Ends Well}
      3~\glspl{sp} & Complete  \\
      6~\glspl{sp} & Unspoilt  \\
      1~\gls{gp} & Living      \\
      2~\glspl{gp} & Magical     \\
    \end{nametable}
    \columnbreak

    Local bandits start capturing people to put in the well.
    The \glspl{sunGuard} do the same with criminals.
    Before long, the word is out, and parents threaten disobedient children with the well.
  \end{multicols}

  Nobody admits to paying into the well, but everyone knows about it, and very old coins are starting to cause inflation.
  \item
  To finally send his crystal-eyes across the land.

  The lich collected rare blue crystals, known as `blue-bloods'.
  These Water \gls{sphere} \glspl{ingredient} shine like the sea, so it adds some gold to make them into beautiful necklaces and \glspl{talisman}.
  \index[mana]{Water!Blue-blood crystals}
  \begin{enumerate}
    \item
    Each necklace gathers rudimentary information about the numbers of people they see (though they cannot hear or understand anything around them).
    \item
    Once a necklace sees someone with \glspl{mp}, they transform the wearer into a monster.
    \item
    After a while, the lich hopes to have gathered good information by word-of-m\ldots \gls{talisman}, and put a nasty wedge between any local \glspl{doula} and the kinds of people who wear fancy blue necklaces found in the forest.
  \end{enumerate}
    \talisman{Blue-Blood Necklace}% Name
      {Devious, Duplicated}% Enhancements
      {Wax}% Action
      {Earth, Water}% Spheres
      {\roll{Strength}{Medicine}}% Resistance
      {Once the necklace activates, it cracks, and the curse begins, slowly.
      Over \showOnset, the wearer becomes massive, hungry, and full of hunger, demanding \arabic{spellPlusOne} extra meals a day}% Summary
      {}% Details
    \showTalisman
  \item
  To expand his mortal connections through books on \gls{alchemy}.
  \begin{enumerate}
    \item
    The lich leaves simple \gls{alchemy} recipes for preserving food and freezing water in libraries and dropped on the road (as if they had fallen off a wagon).
    \item
    The lich can detect that particular potion's use at long range, and slowly encourages anyone who uses the recipes, dropping new books, and making small requests for \glspl{ingredient}.
    \item
    Everyone who tries to make use of the books gets the recipes wrong some of the time, with standard mishaps.
    Most botched alchemical jobs give people in the area headaches, or strange visions; the effects might include any of the level 1 Death spells.%
    \exRef{core}{Core Rules}{Death1}
  \end{enumerate}
\end{dlist}

\end{multicols}

\section{\Glsfmtplural{dragon}}
\label{dragon}

\begin{multicols}{2}

\widePic{loh/dragon}

\noindent
When a dragon lands on a farm, and just starts eating sheep, there are two types of reactions.
Some people scream, because it seems like the right thing to do.
Others do nothing, because there is nothing to be done.

Dragons have little need of sheep.
They can kill local \glspl{basilisk}, or hunt deer with ease.
Still, they descend on farms, eat some animals, and sometimes the farmer, because they get bored.
Nobody ever asks a dragon how their day is going, but if they did, they would often find the dragon bored.

Their association with the sky, and endless destruction, leads people to think of them as divine.
Some even think that the gods made the world just for dragons to play.

\paragraph{Natural Abilities:}
Dragons can sprout a spark of flame from their mouths.
Coupled with the Fire sphere, this allows them to produce fire from their mouths.

Dragons can usually fly, although many become too heavy to simply take off.
After too many sheep dinners, they have to reach a high location in order to take flight.

\paragraph{Ecology:} Dragons are generally solitary creatures, though people have nightmares of this situation changing constantly.
They come from distant lands and like to create fanciful tales of their homeland in order to scare humans or irritate gnomes.
It is said that older dragons sometimes transform themselves into humans and walk abroad in the land, pretending to be alchemists, bards, \glspl{warden} or just whatever strikes their fancy.
Rumours abound of their constant interference in guild politics, though if half the rumours were true there would be more dragons than horses in \gls{fenestra}.

Dragons have their own language, made to communicate with a lipless hiss, and some tail-flicks.%
\footnote{Multiple dragons have claimed that they -- personally -- taught gnolls how to speak.
Despite the obvious lie, the two languages are close enough to be mutually intelligible.}
\index{Languages!Gnolls}
This has provided a lot of false hope to people who think they can negotiate with them.

\dragon

\showStdSpells

\subsection{Plans}

Roll $1D6$ to see what the dragon wants.

\begin{dlist}
  \item
  Just to sleep in peace for the season.
  Nothing will wake it, except someone approaching.
  \item
  To set \pgls{village} ablaze and watch people run about!

  As the troupe approach, they see it setting the fire.
  It circles overhead after, watching the carnage, then leaves with a feint smile.
  \item
  To recover from a fight with another dragon.
  \begin{enumerate}
    \item
    Reports of strange, loud noises emerge.
    \item
    Other reports mention seeing an even larger dragon than the one everyone has heard about (and a different colour).
    \item
    The local dragon approaches \pgls{village} at night and requests meat.
    It will not reveal its injuries, out of pride.
  \end{enumerate}
  \item
  To learn a new language.
  \begin{enumerate}
    \item
    It arrives, grabs someone, and flies away.
    \item
    Sometime later it returns (or finds somewhere to practice this new language), and asks for food by repeating a single sentence -- `I need food', and nothing more.
    \item
    Later still, it returns to ask for different food, with a slightly larger vocabulary.
    It has learned to speak the local human language from its captive, who remains alive, at least until the dragon's language improves.
    \item
    Correcting the dragon's grammar guarantees the captive remains alive.%
    \footnote{This is best done politely.}
  \end{enumerate}
  \item
  To learn more of the Air Sphere.
  Practice makes perfect, so it starts to push people on the road over, or makes noxious clouds to choke them.

  After two scenes, increase the dragon's Air Skill by one.
  \item
  To attempt a larger spell than ever before.
  \begin{enumerate}
    \item
    The dragon identifies various locations of plants which it can use as \glspl{ingredient}, and people take note of where it flies.
    \item
    After a couple of reports, people figure out what the dragon wants, and start using its presence as a sign that the area holds valuable materials.
    \Glspl{guard}, \glspl{seeker}, \glspl{doula}, and farmers start investigating those areas, hoping to use or sell the plants.
    \item
    The dragon eventually uses the materials (dragons can use \glspl{ingredient} as \glspl{boon} simply by swallowing them).
    Find the biggest spell it might cast with some Sphere bonuses, then increase one of its Spheres by~1.
  \end{enumerate}
\end{dlist}

\end{multicols}

\section{\Glsfmtplural{dryad}}
\label{dryad}

\widePic{loh/dryad}

\begin{multicols}{2}

\noindent
As elves grow, their magical abilities can increase to the point where casting spells comes as naturally as song.
They begin to shapeshift, and change the bodies of animals, just for fun (making pets, monstrosities, or both).
One year, they may grow bark-like skin, the next their hair turns to leaves.
Their bodies and experiments change with the seasons.
Around this time, most leave their communities, and live alone, or sometimes stay with others who have grown old and strange.

Once the \gls{dryad} has become sufficiently isolated from others, many stop seeing other sentient creatures as more `special' than anything else.
If gnomes disturb their sleep, they may simply tell the whole community to move.
If a human \gls{village} starts forging roads, they may send a bird to tell them to leave.
And if the \gls{dryad} has trouble hunting deer, it may resort to simply eating a passing trader.
It's all just meat at the end of the day.

Their daily life looks idyllic.
They might spend a full month staring at a flower, in order to follow its life-cycle, or imitate a nearby animal until they can make a perfect imitation, then weave that ability into their spells.

Dryads often cast long spell-songs, which let them know what kinds of creatures are in the area.
They use this knowledge to affect others from afar, turning them into strange creatures, shrinking them, or making them grow useless appendages.

Dryads can easily become violent when their territory is threatened.
Being intelligent creatures, they do not march blindly into battle, but will use magic to pull apart any settlements they feel are in the wrong place.
Some can be bargained with, but it's famously difficult to bargain with creatures who don't have any use for gold, outside information, or friends.

\dryad

\showStdSpells

\showEnc[El]

\begin{itemize}

  \item
  If any character can be heard singing, a nearby \gls{dryad} gifts them extra \glsentrylongpl{fp} for the next 3 \glspl{interval}.
  \item
  Four bandits camped in \pgls{dryad}'s territory, so she summoned mist around them, and attacked them by surprise.

  The troupe find her eating a corpse while her pet bear sits beside her, eating another.
  She just stares at them, while chewing slowly.
  \item
  The \gls{dryad} follows the troupe through the forest, trying not to be seen.
  She listens, and judges them, using all the spells she has to gain information about them.
  If they seem like they're not a threat to the forest, she helps them with any encounters.
  Otherwise, she uses whatever magics she has to curse them.
  \item
  A couple of farmers wandered too far off the path, looking for rare herbs.
  The local \gls{dryad} took offence at the intrusion and vandalism, and turned them into unnatural, morphed creatures.
  They have bark for skin, and long razor-sharp teeth, but cannot speak beyond a murmuring growl.
  Knowing their own \gls{village} will kill them on sight, they stare at their old home, every Sundown, and sigh.
  Then they go out, and hunt with spears, nets, and their natural camouflage.
  \item
  The \gls{dryad} who lives on the hill loves horses.
  As they pass, she sings to them, beckoning them to come and run with her.
  All horses in the troupe, or caravan, start to buck and try go free.
\end{itemize}

\morphhuman

\subsubsection{Plans}
depend on the temperature.
Add +1 during mild \glspl{cycle}, and +2 during warm \glspl{cycle}.

\begin{dlist}
  \item
  To build a palace of ice.
  The size depends on the \gls{dryad}'s skill with the Earth \gls{sphere}.
  If big enough, they may freeze enough water from local streams to stop the stream entirely!
  \item
  To hunt in the snow by moonlight.
  If she sees tracks, the hunt is on!
  \item
  To run with the aurochs.
  She has enchanted them to follow her lead.
  \item
  To summon storms with a rhythm.
  Thunder and lightning all strike on-the-beat, and even the rain sounds like drums.
  \item
  To swap predator and prey.

  She has twisted \glspl{crawler} to steal fruits, and given deer sharp antlers and teeth.
  Over the next few \glspl{interval}, the troupe will encounter (or hear of) many of these twisted creatures.
  \item
  Just to sing uninterrupted.
  She doesn't like to hear noisy people on a road, or to hear the drone of pipes from \glspl{broch}.

  Traders on the road soon learn to be quiet, and the \gls{guard} have some decisions to make, as her spells can travel to the tower without anyone seeing her in the forest.
  \item
  Growing really \emph{weird} plants.

  Her spells affect the gardens of nearby settlements, making the plants stand up and wander on four legs, or form into houses.
  \item
  To wriggle free of these webs.
  She cursed the web-spinning \gls{crawler} to eat its own legs off, but now she's out of mana and can't get out.

  If the \glspl{pc} help her, she will remember them.
\end{dlist}

\end{multicols}

\needspace{6\baselineskip}
\section{\Glsfmtplural{ogre}}
\label{ogre}

\begin{multicols}{2}

\noindent
In many ways, \glspl{ogre} seem like more natural inhabitants of \gls{fenestra} than humans.
They can lounge outdoors comfortably, occasionally clubbing \pgls{crawler} to death, or trying to chase down deer.
The only thing stopping a peaceful life of free-range fruit and arachnid salads is their stomachs, which need constant maintenance.
Their appetites outstrip even goblins.

Goblinoid creatures don't have legal agreements, but they do have certain laws of nature, and one of those is the law of the \gls{ogre} king.
Ogres take charge because they're large.
The standard solution for goblins is warriors in full plate armour.
And the standard solution for warriors in full plate armour is \pgls{ogre} hitting them with a tree.

If a horde of goblins, lead by \pgls{ogre}, becomes hungry, they all start to think the same thing: they must find food soon, or one will eat the other.
This unspoken knowledge pushes any goblinoid horde to constant raids on any nearby farms.

Ogres are not born, but created by a `stuffing'.
This is when one goblin stuffs himself full, then keeps on eating.
Eventually, this larger goblin grows broad shoulders, and a belly, and grow as tall as a human.

\armouredOgre[\npc{\N\M}{Ogre King}]

\newBeast[\N]{Goblins}% Name
  {goblin}% label
  {pale, little humanoids, with extreme hunger}% description
love law and order.
They stare at order with a sense of wonder, and love finding out new laws.
They discuss how these systems work, and how the systems might react to different circumstances.
And then they test every theory at once.

Goblins remain the subject of \glspl{seeker}' theories about whether they really form a `people', like elves or humans, or if the world `goblin' lazily addresses a number of different peoples.
On the one hand, goblins are highly dimorphic, as some sprout claws, fangs, or even tentacles.
On the other hand, these unique advantages seem to sprout from random goblins, rather than being linked to a populace.

Goblins demand a separation of intelligence from cunning, as they have all of the latter, but none of the former.
They nearly-never enter into `combat', in the sense that humans might understand it.
They might poison someone's food, lift heavy rocks to the top of a ravine to drop on people's heads, throw spears at horses, or make a high pitched whistle (audible to dogs and elves, but not to humans) to confuse and divide a group.
If they finally `attack', they will wear their enemies down, lead them through a muddy path, wait for them to slip over; then they descend, ten or twenty at a time.

Goblins always play dirty.

\goblin[\npc{\T\N}{Goblins}]

\newBeast[\N]{Hobgoblins}% Name
  {hobgoblin}% label
  {as big as a man, and twice as dangerous}% description
begins as goblins who bathe in the Sunlight and grow big and strong, and soon seek out weapons worthy of a warrior.
Few reach this stage, as they need enormous amounts of food in order to keep their growth-spurts up.

Out of the hobgoblins, some rare continue feeding until they outgrow even humans, and then become \pgls{ogre}.
Some theorise a later stage in goblin-gorging, but most people don't like to think about it.

\hobgoblin[\npc{\T\N}{Hobgoblins}]

\pic{Studio_DA/ogre}

\exampleGoblinEnc

\subsubsection{The Goblin Rating}
\label{goblin_rating}
\index{Goblin Rating}
determines the quantity of goblins in the area.
They eat standard encounters from the bottom-up.
If the Goblin Rating ever equals or beats a random encounter roll,%
\footnote{See \autopageref{encounters}.}
the encounter is replaced with a number of goblins equal to four times the encounter roll.

\begin{exampletext}
  The troupe journeys down the road where traders reported goblin raids occurring with increasing frequency.
  The Civilization Rating is `7', and the current Goblin Rating is `4'.
  If the \gls{gm} rolls `3', 12 goblins have arrived to eye-up the troupe.
\end{exampletext}

As the Goblin Rating increases, they devour the people in the area bit by bit, and anyone on the outskirts of civilization will neither find boars, nor traders on the road.
Instead, they find goblins, hungry for more.

\subsection{Plans}

Goblins have more energy than focus, so roll $2D6$ to find two different plans for the goblins.

\begin{dlist}
  \item
  To feast on crawlers.
  \begin{enumerate}
    \item
      The troupe spot a group of goblins, sitting in the forest, eating a spear-peppered \gls{crawler}.
    They turn from their meal, then jeer at the troupe.

    ``Back on the road, and bolt!''
    \item
    Goblin-sightings quiet-down for a while, as they learn new recipes.
    However, the Goblin Rating increases by~1.
    \item
    Eventually, they run low on \glspl{crawler}.
    As a result, the griffin population explodes during the next warm season (as noted \vpageref{crawlerBalance}).
  \end{enumerate}
  \item
  To lay siege to \pgls{broch}!
  Each new scene brings $2D6$ more goblins.
  \begin{enumerate}
    \item
    People report faces watching them from high in the forest.
    \item
    The goblins steal a saw, and manage to weaken two trees on the road to the \gls{bothy}.
    The trees will fall onto the road by the next storm, or the goblins can bring them down by climbing up, and swinging back-and-forth together.
    \item
    Goblins sneak out with picks to pull stones from the \gls{broch}'s base, then run away with them.
    \item
    $1D6+4$ hobgoblin cooks arrive with supplies.
    They set up fires in a great circle around the \gls{broch} (just out of sight), and start cooking.

    From this point on, no more supplies reach the \gls{broch} -- the goblins eat every trader, including those with \glspl{guard}.
    If the troupe destroy the food, the goblins will be forced to attack directly.
    \item
    If things are going well for the goblins, the \gls{ogre} king himself arrives.
    If the goblins bring down the \gls{broch}, increase the Goblin Rating by~1.
  \end{enumerate}
  \item
  To annoy the \glspl{pc}.
  \begin{enumerate}
    \item
    A massive troupe of goblins follow the troupe, waiting for them to sleep.
    They dare one of their number to scout ahead and follow the troupe, so the \glspl{pc} may spot the scout, but that won't help them find the rest.
    If they kill the scout, the other goblins will follow the screams, and bully another member into following the troupe.
    \item
    Over the night, they try to steal food, frighten horses, throw rocks, cut ropes, and sing (goblin singing can stop anyone sleeping, no exceptions).
    They will not attack unless the right moment strikes.
    If they become too bored, they run ahead on the road -- roll the next encounter immediately.
    If they encounter a trader, they may eat them.
    If they find \pgls{basilisk}, they will run back to the \glspl{pc} so it eats them instead.
  \end{enumerate}
  \item
  To map the area.
  Each week, the \gls{ogre} king sends $2D6$ goblins to a new area on the map, starting with 1, then 2, and so on.

  As they travel, they leave poisoned horse-biscuits on roads, so that the horses on a caravan's first wagon stop to eat the biscuits, then die shortly afterwards.
  \item
  To make tunnels underneath the nearest settlement (lowest point first).
  If those creatures live underground, the goblins will attempt a surprise-attack.
  If the nearest settlement is human, the goblins will try to dig a tunnel under an important structure.

  The larger the settlement, the longer this plan takes.
  \item
  To brew new potions.
  A goblin druid arrives from a neighbouring realm, deep underground.
  \begin{enumerate}
    \item
    Goblin scouts gather \glspl{ingredient} from all around (any potent plants in the area vanish).
    \item
    The druid experiments with new potion ideas -- various goblins gain tentacles, claws, and a few gain wings.
    \item
    The druid dumps the failed experiments in the local river\ldots upstream from a settlement.
    People there become afflicted with unpleasant spells from the Life Sphere.
  \end{enumerate}
\end{dlist}

\end{multicols}

\section{\Glsfmtplural{hag}}
\label{hag}
\index{Hags|textbf}

\begin{multicols}{2}

\noindent
\Glspl{hag} \glsdesc{hag}.

To practice their magic, they spend their days shape-shifting into monsters with more bone-protrusions than skin, then doing the same to any nearby animals.
They often collect these animals as `pets', but never spend enough time procuring food for them, instead preferring to release them and let them eat people on the road.
After they've finished creating torments for the local population, they like to sit down and have a cup of tea and cake.
After all, `hag' is just a nasty word for a person, and people like cakes.

\hag

\showStdSpells

\subsubsection{Pets}
never stay normal for long, as hags `bless' them with size, strength, and hunger.
The spell is easy, but reversing the spell means starving the pet for a while, which never goes down well.

\ifodd\value{r4}
  \morphcat[\npc{\R\A\M}{Grebo}]
\else
  \morphspider[\npc{\R\A\M}{Mr Stickilegs}]
\fi

With enough time and twisting, farmers sometimes take bets about whether the pet began as a cat or a spider.

\subsection{Plans}

Roll $1D6$ to find the hag's primary plan.

\begin{dlist}
  \item
  To make the people pay for their insulting nick-name for her.
  \begin{enumerate}
    \item
    \Pgls{village} has seen no rain this season -- the clouds always go elsewhere since `old crow-face' moved in.
    \item
    The farm animals miscarry constantly, and the chickens have stopped laying eggs.
    And I think I know who's fault it is.
    Next time I see that wizened-\gls{witch} collecting firewood on the other side of the river, you know what I'm gonna call her?
    \item
    The \gls{warden}'s been pregnant for two years!
    Something has to be done about that battleaxe-bitch!
  \end{enumerate}

  Nobody who went to speak with the hag has returned, but someone needs to figure out what offended her, and make amends\ldots from a safe distance.
  \item
  To watch her babies grow.
  \begin{enumerate}
    \item
    The farmers have caught a strange creature in their nets.
    It looks like a thousand earthworms banded together to form a fake body, then wore a thorn-bush as a coat.
    It writhes, silently, while a man wrestles a live pig towards it.
    Eventually, the man wins, and feeds the pig to the creature.

    When asked, the farmers explain that `the lonely lady by the river' doesn't like when people hurt the things she makes, and it's easier to just give them a meal, and send them on their way.
    \item
    Repeat the encounter a few times.
    The troupe have a 1-in-6 chance of the hag watching them via spells each time.
  \end{enumerate}
  \item
  Just to chat with travellers on the road.

  She introduces herself as `Carnbile', and asks pleasantly for directions, asks about their allies' health (she says she knows them), then wanders away with a hearty farewell.
  \item
  To make the \glspl{warden} know their place.

  Minstrels, criers, and tradesmen over the entire land spread the new law: none may hunt deer, and every able archer must kill an animal which hunts deer, and hand the head to \pgls{warden} as proof of the kill.

  One of the \glspl{warden} pissed off Sunstain by hunting one of her pets, so she twisted the \gls{warden}'s brothers bodies into deer-like, sprightly creatures, incapable of speech.
  The \gls{warden} has placed a high reward on their return, but they stand little chance running around the forest alone in a deer-like shape, with no voice.
  \item
  Just to sleep till \gls{cThree}.

  The hag goes to a secluded area (perhaps a hidden cave) and begins hibernation.
  She leaves a simple message, painted on her front door -- `{All return}'.%
  \footnote{Hags make their own rules in life, which includes rules on how to spell words.}
  If anything in her house goes missing, she will tear apart every \gls{village}, \gls{broch}, and bridge finding the culprit.
  \item
  To get a better postal system.
  She's tired of her enchanted ravens disappearing, and using magic over long ranges demands far too much time preparing plants.

  She slowly interrogates and eliminates every real and potential blockage from her location, to any roads leading out of the region (if no such road exists, she starts there too).

  She builds up a long-term plan, where each part takes at least a week.
  \begin{enumerate}
    \item
    If goblins interfere with the roads, she curses their leader.
    \item
    If she needs help building roads, she threatens \pgls{warden}'s children.
    \item
    If a town could do more to protect its traders, she writes a polite letter, advising them where they can find an excellent quarry, for the building of \glspl{bothy}.
  \end{enumerate}
  The letters talk about local gossip, handsome bards, and the seasonal activities of birds.
\end{dlist}

\end{multicols}

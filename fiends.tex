\chapter{Fiends}

\toggletrue{genExamples}

In a world of wandering monsters, sensible habits let people outsmart most of them.
One can predict beasts, but not calculating, intelligent enemies.

These sentient creatures live outside of civilizations, but feed on them.
Unlike beasts, they do not usually stop to feed after killing one person, but can kill entire caravans, or sometimes whole villages.

\section{Bandits \& Brigands}

\begin{multicols}{2}

When farmers run out of food, or city-dwellers lose their jobs, they have only two option -- banditry or the \gls{guard}.
People generally select whichever comes along to recruit them first.

A good many bandits begin life the same way.

\begin{enumerate}
  \item
  A bad Winter leaves a village without much in the way of supplies.
  \item
  A few people with alternative places to stay leave.
  \item
  A trader arrives, and sees a desolate village, with nothing to trade.

  At this point, the remaining villagers have two options, and they know it:
  \begin{itemize}
    \item
    Let the trader return to town, and inform all the other traders not to bother with that desolate little village on the outskirts ever again.
    \item
    Take the trader's items and survive.
  \end{itemize}
  \item
  The \gls{guard} soon come to destroy the newly formed `bandit village'.
  \item
  Assuming the villagers are sensible, they leave, set up camp somewhere in the forest, and set up some new homes there.
\end{enumerate}

\noindent
These groups usually don't live live long, but some manage to create a long-term place of residence, and plague all the villages around for years, or even decades.

Once they've set themselves up with a long-term base, they form the same relationship to civilization as any other fiends, with one important difference -- they can walk into any town and trade, just as long as nobody recognizes their faces, or their goods.

People sometimes make a distinction between bandits and brigands, where the former begin as farmers who take to robber, and the latter start as members of the \gls{guard}.
The difference becomes most visible in daily life, as farmers-turned-outlaws generally return to farming whenever possible, even if they can't sell their vegetables any more.

\humanfarmer[\npc{\E\Hu}{Bandit}]
\label{bandits}

\humansoldier[\npc{\E\Hu}{Brigand}]
\label{brigands}

\paragraph{Encounters:}


\begin{itemize}
  \item
  The party reach a bothy, and find it already full of `traders', going to sell rare fabrics.

  Before morning, they pull weapons out, and attempt to rob the most valuable items of anyone staying there.
  \item
  One bandit arrives on the road, a long way from civilization.
  She begins shouting demands in a very reasonable voice, reminding everyone that she wants to reach a reasonable, amicable agreement.
  They must throw their valuables out, or the men in the trees will fill the horses full of arrows, and then any people.

  Her long chain armour will protect her from a few arrows, so she has time to jump into the forest.
  She refuses to allow any evidence of the bandits' number.

  If the party (and anyone with them) refuses, she asks them to put someone aside safely, so they can tell the tale of what happened here.

  ``Just to make sure there's less bloodshed in the future''

  In fact there are only five bandits with longbows, and two more with swords.
  \item
  The outlaws lay quietly near a bothy, and simply wait for any traders to leave their gear outside.
  Then they wedge the door shut from the outside with a specially-shaped piece of wood which pushes against the frame, take the carts from outside, and leave the traders with their horses, but no goods, nor carts.
\end{itemize}

\end{multicols}

\section[Lich]{Liches}
\label{liches}

\begin{multicols}{2}

\noindent
Those who study Death magic do not have to die.
They can slow the process of decay, and lock their own souls in their body.
This won't stop death, but it will delay it.
Corpses which stave off decay continue to rot any time the soul exercises the body, so once a lich creates itself, it begins with a fresh countdown to its own complete and natural death.

A lich's activities revolve around staying cold and motionless.
They typically find a lair underground, in the cold depths, and attempt to gather servants who can act on their behalf, and help them gain more power.

Their every night is spent completely aware of their own unmoving body, bored, irritated, but so scared of their own death that they continue lying motionless, with nothing to do but plan, and cast whatever spells require no motion.
Luckily for them, the dead can cast spells without the need to articulate or gesticulate in the usual way.%
\footnote{See \autopageref{dead_communication}.}

Despite the stale mana in caves, liches can gather enough to cast a few \textit{Witness} spells each night, and discover a lot about their surroundings.
Many start to amass their power as information brokers.
Seekers who pull at a source of information too much can find themselves face-to-face with a corpse which speaks directly into their minds.

Once they have a cool, dry place to store their bodies, they begin to amass short and long term armies.
Short-term armies consist of any powerful creatures
Once they have a cool, dry place to store their bodies, they begin to amass short and long term armies.
The short-term armies consist of any powerful creatures in the area (gnomish alchemists, \gls{guard} leaders, dwarven merchants).
The long-term always consists of armies of ghouls `on-ice', and never placed far from the lich's body.

\lich

\showStdSpells

\paragraph{Encounters:}


\begin{itemize}
  \item
  A trader delivers a letter, which contains very old grammar but very modern words.
  It requests a complete map of an area beyond the \gls{edge} which the troupe have recently visited, and offers 20 \glspl{gp} in return.
  The letter comes with an advancement of 20 \glspl{sp}; the coins seem new, but date back seven centuries.

  The letter is signed by an ancient lich, who began life as a dwarven warrior.
  Warriors had reported finding and destroying her body three times, and eventually everyone just stopped looking for her.

  The bag of coins comes with a rare gemstone.
  This lich included this item so that she could track the movements of her money.
  \item
  While journeying through the long, empty tunnels of the \gls{deep}, the troupe spot a funeral procession of a hundred wandering dead in the distance.

  Everyone hears a voice in their head, asking them to stand back.
  If they look aggressive, the voice will offer them money in return for turning back, and turning into a side tunnel, then waiting for the procession to pass.

  This lich has decided to relocate, carried by a hundred ghouls, and three trusted, cunning, ghasts.
  None will attack the troupe, as long as they remain at a safe distance.
  The lich will agree to leave up to 1 \gls{sp} for each \gls{hp} the party has in total.
  \item
  While looting items from some lost city (see \autopageref{lostCities}), the troupe discover a door leading downwards.
  A voice in their head tells them to keep out, but they can see nobody.
  Suddenly, they realise the city really should have had some bodies, or skeletons at least, but it had nothing, and they wonder where all those bodies went\ldots
\end{itemize}

\end{multicols}

\section[Dryad]{Dryads}
\label{dryad}

\begin{multicols}{2}

\widePic{loh/dryad}

\noindent
As elves grow, their magical abilities can increase to the point where casting spells comes as naturally as song.
They begin to shapeshift, and change the bodies of animals, just for fun (making pets, monstrosities, or both).
One year, they may grow bark-like skin, the next their hair turns to leaves.
Their bodies and experiments change with the seasons.
Around this time, most leave their communities, and live alone, or sometimes stay with others who have grown old and strange.

Once the dryad has become sufficiently isolated from others, many stop seeing other sentient creatures as more `special' than anything else.
If gnomes disturb their sleep, they may simply tell the whole community to move.
If a human village starts forging roads, they may send a bird to tell them to leave.
And if the dryad has trouble hunting deer, it may resort to simply eating a passing trader.
It's all just meat at the end of the day.

Their daily life looks idyllic.
They might spend a full month staring at a flower, in order to follow its life-cycle, or imitate a nearby animal until they can make a perfect imitation, then weave that ability into their spells.

\dryad

\showStdSpells

\paragraph{Tactics:}
Dryads often cast long spell-songs, which let them know what kinds of creatures are in the area.
They use this knowledge to affect others from afar, turning them into strange creatures, shrinking them, or making them grow useless appendages.

Dryads can easily become violent when their territory is threatened.
Being intelligent creatures, they do not march blindly into battle, but will use magic to pull apart any settlements they feel are in the wrong place.
Some few can be bargained with, but it's famously difficult to bargain with creatures who don't have any use for gold, outside information, or friends.

\paragraph{Encounters:}

\begin{itemize}

  \item
  If any character can be heard singing, he gifts them extra \glsentrylongpl{fp} for the remainder of the \gls{interval}.
  \item
  Four bandits camped in a dryad's territory, so she summoned mist around them, and attacked them by surprise.

  The party find her eating a corpse while her pet bear sits beside her, eating another.
  She just stares at them, while chewing slowly.
  \item
  The dryad follows the party through the forest, trying not to be seen.
  She listens, and judges them, using all the spells she has to gain information about them.
  If they seem like they're not a threat to the forest, she helps them with any encounters.
  Otherwise, she uses whatever magics she has to curse them.
  \item
  The dryad who lives on the hill loves horses.
  As they pass, she sings to them, beckoning them to come and run with her.
  All horses in the troupe, or caravan, start to buck and try go free.
\end{itemize}

\end{multicols}

\widePic{loh/dragon}

\section{Dragons}
\label{dragon}

\begin{multicols}{2}

\noindent
When a dragon lands on a farm, and just starts eating sheep, there are two types of reactions.
Some people scream, because it seems like the thing to do.
Others do nothing, because there is nothing to be done.

Dragons have very little need of sheep.
They can kill local basilisks, or hunt deer with ease.
Still, they descend on farms, eat farmers' sheep, and sometimes the farmer, because they get bored.
Nobody ever asks a dragon how their day is going, but if they did, they would often find the dragon bored.

Their association with the sky, and endless destruction, leads people to think of them as divine.
Some even think that the gods made the world just for dragons to play.

\paragraph{Natural Abilities:}
Dragons can sprout a tiny spark of flame from their mouths.
Coupled with the Fire sphere, this allows them to produce fire from their mouths.

Dragons can usually fly, although many become too heavy to simply take off.
After too many sheep dinners, many have to reach a high location in order to get going in the air.

\paragraph{Ecology:} Dragons are generally solitary creatures, though people have nightmares of this situation changing constantly.
They come from distant lands and like to create fanciful tales of their homeland in order to scare humans or irritate gnomes.
It is said that older dragons sometimes transform themselves into humans and walk abroad in the land, pretending to be alchemists, bards, nobility or just whatever strikes their fancy.
Rumours abound of their constant interference in politics among all races, though if half the rumours were true there would be more dragons than horses in Fenestra.

The majority of dragons do not learn humanoid languages, but some few learn dwarvish or elvish.
This has provided a lot of false hope to those trying to negotiate with them.

\paragraph{Encounters:}

\begin{itemize}
  \item
  The dragon rests peacefully in the forest.
  As the party approach it, they can slowly back away.
  No amount of noise will wake it, unless they approach it.
  \item
  In the distance, as the troupe approaches a village, the local dragon descends and annihilates it in a torrent of fire.

  Nobody knows why.
  They just do that sometimes.
  \item
  A dragon approaches a village at night, and requests meat.
  She has not been able to see since a fight with another dragon, who removed her eyes.
  She won't share this information willingly, out of pride.
  \item
  A dragon arrives, grabs someone, and flies away.
  \begin{itemize}
    \item
    Sometime later it returns, and asks for food by repeating a single sentence -- `I need food', and nothing more.
    \item
    Later still, it returns to ask for different food, with a slightly larger vocabulary.
    It has learned to speak the local human language from its captive, who remains alive, at least until the dragon's language improves.
    \item
    Correcting the dragon's grammar guarantees the captive remains alive.%
    \footnote{This is best done politely.}
  \end{itemize}
\end{itemize}

\dragon

\end{multicols}

\chapter{Fiends}

\toggletrue{genExamples}

In a world of wandering monsters, sensible habits let people outsmart most of them.
One can predict beasts, but not calculating, intelligent enemies.

These sentient creatures live outside of civilizations, but feed on them.
Unlike beasts, they do not usually stop to feed after killing one person, but can kill entire caravans, or sometimes whole villages.

\section{Bandits \& Brigands}

\begin{multicols}{2}

\noindent
When farmers run out of food, or city-dwellers lose their jobs, they have only two option -- banditry or the \gls{guard}.
People generally select whichever comes along to recruit them first.

A good many bandits begin life the same way.

\begin{enumerate}
  \item
  A bad Winter leaves a village without much in the way of supplies.
  \item
  A few people with alternative places to stay leave.
  \item
  A trader arrives, and sees a desolate village, with nothing to trade.

  At this point, the remaining villagers have two options, and they know it:
  \begin{itemize}
    \item
    Let the trader return to town, and inform all the other traders not to bother with that desolate little village on the outskirts ever again.
    \item
    Take the trader's items and survive.
  \end{itemize}
  \item
  The \gls{guard} soon come to destroy the newly formed `bandit village'.
  \item
  Assuming the villagers are sensible, they leave, set up camp somewhere in the forest, and set up some new homes there.
\end{enumerate}

\noindent
These groups usually don't live live long, but some manage to create a long-term place of residence, and plague all the villages around for years, or even decades.
Once they've set themselves up with a long-term base, they form the same relationship to civilization as any other fiends, with one important difference -- they can walk into any town and trade, just as long as nobody recognizes their faces, or their goods.

People sometimes make a distinction between bandits and brigands, where the former begin as farmers who take to robber, and the latter start as members of the \gls{guard}.

\humanfarmer[\npc{\E\Hu}{Bandit}]
\label{bandits}

When the \gls{guard} start to rob people, they certainly become far more deadly than farmers with a bad attitude.

\humansoldier[\npc{\E\Hu}{Brigand}]
\label{brigands}

\paragraph{Encounters:}


\begin{itemize}
  \item
  The party reach a bothy, and find it already full of `traders', going to sell rare fabrics.

  Before morning, they pull weapons out, and attempt to rob the most valuable items of anyone staying there.
  \item
  One bandit arrives on the road, a long way from civilization.
  She begins shouting demands in a very reasonable voice, reminding everyone that she wants to reach a reasonable, amicable agreement.
  They must throw their valuables out, or the men in the trees will fill the horses full of arrows, and then any people.

  Her long chain armour will protect her from a few arrows, so she has time to jump into the forest.
  She refuses to allow any evidence of the bandits' number.

  If the party (and anyone with them) refuses, she asks them to put someone aside safely, so they can tell the tale of what happened here.

  ``Just to make sure there's less bloodshed in the future''

  In fact there are only five bandits with longbows, and two more with swords.
  \item
  The outlaws lay quietly near a bothy, and simply wait for any traders to leave their gear outside.
  Then they wedge the door shut from the outside with a specially-shaped piece of wood which pushes against the frame, take the carts from outside, and leave the traders with their horses, but no goods, nor carts.
\end{itemize}

\end{multicols}

\section[Dryad]{Dryads}
\label{dryad}

\begin{multicols}{2}

\widePic{loh/dryad}

\noindent
As elves grow, their magical abilities can increase to the point where casting spells comes as naturally as song.
They begin to shapeshift, and change the bodies of animals, just for fun (making pets, monstrosities, or both).
One year, they may grow bark-like skin, the next their hair turns to leaves.
Their bodies and experiments change with the seasons.
Around this time, most leave their communities, and live alone, or sometimes stay with others who have grown old and strange.

Once the dryad has become sufficiently isolated from others, many stop seeing other sentient creatures as more `special' than anything else.
If gnomes disturb their sleep, they may simply tell the whole community to move.
If a human village starts forging roads, they may send a bird to tell them to leave.
And if the dryad has trouble hunting deer, it may resort to simply eating a passing trader.
It's all just meat at the end of the day.

\dryad

\showStdSpells

\paragraph{Tactics:}
Dryads often cast long spell-songs, which let them know what kinds of creatures are in the area.
They use this knowledge to affect others from afar, turning them into strange creatures, shrinking them, or making them grow useless appendages.

Dryads can easily become violent when their territory is threatened.
Being intelligent creatures, they do not march blindly into battle, but will use magic to pull apart any settlements they feel are in the wrong place.
Some few can be bargained with, but it's famously difficult to bargain with creatures who don't have any use for gold, outside information, or friends.

\paragraph{Encounters:}

\begin{itemize}

  \item
  If any character can be heard singing, he gifts them extra \glsentrylongpl{fp} for the remainder of the \gls{interval}.
  \item
  Four bandits camped in a dryad's territory, so she summoned mist around them, and attacked them by surprise.

  The party find her eating a corpse while her pet bear sits beside her, eating another.
  She just stares at them, while chewing slowly.
  \item
  The dryad follows the party through the forest, trying not to be seen.
  She listens, and judges them, using all the spells she has to gain information about them.
  If they seem like they're not a threat to the forest, she helps them with any encounters.
  Otherwise, she uses whatever magics she has to curse them.
  \item
  The dryad who lives on the hill loves horses.
  As they pass, she sings to them, beckoning them to come and run with her.
  All horses in the troupe, or caravan, start to buck and try go free.
\end{itemize}

\end{multicols}

\widePic{loh/dragon}

\section{Dragons}
\label{dragon}

\begin{multicols}{2}

Dragons can be terrifying and extremely damaging but are generally considered to be divine creatures given their association with the sky.
When they land on a farm there is precious little anyone can do but let them take it and hope that the dragon falls back into hibernation soon.
Dragons spend a lot of time in underground realms or far off places inhabited by spirits, demons and semi-divine things of the other side.

Dragons can reach up to sixty feet in length, though a more modest specimen is presented here.

\paragraph{Natural Abilities:}
Dragons' wings allow them to fly, using the Athletics Skill to move in the air instead of the ground.

\paragraph{Ecology:} Dragons are generally solitary creatures, though people have nightmares of this situation changing constantly.
They come from distant lands and like to create fanciful tales of their homeland in order to scare humans or irritate gnomes.
It is said that older dragons sometimes transform themselves into humans and walk abroad in the land, pretending to be alchemists, bards, nobility or just whatever strikes their fancy.
Rumours abound of their constant interference in politics among all races, though if half the rumours were true there would be more dragons than horses in Fenestra.

The majority of dragons do not learn humanoid languages, but some few learn dwarvish or elvish.
This has provided a lot of false hope to those trying to negotiate with dragons.

\begin{boxtext}

  As you start to approach, its golden eyes glitter, and you find yourself paralysed by some unearthly magical force.
A moment later and fire explodes from its mouth, burning everyone in front of you.
  A few survive just enough to start crawling away in retreat, while the dragon considers its next move.

\end{boxtext}

\paragraph{Encounters:} Most dragons encountered will be fast asleep.
Sometimes they rest openly in a forest -- they have the rare claim to power that they can sleep in the open without any real fear from the world.
In such cases, characters will simply need to retreat quietly.

At other times, dragons may be seen taking back a recent kill -- perhaps a deer or a farmer's sheep.
A few dragons take people back to their lair in order to learn about recent news or to learn about human languages.
Once the dragon has learnt what it can, the person is often challenged to a game of riddles for their freedom.  

\dragon

\end{multicols}

\chapter{Fiends}

\toggletrue{genExamples}

In a world of wandering monsters, sensible habits let people outsmart most of them.
One can predict beasts, but not calculating, intelligent enemies.

These sentient creatures live outside of civilizations, but feed on them.
Unlike beasts, they do not usually stop to feed after killing one person, but can kill entire caravans, or sometimes whole villages.

\section{Bandits \& Brigands}

\begin{multicols}{2}

\noindent
Those poor souls who have either avoided joining the \gls{guard} or escaped from it live an unpredictable life beyond the \gls{edge}.
Those who manage to make a proper home for themselves will often farm like any normal villager, and only attack people when most in need.
Banditry typically increases a lot during the Winter.

Random city-dwellers and farmers who have fled rather than being shunted into the \gls{guard} are known as `bandits', while `brigands' have spent time in the \gls{guard} and decided they want nothing more to do with it.

\paragraph{Tactics:}
Brigands will pick a likely spot by the roadside and practice with their arrows for hours on end.
If someone shows up, one of the spotters will give a sign and the group will go silent.
The first people to be hit are fighters.
The second are the horses.
Traders, meanwhile, will often survive an encounter with brigands if they agree to abandon what they have and run along the road.

\humansoldier[\npc{\E\Hu}{Brigand}]
\label{brigands}

Bandits, on the other hand, typically gather as many as they can to attack smaller groups, and extort traders' goods through intimidation.
At other times, hunger drives them to extremes, and they end up attacking anything that looks like it might carry coin or food in its pockets.

\humanfarmer[\npc{\E\Hu}{Bandit}]
\label{bandits}

\end{multicols}

\section[Dryad]{Dryads}
\label{dryad}

\begin{multicols}{2}

\widePic{loh/dryad}

\noindent
As elves grow, their magical abilities can increase to the point where casting spells comes as naturally as song.
They begin to shapeshift, and change the bodies of animals, just for fun (making pets, monstrosities, or both).
One year, they may grow bark-like skin, the next their hair turns to leaves.
Their bodies and experiments change with the seasons.
Around this time, most leave their communities, and live alone, or sometimes stay with others who have grown old and strange.

Once the dryad has become sufficiently isolated from others, many stop seeing other sentient creatures as more `special' than anything else.
If gnomes disturb their sleep, they may simply tell the whole community to move.
If a human village starts forging roads, they may send a bird to tell them to leave.
And if the dryad has trouble hunting deer, it may resort to simply eating a passing trader.
It's all just meat at the end of the day.

\dryad

\showStdSpells

\paragraph{Tactics:}
Dryads often cast long spell-songs, which let them know what kinds of creatures are in the area.
They use this knowledge to affect others from afar, turning them into strange creatures, shrinking them, or making them grow useless appendages.

Dryads can easily become violent when their territory is threatened.
Being intelligent creatures, they do not march blindly into battle, but will use magic to pull apart any settlements they feel are in the wrong place.
Some few can be bargained with, but it's famously difficult to bargain with creatures who don't have any use for gold, outside information, or friends.

\paragraph{Encounters:}

\begin{itemize}

  \item
  If any character can be heard singing, he gifts them a spell-song for the remainder of the adventure -- perhaps something which grants Fate Points, or a spell to calm animals.
  \item
  The dryad follows the party through the forest, trying not to be seen.
  She listens, and judges them, using all the spells she has to gain information about them.
  If they seem like they're not a threat to the forest, she helps them with any encounters.
  Otherwise, it uses whatever magics it has to curse them.
  \item
  Four bandits camped in a dryad's territory, so she summoned mist around them, and attacked them by surprise.

  The party find her eating a corpse while her pet bear sits beside her, eating another.
  She just stares at them, while chewing slowly.

\end{itemize}

\end{multicols}

\widePic{loh/dragon}

\section{Dragons}
\label{dragon}

\begin{multicols}{2}

Dragons can be terrifying and extremely damaging but are generally considered to be divine creatures given their association with the sky.
When they land on a farm there is precious little anyone can do but let them take it and hope that the dragon falls back into hibernation soon.
Dragons spend a lot of time in underground realms or far off places inhabited by spirits, demons and semi-divine things of the other side.

Dragons can reach up to sixty feet in length, though a more modest specimen is presented here.

\paragraph{Natural Abilities:}
Dragons' wings allow them to fly, using the Athletics Skill to move in the air instead of the ground.

\paragraph{Ecology:} Dragons are generally solitary creatures, though people have nightmares of this situation changing constantly.
They come from distant lands and like to create fanciful tales of their homeland in order to scare humans or irritate gnomes.
It is said that older dragons sometimes transform themselves into humans and walk abroad in the land, pretending to be alchemists, bards, nobility or just whatever strikes their fancy.
Rumours abound of their constant interference in politics among all races, though if half the rumours were true there would be more dragons than horses in Fenestra.

The majority of dragons do not learn humanoid languages, but some few learn dwarvish or elvish.
This has provided a lot of false hope to those trying to negotiate with dragons.

\begin{boxtext}

  As you start to approach, its golden eyes glitter, and you find yourself paralysed by some unearthly magical force.
A moment later and fire explodes from its mouth, burning everyone in front of you.
  A few survive just enough to start crawling away in retreat, while the dragon considers its next move.

\end{boxtext}

\paragraph{Encounters:} Most dragons encountered will be fast asleep.
Sometimes they rest openly in a forest -- they have the rare claim to power that they can sleep in the open without any real fear from the world.
In such cases, characters will simply need to retreat quietly.

At other times, dragons may be seen taking back a recent kill -- perhaps a deer or a farmer's sheep.
A few dragons take people back to their lair in order to learn about recent news or to learn about human languages.
Once the dragon has learnt what it can, the person is often challenged to a game of riddles for their freedom.  

\dragon

\end{multicols}

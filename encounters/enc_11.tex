\ifnum\value{temperature}<2

\bear

\stepDay{4}
You wouldn't imagine that something as heavy as a bear would sneak up on you, but anything can sneak up on you in a storm.

\begin{boxtext}
  They say the wind has teeth, perhaps because it bites, but also because it can eat a campfire.
  The tents have been shaking, now they threaten to take off.
\end{boxtext}

Once the troupe's flustered, the bear appears, as it thinks the tents look like nice places to curl up during a storm (and of course, it is correct).

\paragraph{If the troupe harm the bear,}
it flees.
Otherwise, it stalks them for a week, hoping to eat their food.

\paragraph{After the \gls{interval},}
the troupe regain 4~\glspl{mp} from the wind, and another encounter emerges.

\else

\stepDay{2}
This \gls{crawler}, old and tired, needs to feed soon.
But it retains its patience.
It watches the \glspl{pc} from a distance, and places a web in their path, and pulls well back.

\chitincrawler

If the \glspl{pc} spot it (\roll{Wits}{Vigilance}, \tn[10]) or at least avoid screaming then it returns in \pgls{interval} to lay another.
But \emph{this time} it circles behind, rushing into them full speed.


\fi

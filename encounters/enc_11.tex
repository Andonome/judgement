\ifnum\value{temperature}<2

\bear

\stepDay{4}
You wouldn't imagine that something as heavy as a bear would sneak up on you, but anything can sneak up on you in a
\ifnum\value{temperature}=0
  \gls{snow}
\fi
storm.

\begin{boxtext}
  They say the wind has teeth, perhaps because it bites, but also because it can eat a campfire.
  The tents have been shaking, now they threaten to take off.
\end{boxtext}

Once the troupe's flustered, the bear appears, as it thinks the tents look like nice places to curl up during a storm (and of course, it is correct).

\paragraph{If the troupe harm the bear,}
it flees.
Otherwise, it stalks them for a week, hoping to eat their food.

\paragraph{After the \gls{interval},}
the troupe regain 4~\glspl{mp} from the wind, and another encounter emerges.

\else

\stepDay{2}
This \gls{crawler}, old and tired, needs to feed soon.
But it retains its patience, and lays many webs, two miles around.
It wanders, checking them again, and repairing them when needed.

\chitincrawler
\label{crawlerStatblock}

The webs rest around head-height, making them difficult to spot (\roll{Wits}{Vigilance}, \tn[10]).
If anyone caught in the web screams or shouts, the \gls{crawler} rushes back within five \glspl{round}.
Otherwise, it takes \pgls{interval} to return.

\fi

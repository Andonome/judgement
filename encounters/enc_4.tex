
\ifcase\value{cycle}

\stepDay{3}

\begin{boxtext}
  Mist on snow makes a bland morning.
  The world turns into nothing but a series of brown tree-trunk lines which let you know you've walked too far from the road.
  And when hoof-prints start appearing in the ground, this lets you know that you've entered a path laid by aurochs.
\end{boxtext}

The aurochs are only a mile away, and can be tracked easily, though getting back won't be easy.

\or

\stepDay{1}

\begin{boxtext}
  In the misty morning, you hear the the sound of water, and a `crack' sound.
  Then silence.
  `Crack'.
  Silence.
  `Crack'.
\end{boxtext}

The sound comes from nearby goats, butting heads.
Sneaking up on them won't be difficult in the mist: \roll{Dexterity}{Stealth} (\tn[9]).

\or

\stepDay{3}

The bear won't harm the troop, but will look for an opportunity to steal any food she can smell.
She has three cubs nearby.

\setcounter{tn}{\value{Dexterity}}
\addtocounter{tn}{\value{Stealth}}

If the troup have stationed a lookout, they can spot the bear with \roll{Strength}{Vigilance} (\tn).

\bear

\else

\stepDay{3}

\begin{boxtext}
  \Pgls{guard} corpse hangs from a tree, the empty head flapping like a puppet-show in the violentl wind.
  The body has been turned inside-out, but the black leather armour and long, dark hair remain.
\end{boxtext}

\Pgls{woodspy} ate him, and left the remains hanging on the tree.
Searching around the corpse reveals a bag with \lootSmall.

Tracking the \gls{woodspy} won't  be easy, but a \roll{Intelligence}{Survival} roll at \tn[10] lets the troupe follow it three miles to a new tree%
\IfLabelExistsTF{\jobname:harvestWoodspy}%
  {%
    \vpagerefnearby{\jobname:harvestWoodspy}%
      {.}%
      {(use the statblock from \autopageref{\jobname:harvestWoodspy}).}%
  }{.
    \woodspy
  }

Failure means they have become lost; regaining their bearings depends on finding a landmark, such as the sound of pipes from \pgls{broch}.

\fi

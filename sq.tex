\chapter{\Glsfmtplural{sq}}

\label{sidequests}
\index{Weaving Stories}

\declareSQareas{Forest,Town,Roads}
\togglefalse{genExamples}

\section{Segmented Tails}

\begin{multicols}{2}

\noindent
Very few pre-written `adventure modules' manage to entirely escape the sin of `railroading' the \glspl{pc} into a particular course of action.
When the module's plot says `the \glspl{pc} go there', or `then they realize this', the \glspl{pc} never go there, and come to an entirely different realization.

BIND has its own way of telling stories, where each \gls{segment} floats free of the others, and many tales are mixed in together.
To illustrate, I'm going to take some classic stories, and rearrange them as \glspl{sq}, just to see how thy look.

\sidequest[Forest,Roads]{Three Little Pigs}

Each \gls{sq} begins with a summary of each \gls{segment}.
Each \gls{segment} below also shows which \gls{region} it can take place in.

\sqpart{Forest}% AREA
{House of Straw}% NAME
{The troupe find a little straw house, blown apart, next to a bloody pile of bones}% SUMMARY

\begin{exampletext}
  A big, bad, wolf has blown a pig's straw house down, and eaten him.
  The straw has scattered for miles down the road.
\end{exampletext}

\begin{boxtext}
  Little sticks of straw litter the path you've followed.
  Trees above have reeds of straw stuck in their branches.
  In the distance, flies buzz.
\end{boxtext}

This \gls{segment} can take place anywhere, in any part of any forest.
We can't show the little pig's straw house on any map before the game starts, because we don't know where the \glspl{pc} will travel, but it might go onto the map after they encounter the scattered straw house.

If the \glspl{pc} investigate the fly-covered bones of the little pig, they can get a roll to see how much they understand of what just happened here, but the wolf is long-gone, so tracking him is hopeless.

\sqpart{Roads}% AREA
{House of Sticks}% NAME
{A little pig invites the troupe to dinner, just before the big, bad, wolf arrives}% SUMMARY

\begin{exampletext}
  The wolf has tracked down another little pig, but waits until night to feed.
  The pig has no idea something wants to eat him.
\end{exampletext}

\begin{boxtext}
  A cold wind blows.

  A little path at the side of the road leads along to a serene cottage, with a root-vegetable field growing all around it.
  A pig stops his digging, and waves happily to you, then shouts `greetings!', and asks if you'd like to come inside for dinner, although you can barely hear him over the sound of the growing wind.
\end{boxtext}

This scene may take place the very next day, as the troupe exit the forest, and enter the roads.
Or the \glspl{pc} may remain in the forest for a week, caught up on other missions.
In the latter case, you can simply give them other \gls{sq} \glspl{segment} until they exit the forest.

As before, this house might be anywhere on the roads, so it can take place no matter where the \glspl{pc} wander.

\paragraph{If the \glspl{pc} stay for dinner,}
the wind grows stronger, until the house falls in, damaging the \glspl{pc}.
At this point, the wolf jumps in, and grabs the little pig, then runs off with him, into the forest.

If the \glspl{pc} Damage the wolf, he flees immediately.

\paragraph{If the \glspl{pc} chase the wolf into the forest}
then you can pull up the next \gls{sq}'s \gls{segment}, and start it immediately, interrupting the fight with the wolf.
This mixing of various plots helps weave the stories into a tapestry.

\paragraph{If the \glspl{pc} kill the big, bad, wolf}
then the \gls{sq} stops here.
\Glspl{sq} sometimes do that, and that's okay -- sometimes \pgls{sq} finishes early.

\paragraph{If the \glspl{pc} simply leave}
then the pig dies, and the \glspl{pc} may never find out.
But the next \gls{segment} can continue anyway.

\sqpart{Roads}% AREA
{House of Bricks}% NAME
{The big, bad, wolf begins his siege outside the last little pig's brick house}% SUMMARY

\begin{exampletext}
  The wolf has tracked down the last little pig, but cannot enter his house.
  The pig has no idea what to do -- his food will not last forever.
\end{exampletext}

\begin{boxtext}
  A storm has arrived, quite suddenly.
  On the hill ahead, you can see a two-story pinkish-red house made of bricks.
  A little pink face opens the shutters, waves his arms frantically, and calls out.
  You cannot hear him over the storm, but it's clear he calls for help.
\end{boxtext}

Each \gls{sq} with a creature should always contain a full statblock on the same page or the next page, so you can see all the information required for the scene at a glance.

\Person{\npc{\M\A}{Big, Bad, Wolf}}%
  {{3}{3}{4}}% BODY
  {{-1}{3}{-4}}% MIND
  {%
    \set{Brawl}{3}
    \set{Deceit}{2}
    \set{Stealth}{2}
    \set{Survival}{3}
    \set{Air}{3}
    \set{Earth}{1}
  }% SKILLS
  {}% KNACKS
  {Tooth-pick made of bone}% EQUIPMENT
  {\quadruped \hide{4}}% ABILITIES

The lowest line shows a combat-ready summary of the creature.
Next to the \gls{dr}, a secondary \gls{tn} is shown, which is the amount required to get \pgls{vitalShot} on the creature.

The circles show free \glspl{hp}, so you can cross them off if the creature takes Damage.
The second set of tick-boxes shows free \glspl{mp}.

Spellcasters should also have a few standard spells noted below their stat-blocks.
These spells each show a Cost and Range, as you'd expect, but note the words `Resisted' and `Roll'.
Spells which have the former show what \pgls{pc} must roll to resist the spell, with \pgls{tn} based on Goutmare's \roll{Charisma}{Sphere}.
Spells with the latter just show her \roll{Charisma}{Sphere} total, and note how to determine the spell's \gls{tn}.

The spells are intended as a suggestion -- if \pgls{npc} needs another spell, and you know exactly what would fit, you should add one or two more.
Most casters should know at least two spells for each level of each \gls{sphere}.

\showStdSpells

\paragraph{If the \glspl{pc} enter the house,}
the little pig bars the doors, and the storm dies down within the hour.
However, the wolf simply waits for them to leave, and will return the next day.

The last little pig knows this, and will spend the entire night bitterly complaining how awful it is to taste so good.

\paragraph{If the \glspl{pc} try to attack the wolf,}
it will continue blowing, and if some of the \glspl{pc} fall over, he attacks for a moment, then runs back to the forest, then tries again.

\sidequest[Forest,Roads,Town]{Cinderella}

\sqpart{Town}% AREA
{New Recruits}% NAME
{The troupe must accompany new recruits to the \glsentrytext{guard}, to the nearest \glsentrytext{broch}}% SUMMARY

\begin{exampletext}
  When the last of Cindarella's family died, she had no choice but to work at the last place who will have anyone: the \gls{guard}.
\end{exampletext}

\begin{boxtext}
  Tired, hopeless faces sit outside the \gls{court}.
  A woman at the back is crying, but refuses any consolation from the others, snapping at them `I am not a criminal!'.
\end{boxtext}

\Pgls{jotter} spots the troupe in town, and asks them to accompany a brand-new group of \glspl{fodder} to the nearest \gls{broch}.

The \glspl{pc} may not want to, and could argue.
However the scene ends, the important part is just to establish the existence of Cindarella.

\sqpart{Roads}% AREA
{What Lies on the Roads}% NAME
{\Pgls{doula} pretends to be \pgls{warden} from `Franchia' to give the troupe orders}% SUMMARY

\begin{exampletext}
  Goutmare, a mischievous \gls{doula}, wants the \gls{guard} to help her \gls{village}.
  She casts an illusion of \glspl{sunGuard} in the distance, and pretends to be \pgls{warden}.
\end{exampletext}

\begin{boxtext}
  A bright, blue, wagon stands at the side of the road.
\end{boxtext}

The next \gls{segment} establishes a `fairy god-mother', who will later help Cindarella pretend to be a princess from a distant land.
She pretends to come from `Franchia', a distant, magical land which has no monsters, and is therefore, very rich.

Cindarella may already be with the troupe at this point, if they have just come from town.
Or she may meet up with Goutmare the \gls{doula} later.

Within the scene, Goutmare tries to order the troupe to help out a nearby \gls{village}, cutting back its trees.
They might argue she has no authority over the \gls{templeOfBeasts}, or may spend a day helping out, instead of doing what they were ordered to.

\sqpart{Forest}% AREA
{Dressing for the Weather}% NAME
{A group of \glspl{guard} argue with Cindarella that she needs to stop wearing noble clothing}% SUMMARY

\begin{exampletext}
  Cindarella has joined a group of a dozen \gls{guard} on a mission to find some plants in the forest, and they're annoyed that she's wearing bright colours (which makes them all stand out), and insists on speaking in a loud, commanding, voice (just like \pgls{warden}) rather than in the hushed whispers of \glspl{guard} who venture past the \gls{edge}.
\end{exampletext}

Little \glspl{segment} like this provide a conflict to resolve, while laying down some foreshadowing.

\sqpart{Town}% AREA
{\squash The Masquerade Comes to Town}% NAME
{Everyone in town celebrates today with a masked party}% SUMMARY

\begin{exampletext}
  Cindarella, with the help of Goutmare the \gls{doula}, has attended the ball, with that bright-blue wagon.
  She has put on a fake accent, and pretends to be a noble from `Franchia'.
\end{exampletext}

Everyone in town celebrates with a masquerade.
Nobles go to a ball at the citadel, while everyone else remains on the street.

The `\squash' symbol in the title shows that this \gls{segment} should be run at the same time as the next.
Whatever happens in the next \gls{segment}, everyone there should be wearing masks.

The \glspl{pc} will almost certainly not be aware of what Cindarella is up to, because that ploy takes place within the town's citadel.
Various \glspl{segment} may note events within particular locations, but events which happen to the \glspl{pc} must make sense for \textit{anywhere} within \pgls{region}.

\sqpart{Town}% AREA
{The Search for the Franchian Woman}% NAME
{The \gls{warden}'s son searches for that mysterious woman from Franchia}% SUMMARY

\begin{exampletext}
  The \gls{warden}'s son has put out a 50~\glspl{gp} reward for finding the mysterious woman from Franchia.
\end{exampletext}

The \glspl{pc} can find Cindarella eating with Goutmare, at her home, by the \gls{village} they first saw her.
If they engage with the ploy, they can earn the hefty reward.
Otherwise, someone else gets the reward, the \gls{warden}'s son marries her, and further shenanigans ensue.

\sidequest[Forest,Roads,Town]{Snow White}

\sqpart{Forest}% AREA
{The Unspeakable Job}% NAME
{A \gls{guard} \gls{ranger} returns from a shameful job}% SUMMARY

The troupe meet \pgls{ranger}, who was tasked with killing Snow White.
He just abandoned her, and doesn't want to talk about the mission.


\sqpart{Roads}% AREA
{The Missing \Glsfmttext{warden}}% NAME
{Rumours abound that the \glsentrytext{warden}' daughter is missing}% SUMMARY

In town, everyone is talking about the \gls{warden}'s daughter going missing.

\sqpart{Roads}% AREA
{The Apple Seller}% NAME
{The Queen dresses as an apple seller, with poisoned apples}% SUMMARY

The troupe meet the \gls{warden}'s new wife, dressed as an old apple seller, with poisoned apples.
If they try to buy any, she refuses.
If they press her, she sells them one day's worth of food, and the next scene they begin to suffer \glspl{ep}.

\sqpart{Town}% AREA
{Mourning till Sunrise}% NAME
{Dwarves mourn in the local bar}% SUMMARY

The local bar cannot get rid of these dwarves, who deep drinking, fighting, and singing sad songs. 

\sqpart{Town}% AREA
{Iron Shoes}% NAME
{The Queen dances in iron shoes}% SUMMARY

Snow White has been rescued (if not by the \glspl{pc}, then by someone else).
The town watch her coronation, and during the main event, she makes her step-mother dance in irons shoes until she dies.

\end{multicols}

\section{Plotting \Glsfmtplural{sq}}

\begin{multicols}{2}

\noindent
Juggling all these \glspl{sq} and \glspl{segment} threatens to become confusing, so each collection comes with a summary which plots out the \glspl{segment} \emph{per location}.

Have a look at the summaries below.
Notice that the first \gls{segment} comes with the `\sqr' symbol.
Once it's done, you can mark it off with an `X', then mark the next \gls{segment} in the \gls{sq} as ready with a `\sqr'.

Imagine what would happen if the troupe stick around on the Roads for a while; they could fly through the \glspl{segment} from the first \gls{sq}, before the second arrives as a distraction.
Compare that with how the events might unfold if the \glspl{pc} immediately go from the Roads to the Forest.

\end{multicols}

\printAllSideQuests{Forest, Roads, Town}

\bigLine

\begin{multicols}{2}

\subsection{Crafting \Glsfmtplural{segment}}

\Pgls{segment} must exist within these constraints:

\begin{enumerate}
  \item
  The \gls{segment} must make sense anywhere within a broad region.
  \item
  The \gls{segment} must make sense within almost any time frame, including weeks after the previous \gls{segment}.
  \item
  The \gls{segment} must not assume any outcome from a previous \gls{segment}.
  \item
  The \gls{segment} must not assume any reactions from the \glspl{pc}.
\end{enumerate}

\subsection{Pacing}

You can spring \pgls{segment} on the \glspl{pc} any time you feel like it.
When they enter a new region (e.g.`Town', or `Forest') makes a good opportunity for a new \gls{segment}, but you may pull out as many as one per day.
Pulling out two \glspl{segment} in a day can become a problem, as many of the \glspl{segment} only make sense during the morning, or at night.

The standard random encounters can interrupt \pgls{sq} \gls{segment}, which makes for a lot of action when you put them all together.
In fact, it becomes so action-packed that a lot of \glspl{segment} focus on more `human' problems, such as negotiation, gossip, and funny scenes, rather than combat.
The random encounters inject plenty of combat already!

\subsection{\Glsfmtplural{region} Create Focus}

\Pgls{campaign} can focus on particular places by selecting the right \glspl{region}.
\Glspl{sq} which use `Forest, Roads, Town', as their \glspl{region}, will focus on overland travel, because the troupe will have to travel the wide lands between those places.
Conversely, \pgls{campaign} centred on a thieves' guild might have a set of regions all concerned with the local city; e.g. `Slums, Guild Halls, Market, Sewers'.
Or \pgls{campaign} about a fae wilderness could take place by the `Shining Lakes', `Enchanted Mountains', and `Cursed Swamp'.
Having exciting, bespoke events occur inside those locations will highlight them as areas of interest.

\end{multicols}

\section{Disjointed \Glsfmtplural{segment}}

\begin{multicols}{2}

\noindent
As the \glspl{pc} move around, they will slowly `trigger' new \glspl{sq}, and may soon find themselves enmeshed in four or five plot-lines simultaneously.
This may sound like an awful mess, but in practice it works out okay.
The players will usually not recognize that \pgls{segment} has begun a new plot until the second or third \gls{segment}.
The first \gls{segment} often exists merely as foreshadowing.
And if the players don't experience any \glspl{segment} for a while, they will quickly forget about it, until it returns.

All these tangled threads usually result in players seeing just one or two `main plots' at a time.
In some sense, they are always correct -- if the players take \pgls{sq} as the main one, then it has become their primary \gls{sq}.

But once these resolve, the other \glspl{sq} already have their ground-work laid out, in the form of rumours or small, forgettable encounters with \glspl{npc}.

However, you can still get too much of a good thing.
In order to wrap up a campaign cleanly, you will need some `plot-buffer', where the \glspl{pc} can still experience the world, but without starting any grand, new, plots.
This is why a campaign should end on a disjointed \gls{sq}, consisting of random \gls{segment} which have nothing to do with each other.

\sidequest[Forest,Roads,Town]{Random Events}

\sqpart{Roads}% AREA
{The Piper}% NAME
{An unpaid piper enchants all local children to follow him underground}% SUMMARY

The \gls{village}'s children run out, eager to find the source of the enchanted music.
Roll an encounter check as usual.

\sqpart{Forest}% AREA
{Granny's Soup}% NAME
{Baba Yaga has no meat for her soup}% SUMMARY

The party meet an old lady, who complains that she has no meat for her soup.
If the party don't help her out, she will return to her walking house (it has chicken legs) and chase them with spells.

\sqpart{Town}% AREA
{Dragon-Spotting}% NAME
{a great wyrm soars overhead}% SUMMARY

This dragon journeys from somewhere unknown, to somewhere unknown, with business that only a dragon could understand.
Gossip about the dragon will fill every \gls{village} and \gls{bothy} for weeks to come, but nothing will come of it -- the dragon has gone.

This \gls{segment} exists simply because dragons exist within \gls{fenestra}, and sometimes they have places to be.

\stopcontents[sq]

\end{multicols}


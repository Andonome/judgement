\chapter{\Glsfmtplural{sq} \& Story Weaving}

\label{sidequests}
\index{Weaving Stories}

\declareSQareas{Town,Roads,The Hollows}

\begin{multicols}{2}

\noindent
I have a strategy for writing stories, which I call `story weaving'.
You can think of it as a set of \glspl{sq}, woven together to make a tapestry.

The \gls{sq} structure makes much more sense once you see it, so I'm going to explain it through short examples.
You can get a quick feel for the overall plot by looking at the miniature table of contents at the start of every \gls{sq}.

\sidequest[Roads,Town,The Hollows]{The Beast}

\sqpart{Roads}% AREA
{Whispers of the Beast}% NAME
{Farmers approach the troupe, asking them to help slay a beast}% SUMMARY

\begin{exampletext}
  A strange beast has been spotted tearing apart a chitincrawler, and eating it on the spot.
  The farmers feel certain that if it ever attacks them, their archers won't be able to kill it.
\end{exampletext}

This first \gls{segment} of the \gls{sq} (`the Beast') can take place in any \gls{village}, or even anywhere on any road, since traders will hear about `the beast'.
Once the \glspl{pc} step out of their first \gls{broch}, you can spring this on them somewhere within the day.

If they encounter traders immediately, then those traders might tell them the rumours of `the beast'.
Or if they have a mission involving \pgls{village}, you might let them begin that mission, and have the farmers tell them about sightings of the beast late at night.

Once this \gls{segment} has finished, the \glspl{pc} may enter some other region, such as the local town.
This will not cause problems for the overall arch of the \gls{sq}, because the next part does not need to happen at a particular time.

\sqpart{Roads}% AREA
{Call to Arms}% NAME
{Scared farmers ask the troupe to pursue the beast}% SUMMARY

\begin{exampletext}
  The beast arrived, and approached the \gls{village} walls out of curiosity.
  The \gls{village} archers immediately shot arrows into its hide, while field-workers fled behind the \gls{village} walls.

  It left, but looked more insulted than injured.
\end{exampletext}

\begin{boxtext}
  Despite the Sun, the fields lie empty.
  You can see three archers on the left, and six on the right, all standing alert.
\end{boxtext}

The `boxtext' suggests how the scene may look once the \glspl{pc} enter, but you should be ready to discard it when it doesn't work.
If the troupe approach \pgls{village} during the day, the description may work fine, but if they arrive at night, or if they plan to keep walking along the roads, then you will need to come up with your own description.

Perhaps the \glspl{pc} will wander across the roads, stopping only at \pgls{bothy}.
In this case, you might adjust the scene by making traders inform the \glspl{pc} about the event in a nearby \gls{village}.

If the troupe remain on the roads, this \gls{segment} could occur a day after the previous \gls{segment}; or if they journey somewhere past the \gls{edge}, you will have to leave this \gls{segment} dormant while they do something else.
You have no time-limit to worry about; the beast could easily be stirring up rumours among farmers for weeks, so you can run this \gls{segment} at any time.

The players should not feel `railroaded' by this.
The farmers simply gripe because they feel scared, not because the players `have to do the quest'.

\paragraph{If the troupe ignore the farmers' please,}
nothing goes wrong.
The \glspl{pc} have no obligation to follow some pre-ordained plot.
Of course that will mean that the beast remains, wandering the forest, and occasionally scaring farmers.

\paragraph{If the troupe decide to kill the beast,}
they can make a roll to track it down.

The beast has a temporary lair nearby, which will give the troupe a challenging fight, assuming they can find it.

\sqpart{Roads}% AREA
{Guild Rivalry}% NAME
{The \gls{sunGuard} arrive, looking to slay the beast}% SUMMARY

\begin{exampletext}
  A troupe of \glspl{sunGuard} arrive, looking to slay the beast.
  A local \gls{warden} ordered them to find it, and has promised to pay them 10~\glspl{sp} each.
\end{exampletext}

Each of the \glspl{sunGuard} have shiny, expensive armour and arms, but they lack any kind of wilderness skills.

\paragraph{If the \glspl{pc} have killed the beast,}
the \glspl{sunGuard} decide to take the credit.

\paragraph{If the beast remains alive,}
the \glspl{sunGuard} journey out to kill the beast themselves.

This \gls{segment} can work fine, no matter how the \glspl{pc} behaved in previous \glspl{segment}.
This does not mean that the \glspl{pc}' actions have no consequences -- it simply means that other people in \gls{fenestra} will continue to act and plan, without viewing the \glspl{pc} as the centre of their world.

\sqpart{Town}% AREA
{The Beast's Mother}% NAME
{\Pgls{doula} asks the troupe if they have seen her `baby boy'}% SUMMARY

\begin{exampletext}
  \Pgls{doula} created the beast to protect \glspl{village} by killing monsters.
  She has grown fond of it, but doesn't know where it has gone.
  So she decided to come to the best place to pick up rumours around the outer \glspl{village} -- the town.
\end{exampletext}

Again, you can run this \gls{segment} any time the \glspl{pc} enter the local town.
The \gls{doula} may approach them in a bar, or may approach them on the street once she notices their \gls{guard} uniforms and figures out that they must be well-travelled.

The \gls{doula} swears vengeance upon anyone who killed the beast.

\paragraph{If the \glspl{pc} blame the \glspl{sunGuard},}
the \gls{doula} may go after those \gls{sunGuard}.

\paragraph{If the \glspl{pc} say that they killed the beast,}
the \gls{doula} spits in their faces, swears to kill them all, and leaves.

\paragraph{If the \glspl{pc} kill the \gls{doula},}
they will almost certainly face serious legal consequences.
Attacking women in the town's streets incurs the wrath of the \gls{sunGuard}, no matter what the victim said.

However, that death would also stop the last \gls{segment} from happening.
This sometimes happens with \glspl{sq}.

\sqpart{The Hollows}% AREA
{\squash~Vengeance}% NAME
{The \gls{doula} takes vengeance on whoever killed her pet}% SUMMARY

\begin{exampletext}
  Some dwarves owe the \gls{doula} a debt, so she requests that they help her kill whoever killed the beast.

  The \gls{doula} once saved the life of their matriarch, so they will fight fiercely to aid or protect her.
\end{exampletext}

The next time the troupe enter the Hollows (the long caverns which stretch for miles underground, and routinely peak out of holes in the earth), they will find that same \gls{doula}, or perhaps she will find them.

\paragraph{If the \gls{doula} believes the \glspl{sunGuard} killed her pet,}
then she will have sworn vengeance against them, publicly.
Once the \gls{sunGuard} heard about this, they began to pursue her, even into the Hollows below \gls{fenestra}.

In this case, the \glspl{pc} will find the \gls{sunGuard} in a stand-off with the \gls{doula}.

\paragraph{But if the \gls{doula} believes the \glspl{pc} killed her pet,}
she will approach quietly, with cut-throat witchcraft, just ahead of her dwarven companions.

As with every \gls{segment}, this part does not assume any outcome from any previous \gls{segment} in the \gls{sq} (except assuming that the \gls{doula} lives).
No assumptions are made about the location either, except that this segment occurs somewhere in the broad region known locally as `the Hollows'.

In retrospect, the players should notice that their decisions have impacted the world in just the ways one might expect; if they decided that `the beast' had not hurt anyone, so they should leave it alone, then they will find the \gls{doula} chasing after the \gls{sunGuard} instead of them.

\sidequest[Roads,Town]{The Fortune Maker}

\noindent
\Pgls{seeker} uses his ability to divine the future and notices criminal activity.
Soon after, the \gls{sunGuard} begin to arrest people based on his word of what they will do, i.e. they arrest people \emph{before} they commit crimes.
The \glspl{pc} will find the \gls{sunGuard} stalking them before long.

\sqpart{Roads}% AREA
{\squash~Palm Reading}% NAME
{A \gls{seeker} offers to read the troupe's fortune}% SUMMARY

\begin{exampletext}
  A local \gls{seeker} has been wandering across the land, giving out blessings and telling people about their future.
\end{exampletext}

The \gls{seeker} aids the \glspl{pc} with any immediate troubles, then offers to tell them their fortunes through palm-reading.

The `\squash' symbol before this part indicates that it should be run alongside the next available \gls{sq} in the region (`Roads').
They may encounter the \gls{seeker} while fighting someone, or hearing a rumour.
These `squashed', or `blended' \glspl{sq} serve two purposes.
They take an otherwise bland encounter, and melt it into the chaos of another \gls{segment}.
Then on the wider scale, these blended \glspl{segment} blend two \glspl{sq} together, which gives the feeling of a larger narrative as various plot-threads weave together.

The \glspl{pc} may refuse to have their palms read, which means you must adjust a later \gls{segment} in this arc.

\sqpart{Town}% AREA
{Precrime Chains}% NAME
{The troupe witness people condemned for crimes which the \gls{seeker} claims they will commit in the future}% SUMMARY

\begin{exampletext}
  The \gls{seeker} has accused a number of men of crimes that they will commit, and the \gls{sunGuard} have gone to the outer \glspl{village} to arrest them.
  Most are farmers, some are \glspl{guard}.
\end{exampletext}

The troupe pass by people, bound by chains and lead by the \gls{sunGuard}, who keep shouting that they are all innocent.

\paragraph{If the \glspl{pc} complain,}
the \glspl{sunGuard} tell them to take their complaints to the \gls{court} during their trials the next day.

While the this \gls{segment} should make sense almost anywhere in the local town, the town still has pre-set locations, `nailed to the map', such as the \gls{court}.
\Pgls{segment} can reference a set location (such as \pgls{doula}'s cabin, or a goblin warren) but will never assume that the \glspl{pc} will enter that location.

As \pgls{gm}, you can easily find yourself unprepared for the players' sudden changes in direction, their changing goals, and strange theories about the world around them.
But by pulling individual \glspl{segment} from a larger whole, you can retain some semblance of a larger picture, without restricting the \glspl{pc} movements, or the players' decisions.

\sqpart{Roads}% AREA
{Stalking and Clanking}% NAME
{The \gls{sunGuard} follow the troupe, waiting for crimes}% SUMMARY

\begin{exampletext}
  The \gls{seeker} who tells fortunes has informed the local \gls{warden} that the \glspl{pc} will commit crimes in the future.
\end{exampletext}

The \glspl{sunGuard} stalking the \glspl{pc} have no experience stalking people, so they will not present a massive challenge.

\humansoldier[\npc{\T[8]\Hu}{\arabic{noAppearing} \Glspl{sunGuard}}]

The troupe can spot them by rolling \roll{Wits}{Vigilance} against the \glspl{sunGuard}' \roll{Dexterity}{Stealth} (\tn).

Each \gls{sq} with a creature should always contain a full statblock on the same page or the next page, so you can see all the information required for the scene at a glance.

This \gls{segment} repeats until the \glspl{pc} have found a proper resolution, by speaking with the \glspl{sunGuard}, \pgls{warden}, or the \gls{seeker}.
Simply killing the \glspl{sunGuard} will not resolve the situation.

\paragraph{If the \glspl{pc} decided not to let the \gls{seeker} read their palms in the first \gls{segment},}
then they can confirm that the \gls{seeker} sometimes fabricates his predictions based on his hunches.

\paragraph{If the \glspl{pc} try to track down the \gls{seeker},}
have them roll \roll{Charisma}{Tactics} (\tn[12]) whenever they meet someone on the road.
Success indicates that they know the next fork on the road to take, and the \gls{tn} decreases by 2; but failure means the \gls{tn} increases by 1 as the character makes bad guesses about where the \gls{seeker} might have been, and where he will go.

The troupe can travel and continue making rolls, day by day, while other \glspl{segment} from other \glspl{sq} interrupt (but never prohibit) their mission.

\end{multicols}

\section{Plotting \Glsfmtplural{sq}}

\begin{multicols}{2}

\noindent
Juggling all these \glspl{sq} and \glspl{segment} threatens to become confusing, so each collection comes with a summary which plots out the \glspl{segment} \emph{per location}.

Have a look at the summaries below.
Notice that the first \gls{segment} comes with the `\sqr' symbol.
Once it's done, you can mark it off with an `X', then mark the next \gls{segment} in the \gls{sq} as ready with a `\sqr'.

Imagine what would happen if the troupe stick around on the Roads for a while; they could fly through the \glspl{segment} from the first \gls{sq}, before the second arrives as a distraction.
Compare that with how the events might unfold if the \glspl{pc} immediately go from the Roads the Hollows.

\end{multicols}

\printAllSideQuests{Roads, Town, The Hollows}

\begin{multicols}{2}

\subsection{Crafting \Glsfmtplural{segment}}

\Pgls{segment} must exist within these constraints:

\begin{enumerate}
  \item
  The \gls{segment} must make sense anywhere within a broad region.
  \item
  The \gls{segment} must make sense within almost any time frame, including weeks after the previous \gls{segment}.
  \item
  The \gls{segment} must not assume any outcome from a previous \gls{segment}.
  \item
  The \gls{segment} must not assume any reactions from the \glspl{pc}.
\end{enumerate}

\subsection{Pacing}

You can spring \pgls{segment} on the \glspl{pc} any time you feel like it.
When they enter a new region (e.g.`Town', or `Forest') makes a good opportunity for a new \gls{segment}, but you may pull out as many as one per day.
Pulling out two \glspl{segment} in a day can become a problem, as many of the \glspl{segment} only make sense during the morning, or at night.
But the \gls{gm} has to play the game by ear.

The standard random encounters roll occur at the same time as \gls{sq} \glspl{segment}, so when you put everything together, it makes for an action-packed series of events.
In fact, it becomes so action-packed that a lot of \glspl{segment} focus on more `human' problems, such as negotiation, gossip, and funny scenes, rather than combat.
The random encounters inject plenty of combat already!

\subsection{Regions Create Focus}

The regions which a series of \glspl{sq} use show you where a campaign's focus lies.
A campaign which uses the areas `Forest, Roads, Town' will focus on overland travel, because those broad areas have interesting things.

Conversely, a campaign centred on a thieves' guild might have a set of regions all concerned with the local city; e.g. `Slums, Guild Halls, Market, Sewers'.

When the players find exciting events in a region, they will quickly intuit that the place has more interesting events, and start to think in terms of `markets or sewers', rather than `mountains or caverns'.

\end{multicols}

\section{Disjointed \Glsfmtplural{segment}}

\begin{multicols}{2}

\noindent
As the \glspl{pc} move around, they will slowly `trigger' new \glspl{sq}, and may soon find themselves enmeshed in four or five plot-lines simultaneously.
This may sound like an awful mess, but in practice it works out okay.
The players will usually not recognize that \pgls{segment} has begun a new plot until they the second or third \gls{segment}.
The first \gls{segment} often exists merely as foreshadowing.
And if the players don't experience any \glspl{segment} for a while, they will quickly forget about it, until it returns.

All these tangled threads usually result in players seeing just one or two `main plots' at a time.
In some sense, they are always correct -- if the players take \pgls{sq} as the main one, then it has become their primary \gls{sq}.

But once these resolve, the other \glspl{sq} already have their ground-work laid out, in the form of rumours or small, forgettable encounters with \glspl{npc}.

However, you can still get too much of a good thing.
In order to wrap up a campaign cleanly, you will need some `plot-buffer', where the \glspl{pc} can still experience the world, but without starting any grand, new, plots.
This is why a campaign should end on a disjointed \gls{sq}, consisting of random \gls{segment} which have nothing to do with each other.

\sidequest[Roads,Town,The Hollows]{Random Events}

\sqpart{The Hollows}% AREA
{The Treasure Guardians}% NAME
{The troupe find dwarves, who found a gem-seam, but don't want the troupe to know about it}% SUMMARY

A dozen dwarves have found a gem-seam.
Their loud digging means they won't hear the troupe approaching.

Once they spot the \glspl{pc} they have a problem: if the \glspl{pc} leave, they may tell others about the seam.
The dwarves don't want to kill the \glspl{pc} but they may have to in order to ensure their special location remains a secret.

This \gls{segment} should test the players' negotiating skills, as nobody wants to fight with a dozen dwarves.

\sqpart{Town}% AREA
{Festivities Cancelled}% NAME
{The shipment of brandy for the upcoming festival has failed to arrive}% SUMMARY

A search of the road will turn up a dozen bodies and two wagons, hiding just off the road.
Bandits killed them all and pulled the third wagon farther down the road, where they took all the barrels of brandy to their forest lair, then dumped the wagon in the nearby river.

\sqpart{Roads}% AREA
{Dragon-Spotting}% NAME
{a great wyrm soars overhead}% SUMMARY

This dragon journeys from somewhere unknown, to somewhere unknown, with business that only a dragon could understand.
Gossip about the dragon will fill every \gls{village} and \gls{bothy} for weeks to come, but nothing will come of it -- the dragon has gone.

This \gls{segment} exists simply because dragons exist within \gls{fenestra}, and sometimes they have places to be.

\sqpart{The Hollows}% AREA
{The Wealthy Corpse}% NAME
{The troupe find \pgls{seeker}'s corpse in the depths}% SUMMARY

\begin{exampletext}
  \Pgls{seeker} journeyed into this cave to map out the local area, missed a path, became thoroughly lost, broke his leg then starved to death.
\end{exampletext}

The troupe will notice few signs of decomposition on the body.
They will may also notice his broken leg, and the fact that he tried to eat his own leather shoes.

His books are valuable but should by law be returned to the \gls{templeOfCuriosity}.

\stopcontents[sq]

\end{multicols}

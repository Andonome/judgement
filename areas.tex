\chapter{Civilization}

\section{The Map}

\begin{multicols}{2}

Welcome to Fenestra. Let's look around.

\mapentry{Features of the Land}

Take a pencil, a blank sheet of paper, and four dice.

\begin{itemize}
\item
  Imagine a circle spanning your page.
\item
  Draw 6 points along the circle, evenly spaced.
\item
  Label them 1-6.
\item
  Roll $1D6$ and draw mountains around the point with the number you rolled.
\item
  Draw hills outside of the mountains, until you reach the nearest two points.
\item
  Take the opposite point -- there lies the water.
\item
  Roll $1D6$ to find out the type of watery area:

  \begin{enumerate}
  \item
    Swamp.
  \item
    Lake.
  \item
    Large lake, which also covers the centre of the map.
    Add $1D3$ islands.
  \item
    Sea, extending to the edge of the map.
    Add $1D6-1$ islands, then give half of them a walled village.
  \item
    Sea, which extends to the centre of the map, then out to the edge.
    Add $1D6$ islands, each with a walled village.
  \item
    Vast sea. It stretches from this point, to the next numbered-point,%
    \footnote{All points go in a circular order, so after 6 comes 1.}
    and then out to the edge of the map.
    Add $1D6+1$ islands to your water, and give each a walled village.
  \end{enumerate}
\item
  Draw a dotted line between any islands, indicating sea-routes.
\item
  Draw contour-lines around any lakes or sea.
\end{itemize}

Forests cover all lands beyond the mountain, but do not draw them. Just
know that they cover everything.

\mapentry{Towns \& Rivers}

Your map has tiny human settlements, huddling in the little pockets of civilization they can create.
Much of the travel takes place along rivers, which let people travel faster than any other mode of transport.

\begin{itemize}
\item
  Roll $4D6$ and add one town to every point which comes up on the dice.
  Ignore repeats.

  \begin{itemize}
  \item
    Draw a small circle to indicate the town.
  \item
    Towns on mountains indicate dwarves, in which case, place their symbol by the town: `\Dw'.
  \end{itemize}
\item
  Draw a river from the mountains to each town.

  \begin{itemize}
  \item
    Each river starts as far away from the others as they can.
  \item
    Rivers often join, but never split.
  \item
    When towns sit behind a lake, the river meets the lake instead.
  \end{itemize}
\item
  Extend each river from towns to the water, weaving through every numbered point along the way.
\item
  For each town, add $1D6 \times 2$ walled villages around it.
  \begin{itemize}
  \item
    Most walled villages sit next to water or rivers.
  \item
    All walled villages have a dotted-line road to their town.
  \end{itemize}
\item
  Where the road crosses water, draw a small bridge.
\item
  Roll $2D6$; each point you roll has a route leading off the map, to lands uncharted.
  Draw the routes with a dotted line.
\end{itemize}

\mapentry{Town Character}

For each town, roll 1D6, and add a number equal to the total walled villages surrounding it.
If it has 4 walled villages, roll $1D6+4$.

\begin{multicols}{2}

\begin{enumerate}
\setcounter{enumi}{2}
\item
  Fresh town -- every day, stone and wood comes in to build a little more each day.
  Occasionally, parts of the wall collapse and archers flock to protect it.
\item
  Trackers
\item
  Deadly and magical plants
\item
  \ifodd\value{r4}
    Ale (and the guild has tight control over the area)
  \else
    White rock (actually limestone)
  \fi
\item
  \ifodd\value{r4}
    Wood-crafters
  \else
    Metallurgists
  \fi
\item
 Bards
\item
  Madness (neighbouring witches are to blame)
\item
  Witchcraft
\item
  \ifodd\value{r4}
  Friendly with the next town -- the locals go back and forth all the
    time.
  \else
    Cartographers
  \fi
\item
  Statues of heroes and gods litter the roads around.
\item
  Beast Arenas (where people watch the beasts of the forest fight)
\item
  Pious monks pray on every corner.
\item
  Bath houses
\item
  War brews with the next town.
  Soldiers will march soon\ldots

  Each one has 1D3 military outposts guarding against the other.
\item
  Demi-human suburbs, where all manner of other races live.
\item
  The grandest library in the area.
\end{enumerate}

\end{multicols}

\mapentry{They call it\ldots{}}

Select a name by combining what everyone knows it for, with the type of
land.
Write the town's name next to it, and underline it.

\begin{itemize}
\item
  People know a town in the hills for its library. That's why they call
  it `Page Valley'.
\item
  A swamp-edge town, famous for its bards, known as `Songmarsh'.
\item
  A town by the sea, famed for Witchery.
  They call it `Hexwave'.%
  \footnote{If you think these names sound stupid, you should look up the meaning of your hometown's name.}
\end{itemize}

\mapentry{It is Ruled by\ldots{}}

For each town, roll 1D6 to determine its ruler.

\begin{enumerate}
\item
  Multiple warring factions -- roll twice more, with +1 to the second roll.
\item
  Warlord, amassing an army to expand their power.
  Shipments of weapons come from the road to other lands.
\item
  Opulent noble, who demands 20\% tithes from all who enter.
  The local oddities annoy and fascinate them equally (from step \ref{mapOddities}).
\item
  Sorcerer king with no experience or business ruling.
  He once killed a fiend, how hard could balls and arranged marriages be?
  (roll 1D6 to find which type, on step \ref{fiendTypes})
\item
  Spoilt noble who has never seen a peasant.
\item
  A servant of the next fiend you roll.
  All who oppose them die on the road (step \ref{fiendTypes}).
\item
  A crime-lord, with a history of theft, now risen to fortune\ldots but not nobility.
\end{enumerate}

Note the town's ruler under its name.

\mapentry{Lonely Roads}

\begin{itemize}
\item
  At each town, draw a road to the next town.
\item
  If you have more than 3 towns, go half-way along each road, and
  connect to half-way along the opposite road (2 towns along).
\item
  If you have a crossroads, place a Lonely Tavern (see \autopageref{lonelyTaverns}).
\item
  If any islands sit alone and unconnected, connect them by sea and road to a town.
\end{itemize}

\mapentry{Oddities}
\label{mapOddities}

\begin{itemize}
  \item
  At every unoccupied point, place a hidden element (roll $1D6$).
  \item
  If the map's centre has no water, place a hidden element there too.
\end{itemize}

\begin{multicols}{2}
\begin{enumerate}\raggedright
\item
  Witch commune (\Hu)
\item
  Elves (\El)
\item
  Gnomes (\Gn)
\item
  Goblin warren (\N)
\item
  Gnolls (\Nl)
\item
  Lost city (\D) (\autopageref{lostCities})
\end{enumerate}
\end{multicols}

\begin{itemize}
  \item
  If you roll 4 or more, draw another river from the mountains to the oddity.
  \item
  This river then turns towards the edge of the map, until it intersects with another river, or leaves the area entirely.
\end{itemize}

\mapentry{Fiends in the Forest}

Dangerous people, and stranger creatures, live in the forests, controlling large territories.

\begin{itemize}
\item
  Roll 3D6: each point which comes up has one fiend between it and the
  next point.
\item
  Place the fiend on your map with the next number available, starting
  with `7'.
\item
  For each fiend, roll three dice, to determine its type, wishes, and
  ability.
\end{itemize}

\mapentry[fiendTypes]{Who are They?}

Roll $1D6$ to identify the fiend, then add this number to the fiend's next two rolls.

\begin{multicols}{2}
\begin{enumerate}
\item
  Bandit troupe
\item
  Ogre king
\item
  Dragon
\item
  Dryad
\item
  Lich
\item
  Hag
\end{enumerate}
\end{multicols}

Write the fiend's name on the map, and underline it.

\mapentry{The Fiend Has}

(1D6 plus fiend's number)

\begin{enumerate}
\setcounter{enumi}{1}
\item
  gained the sympathies of some local villagers, who will help them.
\item
  many servants across the land -- roll a $D6 + 1$ four times, and place a hidden camp in each number which comes up.
  Each one has a dozen soldiers.
  Mark the camps with an `X'.
\item
  made a temporary alliance with the last fiend on the list.
\item
  imprisoned a gnomish alchemist, and can force him to cast spells.
\item
  complete plate armour!
\item
  a powerful magical item.
\item
  myriad tunnels underground: currently empty, mostly. But nobody knows
  the full extent of them, only that myriad openings exist, most of
  which have been covered by a shallow layer of topsoil.
\item
  a small fortress on an island. No one can approach safely, or without
  being noticed.
\item
  a powerful magical item.
\item
  cast a spell which forces those who venture close to forget what they
  came for.
\item
  a garden which grows all manner of magical and deadly plants.
\end{enumerate}

\mapentry{The Fiend Wants}

(1D6 plus fiend's number)

\begin{enumerate}
\setcounter{enumi}{1}
\item
  to return to society, without giving up their gold.
\item
  to cement an alliance with the next fiend.
\item
  something beautiful to look at.
\item
  standard medical equipment.
\item
  to kill the next fiend, and take all they have.
\item
  standard supplies from villages. It's so hard to find someone who
  delivers!
\item
  a massive pot, three bags of thyme, a gnomish cook, and four fresh men.
\item
  to acquire magical ingredients.
\item
  to find out where those nearby oddities live, and destroy them (from stage \ref{mapOddities} of the map).
\item
  to watch birds in peace (do not make noise on the road).
\item
  a child to raise as her own.
\end{enumerate}

\mapentry{Name the Area}

Combine the fiend with the highest number, with the local water-type.
For example:

\begin{itemize}
\item
  `Hag' + `lake' = `Haglake'
\item
  `Dryad' + `islands' = `Drylands'
\item
  `Lich' + `sea' = `Portlich' (stress on the first syllable!)
\end{itemize}

\mapentry{Depth}

\begin{itemize}
  \item
  Think about what the local populations say about the local oddities from step \ref{mapOddities} -- do they know what lives there?
  \item
  Where do the routes off the map lead?
  Can people access them?
  Or has the area been cut off from other lands entirely, with the last road out in some uncharted area, without a road leading to it?
  \item
  What gossip would travellers hear on the road about local fiends and oddities?
  \item
  Fill in the missing names.
  Human rulers typically take their town's name as their last name, so someone in charge of a town called `Palemarsh' might be called  `Lord Cartpike Palemarsh', or similar.
  \item
  Pull out a pen and start adding depth and shading to the map.
\end{itemize}

\end{multicols}

\section{Strange Places}

\begin{multicols}{2}

\subsection{Lonely Taverns}
\label{lonelyTaverns}

These taverns exist on long stretches of road, far from any town, and
charge high prices for a drink. They must live off traders passing
through, and survive whatever the forest brings out. Normal people don't
stay for long. Those who stay a while often have problems with the local
law, as these places often make their own laws. Barkeeps punish any
robbery close to the tavern harshly, but don't often care about what
people do around the towns.

\mapentry{The Barkeep}

Roll $1D6$ to find this season's barkeep (they change all the time):

\begin{enumerate}
\item
  A veteran of the Night Guard, with a hundred war-stories. Of course
  when he tells them, nobody can get a drink, so don't ask!
\item
  Someone from one of the oddities (stage \ref{mapOddities}) -- perhaps a gnome, or someone
  descended from a nearby lost city, who speaks about plans to find it, and return.
\item
  An outlander from so far away, nobody has ever heard of it. Every
  story she tells sounds made-up, but the strange accent shows she really
  does come from somewhere distant.
\item
  A powerful mage who swore an oath never to use magic again. He won't
  say why.
\item
  A collective -- you stay as long as you like, earn your keep, then go
  when you please. Sometimes in the colder Seasons, the place just lies
  barren.
\item
  A dwarf who records all he can. The patrons say he works as a spy for
  someone, but they disagree about whom.
\end{enumerate}

\mapentry{The Menu}

Roll $3D6$ -- the \glspl{pc} can order any of these meals.

\newcommand\menuItem[3][(\arabic{r12} \glspl{cp})]{%
  \randomdozen%
  \randomthree%
  \randomfourB%
  \ifodd\value{enumi}
    \randomthreeC%
    \randomfour%
  \fi
  \item
  \textbf{#2:}
  #3
  #1
}

\begin{enumerate}
  \ifodd\value{r3}
    \menuItem[]{Bugger-all}{the first few barrels went rotten, and now someone's stolen the entire pot of soup.
    The menu will be limited for the day.}
  \else
    \menuItem[]{Get bent}{the barkeep's in a foul mood, because they need a day off.}
  \fi
  \menuItem{Griffin-wing}{freshly killed this morning, after the griffin tried to fly away with a gnomish patron.}
  \menuItem{Mystery-stew}{why are you hesitating?
  It goes rotten quick, so get eating!
  \footnote{Chitincrawler `meat' (webbing as sauce!)}
  }
    \ifnum\value{r4}<2
      \newcommand\morningSoup{uproot}
    \else
      \newcommand\morningSoup{marching_mushroom}
    \fi
    \menuItem{Sunrise Soup}{the chef found a new plant this morning, and he's already learning how to cook it!
    \footnote{In fact this is \nameref{\morningSoup}, see \autopageref{\morningSoup}, for the effects.}
    }
  \menuItem{Deer}{thank the man in black, sitting in the corner.}
  \ifodd\value{r4}
    \menuItem{Dwarf-beard}{actually just a type of seaweed, left as payment by a local trader; but it tastes just like the real thing!}
  \else
    \menuItem{Eye-Spy}{made with actual woodspy.}
  \fi
\end{enumerate}

\subsection{Lost Cities}
\label{lostCities}

People call them `stone dragons', to remind the young how deadly an apparently empty city becomes.
Strange creatures always take residence in them.

Before the \glspl{pc} enter, roll dice to make the town with steps \ref{lostCataclysm} to \ref{lostTowers}.

Once the \glspl{pc} enter, they make some rolls each \gls{interval}.

\begin{itemize}
  \item
  Roll to see when dweller interactions occur next.
  During this time, the \glspl{pc} can sneak about more easily.
  (\nameref{lostCries})
  \item
  The \glspl{pc} then try to move quietly through the city, without drawing the attention of those who dwell there.
  (\nameref{lostWhispers})
  \item
  If the \glspl{pc} fail to sneak, the chase is on!
  (\nameref{lostChase})
  \item
  If the \glspl{pc} go unnoticed, they can attempt to forage.
  (\nameref{lostForaging})
  \item
  Most \glspl{pc} will struggle with foraging, but they can get better at it by climbing the towers from step \ref{lostTowers}, or splitting up and making individual rolls.
\end{itemize}

\begin{figure*}[b!]

\begin{nametable}[c|p{.4\textwidth}|L|L]{Foraging}

  \textbf{Roll} & \textbf{Place} & \textbf{Prize 1} & \textbf{Prize 2}\\\hline

  \textbf{6} &
    Under a hidden floorboard -- broken patches show something underneath &
    \lootJewellery &
    \lootMagic \\

  \textbf{5} &
    Hidden room, behind an old bookcase &
    \lootJewellery &
    \lootJewellery, and \lootBig \\

  \textbf{4} &
    In empty home, inside a dark and empty doorway &
    \lootJewellery, and \lootJewellery &
    \lootJewellery, and \lootBig \\

  \textbf{3} &
    Inside a largely preserved house.
    The family died peacefully, somehow\ldots &
    \lootMedium{} and, surprise, woodspy! (\autopageref{woodspy}) &
    \lootJewellery \\

  \textbf{2} &
    Behind a fully stone door, clearly used as a safe.
  Opening it requires Intelligence + Crafts (\gls{tn} 12 to do so quietly, otherwise, \gls{tn} 7) & \lootJewellery & \lootMagic  \\

  \textbf{1} &
    Lying under a pile of human bones &
    \lootBig &
    \lootMagic \\

\end{nametable}

\end{figure*}
\mapentry{The Cataclysm}
\label{lostCataclysm}

Everyone fled the city long ago.
The neighbouring towns don't always remember exactly where the city lies, but they always remember the tale of how it fell.

\begin{enumerate}
  \item
  They burnt the wrong witch, and her enraged sister came to the city.
  She made plants grow between every brick, and pulled down hailstones in the shape of carrots.
  They killed her in the end, but the walls lay cracked beyond repair, and the central citadel had fallen.
  \item
  A fire started.
  As people banded together to walk outside and escape the smoke, hungry residents of the forest began to watch them, and pick them off, bit by bit.
  Some stayed in the nearby villages, but they could not keep everyone inside, so the rest tried to take boats or walk away.

  Many who remained died of smoke inhalation.
  The rest found that walls without guards do little good, as the fire died out, and beasts began to creep in.

  By the time some inhabitants dared to return to the blackened rocks of their city, the forest had claimed the town as its own.
  Then the city's new dwellers grew hungry, and one day the villages disappeared.
  \item
  A dragon came, full of hatred and hunger.
  They say it still sleeps somewhere in the city, but who knows?
  \item
  Underground tunnels opened, and little goblins popped up.
  Others followed soon, and one day, not long after, a horde ascended from the \gls{deep} to feast upon the city.
  \item
  Foolishness and bad luck lead to an unlikely series of calamities.
  The local lords began a war with another civilization, and lost.
  With fewer soldiers about, people suffered more casualties from the forest.
  Less food from the farms lead to theft, and the lords demanded executions.

  When the city's fighting-beasts escaped their cages, and fled from the arena through the streets, people decided they could take no more, and many left.
  With a reduced population, limited food, and walls too long to properly guard, the city collapsed in on itself, bit by bit.
  \item
  Local witches had warned not to expand the villages -- the nearby villagers had strange ideas, and too many developed sorcery of some kind, when the secrets were whispered to them in dreams.

  No great even happened here -- the population simply shrank more than it grew, with a thousand little catastrophes.

  Some say spirits haunt the place.
  Others warn not to drink the river water.
  Whatever the truth, that land is cursed.
\end{enumerate}

\mapentry{Wandering Dwellers}

Roll $3D6$ -- each number which comes up determines one type of city-dweller.
Ignore any repeats.

\begin{enumerate}
  \item
  \textbf{Oozes} of all types slide across the streets, which seem strangely clean, and free of debris or foliage.
  \item
  \textbf{Chitincrawlers} hide in every abode.
  Verdant berries of every colour have encouraged various deer into the area, but every shadow gleams with thick webbing.

  $1D6+8$ live here here in total.
  \item
  \textbf{Griffins} look down from every tower.
  The high towers make perfect perches to surveille the area, and the degrading wood helps to make nests.

  $1D6+4$ live here in total.
  \item
  \textbf{Crazed Witches} occasionally enter these cities, looking for spell components.

  Roll $1D6$ to find their number.
  \item
  \textbf{Demilich} covens help these undead sorcerers study with their own kind.
  However, their lack of basic empathy makes them dangerous to each other -- none of them really trust the others, so they share information slowly, always hinting that they have more to teach while masking their true abilities.

  Roll $1D3 + 1$ to determine the number of demiliches.
  \item
  \textbf{Dragon} eggs make for powerful magical items, so many dragons like nesting in strange and dangerous areas.
  Of course, only the largest of buildings, such as town halls, or theatres, can house such a massive creature.
\end{enumerate}


\mapentry{Dweller Relations}
Each dweller in a lost city has a relationship with the others, depending on their relative numbers.
Those with a higher number control or prey on those with a lower number.

\begin{enumerate}
  \item
  Dwellers within \arabic{enumi} step of each other become allies if sentient, and otherwise ignore each other.
  \item
  A dweller numbered \arabic{enumi} greater than another becomes aggressive, and the two begin to fight.
  \item
  Dwellers \arabic{enumi} ahead of another kill the lower form, and use the bodies to feed, or cast spells.
  \item
  Dwellers \arabic{enumi} ahead of another use magic or threats to control the lesser dweller.
  \item
  Any dweller \arabic{enumi} ahead of another will ignore them entirely.
\end{enumerate}

Draw a line, triangle, or square, to show the relations of the dwellers in the lost city.

\mapentry{Tall Towers in the Labyrinth}
\label{lostTowers}

Wooden rooves rot, leaving bear stone walls across much of the fallen city.
Some walls collapse, opening new passages, while trees grow and thorny bushes grow to block doors and streets.
A twisted, mossy, labyrinth forms.

Despite the rot and chaos, some buildings still stand tall.
Climbing these buildings gifts a wide perspective of the city, which grants a bonus to the \glspl{pc} foraging rolls.

Roll $4D6$: one for each tower.

\begin{enumerate}
  \item
  This tower has fallen to the ground, leaving nothing worth climbing.
  \item
  This tower has one wall remaining -- a Speed + Athletics check (\gls{tn} 8) allows the climber to see a little in all four directions.
  \begin{itemize}
    \item
    During daylight, the perspective adds +1 to all foraging rolls.
  \end{itemize}
  \item
  This tower has a single candle, and the light is clearly visible at night.
  If anyone enters, any sentient city-dwellers in the area will see them blocking the light, and investigate.
  (If there are many, pick the lowest numbered dweller)
  \begin{itemize}
    \item
    If no sentient life exists in the area, the candle has been left by a traveller who has left a note stating which creatures he has seen.
    \item
    All further towers on the roll of a \arabic{enumi} create an encounter with the dwellers with the lowest number.
    \item
    During daylight, the perspective adds +1 to all foraging rolls.
  \end{itemize}
  \item
  This tower stands tall enough to see all around.
  The bones of dead humans fill the stairway.

  Any movement which disturbs the bones will send them falling down the stairs with a clank and a thud.
  \Glspl{pc} must roll Dexterity + Stealth (\gls{tn} 11) to move past the bones without issue, or else waste \pgls{interval} moving everything down by hand (in this case they roll Dexterity + Crafts, \gls{tn} 7).
  \begin{itemize}
    \item
    During daylight, the perspective adds +2 to all foraging rolls.
  \end{itemize}
  \item
  The tallest of towers allow anyone inside to see all around.
  They can receive a complete list of towers in the area.
  \begin{itemize}
    \item
    During daylight, the perspective adds +2 to all foraging rolls.
    \item
    The first time \pgls{pc} climbs to this height, they see two dwellers interacting.
  \end{itemize}
  \item
  The tallest towers often have the least stability.
  The \gls{pc} who climbs this far will receive the same benefits as above, but if their \gls{weight} is above 5, the structure begins collapsing.

  They can run down (Speed + Athletics, \gls{tn} 9), but failure indicates the building has collapsed upon them, inflicting $4D6$ Damage.

  Or the \gls{pc} can elect to jump, and try to grab something nearby (Speed + Athletics, \gls{tn} 12), but failure indicates that they will suffer a nasty fall of $2D6$ Damage.
\end{enumerate}

\bigLine

\subsubsection{Cries in the Night}
\label{lostCries}

When fiends interact, they often make noise.
Witches casting spells upon local oozes, or dragons ordering minions about can't be done quietly; although the \glspl{pc} will probably not understand what the noises mean.

Every $1D6$ \glspl{interval}, the \glspl{pc} hear an interaction.

\subsubsection{Whispers in the Cold Labyrinth}
\label{lostWhispers}

When the \glspl{pc} want to move through a lost city, they test Dexterity + Stealth to remain quiet.
The \gls{tn} is normally 8, but often easier.

\begin{itemize}
  \item
  -1 if moving at night.
  \item
  -1 in the rain.
  \item
  -1 in a storm.
  \item
  -2 if the fiends are busy with something.
  \item
  -4 if hiding indoors.
  \item
  +5 if lighting a fire.
\end{itemize}

\subsubsection{The Chase}
\label{lostChase}

Any time the \glspl{pc} accidentally make a noise, they can try to gingerly run and hide.

\begin{itemize}
  \item
  They can try to escape without making more noise with another Speed + Stealth check, this time at \gls{tn} 10.
  \item
  The dweller with the lowest number arrives the first time the \glspl{pc} fail a check, then the next dweller on the list, and so on.
  \item
  Once all dwellers have been alerted, the first two arrive to investigate the noise at the same time.
  From then on, all dwellers come to investigate at every noise.
  \item
  Up to $1D3$ dwellers emerge at a time.
\end{itemize}

\subsubsection{Foraging Quietly}
\label{lostForaging}

Each time the \glspl{pc} forage, they roll Intelligence + Vigilance (\gls{tn} 12) to figure out where the best loot lies.
If they succeed, roll 2D6 -- the first die determines the place, and the second determines the prize!

\begin{itemize}
  \item
  Once they have a prize, cross it off the list.
  \item
  If they get the same number again, they will find the second prize with that number.
  \item
  The third time they roll a number, they find nothing -- lost cities only hold so many treasures.
  \item
  If the \glspl{pc} split up, each one can perform individual rolls, so they will loot far more efficiently, but run more chance of one getting caught alone.

\end{itemize}

\end{multicols}

\needspace{12em}
\hrulefill

\begin{multicols}{2}

\jelly

\jelly

\jelly

\griffin

\chitincrawler

\demilich

\dragon

\end{multicols}

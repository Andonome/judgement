\chapter{Forest's Edge}

\begin{multicols}{2}

\subsection{The Map}

Welcome to Fenestra. Let's look around.

\mapentry{Features of the Land}

Take a pencil, and a blank sheet of paper.

\begin{itemize}
\item
  Imagine a circle spanning your page
\item
  Draw 6 points along the circle, evenly spaced.
\item
  Label them 1-6.
\item
  Roll $1D6$ and draw mountains around the point with the number you rolled.
\item
  Draw hills outside of the mountains, until you reach the nearest two
  points.
\item
  Take the opposite point - there lies the water.
\item
  Roll $1D6$ to find out the type of watery area:

  \begin{enumerate}
  \item
    Swamp.
  \item
    Lake.
  \item
    Large lake, which also covers the centre of the map.

    \begin{itemize}
    \item
      Add $1D3$ islands.
    \end{itemize}
  \item
    Sea, extending to the edge of the map.

    \begin{itemize}
    \item
      Add $1D6-1$ islands, then give half a walled village.
    \end{itemize}
  \item
    Sea, which extends to the centre of the map, then out to the edge.

    \begin{itemize}
    \item
      Add $1D6$ islands, each with a walled village.
    \end{itemize}
  \item
    Vast sea. It stretches from this point, to the next numbered-point,%
    \footnote{All points go in a circular order, so after 6 comes 1.}
    and then out to the edge of the map.
    \begin{itemize}
    \item
      Add $1D6+1$ islands to your water, and give each a walled village.
    \end{itemize}
  \end{enumerate}
\item
  Draw a dotted line between all islands and nearby towns, indicating
  sea-routes.
\item
  Draw contour-lines around any lakes or sea.
\end{itemize}

Forests cover all lands beyond the mountain, but do not draw them. Just
know that they cover everything.

\mapentry{Towns \& Rivers}

Your map has tiny human settlements, huddling in the little pockets of civilization they can create.
Much of the travel takes place along rivers, which let people travel faster than any other mode of transport.

\begin{itemize}
\item
  Roll 4D6 and add one town to every point which comes up on the dice.
  Ignore repeats.

  \begin{itemize}
  \item
    Draw a small circle to indicate the town.
  \item
    Towns on mountains indicate dwarves.
  \end{itemize}
\item
  Draw a river from the mountains to each town.

  \begin{itemize}
  \item
    Each river starts from a separate point in the mountains.
  \item
    Rivers join after leaving a town, but never split.
  \item
    When towns sit behind a lake, the river meets the lake instead.
  \end{itemize}
\item
  Extend each river from a town to the water, weaving through every
  numbered point along the way.
\item
  For each town, add $1D6 \times 2$ walled villages around it.

  \begin{itemize}
  \item
    Most walled villages sit next to water or rivers.
  \item
    All walled villages have a road to their town.
  \item
    Nearby walled villages have roads leading to each other.
  \end{itemize}
\item
  Draw a dotted line between each town, indicating a road.
\item
  Roll $1D6$ - this point has a route leading off the map, to lands
  uncharted. Draw the route with a dotted line.
\end{itemize}

\mapentry{Everyone knows it for\ldots{}}

For each town, roll 1D6, and add a number equal to the total walled villages surrounding it.
If it has 4 walled villages, roll $1D6+4$.

\begin{multicols}{2}

\begin{enumerate}
\setcounter{enumi}{2}
\item
  Cartographers
\item
  Deadly and magical plants
\item
  Bath houses
\item
  Trackers
\item
  White rock
\item
  Craftsmen
\item
  Bards
\item
  Madness
\item
  Witchcraft
\item
  Beast Arenas
\item
  Friendly with the next town - the locals go back and forth all the
  time.
\item
  Statues of heroes litter the roads around.
\item
  Pious monks pray on every corner.
\item
  War brews with whoever lives on the next point on the map. Once the
  arrow supplies arrive, soldiers will march.

  \begin{itemize}
  \item
    Each one has 1D3 military outposts guarding against the other.
  \end{itemize}
\item
  Demi-human suburbs, where all manner of other races live.
\item
  The grandest library in the area.
\end{enumerate}

\end{multicols}

\mapentry{They call it\ldots{}}

Select a name by combining what everyone knows it for, with the type of
land.

\begin{itemize}
\item
  People know a town in the hills for its library. That's why they call
  it `Page Valley'.
\item
  A swamp-edge town, famous for its bards, known as `Songmarsh'.
\item
  A town by the sea, famed for Witchery. They call it `Hexwave'.
\end{itemize}

Write the town's name on the line next to its number.

\mapentry{It is Ruled by\ldots{}}

For each town, roll 1D6 to determine its ruler.

\begin{enumerate}
\item
  Multiple warring factions - roll twice more, with +1 to the first
  roll.
\item
  Warlord, amassing an army to expand their power.

  \begin{itemize}
  \item
    Shipments of weapons come from the road to other lands.
  \end{itemize}
\item
  Opulent noble, who demands 20\% tithes from all who enter.

  \begin{itemize}
  \item
    The local hidden Oddities annoy and fascinate them equally.
  \end{itemize}
\item
  Sorcerer king with no experience or business ruling.

  \begin{itemize}
  \item
    He once killed a fiend (roll 1D6 below to find which type), how hard
    could balls and arranged marriages be?
  \end{itemize}
\item
  A servant of the next fiend. All who oppose them die on the road (see
  below).
\item
  Spoilt noble who has never seen a peasant.
\item
  A crime-lord, with a history of theft, now risen to fortune\ldots but
  not nobility.
\end{enumerate}

Note the town's ruler under its name.

\mapentry{Lonely Roads}

\begin{itemize}
\item
  At each town, draw a road to the next town.
\item
  If you have more than 3 towns, go half-way along each road, and
  connect to half-way along the opposite road (2 towns along).
\item
  If you have a crossroads, place a Lonely Tavern (see \autopageref{lonelyTaverns}).
\end{itemize}

\mapentry{Hidden Oddities}

Every unoccupied point, plus the map's centre, receives a hidden
element:

\begin{multicols}{2}
\begin{enumerate}
\item
  Goblin warren
\item
  Gnolls
\item
  Lost city
\item
  Witch commune
\item
  Elves
\item
  Gnomes
\end{enumerate}
\end{multicols}

\mapentry{Fiends in the Forest}

Dangerous people, and stranger creatures, live in the forests, controlling large territories.

\begin{itemize}
\item
  Roll 3D6: each point which comes up has one fiend between it and the
  next point.
\item
  Place the fiend on your map with the next number available, starting
  with `7'.
\item
  For each fiend, roll three dice, to determine its type, wishes, and
  ability.
\end{itemize}

\mapentry{Who are They?}

\begin{multicols}{2}
\begin{enumerate}
\item
  Bandit troupe
\item
  Ogre king
\item
  Dragon
\item
  Dryad
\item
  Lich
\item
  Hag
\end{enumerate}
\end{multicols}

Roll $1D6$ twice, and add the fiend's number to each roll, to find out its desires and abilities.

\emph{You roll a `3', indicating a dragon. You roll `2, 6'. 3+2 = 5, so
this dragon needs standard medical supplies. Perhaps it has a
tooth-ache. 3+3 = 6, so this dragon wears full plat armour}

\mapentry{The Fiend Wants}

\begin{enumerate}
\item
  to return to society, without giving up their gold.
\item
  to cement an alliance with the next fiend.
\item
  something beautiful to look at.
\item
  standard medical equipment.
\item
  standard supplies from villages. It's so hard to find someone who
  delivers!
\item
  a massive pot, three bags of thyme, a gnomish cook, and four fresh
  men.
\item
  to acquire magical ingredients.
\item
  to find out where those nearby hidden oddities live\ldots{}
\item
  to kill the next fiend, and take all they have.
\item
  to watch birds in peace (do not make noise on the road).
\item
  a child to raise as her own.
\end{enumerate}

\mapentry{The Fiend Has}

\begin{enumerate}
\item
  gained the sympathies of some local villagers, who will help them.
\item
  great many soldiers - $1D3 \times 30$ of them exist, with around 15 in each
  camp.
\item
  made a temporary alliance with the last fiend on the list.
\item
  imprisoned a gnomish alchemist, and can force him to cast spells.
\item
  complete plate armour!
\item
  a powerful magical item.
\item
  myriad tunnels underground: currently empty, mostly. But nobody knows
  the full extent of them, only that myriad openings exist, most of
  which have been covered by a shallow layer of topsoil.
\item
  a small fortress on an island. No one can approach safely, or without
  being noticed.
\item
  a powerful magical item.
\item
  cast a spell which forces those who venture close to forget what they
  came for.
\item
  a garden which grows all manner of magical and deadly plants.
\end{enumerate}

\mapentry{Name the Area}

Combine the fiend with the highest number, with the local water-type.
For example:

\begin{itemize}
\item
  `Hag' + `lake' = `Haglake'
\item
  `Dryad' + `islands' = `Drylands'
\item
  `Lich' + `sea' = `Portlich' (stress on the first syllable!)
\end{itemize}

\mapentry{Inking}

Pull out a pen and emphasise what you can.

\end{multicols}

\section{Strange Places}

\begin{multicols}{2}

\subsubsection{Lonely Taverns}
\label{lonelyTaverns}

These taverns exist on long stretches of road, far from any town, and
charge high prices for a drink. They must live off traders passing
through, and survive whatever the forest brings out. Normal people don't
stay for long. Those who stay a while often have problems with the local
law, as these places often make their own laws. Barkeeps punish any
robbery close to the tavern harshly, but don't often care about what
people do around the towns.

Roll $1D6$ to find the barkeep:

\begin{enumerate}
\item
  A veteran of the Night Guard, with a hundred war-stories. Of course
  when he tells them, nobody can get a drink, so don't ask!
\item
  Someone from one of the hidden oddities - perhaps a gnome, or someone
  descended from a nearby lost city.
\item
  An outlander from so far away, nobody has ever heard of it. Every
  story he tells sounds made-up, but the strange accent shows he really
  does come from outside.
\item
  A powerful mage who swore an oath never to use magic again. He won't
  say why.
\item
  A collective - you stay as long as you like, earn your keep, then go
  when you please. Sometimes in the colder Seasons, the place just lies
  barren.
\item
  A dwarf who records all he can. The patrons say he works as a spy for
  someone, but they disagree about whom.
\end{enumerate}

\end{multicols}

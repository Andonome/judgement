\chapter[House of Law]{Laws}
\section{Morale}
\label{morale}
\index{Morale}
\begin{multicols}{2}

\noindent
Unsure if your \glspl{npc} want to fight?
Roll their Combat or Brawl Skill at \gls{tn} 7, plus the modifiers in the Morale Chart, before combat starts.
This group roll counts for everyone, so if the group roll a total of \gls{tn} 7, but one member is wounded, that member will fail the roll and flee.
Of course on the next round, this may prompt others to flee, as it changes the proportions of creatures to \glspl{pc}.

You can use a single roll for an entire combat -- the \gls{gm} simply keeps that roll hidden.
If the enemy rolls a `12', all of them will probably fight until they die.
If they roll a `7', they may start to flee once wounded, and then more will flee once only half remain (but they continue to recheck only at the start of a round).

Most combats will end with one side or the other running away -- few troops want to fight to the last man when they could potentially be safe at home by the end of the day.

The players do not take morale checks -- they decide when it's time to run away by the look of the situation.
Usually a good time is when all the \gls{fp} have run out.
\footnote{The \glsentrytext{gm} may also wish to cut all Morale checks for any \glspl{npc} with remaining \glsentrytext{fp}.}

When an enemy flees the scene after a fight has begun, characters still gain full \glspl{xp} for the fight, since they still `defeated' the enemy.

On a tie, the \gls{gm} finally gets to decide what should happen.

\end{multicols}

\moralechart


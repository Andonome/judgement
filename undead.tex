\chapter[The Undead]{The Undead \D}
\index{Undead}

\section{Properties of the Dead}
\label{undead}

\begin{multicols}{2}

\noindent
Undead creatures have certain properties in common.

\subsection{Feeding}
Firstly they imperceptibly feed from the souls of the living.
This is not performed with the mouth but by merely being close to dying things and absorbing them before they can wander to the next realm.

The undead do not regain mana over time.
Rather, intelligent undead who use magic must kill to regain mana.
Every dead creature within their vicinity regains them 1 \gls{mp}, plus the creature's Intelligence Bonus (if positive).

The `range' of this ability is equal to five steps, plus five steps for each Wits Bonus of the undead thing consuming the soul (again, if positive).
Ties go to whichever of the dead has the highest Intelligence Bonus, then Wits Bonus.

\subsection{Senses}

Undead eyes generally do not work, instead they `see' the souls of people shining outward.
Inanimate objects such as books, or even fellow undead, are not so clearly seen; the undead can avoid bumping into these objects but have great trouble reading anything or working fine machinery.
However, they can operate in complete darkness and even fight without penalty, using the light of living people's souls to see them.
They can also see living beings from a great distance due to the soul-light they emit.

Undead also feel no pain and suffer little from scrapes and bruises.
As a result, they automatically have a \gls{dr} of 2 which stacks with armour in the usual way.%
\footnote{See page \pageref{stackingarmour}.}
This counts as Complete armour, but not Perfect -- shots through their eyes or attacks which sever muscles still debilitate them.

\subsubsection{Communication}

Ageing corpses -- even those that age fairly well -- lose their ability to speak entirely.
Any ghast who wishes to speak will have to resort to either magic, writing, or some other system, because a dead tongue and dead lungs can never articulate things properly.
This handicap never seems to fluster the undead when they want to use magic -- they seem to have some kind of `speech of the soul', which works fine without an actual tongue.

\subsection{Bodies}

When the undead are newly created, they are clumsy, as they are not used to their own bodies, and suffer a -2 penalty to Dexterity.
Shortly afterwards, rigour mortis sets in, and then decay.
Any undead more than a few hours old gain a -2 penalty to their Speed Bonus, but lose the Dexterity penalty.

The undead do not tire -- they take no Fatigue Points.
They can walk or dig or fight endlessly, without complaint.
They enjoy feeding on souls, but it is not required for them to continue moving.

\subsubsection{Aggression}
The undead natural aggression translates into a basic Brawl score of +2.

\end{multicols}

\section{Classifications of the Undead}

\begin{multicols}{2}

\noindent
The undead are native to nowhere.
They exist purely as a result of necromantic spells.
They can take any number of forms and have varied Traits -- the following are presented just quick reference.

\best[\D\E]{Demilich}\label{demilich}
\index{Undead!Demilich}

Any necromancer worth their salt eventually joins the dead, but never really dies.
Demiliches are the first stages of a lich, as they ascend to become a machine of pure power, gathering undead forces, powerful spells, and a formidable lair.

\paragraph{Ecology:} While in theory these creatures can live anywhere, most live in secluded areas.
Deep caves, horrible deserts, or icy mountains provide excellent spots for the dead because they are such difficult areas for the living.
Snowy wastelands, filled with frozen corpses, ready to walk again once called, can provide the perfect location for an ice palace.

\paragraph{Encounters:}

\begin{itemize}

  \item
  The demilich covers a dozen corpses in pitch and sends them into the village.
  Once the houses start to burn, it plans to send ghasts to pick off fleeing villagers.
  The dead move easily through the smoke, but living creatures suffer 2 Fatigue Points each round they wander between houses, and 4 when inside the houses.
  \item
  A demilich wanders with its horde, and spots the characters a mile away.
  It begins by cursing them, draining them of their \glsentrylongpl{fp}.
  If they flee, it wanders after them, keeping its small enclave of ghouls with it at all times.

\end{itemize}

\paragraph{Tactics:}
Step 1: Gather corpses to make ghouls.
Step 2: Use ghouls to gather more powerful corpses.
Step 3: Repeat.

When Demiliches arise, they require somewhere as a base of operations.
Too far from civilization is bad as they cannot gather the corpses necessary to raise an army.
Too close is also bad as they cannot hide.

Often they turn the first corpses they find into ghasts -- powerful, sentient, undead.
They occasionally go on missions for more powerful corpses, such as strong humans, bears, ogres, or horses.

\demilich

\best[\D]{Ghoul}
\label{ghoul}
\index{Undead!Human}

Ghouls are the bread and butter of necromancers.
When they first rise from the dead, they stumble clumsily, and chase after any living humanoid.
With nobody nearby, some just stand there, and most wander aimlessly.
Within hours, the undead get used to their form but their bodies seize up from their dead state, and they start to give off the standard stench of death.

The given example is a standard human ghoul.
To make ghouls of other races, simply change the default Strength bonus -- dwarves have Strength 0, gnomes have Strength -2.

\ghoul[\npc{\D}{Ghoul}]

\paragraph{Tactics:}
Ghouls attack in swarms and almost always grapple their targets as a first attack in order to claw and bite in later attacks.
Where most creatures wouldn't grapple people mid-battle, because it makes them vulnerable, these non-sentient undead have no sense of self-preservation.
They grab, bite, and the other dead have an easier time assaulting the target after that.%
\iftoggle{core}{%
  \footnote{See the core rules, page \pageref{grappling}, for grappling rules.}
}{}

\paragraph{Encounters:}
The undead are created, and their creator is rarely far from them.
Most encounters will have them directed, and created with some specific purpose in mind.

Since the undead are slow, they will generally attack when people are cornered, or the undead will be used to gain some specific ground, or attack a given resource.

\best[\D]{Ghast}
\label{ghast}
\index{Undead!Human Ghast}

Such creatures are often masterworks created by a necromancer as a personal body-guard or the lead warrior in an army.

\ghast[\npc{\D\E}{Ghast}]

\paragraph{Ecology:} They are fully sentient, but often too intent upon feeding on humanoid souls to do much except obsess over murder.
Occasionally, one will escape and control its urges enough to find a secluded spot and simply try to exist.
These creatures often end up haunting local crypts, mines or other forgotten areas.

\paragraph{Encounters:}

\begin{itemize}

  \item
  Two ghasts hide among a pack of standard ghouls.
  They wander slowly, at the back of the pack, then suddenly unsheathe their swords and attack ferociously.
  \item
  A ghast kills travellers, then lays down with the dead, pretending to be one of their number.
  When the party arrive and examine the dead, it jumps the weakest member while they're off-guard.
  \item
  Using its ability to see the living at all times, a ghast stalks the party from a long distance.
  It waits until they enter combat with some other force, and jumps in as soon as they become injured.

\end{itemize}

Ghasts under the control of a necromancer can plan admirably, and often pretend to be yet another one of the dead.

Independent ghasts tend to stalk prey from afar, and will often not attack adventurers while they are strong.
Instead, they continue stalking until other problems arise, and go for the kill if the travellers are ever wounded.

\end{multicols}



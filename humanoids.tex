\section[Sentient Creatures]{Sentient Creatures \E}

\begin{multicols}{2}

\noindent
These are the five sentient races which change the face of Fenestra.
A few examples of each will be presented -- these will not match up with \glspl{pc}' starting Attributes for a number of reasons.
Elves are very long lived, but players will only be able to portray young elven characters -- similar but less extreme things could be said of gnomes or dwarves.
The examples given will be of common examples of the races, together with a soldier or some other adventuring archetype, and a magic user.
When crafting your own encounters, individual examples of a race should still have unique Traits -- these are simply provided for fast reference.

\subsection[Dwarves]{\Dw\ Dwarves}
\index{Dwarves}
\label{best_dwarves}

\begin{boxtext}

  You can't read dwarvish, but writing on cavern walls, in blood never, looks friendly.
  Nevertheless, there's no going back.

\end{boxtext}

Dwarves can make excellent underground enemies or friends depending upon the characters' missions.
Generally, if the characters are law-abiding types and have the good grace to send a letter ahead of themselves politely asking for aid, they will receive at least a Spartan welcome.
If they share any enemies with the dwarves, they can expect a hero's welcome, including free weapons, food, and the promise of ale if they return.

Dwarven traders also travel the land to sell their strong ales or purchase products otherwise unavailable.
They might even be stuck into an adventure as part of another culture as some few villages have hosted dwarves for a decade or more as part of the cultural exchanges which dwarven matriarchs occasionally drive.

\begin{boxtext}

  Your friendliest voice bounces down the tunnel, and movement starts.
  It sounds like there could be ten of them, but it's hard to tell with all the echoing.

  Only six come out, all with grimy faces, but shining, well-polished plate armour.
  The first says ``I know this isnae your faults lads, but if yous leave, it'll be little stories going up before long''.

  ``Death before taxes!'', he shouts.

  Looking back, the runes have already cast a forcefield across the tunnel behind you, and the dwarves have raised their crossbows.

\end{boxtext}

\best[\M\Dw]{Trader}\label{dwarven_trader}

Your standard dwarvish citizen is a gruff male who works with his hands.

\dwarventrader

\best[\M\Dw]{Soldier}\label{dwarven_soldier}

Dwarvish soldiers sport proud suits of plate armour, making them nearly impenetrable to normal weapons.

\dwarvensoldier[\npc{\M\Dw}{Soldier}]

\paragraph{Tactics:} Dwarven  warriors typically \textit{charge} and trust to their plate armour to keep them safe in battle.%

\best[\F\M]{Runemaster}
\label{dwarven_runemaster}

Every dwarven citadel hosts at least one rune master, who uses their spells to give blessings, and keep the edges of stronghold free from enemies.
The greatest of runemasters can protect entire settlements with their wide-ranging spells, and advise the matriarchs.
While most are female, some males have been taught by their mothers.
`Runemaster', is among the most prestigious roles a dwarven male might attain.

\dwarvenrunemaster

Runemasters typically prepare their spells before any battle commences.
Commonly made magical items include a runic breastplate which stores 3 MP and can cast  a blessing to restore $1D6+2$ FP (as per Fate 2).

\subsection[Elves]{\El\ Elves}
\index{Elves}
\label{best_elves}

\begin{boxtext}

  Most of the elves are engaged in a communal song, but two stand at the side arguing.
  They look like children, except for the eyes which appear old and stressed.
  The leaves all around seem to sway and faintly grow in response to the song.

\end{boxtext}

Elves can forget about the outside world and forget to protect themselves quite easily, perhaps not noticing that in the last fifty years a new human settlement complete with an army has appeared, or that gnolls have invaded the area.
If characters even encounter a settlement, the elves will probably initially treat them like any other passing animal or be surprised by the idea that there are more humans about.
Travelling elves are usually less snobbish and are better at imitating human customs such as wearing clothing.
Many take to the human roads to wander the earth, trying to bring back the wisdom of gnomish villages to their lands, and occasionally stopping in human towns and exchanging their jewellery for a bottle of wine and a meal.

\begin{boxtext}

  As soon as you speak with the arguing elves one looks aside, then points to the setting sun.
  You turn to look at it for a moment and it instantly sets, leaving the forest dark, without a hint the elves were there.

  Your guide informs you that you've been staring at the Sun vacantly since you spoke to the elf ealier.
  Just staring, and not saying anything to anyone.

\end{boxtext}

\paragraph{Male Names:} Minyon, R\'{u}natar, Telma, Norson.
\paragraph{Female Names:} R\'{i}a, Malt\"{e}, Maiw\"{e}, Fail\"{e}, Telw\"{e}.

\widePic{loh/dryad}

\best[\F\M]{Dryad}
\label{dryad}

Various elves, usually towards the end of their lives, lose any intimate connection with their bodies and shapeshift constantly, using their natural abilities.  Sometimes these creatures will live alone in the woods, preferring the company of simple animals rather than the increasingly naive and ignorant lives of the much younger elves in their communities.  Others, more malicious, have been known to frequent the shores of lakes and eat anything that comes by, including other humanoids.  These long-lived creatures view themselves as so far beyond any other creature that they see little difference between the tweeting of birds and the simple conversation of humans, so eating their flesh underwater is simply one short step away.

When dryads reproduce, the magic in their veins is often so strong that the child is part woodland and part elf, making them almost an entirely different species.
Each dryad has different abilities, but high levels of the Polymorph and Aldaron spheres are always attained.
Song magic is also popular among the dryads, allowing them to stay safe by reading what the future holds and to bless people whom they find entertaining.

Dryads who are not malicious often represent themselves as demigods, accepting offerings from gnolls, humans or gnomes and giving blessings or guidance in return.

\dryad

\paragraph{Tactics:}
Dryads always have a penalty of 5 MP to keep their dryad form in place, though some remove it to revert to elf-form or change into a human.
Their knowledge of the Fate sphere also allows them to keep their Fate Points stocked up at all times.
When fights break out, most dryads will take some elemental form, or turn into a bird and fly away.
They prefer to summon animals to their aid, or approach lone people, then enchant them to wade into the water where they cannot move so easily.

Dryads can easily become violent when their territory is threatened.
Being intelligent creatures, they do not march blindly into battle, but will use magic to pull apart any settlements they feel are in the wrong place.
Some few can be bargained with, but it's famously difficult to bargain with creatures who don't have any use for gold, outside information, or friends.

\paragraph{Encounters:}

\begin{itemize}

  \item
  If any character can be heard singing, he gifts them a spell-song for the remainder of the adventure -- perhaps something which grants Fate Points, or a spell to calm animals.
  \item
  The dryad, unhappy with the party encroaching upon his territory, sings a \textit{Forest's Call} spell upon a \gls{pc}.%
  \exRef{core}{the core rules}{forestsCall}
  If approached, he flees, but always stays close to the party.
  Once the spell takes effect and animals attack the troupe, the dryad begins polymorphing the party into animals.
  \item
  The dryad follows the party through the forest, trying not to be seen.
  She listens, and judges them, using all the spells she has to gain information about them.
  If they seem like they're not a threat to the forest, she helps them with any encounters.
  Otherwise, it uses whatever magics it has to curse them.
  \item
  Four bandits camped in a dryad's territory, so she summoned mist around them, and attacked them by surprise.

  The party find her eaten a corpse while her pet bear sits beside her, eating another.

\end{itemize}

\best[\F\M]{Wanderers}
\label{elf}

Given their long lives, most elves do surprisingly little.
Nevertheless, after enough decades, almost all pick up a few crafts, or at least learn some social graces.
Almost all will know basic Wyldcrafting in order to survive.

\paragraph{Encounters:}
Most of the elves characters might meet will be wanderers who want to see the wider world.
Most will come across as tourists, which looks rather strange in a world with almost no tourism.
Others, with less social skills, typically come across as patrons of a zoo.

They have a reputation as being assassins, which most work hard against, although many still practice with their short, thin blades enough to keep themselves safe.

\paragraph{Tactics:} When they need to fight, elvish blades are fast.
They prefer fighting in the darkness as their heightened senses give them a distinct advantage.
While elvish blades are sharp, they do not have the strength to damage people wearing heavy armour, so most avoid any fights with those in heavy armour.

\elf[\npc{\E\El}{Wanderer}]

\best[\F\M]{Enchanter}
\label{elven_enchanter}

Elven enchanters have hundreds of years to perfect not only their natural magics but also outside magic Paths -- often the Path of Song.

\elvenenchanter[\npc{\E\El}{Enchanter}]

\paragraph{Tactics:} Enchanters can grasp at people's minds, confusing people or sending them to sleep with a TN of 13.
They typically turn groups against each other, converting one side to their service through mental Domination before casting Confusion upon the rest.

\subsection[Gnolls]{\Nl\ Gnolls}
\index{Gnolls}
\label{best_gnolls}

\best[\E\Nl]{Hunter}
\label{gnoll_hunter}

Adventures containing gnolls will almost certainly be martial in nature.  They can be insular and very tribal, and few characters will count as being part of any gnoll's `in-group', not even other gnolls.

\gnollhunter[\npc{\M\E\Nl}{Gnoll Hunter}]

\pic{loh/gnoll}

\paragraph{Encounters:} Gnoll encounters have to be viewed entirely in terms of the local territories.
If the territory belongs to the gnolls, they will do as they please to the characters, but are unlikely to be preparing for a fight.
If the territory does not belong to the gnolls, they will be polite, and immediately explain their reasons for where they are, and explain they intend to leave as soon as their business is concluded.
\paragraph{Tactics:}
If the territory is disputed, the gnolls are dangerous, and eternally prepared for a dirty fight; they will strike at night, throw spears over an area, then rush to kill the largest target they can see in unison.

Your standard gnoll is well equipped to hunt, gather food and deter intruders into their territory.
Women usually take their children out on hunting missions to train them from an early age, although they are permitted to run away if battles with humanoids rather than hunting arises.

\begin{boxtext}

  Approaching the settlement, dozens of dogs run out barking.
  They look massive, almost large enough for a gnome to mount and ride.
  More barking from behind them turns out to be a gnoll shouting orders, and the dogs stop just shy of trying to tear you apart.
  The entire village come out of their tents, and away from their fires, and form a semi-circle in front of you; perhaps 20 in total.

\end{boxtext}

\best[\E\Nl]{Shaman}
\label{gnoll_shaman}

Gnoll shamans typically follow Laiqu\"{e} or Qualm\"{e}.  They are a rarity in any gnoll society but always accorded respect when they are present.  They are not permitted to enter war councils but are also immune to all challenges except from other shamans.

\gnollshaman[\npc{\E\Nl}{Shaman}]

\begin{boxtext}

  The gnolls don't respond, just stare.  
  Slowly, an old creature wanders forward, and the semi-circle part for him.
  He looks like a proper old dog, but the bones piercing his ear show him to be a priest of Qualm\"{e}.

  ``How may we help?'', he asks innocently, while the other gnolls wait patiently.

\end{boxtext}

\subsection[Gnomes]{\Gn\ Gnomes}
\index{Gnomes}
\label{best_gnomes}

Many gnomes few have taken to thievery in human towns where they can dress as children and confuse people with their illusions, though they will often leave town once it is generally known that there is a gnomish illusionist about -- their tricks are much easier to spot once everyone is on the lookout for them.

\begin{boxtext}

  ``You got me'', the little illusionist says as he steps out of the bushes.
  ``Okay'', he concedes, ``I'll show you the jewels I stole if you don't hurt me''.
  The chitincrawler sitting beside him slowly reveals itself to be a normal, boring bush.

\end{boxtext}

\best[\E\Gn]{Gnome}
\label{gnomish_citizen}

A majority of gnomes are farmers, often with a light interest in alchemy and academic literature in general.  Some adventurous (or simply poor) gnomes end up as thieves in large-scale cities.

\gnome[\npc{\E\Gn}{Gnome}]

\best[\Gn\E]{Gnomish Illusionist}\index{Illusionist}
\label{gnomish_illusionist}

Most communities of gnomes hold at least one specialised illusionist.  When not studying they farm, bore people with philosophical questions and smoke extraordinarily long pipes.  On occasion, some will specialise in the invocation sphere in order to hunt animals.

\gnomishillusionist[\npc{\E\Gn}{Gnomish Illusionist}]

\paragraph{Tactics:} Illusion goes \emph{way} beyond casting illusions.
If people expect an illusionist to be present, illusionists will make a human warrior look like a badly-made illusion in the hopes of starting confusion, or perhaps even a fight.
If they're in basilisk territory, they will make the illusion of basilisk droppings before making an illusion of a basilisk's roar.
If an actual basilisk appears, an illusionist can make it appear that a chasm is blocking the party's escape, or make a real chasm appear as if it were solid ground.

\begin{boxtext}

  As you follow the little man, you notice a strange movement in his face.
  Suddenly, he disappears, and the bush behind transforms into a nura creature, and rushes towards you.

\end{boxtext}

\subsection[Humans]{\Hu\ Humans}
\index{Humans}

Being the most prolific race in the area, humans will most likely form the centrepiece of any campaign.  Elves do not have cities with taverns, open to anyone for trade.  Gnomes actively hide their towns until they are sure that they will not be attacked, and live so much underground that adventurers may well camp overnight by a gnomish village without either party being aware of the other.  Humans are not always hospitable, but their great towns are geared towards trade to the point where anyone with enough tradeable goods is pushed in as if by some invisible hand into the local marketplace and then an inn.

\paragraph{Encounters:} Human roads can be distinguished by two key features -- their poor quality, and the fact that they are everywhere.
On these roads, two primary types of humans wander: traders and bandits.
Traders are far more common, and easy to spot due to their goods.
They always want more company, as it keeps them safe, and sometimes pay the \gls{guard} a small sum in order to stay with them while travelling.

Bandits, on the other hand, will only present themselves if they think they can win a fight.

\best[\Hu\E]{Human Trader}
\label{human_trader}

At least half the humans in any given area are farmers.
Most organised militia are comprised of ex farmers who expect to return to farms, either to their own after sufficient payment or to new farms in conquered territory.
Humans may not have the greatest empathy with animals, but farmers certainly spend more time around inhuman animals than any other sentient race.

\humantrader[\npc{\E\Hu}{Trader}]

\best[\Hu\E]{Priest}\label{human_priest}

Human priests might follow any one of the gods.
For simplicity's sake, a priest of V\'{e}r\"{e} is presented below.

\humanpriest[\npc{\E\Hu}{Priest}]

Often a priest will accompany bands of warriors, weather they are going to war or simply moving in to defend a village against a recent threat.
Any such warriors accompanied by a priest will have a full allotment of Fate Points.

\best[\T\Hu\E]{\Glsfmttext{guard}}
\label{human_soldier}

The \gls{guard} -- civilization's last hope (at least according to the captains) -- exist in every corner of Fenestra.
Many consider them a menace, as the requirements to join don't include table-manners, common courtesy, or cleanliness.

\humansoldier[\npc{\E\Hu}{\Glsfmttext{guard}}]

\best[\T\Hu\E]{Bandits \& Brigands}

Those poor souls who have either avoided joining the \gls{guard} or escaped from it live an unpredictable life beyond the \gls{edge}.
Those who manage to make a proper home for themselves will often farm like any normal villager, and only attack people when most in need.
Banditry typically increases a lot during the Winter.

Random city-dwellers and farmers who have fled rather than being shunted into the \gls{guard} are known as `bandits', while `brigands' have spent time in the \gls{guard} and decided they want nothing more to do with it.

\paragraph{Tactics:}
Brigands will pick a likely spot by the roadside and practice with their arrows for hours on end.
If someone shows up, one of the spotters will give a sign and the group will go silent.
The first people to be hit are fighters.
The second are the horses.
Traders, meanwhile, will often survive an encounter with brigands if they agree to abandon what they have and run along the road.

\humansoldier[\npc{\E\Hu}{Brigand}]
\label{brigand}

Bandits, on the other hand, typically gather as many as they can to attack smaller groups, and extort traders' goods through intimidation.
At other times, hunger drives them to extremes, and they end up attacking anything that looks like it might carry coin or food in its pockets.

\humanfarmer[\npc{\E\Hu}{Bandit}]
\label{bandit}

\widePic{loh/dragon}

\subsection[Dragons]{\E\ Dragons}
\label{dragon}

Dragons can be terrifying and extremely damaging but are generally considered to be divine creatures given their association with the sky.
When they land on a farm there is precious little anyone can do but let them take it and hope that the dragon falls back into hibernation soon.
Dragons spend a lot of time in underground realms or far off places inhabited by spirits, demons and semi-divine things of the other side.

Dragons can reach up to sixty feet in length, though a more modest specimen is presented here.

\paragraph{Natural Abilities:}
Dragons' wings allow them to fly, using the Athletics Skill to move in the air instead of the ground.

\paragraph{Ecology:} Dragons are generally solitary creatures, though people have nightmares of this situation changing constantly.
Most of the time none are active in Fenestra, though an unknown number are presumed to sleep there in the Liberty region.
They come from foreign lands and like to create fanciful tales of the places beyond in order to scare knights or irritate academics.
It is said that older dragons sometimes transform themselves into humans and walk abroad in the land, pretending to be alchemists, bards, nobility or just whatever strikes their fancy.
Rumours abound of their constant interference in politics among all races, though if half the rumours were true there would be more dragons than horses in Fenestra.

The majority of dragons do not learn humanoid languages, but some few learn dwarvish or elvish.
This has provided a lot of false hope to those trying to negotiate with dragons.

\begin{boxtext}

  As you start to approach, its golden eyes glitter, and you find yourself paralysed by some unearthly magical force.
A moment later and fire explodes from its mouth, burning everyone in front of you.
  A few survive just enough to start crawling away in retreat, while the dragon considers its next move.

\end{boxtext}

\paragraph{Encounters:} Most dragons encountered will be fast asleep.
Sometimes they rest openly in a forest -- they have the rare claim to power that they can sleep in the open without any real fear from the world.
In such cases, characters will simply need to retreat quietly.

At other times, dragons may be seen taking back a recent kill -- perhaps a deer or a farmer's sheep.
A few dragons take people back to their lair in order to learn about recent news or to learn about human languages.
Once the dragon has learnt what it can, the person is often challenged to a game of riddles for their freedom.  

\dragon

\end{multicols}



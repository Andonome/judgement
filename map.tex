\chapter[The Cartographers]{Cartography}
\label{civilization}
\label{makingTheMap}

%Welcome to \gls{fenestra}.
%Let's look around.

Fantasy stories have two special characters -- magic, and the landscape.
The audience discover both, just as they learn about the other characters.
As \pgls{gm}, your character is the landscape, and making a map is your character creation.

If you were to run through a standard series of encounters alone,%
\footnote{You can make some random characters with the \textit{Book of Stories}.}
it might go something like this:

\begin{multicols}{2}

\begin{enumerate}
  \item
  Starting with \autoref{yourMap}, you outline your map, with roads, rivers, and ten points of interest.
  \item
  You can add details to each point on the map at any time, but to begin with, you only need the first location: a small human settlement, consisting of a single~\glspl{broch} which guards two \glspl{village}.

  You find the character of the \glspl{village} by rolling on the chart in \autoref{mapCharacter}.
  \begin{itemize}
    \item
    The first \gls{village} has a moat and a watermill, so you name it `Millwall', and mark in on your map by a river.
    \item
    The second \gls{village} has copper spikes placed on its walls, so you name it `Copcrown', and this point on the map is complete.
  \end{itemize}
  \item
  Life in the \gls{guard} means that every dawn arrives with a mission, so you turn to \autoref{encounters}: `\nameref{NGmissions}' to find out what lies ahead.
  \begin{itemize}
    \item
    Your first mission is cleaning the \gls{broch}.
    These small tasks are simply flavour, or an excuse to show new players how to roll for an action, so you jump to the second day, and roll another mission.
    \item
    The second mission looks more promising: the troupe will need to investigate an area with \pgls{ogre}, some days' \gls{journey} away.

    The journey will take them to \pgls{village}, then \pgls{bothy}, where they will have to leave the road, and venture past the \gls{edge}.
  \end{itemize}
  \item
  You'll need more information about that \gls{ogre}, so you return here to \autoref{makingTheMap}, \autoref{mapCharacter}, to fill out more of the map.

  The dice say that the \gls{ogre} a set of complete plate armour, and a mushroom farm which extends down into the \gls{deep}.
  That might cause complications later!
  \item
  You have enough points on the map filled, so you'll just need to roll up the random events for the journey, so you plan a route, and turn back to \autoref{encounters}: \nameref{encounters}.
  \item
  A single roll of $3D6$ tells you that the first road-encounter will come after 2~\glspl{interval}: \pgls{woodspy} and rain.
  \item
  The troupe will pass Millwall, and \Glspl{village} have their own random events in `\nameref{villageEvents}', \vpageref{villageEvents}.

  Today, Millwall holds a funeral for a fallen \gls{guard}.
  \item
  As the troupe bed down in a farmer's barn, the encounter occurs; the rain starts to fall while \pgls{woodspy} has entered the \gls{village}: it begins feeling inside the barn for signs of an opening.

  Due to heavy rain, the troupe fail to wake, and the \gls{woodspy} snatches up the smallest member, then carries them into the night\ldots
  \item
  The next day, the troupe wake one member short, and wonder if they simply deserted the \gls{guard}.
  After breakfast, they leave for the \gls{edge}.
  \item
  You roll a second encounter: after 4 more \glspl{interval} (i.e. one day) a mist will arrive, with goblins.
\end{enumerate}

Every movement the troupe make creates some situation -- often a small detail, sometimes a catastrophe.
You can use this gentle hum of chaos to generate a flurry of fast-paced problems, or as the background noise to other plots and schemes.

\end{multicols}

\section{Rivers \& Roads}
\label{yourMap}

\begin{multicols}{2}

\mapentry{The Pentacle \& the Points}
provide the map's layout.
Take a pencil, blade, and dice, then cut out the example map on the left.

Go along each point, from 1 to 10.
At each point, roll $1D6$ and add that point's number to find out what's on that point on the map.
You might just roll $10D6$ all at once, and then pick them in left-to-right order.
This process should feel a little sloppy.

\begin{multicols}{2}

\begin{enumerate}
  \raggedright
  \setcounter{enumi}{1}
  \item
  (D)~Dragon on a mountain
  (\autopageref{dragonPoint})
  \item
  (\Hu)~Humans on a coast, with $1D6$ islands
  (\autopageref{humanPoint})
  \item
  (\Hu)~Humans
  \item
  (\Hu)~Humans on a coast
  \item
  (\Dw)~Dwarves a mountain
  (\autopageref{dwarvenPoint})
  \item
  (\Hu)~Humans
  \item
  (\El)~Elves
  (\autopageref{elvesPoint})
  \item
  (B\Hu)~Bandits
  (\autopageref{banditsPoint})
  \item
  (\Gn)~Gnomes on hills
  (\autopageref{gnomePoint})
  \item
  (\Nl)~Gnolls
  (\autopageref{gnollPoint})
  \item
  (\N)~Ogre
  (\autopageref{ogrePoint})
  \item
  (D\El)~Dryad
  (\autopageref{dryadPoint})
  \item
  (H)~Hag
  (\autopageref{hagPoint})
  \item
  (\D)~Lich
  (\autopageref{lichPoint})
  \item
  (\Hu)~Humans
\end{enumerate}

\end{multicols}

\begin{exampletext}
  For example, if your first die reads `\dicef{4}', then the result is $4+1 = 5$, and you can write `humans', or just `\Hu' on the map.

  If the next roll is `\dicef{6}', then you would write `elves' or `\El' next to point 2 on the map.
\end{exampletext}

Each dot has four neighbours.
The highest neighbour is the neighbour with the highest number, and the lowest neighbour, is the neighbour with the lowest number.

\mapentry{Mountains \& the Coastline}
go onto the map first.
They always form a line where the earth has cracked.

\begin{itemize}
  \item
  If only one point has a mountain, extend it with a long line of mountains, staying outside of the other points.
  \item
  If you have more than one point with a mountain, connect them all with a mountain-range.
  This range should avoid going through all other points, but might cut across the centre of the map.
  \item
  Place a valley of hills near the centre of the mountain range.
\end{itemize}

\mapentry{Rivers}
\label{mapRivers}
come from mountains, then flow to the sea.

\begin{enumerate}
  \item
  At each point with a mountain, draw a river to every neighbour without a mountain.
  Draw your rivers with wavy, lazy, lines.
  \item
  Each river flows down-hill, towards the highest neighbour.
  They avoid mountains and avoid doubling-back to a point they've already been at.
  \item
  Rivers begin small, but when another river joins it, they run faster and wider.

  Draw these rivers with a \emph{thick} line.
  \item
  When the strength of a third river joins, it becomes a canyon, and it will flood during the warm seasons.

  Draw the canyon with a parallel~line.
  \item
  If a fourth river adds its strength, they all form a loch.
  
  The loch exits towards the lowest neighbour as another thick line (with the strength of two rivers).
  \item
  Once a river has nowhere else to go, it flows off the map, to uncharted lands.
\end{enumerate}

\mapentry{Hills \& Bogs}
provide water to any point without a river, or a coast.
If you don't have any points without a river, just skip this step.

\begin{enumerate}
  \item
  The highest point without water has a bog between it and its highest neighbour.
  The bog is three miles across.
  \item
  All outer points (i.e. `1 to 5') without water receive some small hills.
  \item
  The hills provide small streams, which journey to the highest point without water, as per step \ref{mapRivers}, above.
  \textit{However}, they will avoid the bog.
\end{enumerate}

\mapentry{\Glsfmtplural{lonelyRoad}}
\index{Roads}
\label{drawRoads}

Humans always litter the land with roads, and roads out always start from \glspl{broch}.

\begin{enumerate}
  \item
  At each human settlement, draw $1D6$ tall \glspl{broch}.
  They sit around 5 miles apart, and usually form triangles.
  \item
  Connect each human settlement to every other settlement with \pgls{lonelyRoad}.
  \begin{itemize}
    \item
    If a road crosses a small river, draw a small bridge.
    \item
    If it crosses a larger river, drawn with two lines, draw a larger bridge.
    \item
    If a canyon or loch blocks a road, people may find another way around, or may just have to row across the river, or descend into the canyon.
  \end{itemize}
  \item
  The human settlement with the highest number has a road leading out of the map, towards the closest border.
  \label{roadOut}
  \item
  If a dwarvish settlement has human neighbours, it connects to the nearest road.
  \item
  Label some roads with their length in miles.
  \begin{itemize}
    \item
    The short distance between points 6 and 7 is about 10 miles.
    \item
    The medium distance between points 1 and 6 is about 15 miles.
    \item
    The long distance between points 1 and 2 is about 25 miles.
  \end{itemize}
  \item
  If any road stretches for 40 miles or more, draw \pgls{lonelyTavern} half-way along that road with an `L' (details \vpageref{lonelyTaverns}).
  \item
  Long roads have \glspl{bothy} placed along them for travellers to rest.
  \label{bothyRolls}
  For every road over 10 miles long, place \pgls{bothy} every $1D6+6$ miles from the lowest point
  (you might leave this step until the \glspl{pc} use this road).
\end{enumerate}

\bigLine

\paragraph{Do not draw trees on the map,}
just know that they cover everything.
A twilight of towering trees covers everything which people have not cut down, but if you draw a few trees, the rest of the map will appear naked by comparison.

\paragraph{The map's name}
should come from a distinctive feature.
A good starting point is the most prolific feature of the map.
For example, a map with a bog and three elvish settlements ranges might receive the name `Faebog', or one with two dragons and a lot of lochs might receive `Lochscale'.

\end{multicols}

\section{Zooming In}
\label{mapCharacter}

\begin{multicols}{2}

\noindent
You can add detail to every point on the map once the \glspl{pc} arrive, but there's no need to prepare everything too far ahead of time.

Each point on the map needs more detail -- but only once the \glspl{pc} arrive.
The \gls{campaign} begins on the lowest human settlement.%
\footnote{If there are no human settlements on your map, you have found a rare place in \glsentrytext{fenestra}.
The troupe are likely lost, and will need to return to civilization, which will not be easy without any roads!}
Create a few details with the random lists \vpageref{humanPoint}.

Whenever you find out a new location's details, note them down as a key to the main map, and check if you can add any features to the main map.
Alternatively, you can start a smaller map which covers only one or two points.

\subsection{Civilization}

The civilized lands have people of all types.
Elves, humans, gnolls, dwarves, or gnomes -- any place with a population that either sustains itself, or trades with others, counts as `civilization'.

\subsubsection[Dwarven Settlements]{\Dw\ Dwarven Settlements}
\label{dwarvenPoint}
begin with 100 dwarves, and a mushroom garden to sustain them.
Roll $3D6$, then combine each pair of 2 dice to find each unique feature.
Whenever you roll 8 or more, add 100 dwarves to the population.

\begin{exampletext}
  For example, on the roll `\dicef{2} \dicef{4} \dicef{3}', you would use the results for:

  \begin{itemize}
    \item
    $2+4 = 6$
    \item
    $4+3 = 7$
    \item
    $3+2 = 5$
  \end{itemize}

  The unique results are `5', `6', and `7'.

  \labelledDiceTrio{2}{4}{3}{Magma Stream}{Tin Seam}{Polite Passage}
\end{exampletext}

\begin{enumerate}
  \stepcounter{enumi}
  \item
  A king rules this population.
  Dwarves consider this bad luck, but the king promises to have a daughter and let her take his place in the next generation.%
  \exRef{stories}{Stories}{dwarven_structure}
  \item
  Gold seams run through the mountain, allowing the dwarves to make precious items, and to trade with overland peoples.
  \item
  A Hall of Records stands at the base of this mountain, guarded by heavy doors.
  It works a little like a library, but only contains factual accounts of the things dwarves take interest in, such as facts about rocks, the lineage of queens, descriptions of how to farm on mountains, and techniques to measure altitude.

  Each book has a minimum of three seals, pressed a dwarven record-keeper, who has verified the authenticity of every statement in the book personally.
  \item
  The `Polite Passage', is the name dwarves use for the long, thin, bridges they create, with only room for a single dwarf to walk at a time.
  Settlements have one for entering, and another for exiting.

  Larger rooms around the mountain's exterior store wide equipment, such as wagons.
  \item
  Magma streams in the mountain's heart flow eternally, and mix with underground rivers, producing a great plume of steam at the top.
  Some tunnels contain steam-traps, where a metal grate will ring the `dinner bell', and allow hot steam to flow in to cook the intruding predator.
  \item
  A tin seam allows dwarves to mine, and trade with nearby settlements.
  \item
  In the dark \gls{deep} below the mountain, extensive underground tunnels, filled with fungi, moss, living oozes, skein, and umber hulks.

  With enough traps laid out for the monsters, this creates a kind of rough garden, or `hunting ground' for the dwarves to gather additional food.
  It also provides dyes, which dwarves use to turn their beards and hair green, yellow, or blue, depending on their wealth and gender.\index[Dye]
  \item
  $1D6$ farms extend outside the mountains, across the fertile lands at the base.
  The dwarves create them with high stone walls, and polish each brick until it has nothing to hold onto, and no way to climb up.

  Inside, may allow as much space as any \gls{village} to grow vegetables for the mountain.
  \item
  An elevated barley garden around a third of the way up the mountain, provides food for the settlement.
  Nobody can see the gardens from below, as it grows on steps carved into the mountain.

  A series of rock-bells will sound when anything steps on them, but do not ring in response to the wind.
  \item
  An iron seam allows dwarves to create quality weapons and armour.
  If they have no road out, then they only make rare, wholesale trades.
  \item
  Half way up the mountain, shepherds take sheep and goats on long walks, then put them safely away at night, to sleep in stone rooms.
\end{enumerate}

\subsubsection[Elven Grounds]{\El\ Elven Grounds}
\label{elvesPoint}
centre themselves around a couple of powerful spellcasters who provide safety and often food.
Roll $3D6$, then combine each pair of 2 dice to find each unique feature of the local elven territory.
Take the highest roll, and multiply it by 5 to find the population.

\begin{exampletext}
  For example, a roll of `\dicef{4} \dicef{4} \dicef{6}' would show a powerful enchanter lives here, who tames griffins for the other elves to ride
  (discard duplicate rolls).
  Another elder has constructed a massive tree-dwelling, with rooms inside and paths between the various trees.

  In total, ($\dicef{6}\dicef{4}\times 5 =$) 50 elves live here.

  \labelledDiceTrio{4}{4}{6}{Enchanter}{Tree-house}{Enchanter}

\end{exampletext}

\begin{enumerate}
  \stepcounter{enumi}
  \item
  An elder, focussed on the Force Sphere, often casts portal spells around the many bushes and briers in the area, warping space.
  Anyone entering may accidentally emerge at the other side of the elvish lands.
  \roll{Wits}{Wyldcrafting} to notice, \tn[7] in the day, or \tn[12] at night.

  (Name fragment: `\textit{ando}')
  \item
  Expert archers leave their marks on every tree.
  They always collect every arrow, but the marks remain.

  (Name fragment: `\textit{pilin}')
  \item
  An elder, focussed on the Life Sphere, often casts spells around the perimeter to make anything there shrink.
  Replace every second encounter with a miniaturized version (e.g. miniaturized \glspl{crawler} or bears).
  The spell does not affect elves or gnomes.

  (Name fragment: `\textit{pitya}')
  \item
  An elder, focussed on the Light Sphere, often casts illusions of elves.
  Anyone entering will find $2D6$ dancing apparitions scattered around the area, but not responding to them.

  (Name fragment: `\textit{pirne}')
  \item
  The elves mostly live underground.
  Many of the elves are proficient in the Earth \gls{sphere}, and use it to make rocks or slate partially transparent.
  The glass rooves allow them to live underground, with Sunlight gently dabbing the dwellings.

  (Name fragment: `\textit{cal--}')
  \ifodd\value{r3}
    \item
    An elder with the Life Sphere creates enchanted gardens around the area, but always puts them in a new location each time (so the soil never becomes depleted).

    This elder has taken to shape-shifting, and will soon leave the other elves (or the elves will leave them), to wander free as \pgls{dryad}.

    (Name fragment: `\textit{lelya}')
  \else
    \item
    A wide loch spans the area, providing fish.
    (Name fragment: `\textit{ailin}')
  \fi
  \item
  Useful plants grow all over the region.
  Each of the three dice you have rolled indicates one common plant in the area:
  \randomPlants
  Find the plants \vpageref{plants}.

  (Name fragment: `\textit{tarwa}')
  \item
  An elder, focussed on the Mind Sphere, often casts spells around the perimeter, which confuse anyone who enters.
  They lose a portion of the day, wander aimlessly, and emerge somewhere else, without any memory of what they have done.

  The elder occasionally enchants griffins in the area, allowing a few elves to fly on their backs.

  (Name fragment: `\textit{vanwa}')
  \item
  A peat-bog furnishes the elves here with iron, allowing them to craft iron weapons.

  (Name fragment: `\textit{norna}')
  \item
  Slow spells from the Life Sphere have crafted massive tree-houses around the area.
  When the lights are on, anyone can see them.
  When the lights go out, they appear like normal trees.

  (Name fragment: `\textit{alda}')
  \item
  Expert artisans live in this community.
  Local clay deposits allow them to craft excellent pottery, including ceramic armour.
  If they have dwarven neighbours, they also trade for gold and silver, and make fine jewellery.

  (Name fragment: `\textit{namba}')
\end{enumerate}

\paragraph{The settlement's name}
comes from combining the name fragments in each feature.
Combine two `a' sounds wherever possible.

\begin{boxtable}[YcYcY]
  vanwa &   &      & = & Vanwa \\
  pitya & + & ando & = & Pityando \\
  vanwa & + & alda & = & Vanwalda \\
  lelya & + & alda + namba & = & Lelyalda Namba \\
\end{boxtable}

\subsubsection[Gnoll Grounds]{\Nl\ Gnoll Grounds}
\label{gnollPoint}

Gnolls typically wander back and forth around a territory.
They will use roads when they see them, but never like to rely on them.

The tribe has $2D6 \times 5$ members, and the same number of hunting dogs.
Roll 3 dice and check every combination of 2 to find the characteristics the gnolls who wander this ground.

\begin{enumerate}
  \stepcounter{enumi}
  \item
  The tribe have taken in a bear-cub, and raised it as one of their own.
  It hunts with them, and understands when to be quiet, and when to alert people to danger.
  \item
  This tribe suffer from isolation.
  Where most tribes have members who learn the neighbouring languages, but these gnolls only know how to trade with explicit allies, and still cannot speak their languages properly.
  \item
  The lowest neighbour with \pgls{fiend} has killed a few of these gnolls, and they want revenge.
  \item
  The highest neighbour with civilization (i.e. any spot without \pgls{fiend}) pays well for the gnolls' hunting dogs, so many of the tribe members have items from them, along with a general feeling of kinship.

  For example, if the highest neighbouring point near them has elves, then they will have elven jewellery, and plenty of elven songs (though the thick accent may not make this obvious).
  \item
  The tribe have figured out how to use local plants as hair-dye, and they have decided that this is the best thing ever!
  \index[Dye]
  \item
  A powerful druid advises the tribe.
  \item
  These gnolls have some of the best hunting dogs around, and plenty of them.
  Double the number of hunting dogs, so they have twice as many dogs as tribe-members.
  \item
  The gnolls know all the local plants, including where to find marching mushrooms and screechmoss (\autopageref{marching_mushroom}).
  \item
  The tribe keep a herd of aurochs, and wander constantly in order to let them graze.
  They have no fences, but they can still stop the aurochs wandering away -- each one of their dogs knows to stop them escaping.
  \item
  The tribe keep a small herd of sheep.
  \item
  Everyone has contracted \nameref{Torpid Flesh}.
  They are hairless, and look half-dead
  (see \autopageref{Torpid Flesh}).
\end{enumerate}

\subsubsection[Gnomish Warrens]{\Gn\ Gnomish Warrens}
\label{gnomePoint}
interact in subtle ways with their four neighbouring points.
Every point which might benefit the gnomes results in another warren, with some new specialization.

Each warren on the point connects to a nexus cavern below,%
\footnote{Or as the gnomes call it, a `gnexus' (pronounced `neksus').}
which allows the warrens to trade, and (on rare occasions) fight.

\paragraph{Beneficial neighbours}
mean another warren.

\begin{itemize}
  \item
  If they have gnomish neighbours, they enlarge the natural tunnels beneath them to make an underground road.
  The ups and downs mean the path is twice as long below ground as above.
  It is also not without danger, but underground roads have far fewer creatures than above-ground.

  This also allows a gnexus large enough to sustain a complete warren, far below ground.
  \item
  For each neighbouring point with mountains, a new warren is built with an underground stream-boat-ride, which goes from the mountains.
  \item
  If a neighbour has a loch, another warren arises with a grotto to lay traps for fish and eels.
  These traps catch so much that nobody else has much luck fishing.
  \item
  If gnolls live nearby, a warren trades for hunting dogs, and occasionally ride them when going out.
\end{itemize}

\paragraph{Neutral neighbours}
do not add a warren, but do add their own complications.

\begin{itemize}
  \item
  If they have a road nearby, they construct a series of underground bothies, each 6 miles apart, to reach the road.
  They can walk above ground, but always have somewhere to rest along the way.
  \item
  If \pgls{dryad} lives nearby, they create irritating songs which get stuck in your head, and sing them constantly.  Dryads hate this kind of thing.
  \item
  If \pgls{ogre} lives nearby, they create pit-traps with spikes at the bottom.
  When \pgls{ogre} steps in one, they cannot walk straight for a week.
  If a goblin steps in one, it usually dies.
  (\roll{Wits}{Wyldcrafting} \tn[12] to notice, $1D6+1$ Damage for failure)
  \item
  If \pgls{hag} lives nearby, the gnomes grow wheat, and use an underground mill to make flour, and finally, to bake cakes.  Cakes and a little flattery always works on old crones.
  \item
  If elves live nearby, they arrange for a magical dual (just for sport), every \gls{Atya}.

  The gnomes also add a road between them and the elves.
  Most elves don't appreciate roads, but the road serves as a useful warning that if anyone attacks the gnomes, they may attack the elves next.
\end{itemize}

For each warren, roll $1D6$ to find its political structure.

\begin{dlist}
  \item
  Do-opoloy (you decide how something works by making it work)

  (population: 20)
  \item
  Thaumocratic (alchemists rule)

  (population: 40)
  \item
  Anarcho-syndicalist (which consists of debating how anarcho-syndicalism works)

  (population: 60)
  \item
  Technocratic (whoever understands most about a subject decides how it's done)

  (population: 80)
  \item
  Direct Democracy (dinner takes about 3 hours each night)

  (population: 100)
  \item
  Indirect democracy (people vote once a week, documentation is crucial)

  (population: 120)
\end{dlist}

\subsubsection[Human Settlements]{\Hu\ Human Settlements}
\label{humanPoint}
\index{Towns}
begin with \pgls{broch}, which protects two \glspl{village}.
If the farmers manage to survive long enough, the \gls{guard} will build another \gls{broch} within a few miles, which give a protective area for a couple more \glspl{village} between the two guardian towers.
A third \gls{broch} completes a triangle, which means the settlement has a safe space in the centre for a town.

You have already placed $1D6$ \glspl{broch} here, in step \vref{drawRoads}.

\begin{enumerate}
  \item
  For each \gls{broch}, place 2~\glspl{village}.
  \begin{itemize}
    \item
    Most \glspl{village} stay between \glspl{broch}.
    \item
    All \glspl{village} must be within 5~miles of \pgls{broch}.
    \item
    \Glspl{village} prefer to sit beside a river, but avoid the first few miles downriver of another \gls{village} (otherwise they risk drinking waste-water).
  \end{itemize}
  \item
  If the settlement has more than 2~\glspl{broch}, place a town in the centre.
  You can fill in the town's details when needed (\vpageref{mapTown}).
  \item
  Connect each town \gls{broch} and \gls{village} by plotting roads between.
\end{enumerate}

\paragraph{\Glsfmttext{village} Features}
\label{villageFeatures}
should make the \gls{village}'s name, so everyone can remember the place easily.

Roll $3D6$, then combine each pair of 2 dice to see how the \gls{village} looks.
The population equals $1D6 \times 50$ (use the first die).

\encVillageFeatures

Note down the \gls{village}'s features, and combine them with the landscape to make a name.

\labelledDiceTrio{2}{3}{1}{moat}{copper spikes}{wire chimes}

The result above could indicate \pgls{village} of 100 people called `Ringchime' (where `ring' indicates a ring of water around the \gls{village}).
\Pgls{village} with \nameref{screechMoss} nestled in hills could receive the name `Mossdale', or `Screechvale'.

\subsection{Fiends}
\label{mapFiends}

\subsubsection{Bandit \Glsfmtplural{village}}
\label{banditsPoint}
\index{Dragons}

\Pgls{village} left alone, with broken roads and the threat of starvation, has only one option -- banditry.
\Gls{fenestra}'s thick forests create a natural barrier around any \gls{village} which wants to remain hidden.

One road leads out of their settlement and joins with the nearest main road in secret; it splits into many roads at the last moment, so none of them look well-trodden.
The bandits cover their tracks with foliage once they enter the main road, and only leave their private road at night.

A total of 200 villagers live in the commune, but only 50 regularly exit to perform robberies.
See \autopageref{bandit} for details on the \gls{fiend}.

Roll $3D6$, and apply every result.

\begin{dlist}
  \item
  Ten of the bandits have a suit of partial plate armour.
  \item
  Fences among the \gls{village} with the lowest number help the bandits sell their wares.
  Without this aid, they must journey to settlements themselves, or make deals with people from other lands, accessible only through the long road out of the map.
  (see step \vref{roadOut})
  \item
  The bandits have brokered a temporary alliance with whatever \gls{fiend} has the highest number.
  If there are no other \glspl{fiend} on this map, the alliance is with \pgls{fiend} in another land -- accessible by the long road.
  \item
  An imprisoned gnomish alchemist performs tricks and makes \glspl{talisman} for the bandits.
  \item
  They hold an old \gls{broch}, abandoned by the \gls{guard}.
  It watches the private bandit road, and always has $1D6$ bandits watching, with crossbows.

  At night, the \gls{broch} can send signals back to the bandit \gls{village}.
  \item
  \Pgls{witch} leads the bandits from the rear, arming them with \glspl{talisman} and information.
  %\item
  % A fake \gls{village}, upon a rocky area which leaves no footprints.
  % Only a dozen bandits stay inside, but it looks sufficiently lived-in that anyone could think they had found the real bandit \gls{village}, and not look any further.
\end{dlist}

\subsubsection{Dragon Coves}
\label{dragonPoint}
\index{Dragons}

Dragons may live in deep caverns, shallow caverns, or sometimes just lay about in a Sunny patch of forest, without any roof.
In any case, they like to live close to mountains -- surrounding trees can make flight difficult, while having a tall crag to leap from helps them take flight.

Their destructive habits and association with the sky make people think they are divine.
Of course, in \gls{fenestra}, this is not a good thing.

See \autopageref{dragon} for details on the fiend.

Roll $3D6$, and apply every result.

\begin{dlist}
  \item
  An abandoned dwarvish settlement.
  Roll up a dwarven settlement, then replace those dwarves with a dragon (\vpageref{dwarvenPoint}).
  \item
  Complete plate armour, made by gnomes under threat of death.
  \iftoggle{core}{(See the core rules, \autopageref{bandingArmour}.)}{}
  \item
  Knowledge of the local languages.
  Dragons without this know only elvish (it doesn't change much).
  \item
  A glass-smelting workshop (everyone needs a hobby).
  Glass statues litter the dragon's lair.
  \item
  A portal to another realm, filled with glass flowers which function as every type of \gls{ingredient}.

  The suffocating heat in that strange desert inflicts 3~\glspl{ep} every \gls{interval}.
  \item
  Three eggs, which will hatch next season.
  Once that happens, the dragon will need to venture out to feed her newborn.
\end{dlist}

\subsubsection{Dryad Gardens}
\label{dryadPoint}
\index{Dryads}

Dryads concern themselves with plants and animals more than people\ldots but they also consider most people to be animals.

See \autopageref{dryad} for details on the fiend.
Roll $3D6$, and apply every result.

\begin{dlist}
  \item
  A dozen ex-farmers, now transformed into strange mutant creatures.
  \item
  A maze of venomous thorns envelopes the entire area.
  Touching them inflicts \glspl{ep}.
  \item
  The dryad makes \glspl{talisman}, just for fun, and leaves them as gifts for travellers who behave politely.
  Other \glspl{talisman} find their way to less polite travellers.

  Sensible people leave all of them alone; one can never tell what a dryad considers good behaviour, or what they consider a `reward'.
  \item
  Deep caverns exit in the centre of the dryad's lair.
  The dryad sometimes journeys down, to bring up umber hulks and oozes, and observe them, then release them into the wild.

  These caverns go down into the \gls{deep}, and eventually connect to each other point which has caves.
  \item
  $1D3$ \glspl{basilisk} live with, and love, the dryad.
  They will protect the dryad with their lives.
  (find \glspl{basilisk} on \autopageref{basilisk})
  \item
  $1D6$ griffins, used as protectors and steeds.
  (find griffins on \autopageref{griffin})
\end{dlist}

\subsubsection{Hag Cottage}
\label{hagPoint}
\index{Hags}

Here, an old lady lives alone in a hut.
Of course, in \gls{fenestra}, only one type of old lady lives alone.
See \autopageref{hag} for details on the fiend.

Roll $3D6$, and apply every result.

\begin{dlist}
  \item
  The \gls{hag} has a dozen griffins circling her hut at all times.
  Each one will kill anyone who gets close to the hut, but will not attack them before they approach.
  \item
  The \gls{hag} keeps 4 \glspl{crawler} as pets.
  They spin webs across every window and tree.
  (find \glspl{crawler} on \autopageref{chitincrawler})
  \item
  Mouthdiggers create pot-holes around the entire area, and ensure nobody attacks from below.
  (find mouthdiggers on \autopageref{mouthdigger})
  \item
  A garden of carrots and \nameref{screechMoss} encircles the cottage.
  It screams the moment someone steps on it.
  (find \nameref{screechMoss} on \autopageref{screechMoss})
  \item
  Spells have ravaged the soil in all directions.
  The trees have no leaves, the grass looks half-yellowed, and no animals inhabit the area except some small insects.

  The \gls{hag} now has to take long walks to grow food, which continuously degrades more and more of the landscape.
  \item
  An assortment of potion bottles line her shelves.
  Each one has a poorly-written label, or the wrong label.
\end{dlist}

\subsubsection{Ogre Hovels}
\label{ogrePoint}
\index{Ogres}
\index{Goblins}

The \gls{ogre} has $2D6 \times 5$ goblins, ready to raid all the local \glspl{village}.

See \autopageref{ogre} for details on the fiend.
Roll $3D6$, and apply every result.

\begin{dlist}
  \item
  The goblin population grew until they ate an entire \gls{village}, and then consumed every \gls{village} in this area.
  They laid siege to the city here, and then managed to enter by using a catapult and primitive parachutes.

  The city was half-burnt, and remains inhabited by the goblin horde, and their king.
  Everyone remembers this area, and takes it as a warning to deal with small problems before they become big problems.

  Check the details for this lost city \vpageref{lostCities}.
  \item
  Long caverns below provide the goblins a mushroom farm.
  They extend to a cold region, and eventually into the \gls{deep}.
  \item
  The \gls{ogre} has a set of complete plate armour, created by a blacksmith he still holds prisoner in an abandoned \gls{broch}.
  \item
  The goblins have acquired 10 crossbows and 100 quarrels, and have started to figure out how to use them.
  \item
  The \gls{ogre} king has a pack of $3D6$ hobgoblins.
  \item
  A travelling dryad, who found the goblins rude, erected a living hedge-maze, as vengeance for their bad manners.
  It changes a little each season, so the route out changes with it.
  And every cold season, the hedge maze vanishes, to reveal a horde of starving goblins.
\end{dlist}

\subsubsection{Lich Lairs}
\label{lichPoint}
\index{Liches}

Every \gls{lich} has a set of caverns to stay.
Finding \pgls{lich} in the cold, twisting \gls{deep} makes a near-impossible challenge, as they change location at random.

See \autopageref{lich} for details on the fiend.
Roll $3D6$, and apply every result.

\begin{dlist}
  \item
  The \gls{lich} waited for centuries, building an army of ghouls.
  When a nearby human town weakened, the \gls{lich} attacked.

  He remains in the town to this day.
  The local \glspl{village} have rotted, but 3~\glspl{broch} remain standing.
  Create a lost city for the \gls{lich}, on \vpageref{lostCities}.
  \item
  An old \gls{broch}, lost to the \gls{guard}, now used as an alchemical workshop.
  \item
  The \gls{lich} keeps a crew of $3D6$ ghasts, ready to follow orders.
  \item
  A cabal of ten necromancers-in-training visit with bodies to feed the \gls{lich}'s army, in exchange for tutelage.
  They frequent local towns, and pick up the dead with a cart.
  \item
  The \gls{lich} has complete plate armour for himself and any steed.
  \item
  When \pgls{basilisk} attacked, he killed it, turning it into an undead steed.
  It remains inside the city, waiting for the \gls{lich} to summon it.
\end{dlist}

\end{multicols}

\section{Stone Walls}
\label{mapStrangePlaces}

\begin{multicols}{2}

\subsection{Towns \& Cities}
\label{mapTown}
depend on the surrounding hamlets to feed them, and those hamlets depend on the surrounding \glspl{village} and \glspl{broch}.

To explore the town, roll $1D6$ plus the number of \glspl{village} around it three times, and accept each result once.
For example, if a human settlement with 2~\glspl{broch} will have 4~\glspl{village}, so a roll of \dicef{4} \dicef{4} \dicef{5} would mean accepting results \textbf{8} and \textbf{9}.

\begin{enumerate}
\setcounter{enumi}{6}
  \item
  The outer stone wall has three layers, structured like a maze.
  Sporadic arches mean that even griffins can struggle to leave quickly.
  \Glspl{monster} which enter often become lost, or at least trapped long enough for the \gls{sunGuard} to pick them off.
  \item
  White rock (actually limestone).
  \item
  Statues of heroes litter the roads all around town.
  The \gls{warden} loves a good story of battle.
  \item
  The local \gls{templeOfPoison} thinks the \gls{warden} has become too big for her boots and needs to die.
  They begin by ingratiating themselves with the \gls{warden}'s children, then start planting clues about one of the \gls{warden}'s other enemies making a grab for power.
  \item
  The \gls{warden} has a secret alliance with whatever lies on point 9 of the map.
  \item
  `Demi-human suburbs', where various non-humans live.
  \item
  A fight about marriages and road taxes has caused an argument between the town \gls{warden} and the next point with a human settlement (if there are no human settlements with a higher number, the next is on the next map).
  The \gls{sunGuard} will march to war within three months\ldots
  \item
  A grand \gls{court} where the town \gls{warden} finds beasts guilty, and has them fight for entertainment.

  Live beasts fetch a premium price -- 1~\gls{sp} per \gls{cr}.
  \item
  The grandest \gls{templeOfCuriosity} known, with books (and soaps) of every size and type.
  Some say alchemical gateways to other lands hide within the library, but those people always mysteriously vanish before long.
  \item
  The \gls{warden} is \pgls{witch}, and does not feel shy about using his skills publicly.
\end{enumerate}

\paragraph{Name the settlement}
something which fits the entire area.
If it has half a dozen \glspl{village} with mills and a swamp, you might call it `Millrot'.
If the local town's ruler practices open \gls{witchcraft}, it might be `Hexward'.
Settlements often receive the same name as their town, if they have one.

Try some word-association with the town's theme or location.
Write the settlement's name next to it, and underline it.

\subsection{Lost Cities}
\label{lostCities}
\index{Lost Cities}

\widePic[t]{Nelness/city}

People call them `\gls{yonder}'s house', as a reminder that most people who enter never return, but remain with \gls{yonder}.
People also call them `land \glspl{deep}', because these spaces act according to their own rules; they do not have the usual wandering monsters, but some predators always take residence in them, along with the fiend who ruined the city.

Before the \glspl{pc} enter, you should make some rolls to find out the nature of the town.
What creatures live there?
Which towers are still standing?
Roll dice to find out which tall towers still stand in the city, in steps \vrefrange{lostDwellers}{lostTowers}.

Once the \glspl{pc} enter, they can attempt to look for valuables, quietly, in steps \vrefrange{lostWhispers}{lostForaging}.

Entering a lost city will go something like this:

\begin{exampletext}
  The \gls{gm} knows the troupe are headed to the lost city, where \pgls{ogre} stays with his goblin horde.

  First, she rolls up the inhabitants.
  `\dicef{3} \dicef{3} \dicef{6}' indicates two city-dwellers: \glspl{crawler} and a coven of $1D3+1$ demiliches.
  One more roll shows that means a coven of 3 demiliches have made a little secluded spot for themselves in the lost city, unbothered by the goblins.
  Perhaps they use the \glspl{crawler} as a source of \glspl{ingredient}?

  Next, she rolls $4D6$ to find the towers, and finds `\dicef{2} \dicef{2} \dicef{6} \dicef{4}' -- that means that this lost city still has 3 tall towers standing
  (towers let the \glspl{pc} take in an overview of the city).

  As the troupe enter, she describes the scene to the players.

  \begin{boxtext}
    The city seemed massive from outside, but here at the rusted gate, you can only see doorways and windows lying open like black eyes.
    Thorny bushes with little multicoloured berries grow all down the street ahead, where the Sunlight strikes brightest.
    Dropping around them suggest deer frequently come here to feed.
  \end{boxtext}

  The troupe moves quietly towards the location their map claims the old guild halls once stood, by the market.
  But before hunting for loot, they roll \roll{Dexterity}{Stealth} (\tn[8]) to stay silent.
  The group roll succeeds, so they move quietly, then roll \roll{Intelligence}{Vigilance} (with a +3 Bonus for the map) at \tn[12].

  The first roll fails, so the \gls{gm} mentions that they can cover more ground by splitting up if they want.
  But \pgls{interval} has already passed while they hunted, and the Sun is beginning to set.

  The troupe is now deep into the lost city, so they can wait in the dark, or navigate back out again, replacing the usual city encounters with the wandering monsters outside.
  They decide to stop foraging, and hide in an abandoned house, then wait to see what morning brings.

  During the night, the \gls{gm} must keep track of what the dwellers in the lost city get up to.
  Since the city has demiliches and \glspl{crawler}, it makes sense that the demiliches would want to use the \glspl{crawler}' bodies as \glspl{ingredient} for \gls{witchcraft}.
  The \gls{gm} describes the sounds of demiliches casting Death magic on \glspl{crawler}, so they can use their bodies to make \glspl{boon}.

  They probably don't make much sound -- shuffling, grappling, and swooshing cloaks as the necromancers gesticulate to cast spells.

\end{exampletext}

\mapentry{Finding the City}

It may seem difficult to lose a city, but when grass grows over the roads, and the wooden \glspl{village} crumble, cities can become almost invisible.
If the troupe only know about a city's location through legend, they will have to roll \roll{Intelligence}{Wyldcrafting} at \tn[14] to locate the city.

Cities which have fallen more recently often still have a visible road, and will not require any roll to find.

\mapentry{Wandering Dwellers}
\label{lostDwellers}

Each city has some population of the servants of the local fiend.
If \pgls{lich} lives in the city, it has a lot of ghouls; if \pgls{ogre} king lives here, it has goblins.

Additionally, lost cities attract other creatures which end up taking residence.

Roll $3D6$ -- each number which comes up determines one type of city-dweller.
Ignore any repeats.

\begin{dlist}
  \item
  \textbf{\Pgls{seeker} of the \gls{templeOfCuriosity}} has enlisted $1D6+4$ fully armoured \glspl{sunGuard} as part of a reconnaissance mission for \pgls{warden}.
  A camp sits nearby, with two more \glspl{sunGuard}, six donkeys, and 20 days' rations.

  The \gls{seeker} won't like anyone interfering with his business, so if he encounters the troupe, roll a Morale check (see \vpageref{morale}); on a pass, he tries to kill them.
  \item\label{lostOoze}
  \textbf{Oozes} of all types slide across the streets, which seem strangely clean, and free of debris or foliage.
  \item
  \textbf{\glspl{crawler}} hide in every abode.
  Verdant berries of every colour have encouraged deer into the area, but every shadow gleams with thick webbing.

  $1D6+8$ live here in total.
  \item
  \textbf{Griffins} look down from every tower.
  The high towers make perfect perches to surveille the area, and the degrading wood helps to make nests.

  Each tower has 2 griffins at the top.
  \item
  \textbf{\Glspl{woodspy}} have not only made this place their home, they have taken to imitating the items in the area, such as stools, piles of books, or chests.

  $1D6+4$ live here in total.
  \item\label{lostDemilich}
  \textbf{Demilich} covens help these undead sorcerers study with their own kind.
  However, their lack of basic empathy makes them dangerous to each other -- none of them really trust the others, so they share information slowly, always hinting that they have more to teach while masking their true abilities.
  Whenever one is wounded while another is not, they both attack each other (the first has spotted the right moment to slay a potential enemy, and the second knows what the first is thinking, as they all operate by the same unempathic principles).

  Roll $1D3 + 1$ to determine the number of demiliches.
\end{dlist}

\mapentry{Tall Towers in the Labyrinth}
\label{lostTowers}

Wooden rooves rot, leaving bear stone walls across much of the fallen city.
Some walls collapse, opening new passages, while trees grow and thorny bushes grow to block doors and streets.
A twisted, mossy, labyrinth forms.

Despite the rot and chaos, some buildings still stand tall.
Once \pgls{pc} climbs to the top of a tower, two things happen:


\begin{itemize}
  \item
  The tower's vast perspective grants a +2 Bonus to foraging, as they spot good places to investigate (such as places which look like guild houses, or less-damaged areas of the city).

  This only works while the character can see, so it probably only works during daylight.
  \item
  Any undead in the city can clearly see the character, and will begin moving towards them.
  \item
  There is a 1-in-6 chance of seeing the dwellers of the city interact with each other (start lowest to highest).

  If the city has a curious \gls{seeker} and some acidic oozes, then the character may see the \gls{seeker} running away from the oozes.
  Or if the city has \glspl{crawler} and griffins, they may see a griffin caught in a web.
\end{itemize}

Climbing these buildings gifts a wide perspective of the city, which grants a +2~Bonus to the \glspl{pc}' foraging rolls.

Roll $4D6$ -- each number which comes up indicates a tower in the lost city.

\begin{dlist}
  \item
  This old \gls{templeOfPoison} had lavish beds up at the top, though the only current inhabitants are dead spiders and rat-droppings.

  The kitchen has a basement, filled with sealed clay pots.
  The wooden ladder has rotted, and will break once anyone of \gls{weight} 6 or more steps onto it.

  Anyone investigating the basement should make an \roll{Intelligence}{Vigilance} roll (\tn[10]).
  Success means they have found a secret stash of 20~\glspl{sp}, hidden in a honey pot.
  Failure means that a clay pot has gone rotten, and explodes; the character has contracted Corpse Hands (\vpageref{Corpse Hands}), and they will start to feel the effects after 2~\glspl{interval}.
  \item
  This tower has one wall remaining -- a \roll{Speed}{Athletics} check (\tn[8]) allows the climber to see a little in all four directions.
  \item
  This tower once held the \gls{sunGuard}, but now has a single servant of the local fiend (usually goblins or ghouls).
  If the \glspl{pc} do not take it by surprise, it will shout for aid (the undead can shout to other undead silently).
  \item
  This tower stands tall enough to see all around.
  The bones of dead humans fill the stairway.

  Any movement which disturbs the bones will send them falling down the stairs with a clank and a thud.
  \Glspl{pc} must roll \roll{Dexterity}{Stealth} (\tn[11]) to move past the bones without issue, or else waste \pgls{interval} moving everything down by hand (in this case they roll \roll{Dexterity}{Crafts}, \tn[7]).
  \item
  The old citadel still stands, despite decay.
  Parts of the original \gls{warden}'s home have rotten flooring, so moving through it requires a \roll{Wits}{Crafts} roll (\tn[8]) to notice weak areas of flooring, and the best way to walk.
  Failure indicate that the character falls two~\glspl{step}%
  \exRef{core}{Core Rules}{falling}
  and makes an awful sound.
  \begin{enumerate}
    \item
    On the second story, characters will encounter the city dweller with the lowest number.
    \item
    If anyone gets to third story (which requires three rolls), then they will find a cupboard with \lootBig.
  \end{enumerate}
  \item
  This tower holds an empty library, once part of \pgls{templeOfCuriosity}.
  It has no books, and little stability left.

  Upon entering, \glspl{pc} can make a \roll{Wits}{Crafts} roll to understand if this tower will hold their own weight (the rolls tells the \gls{pc} nothing about others).
  Anyone with \pgls{weight} above 5 prompts the structure to collapse, but only once they have reached the top.

  If the tower begins to collapse, anyone inside can run down (\roll{Speed}{Athletics}, \tn[9]), but failure indicates the building has collapsed upon them, inflicting $4D6$ Damage.

  Or the \gls{pc} can elect to jump, and try to grab something nearby (\roll{Speed}{Athletics}, \tn[12]), but failure indicates that they will suffer a nasty fall of $2D6$ Damage.
\end{dlist}

\bigLine

\mapentry{Creeping in the Cold Labyrinth}
\label{lostWhispers}

When the \glspl{pc} want to move through a lost city, they roll \roll{Dexterity}{Stealth} to remain quiet.
The \gls{tn} is normally 8, but often easier.

\begin{itemize}
  \item
  -1 if moving at night.
  \item
  -1 in the rain.
  \item
  -1 in a storm.
  \item
  -2 if the dwellers are busy with something.
  \item
  -4 if hiding indoors.
  \item
  +5 if lighting a fire.
\end{itemize}

The undead are a separate matter.
Even if the troupe move around silently, any active undead will spot them, unless they have some way of hiding from the sight of the dead.%
\footnote{See \autopageref{undead_senses}.}

\paragraph{When the roll fails,}
\label{lostChase}

Then they have an immediate encounter, in this order:

\null
\begin{enumerate}
  \item
  The dwellers with the lowest number.

  \textit{For example, a city with oozes and griffins would mean that if the \glspl{pc} make a noise, then an acidic ooze wanders out of a house, and heads towards them.
  Then on the next failed check to creep around, a griffin would attack.}
  \item
  $2D6 + 8$ of the fiend's servants encounter them, and attack.

  \textit{Usually this means a lot of goblins, or a lot of ghouls.}
  \item
  The fiend itself has awakened, and begins planning an assault.
  They will probably not approach the \glspl{pc}, but will find a vantage point to see the \glspl{pc} from, and co\"ordinate attacks.
\end{enumerate}

\paragraph{Running away,}
requires a \roll{Speed}{Stealth} roll, at a \gls{tn} equal to the enemy's \roll{Speed}{Vigilance}.

\paragraph{Trying to fight}
means making more noise.
Every $1D6$ rounds, another encounter occurs.

\mapentry{Foraging Quietly}
\label{lostForaging}

Each time the \glspl{pc} forage, they roll \roll{Intelligence}{Vigilance} (\tn[12]) to figure out where the best loot lies.
If they succeed, roll $2D6$ -- the first die determines the place, and the second determines the prize!

\begin{itemize}
  \item
  Once they have a prize, cross it off the list.
  \item
  If they get the same number again, they will find the second prize with that number.
  \item
  The third time they roll a number, they find nothing -- lost cities only hold so many treasures.
  \item
  If the \glspl{pc} split up, each one can perform individual rolls, so they will loot far more efficiently, but run more chance of one getting caught alone.

\end{itemize}

\subsubsection{Climbing the Walls}

\Glspl{pc} might feel a temptation to climb the outer walls to get a proper perspective of the city, but doing so leaves them visible to every dweller within.

They gain a +2 Bonus to foraging rolls, but one of the dwellers with the highest number moves (or flies) to attack them.

\end{multicols}

\foragingChart

\section{Depth}
\label{mapDepth}

\begin{multicols}{2}

\subsubsection{Finishing the Map}

\noindent
Once you have each area detailed, you have a complete map, but you can always add more details.

\begin{itemize}
  \item
  Draw contour-lines inside any lochs or sea.
  \item
  What are the roads' names?
  \item
  Do you have any points with civilization, but without a road connecting them to anywhere?
  What would the neighbouring points think about people they occasionally see, but cannot reach by road?
  What rumours would the troupe hear about those places?
  \item
  Select two neighbouring points on your map, and make a new map based on them.
  Add detail, put names on everything, and maybe add some misleading elements.

  If you want to put forests on this map, make it an abstract shape with a clear border, and make sure it goes right up to the edges of the map.
  Everyone in \gls{fenestra} knows they inhabit a tiny island of civilization, walled in by a towering, green, sea.
  \item
  Follow a road leading off your map.
  Where does it go?

  Start a new map, and connect them by the roads which venture past the page.
  Repeat until you charter the world.
\end{itemize}

\bigLine

\subsubsection{Player Connections}

How are the \glspl{pc} connected to this land?
Make a numbered list of possible past events, with a default and non-default race for each entry.

\begin{itemize}
  \item
  If a human rolls a point where humans live, or a gnoll rolls a point where gnolls live, then they come from that place -- perhaps from a town, or \glspl{village} surrounding it.
  \item
  Otherwise, they may have a connection to a nearby fiend, living between the points.
  Perhaps the fiend destroyed their home and family.
  Perhaps they suffered an attack on the road which still haunts their dreams.
  \item
  Non-humans rolling a human town might have worked with whatever the town is famous for (step \vref{mapCharacter}) -- a gnome might have worked as a tracker, or come to the city to listen to its famous bards.
  Or perhaps a gnoll helped keep the great beasts in a town's \gls{court}.
  \item
  Humans rolling an elven settlement might have been raised there, or lived there for some time after fleeing the law.
\end{itemize}

When the players create their \glspl{pc}, have each one roll on your chart to determine their connection to the land.

\needspace{8em}
Your chart might look something like this:
\begin{enumerate}
  \item
  \underline{Horseshoe Valley}
  \begin{description}
    \item[\textbf{Humans:}] you grew up in a little \gls{village} by Horseshoe Valley, and want to seek your fortune in the wider world.
    \item[\textbf{Others:}] you came as a trader to purchase iron goods, but soon grew scared of travelling the road alone.
    Life in the \gls{guard} now seems safer than travelling with all those goods.
  \end{description}
  \item
  \underline{Elven Meadows}
  \begin{description}
    \item[\textbf{Elves:}]
    you come from the elven villages.
    While the elders gave their blessings, they made you swear never to bring outsiders back home.
    \item[\textbf{Others:}]
    the local \gls{hag} -- `Thingizard' -- killed your family and tribe.
    You alone escaped, and now dream impotently of revenge.
  \end{description}
  \item
  And so on\ldots
\end{enumerate}

\end{multicols}


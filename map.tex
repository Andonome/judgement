\chapter[The Cartographers]{Cartography}
\label{civilization}

Welcome to \gls{fenestra}.
Let's look around.

\section{Your Map}
\label{yourMap}

\begin{multicols}{2}

\noindent
Fantasy stories have two special characters -- magic, and the landscape.
The audience discover both, just as they learn about the other characters.

Magic typically remains a mysterious character, but the landscape should always feel open.
As \pgls{gm}, your character is the landscape, and making a map is your character creation.

\mapentry{The Pentacle}

Tear out an example map on the left for the basic map.
You'll need a pencil (or pen, if you're feeling brave) and dice.

At each point, roll $1D6$ and add the number at the point, then add the entry to the map.

\begin{multicols}{2}

\begin{enumerate}
  \setcounter{enumi}{1}
  \item
  Humans
  (\vpageref{humanPoint})
  \item
  Bandits
  (\vpageref{banditsPoint})
  \item
  Dragon
  (\vpageref{dragonPoint})
  \item
  Humans
  \item
  Humans
  \item
  Humans
  \item
  Elves
  (\vpageref{elvesPoint})
  \item
  Dwarves
  (\vpageref{dwarvesPoint})
  \item
  Gnomes
  (\vpageref{gnomePoint})
  \item
  Dryad
  (\vpageref{dryadPoint})
  \item
  Ogre
  (\vpageref{ogrePoint})
  \item
  Gnolls
  (\vpageref{gnollPoint})
  \item
  Hag
  (\vpageref{hagPoint})
  \item
  Lich
  (\vpageref{lichPoint})
  \item
  Humans
\end{enumerate}

\end{multicols}

\begin{exampletext}
  For example, if your first die read `\dicef{4}', then the result is $4+1 = 5$, and you can write `humans', or just `\Hu' on the map.

  If the next roll is `\dicef{6}', then you would write `elves' or `\El' next to point 2 on the map.
\end{exampletext}

Each dot has four neighbours.
The highest neighbour is the neighbour with the highest number, and the lowest neighbour, is the neighbour with the lowest number.

\mapentry{Rivers}

\paragraph{If your map has dwarves or a dragon,}
then they live in mountains.

\begin{enumerate}
  \item
  Draw a mountain range around the point with the dwarves.
  \item
  At each mountain, draw a river to each of the four neighbours, unless that neighbour has a mountain.

  River always begin from separate mountain-locations.
  They might later join together, but rivers never split apart.
  \item
  At each point with a river, extend the river to the lowest neighbour.
  \item
  Rivers begin as a single line, but when they join, they flow faster and wider -- draw them with two parallel lines.
  \item
  If three or more rivers join together, they form a lake, which exits towards the lowest neighbour.
  It still has the strength of many rivers, so continue the exit river with two lines.
\end{enumerate}

\paragraph{If your map has no dwarves nor dragon,}
then it has no mountains.

\begin{enumerate}
  \item
  Points 1 and 2 become islands in a sea.
  \item
  The coast covers points 5, 6, and 7, then wanders off the map.
  \item
  Some well-springs open up on point 10, then join together to form a river.
  The river goes down through points 4, and 3, then goes towards point 2 and enters the sea.
\end{enumerate}

\paragraph{If your map has any points without water,}
then the highest of them becomes a swamp.

\mapentry{Roads}
\index{Roads}

\begin{enumerate}
  \item
  Each human settlement has a road its two nearest human neighbours (ties are broken by going to the highest point).
  Draw the roads with dotted lines, and avoid passing through any other points along with way.
  \item
  If a road crosses a small river, draw a small bridge.
  If it crosses a larger river, drawn with two lines, draw a larger bridge.
  \item
  The human settlement with the highest number has a road leading out of the map, towards the closest border.
  This road stays as far away as it can from other settlements.
  \label{roadOut}
  \item
  If a human settlement has dwarvish neighbours, connect the dwarvish neighbour with the highest number.
  \item
  Label a few roads with the miles in length.

  \begin{itemize}
    \item
    The short distance between points 6 and 7 is about 20 miles.
    \item
    The medium distance between points 1 and 6 is about 30 miles.
    \item
    The long distance between points 1 and 2 is about 50 miles.
  \end{itemize}
  \item
  If any road stretches for 50 miles or more, place a Lonely Tavern half-way along that road (details \vpageref{lonelyTaverns}).
  \item
  The road between points take a long time to walk.
  Put \pgls{bothy} on the road for travellers to rest, and then another for each point the road passes.
\end{enumerate}


\bigLine

\paragraph{Do not draw trees on the map,}
just know that they cover everything.
Drawing a few trees or a little forest suggests that everywhere else has open plains, but open areas are rare in \gls{fenestra}.
A twilight of towering covers almost everything which people have not cut down.

\paragraph{The map's name}
should come from a distinctive feature.
A good starting point is the most prolific feature of the map.
For example, a map with a swamp and three elvish settlements ranges might receive the name `Faebog', or one with two dragons and a lot of lakes might be called `Lochscale'.

\paragraph{One or two details}
should be added to the map.
Pick where the \glspl{pc} should start, and add put in some detail on that area, and perhaps another -- each type of inhabitant has a system to add details below.

A map doesn't need every area filled with perfect detail from the start, so you can leave more distance regions until you think the \glspl{pc} might head there.
And if the troupe lunge to the far end of the map, you can take a quick break, and roll up the details for the next place once they arrive.

\end{multicols}

\section{Zooming In}
\label{mapCharacter}

\begin{multicols}{2}

\noindent
Each point on the map needs more detail once the \glspl{pc} arrive.
Select your campaign's starting point (almost certainly a human settlement) and find out the details.
You might fill in a neighbouring point, if you think the \glspl{pc} could journey there within the first session.

Whenever you find out a new location's details, note them down as a key to the main map, and check if you can add any features to the main map.

\subsection{Civilization}

The civilized lands have people of all types.
Elves, humans, gnolls, dwarves, or gnomes -- places with a population that either sustains itself, or trades with others.

\subsubsection{Dwarven Settlements}
\label{dwarvesPoint}

A population of 100 dwarves lives here, with a mushroom garden sustaining them.

Roll $3D6$, then combine each pair of 2 dice to find each unique feature.
Whenever you roll 8 or more, add 100 dwarves to the population.

\begin{exampletext}
  For example, on the roll `\dicef{2} \dicef{4} \dicef{2}', you would use the results for:

  \begin{itemize}
    \item
    $2+4 = 6$
    \item
    $4+2 = 6$
    \item
    $2+2 = 4$
  \end{itemize}

  The unique results are `6', and `4'.
\end{exampletext}

\begin{enumerate}
  \stepcounter{enumi}
  \item
  This population is ruled by a king.
  Dwarves consider to be bad luck, but the king promises to have a daughter and let her take his place in the next generation.
  \item
  Gold seams run through the mountain, allowing the dwarves to make precious items, and to trade with various races outside.
  \item
  A Hall of Records stands at the base of this mountain, guarded by heavy doors.
  It works a little like a library, but only contains factual accounts of the things dwarves take interest in, such as facts about rocks, the lineage of queens, descriptions of how to farm on mountains, and techniques to measure altitude.

  Each book has a minimum of three seals, pressed a dwarven record-keeper, who has verified the authenticity of every statement in the book personally.
  \item
  The `Polite Passage', is the name dwarves use for the long, thin, bridges they create, with only room for a single dwarf to walk at a time.
  Settlements have one for entering, and another for exiting.

  Larger rooms around the mountain's exterior store wide equipment, such as wagons.
  \item
  Magma streams in the mountain's heart flow eternally, and mix with underground rivers, producing a great plume of steam at the top.
  Various tunnels contain steam-traps, where a metal grate will ring the `dinner bell', and allow hot steam to flow in to cook the intruding predator.
  \item
  A tin seam allows dwarves to mine, and trade with nearby settlements.
  \item
  In the dark \gls{deep} below the mountain, extensive underground tunnels, filled with fungi, moss, living oozes, olms, and umber hulks.

  With enough traps laid out for the monsters, this creates a kind of rough garden, or `hunting ground' for the dwarves to gather additional food.
  \item
  A few farms extend outside the mountains, across the fertile lands at the base.
  The dwarves create them with high stone walls, and polish each brick until it has nothing to hold onto, and no way to climb up.

  Inside, may allow as much space as any \gls{village} to grow vegetables for the mountain.
  \item
  An elevated barley garden around a third of the way up the mountain, provides food for the settlement.
  Nobody can see the gardens from below, as it grows on steps carved into the mountain.

  A series of rock-bells will sound when anything steps on them, but do not ring in response to the wind.
  \item
  An iron seam allows dwarves to create excellent weapons and armour.
  If they have no road out, then they only make rare, wholesale trades.
  \item
  Half way up the mountain, shepherds take sheep and goats on long walks, then put them safely away at night, to sleep in stone rooms.
\end{enumerate}

\subsubsection{Elven Grounds}
\label{elvesPoint}

A group of $2D6 \times 10$ elves lives here.

Roll $3D6$, then combine each pair of 2 dice to find each unique feature of the local elven territory.

\begin{enumerate}
  \stepcounter{enumi}
  \item
  An elder, focussed on the Force Sphere, often casts portal spells around the many bushes and briers in the area, warping space.
  Anyone entering may accidentally emerge at the other side of the elvish lands.
  \roll{Wits}{Wyldcrafting} to notice, \tn[7] in the day, or \tn[12] at night.
  \item
  Expert archers leave their marks on every tree.
  They always collect every arrow, but the marks remain.
  \item
  An elder, focussed on the Life Sphere, often casts spells around the perimeter to make anything there shrink.
  Replace the first two standard encounters with a miniaturized version (e.g. miniaturized chitincrawlers or bears).
  Elves know how to avoid the effects, and never suffer from them.
  \item
  An elder, focussed on the Light Sphere, often casts illusions of elves.
  Anyone entering will find $2D6$ dancing apparitions scattered around the area, but not responding to them.
  \item
  An elder with the Life Sphere creates enchanted gardens around the area, but always puts them in a new location each time (so the soil never becomes depleted).
  \item
  Useful plants grow all over the region.
  Each of the three dice you have rolled indicate one common plant in the area:
  \begin{dlist}
    \item
    Bedleaves
    \item
    Bloodwood
    \item
    Dryad's Kiss
    \item
    Mage Oak
    \item
    Seekmist
    \item
    Whistling Cane
  \end{dlist}
  (find the plans no \vpageref{plants})
  \item
  An elder, focussed on the Mind Sphere, often casts spell around the perimeter, which confuse anyone who enters.
  They lose a portion of the day, wander aimlessly, and emerge somewhere else, without any memory of what they have done.

  The elder occasionally enchants griffins in the area, allowing a few elves to fly on their backs.
  \item
  A peat-bog furnishes the elves here with iron, allowing them to craft iron weapons.
  \item
  Slow spells from the Life Sphere have crafted massive tree-houses around the area.
  When the lights are on, anyone can see them.
  When the lights go out, they appear like normal trees.
  \item
  Expert artisans live in this community.
  Local clay deposits allow them to craft excellent pottery, including ceramic armour.
  If they have dwarven neighbours, they also trade for gold and silver, and make fine jewellery.
  \item
  Xenophobia has gripped the elves here hard.
  They will kill anyone who enters on sight, and everyone in the land knows it.
\end{enumerate}

\subsubsection{Gnoll Grounds}
\label{gnollPoint}

Gnolls typically wander back and forth around a territory.
They will use roads when they see them, but never like to rely on them.

The tribe has $2D6 \times 5$ members, and the same number of hunting dogs.
Roll $2D6$ twice to find a couple of characteristics for the gnolls who wander this ground.

\begin{enumerate}
  \item
  The tribe have taken in a bear-cub, and raised it as one of their own.
  It hunts with them, and understands when to be quiet, and when to alert people to danger.
  \item
  This tribe suffer from isolation.
  Where most tribes have a few members who learn the neighbouring languages, these gnolls only know how to trade with explicit allies, and still cannot speak their languages properly.
  \item
  The lowest neighbour with a fiend has killed a few of these gnolls, and they want revenge.
  \item
  The highest neighbour with civilization (i.e. any spot without a fiend) pays well for the gnolls' hunting dogs, so many of the tribe members have items from them, along with a general feeling of kinship.

  For example, if the highest neighbouring point near them has elves, then they will have elven jewellery, and plenty of elven songs (though the thick accent may not make this obvious).
  \item
  The lowest neighbour with civilization has started a fight with this tribe, and they want revenge.
  Currently, the tribe wants to identify which humans make decisions, and once they know, they will begin the assassinations.

  Their main barrier is that humans change clothes and how they smell regularly, making them hard to re-identify.
  \item
  The tribe have figured out how to use local plants as hair-dye, and they have decided that this is the best thing ever!
  \item
  A powerful druid advises the tribe.
  \item
  These gnolls have some of the best hunting dogs around, and plenty of them.
  Double the number of hunting dogs, so they have twice as many dogs as tribe-members.
  \item
  The gnolls know all the local plants, including where to find marching mushrooms and screechmoss (\autopageref{marching_mushroom}).
  \item
  The tribe keep a herd of aurochs, and wander constantly in order to let them graze.
  They have no fences, but they can still stop the aurochs wandering away -- each one of their dogs knows to stop them escaping.
  \item
  The tribe keep a small herd of sheep.
  \item
  Everyone has contracted \nameref{Torpid Flesh}.
  They are hairless, and look half-dead
  (see \autopageref{Torpid Flesh}).
\end{enumerate}

\subsubsection{Gnomish Warrens}
\label{gnomePoint}

Gnomish warrens interact in subtle ways with every point around them.
To find out about a gnomish warren, you should finish every point around it first.
Every point which might benefit the gnomes results in another warren, with some new specialization.

When multiple warrens exist on a single point, they each connect to a nexus cavern below, which allows the gnomes to trade.

\begin{itemize}
  \item
  If they have gnomish or dwarvish neighbours, they enlarge the natural tunnels beneath them to make an underground road.
  The ups and downs mean the path is twice as long below ground as above.
  It is also not without danger, but underground roads have far fewer creatures than above-ground.
  (add another warren)
  \item
  If this point has neighbours on a mountain, the gnomes have crafted an underground stream boat-ride, which goes from the mountains, to their warren.
  (add another warren)
  \item
  If they have a road nearby, they construct a series of underground bothies, each 6 miles apart, to reach the road.  They can walk above ground, but always have somewhere to rest along the way. (add another warren)
  \item
  If a lake lies in the area, they construct a grotto to lay traps for the creatures there.  These traps catch so much that nobody else has much luck fishing. (add another warren)
  \item
  If gnolls live nearby, they trade for hunting dogs, and occasionally ride them when going out. (add another warren)
  \item
  If a dryad lives nearby, they create irritating songs which get stuck in your head, and sing them constantly.  Dryads hate this kind of thing.
  \item
  If an ogre lives nearby, they create pit-traps with spikes at the bottom.  When an ogre steps in one, they cannot walk straight for a week.  If a goblin steps in one, it usually dies.
  \item
  If a hag lives nearby, the gnomes grow wheat, and use an underground mill to make flour, and finally, to bake cakes.  Cakes and a little flattery always works on old crones.
  \item
  If elves live nearby, they arrange for a magical dual (just for sport), every \gls{Atya}.

  The gnomes also add a road between them and the elves.
  Elves don't appreciate roads, but the road serves as a useful warning that if anyone attacks the gnomes, the elves may fall soon after.
\end{itemize}

\paragraph{Politics in the Warrens}

Each warren governs itself a little differently.

The number of gnomes in a given warren equals $2D6\times 10$.
The first die indicates the political structure this warren has adopted.

\begin{dlist}
  \item
  Do-opoloy (you decide how something works by making it work)
  \item
  Thaumocratic (alchemists rule)
  \item
  Anarcho-syndicalist (this mostly consists in debating how anarcho-syndicalism works)
  \item
  Technocratic (whoever understands most about a subject decides how it's done)
  \item
  Direct Democracy (dinner takes about 3 hours each night)
  \item
  Indirect democracy (people vote once a week, documentation is crucial)
\end{dlist}

\subsubsection{Human Settlements}
\label{humanPoint}
\index{Towns}

\begin{enumerate}
  \item
  Start with $1D6$ to determine the size of the settlement.
  \begin{dlist}
    \item
    2 \glspl{village} and \pgls{broch} forming a triangle.
    The \gls{broch} stands away from the road, protecting the \glspl{village} from whatever might emerge from the \gls{edge}.

    (civilization rating: 10)
    \item
    4 \glspl{village} with 2 \glspl{broch}.
    (civilization rating: 12)
    \item
    6 \glspl{village} form a circle around a small town.
    3~\glspl{broch} sit opposite the road, to protect the \glspl{village} from the \gls{edge}.

    One of the \glspl{village} began as \pgls{broch} and still has that tall tower.
    \Pgls{warden} stays there, ruling over this tiny kingdom.

    (civilization rating: 14)
    \item
    8 \glspl{village} surround a town, with room to safely grow crops and raise cattle outside its walls.
    4~\glspl{broch} help to harden the outer barrier of settlements which protects the town.

    The local \gls{warden} has plenty of profits coming in, and can afford his own \glspl{sunGuard}, to ensure everyone pays their taxes, so that the \gls{warden} can pay for the \gls{guard}.
   
   (civilization rating: 16)
    \item
    10 \glspl{village} and 5~\glspl{broch} keep the town safe.
    A healthy flow of traders ensures the \glspl{village} by the roads stay relatively clear -- anything which wants to attack will try to stalk those traders before attacking the outer \glspl{village}.

    A couple of the outer \glspl{village} have their own \glspl{warden}.

    (civilization rating: 18)
    \item
    12 \glspl{village} and 6~\glspl{broch} surround a massive town.
    Various hamlets allow farmers to raise beasts in peace.
    The network of roads has plenty of hand-crafted signs, telling people how to get to various areas.

    The town supports a minor army of \glspl{sunGuard}, and the area exports plenty of food.
    Each \gls{village} has \pgls{warden} associated with it, though the \glspl{warden} stay in the inner hamlets.

    (civilization rating: 20)
  \end{dlist}
  \item
  Each \gls{village} and \gls{broch} has a small radius of flattened land, without trees, stretching up to a mile from its walls.
  \item
  Once you have some \glspl{village} and \glspl{broch} down, draw roads between them all.
\end{enumerate}

To discover a settlement's most famous feature, take the number of \glspl{village} and add $1D6$.
For example, if a settlement has 4~\glspl{village}, roll $1D6+4$.

\begin{enumerate}
\setcounter{enumi}{2}
  \item
  Newly built -- every day, stone and wood comes in to build a little more each day.
  Occasionally, parts of the wall collapse and archers flock to protect it.
  \item
  Trackers
  \item
  Deadly and magical plants
  \item
  Ale.
  The \gls{templeOfPoison} has tight control over the area (\autopageref{god:Poison})
  \item
  White rock (actually limestone)
  \item
  \ifodd\value{r4}
    Wood-carvers
  \else
    Metallurgists
  \fi
  \item
  Bards
  \item
  Madness (neighbouring witches are to blame)
  \item
  \Glspl{doula}
  \item
  Cartographers
  \item
  Statues of heroes litter the roads around.
  \item
  A grand \gls{court} where the town \gls{warden} finds beasts guilty, and has them fight for entertainment.
  \item
  Bath houses.
  \item
  `Demi-human suburbs', where all manner of other races live.
  \item
  A fight about marriages and road taxes has caused an argument between the town \gls{warden} and the highest neighbour.
  The \gls{sunGuard} will march to war soon\ldots
  \item
  The \gls{templeOfCuriosity} in the land.
\end{enumerate}

\paragraph{They call it\ldots{}}

Select a name for the settlement by combining what everyone knows it for, with the type of land.
Write the settlement's name next to it, and underline it.

Any towns might receive the same name, or something related.

\begin{itemize}
  \item
  People know a town in the hills for its library. That's why they call
  it `Page Valley'.
  \item
  A town by the sea, famed for Witchery.
  They call it `Hexwave'.
  \item
  `Bards' + `Islands' = `Lyrisles'
  \item
  \ifodd\value{r4}
    `Metallurgists' + `forest' = `Iron Basin' (perhaps a dried-up lake sits nearby)
  \else
    `Wood-carvers' + `forest' = `Taming Woods'%
  \fi
  \footnote{If you think these names sound stupid, you should look up the meaning of your hometown's name.}
\end{itemize}

Try some word-association with the town's theme or location.

\paragraph{It is Ruled by\ldots{}}
\index{Towns!Rulers}

If this point has a town, roll $1D6$ to determine its ruler.
Write down the information on a key, referencing each point by its number.

\begin{dlist}
  \item
  Multiple \underline{warring factions} -- roll twice more, with +1 to the second roll.
  \item
  A \underline{Warlord}, recently out of the \gls{guard}, amassing an army to expand their power.
  Shipments of weapons come from the road to other lands
  (see \vref{roadOut}).
  \item
  A \underline{greedy \gls{warden}}, who demands 10\% tithes from all who enter.
  Any neighbouring fiends annoy and fascinate them equally.
  \item
  A new \underline{Sorcerer king} with no experience or business ruling.
  He once killed a fiend, how hard could ruling \pgls{court} and arranged marriages be?

  The sorcerer has killed the same type of fiend as the highest fiend on the map.
  \item
  A \underline{Spoilt noble} who has never seen a farmer in their life.
  \item
  A \underline{servant of the next fiend} on the map (counted by point number).
  All who oppose this \gls{warden} disappear on the road.
  \item
  A \underline{Crime-Lord}, with a history of theft, now risen to fortune\ldots but not nobility.
\end{dlist}

\subsection{Fiends}
\index{Fiends}

\subsubsection{Bandits}
\label{banditsPoint}

An independent \gls{village} has broken away from the rest, and sustain themselves by robbing others on the road.

One road leads out of their settlement and joins with the nearest main road in secret; it splits into many roads at the last moment, so none of them look very used.
The bandits cover their tracks with foliage once they enter the main road, and only leave their private road at night.

A total of 200 villagers live in the commune, but only 50 regularly exit to perform robberies.

\begin{dlist}
  \item
  Ten of the bandit has a suit of plate armour.
  \item
  Fences among the \gls{village} with the lowest number help the bandits sell their wares.
  Without this aid, they must journey to settlements themselves, or make deals with people from other lands, accessible only through the long road out of the map.
  (see step \vref{roadOut})
  \item
  The bandits have brokered a temporary alliance with whatever fiend has the highest number.
  If there are no other fiends on this map, the alliance is with a fiend in another land -- accessible by the long road.
  \item
  An imprisoned gnomish alchemist performs tricks and makes \glspl{talisman} for the bandits.
  \item
  They hold an old \gls{broch}, abandoned by the \gls{guard}.
  It watches the private bandit road, and always has $1D6$ bandits watching, with crossbows.
  \item
  \Pgls{doula} leads the bandits from the rear, arming them with \glspl{talisman} and information.
\end{dlist}

\subsubsection{Dragon Coves}
\label{dragonPoint}

Dragons may live in deep caverns, shallow caverns, or sometimes just lay about in a Sunny patch of forest, without any roof.
Their destructive habits and association with the sky make people think they are divine.
Of course, in \gls{fenestra}, this is not a good thing.

\begin{dlist}
  \item
  An abandoned dwarvish settlement.
  Roll up a dwarven settlement, then replace those dwarves with a dragon (\vpageref{dwarvenPoint}).
  \item
  Complete plate armour, made by gnomes under threat of death.
  \iftoggle{core}{(See the core rules, \autopageref{stackingarmour}.)}{}
  \item
  Knowledge of the local languages.
  Dragons without this know only elvish (it doesn't change much).
  \item
  A glass-smelting workshop (everyone needs a hobby).
  Glass statues litter the dragon's lair.
  \item
  A portal to another realm, filled with glass flowers which function as every type of \gls{ingredient}.
  \item
  Three eggs, which will hatch next season.
  Once that happens, the dragon will need to venture out to feed her newborn.
\end{dlist}

\subsubsection{Dryad}
\label{dryadPoint}

Dryads concern themselves with plants and animals more than people\ldots but they also consider most people to be animals.

Roll $3D6$, and apply every result.

\begin{dlist}
  \item
  A dozen ex-farmers, now transformed into strange mutant creatures.
  \item
  A maze of venomous thorns envelopes the entire area.
  \item
  The dryad makes \glspl{talisman}, just for fun, and leaves them as gifts for travellers who behave politely.
  Other \glspl{talisman} find their way to less polite travellers.

  Sensible people leave all of them alone; one can never tell what a dryad considers good behaviour.
  \item
  Deep caverns exit in the centre of the dryad's lair.
  The dryad sometimes journeys down, to bring up umber hulks and oozes, and observe them, then release them into the wild.
  \item
  $1D3$ basilisks love the dryad, and will protect them with their lives.
  \item
  $1D6$ griffins, who look after the dryad, stay with the dryad, and allow her to fly.
\end{dlist}

\subsubsection{Hag Cottage}
\label{hagPoint}

Here, an old lady lives alone in a hut.
Of course, in \gls{fenestra}, only one type of old lady lives alone.

Roll $3D6$, and apply every result.

\begin{dlist}
  \item
  The hag has a dozen griffins circling her hut at all times.
  Each one will kill anyone who gets close to the hut, but will not attack them before they approach.
  \item
  The hag keeps 4 chitincrawlers as pets.
  They spin webs across every window and tree.
  (\vpageref{chitincrawler})
  \item
  Mouthdiggers create pot-holes around the entire area, and ensure nobody attacks from below.
  (\vpageref{mouthdigger})
  \item
  A garden of carrots and \nameref{screeching_moss} encircles the cottage.
  It screams the moment someone steps on it.
  (\vpageref{screeching_moss})
  \item
  The soil in all directions has been ravaged by constant spells to grow food.
  The trees have no leaves, the grass looks half-yellowed, and no animals inhabit the area except some small insects.

  The hag now has to take long walks to grow food, which continuously degrades more and more of the landscape.
  \item
  An assortment of potions bottles line her shelves.
  Each one has a poorly-written label, or the wrong label.
\end{dlist}

\subsubsection{Ogre}
\label{ogrePoint}

The ogre has $2D6 \times 5$ goblins, ready to raid all the local \glspl{village}.

Roll $3D6$, and apply every result.

\begin{dlist}
  \item
  The goblin population grew until they ate an entire \gls{village}, and then consumed every \gls{village} in this area.
  They laid siege to the city here, and then managed to enter by using a catapult and primitive parachutes.

  The city was half-burnt, and remains inhabited by the goblin horde, and their king.
  Everyone remembers this area, and takes it as a warning to deal with small problems before they become big problems.

  Create a lost city for the ogre, on \vpageref{lostCities}.
  \item
  Long caverns below provide the goblins a mushroom farm.
  They extend to a cold region, and eventually into the \gls{deep}.
  \item
  The ogre has a set of complete plate armour, created by a blacksmith he still holds prisoner in an abandoned \gls{broch}.
  \item
  The goblins have acquired 10 crossbows and 100 quarrels, and have started to figure out how to use them.
  \item
  The ogre king has a pack of $3D6$ hobgoblins.
  \item
  A travelling dryad, who found the goblins rude, erected a living hedge-maze, as vengeance for their bad manners.
  It changes a little each season, so the route out changes with it.
  And every cold season, the hedge maze vanishes, to reveal a horde of starving goblins.
\end{dlist}

\subsubsection{Lich}
\label{lichPoint}

Every lich has a set of caverns to stay.
Finding a lich in the cold, twisting \gls{deep} makes a near-impossible challenge, as they change location at random.

Roll $3D6$, and apply every result.

\begin{dlist}
  \item
  The lich waited for centuries, building an army of ghouls.
  When a nearby human town weakened, the lich attacked.

  He remains in the town to this day.
  The local \glspl{village} have rotted, but 3~\glspl{broch} remain standing.
  Create a lost city for the lich, on \vpageref{lostCities}.
  \item
  An old \gls{broch}, lost to the \gls{guard}, now used as an alchemical workshop.
  \item
  The lich keeps a crew of $3D6$ ghasts, ready to follow orders.
  \item
  A cabal of ten necromancers-in-training visit with bodies to feed the lich's army, in exchange for tutelage.
  They frequent local towns, and pick up the dead with a cart.
  \item
  The lich has complete plate armour for himself and any steed.
  \item
  When a basilisk attacked, he killed it, turning it into an undead steed.
  It remains inside the city, waiting to be summoned.
\end{dlist}

\end{multicols}

\section{Depth}

\begin{multicols}{2}

\subsubsection{Finishing the Map}

\noindent
Once you have each area detailed, you have a complete map.
But you can always add a few details.

\begin{itemize}
  \item
  Find a small section of your map, covering only two points, and make a new map based on it to give to your players.
  \item
  Draw contour-lines inside any lakes or sea.
  \item
  What are the roads' names?
  \item
  Do you have any points with civilization, but without a road connecting them to anywhere?
  What would the other place think of them?
  Would other have rumours about them?
  \item
  What gossip would travellers hear on the road about the local fiends?
  \item
  Follow a road leading off your map.
  Where does it go?

  Start a new map, and connect them by the roads which venture past the page.
  Repeat until you charter the world.
\end{itemize}

\bigLine

\subsubsection{Player Connections}

How are the \glspl{pc} connected to this land?
Make a numbered list of possible past events, with a default and non-default race for each entry.

\begin{itemize}
  \item
  If a human rolls a point where humans live, or a gnoll rolls a point where gnolls live, then they come from that place -- perhaps from a town, or \glspl{village} surrounding it.
  \item
  Otherwise, they may have a connection to a nearby fiend, living between the points.
  Perhaps the fiend destroyed their home and family.
  Perhaps they suffered an attack on the road which still haunts their dreams.
  \item
  Non-humans rolling a human town might have worked with whatever the town is famous for (step \vref{mapCharacter}) -- a gnome might have worked as a tracker, or come to the city to listen to its famous bards.
  Or perhaps a gnoll helped keep the great beasts in a town's \gls{court}.
  \item
  Humans rolling an elven settlement might have been raised there, or lived there for some time after fleeing the law.
\end{itemize}

When the players create their \glspl{pc}, have each one roll on your chart to determine their connection to the land.

\needspace{8em}
Your chart might look something like this:
\begin{enumerate}
  \item
  \underline{Horseshoe Valley}
  \begin{description}
    \item[\textbf{Humans:}] you grew up in a little \gls{village} by Horseshoe Valley, and want to seek your fortune in the wider world.
    \item[\textbf{Others:}] you came as a trader to purchase iron goods, but soon grew scared of travelling the road alone.
    Life in the \gls{guard} now seems safer than travelling with all those goods.
  \end{description}
  \item
  \underline{Elven Meadows}
  \begin{description}
    \item[\textbf{Elves:}]
    you come from the elven villages.
    While the elders gave their blessings, they made you swear never to bring outsiders back home.
    \item[\textbf{Others:}]
    the local hag -- `Thingizard' -- killed your family and tribe.
    You alone escaped, and now dream impotently of revenge.
  \end{description}
  \item
  And so on\ldots
\end{enumerate}

\end{multicols}

\section{Strange Places}

\begin{multicols}{2}

\subsection{Lonely Taverns}
\label{lonelyTaverns}
\index{Lonely Taverns}

These taverns exist on long stretches of road, far from any town, and
charge high prices for a drink. They must live off traders passing
through, and survive whatever the forest brings out.

Normal people don't stay for long.
Those who stay a while often have problems with the local
law, as these places often make their own laws.
Barkeeps punish any robbery close to the tavern harshly, but don't often care about what people do around the towns.
This makes these taverns a safe intermediate location where anyone can talk in peace.

Of course, bandits won't announce themselves as such when speaking with \glspl{guard}, but then the \glspl{guard} often won't announce their employment either, no matter how obvious that sword on their back makes them.

\subsubsection{The Barkeep}

Roll $1D6$ to find this season's barkeep (they change all the time):

\begin{dlist}
  \item
  A veteran of the \gls{guard}, with a hundred war-stories. Of course
  when he tells them, nobody can get a drink, so don't ask!
  \item
  A dozen people from point 4 of the map (though some may not wish to reveal where they come from).
  \item
  An outlander from a land so far away, nobody has ever heard of it.
  Every story she tells sounds made-up, but the strange accent shows she really does come from somewhere distant.
  \item
  A powerful mage who swore an oath never to use magic again.
  He won't say why.
  \item
  A collective -- you stay as long as you like, earn your keep, then go
  when you please. Sometimes in the colder Seasons, the place just lies
  barren.
  \item
  A dwarf who records all he can.
  The patrons say he works as a spy for someone, but they disagree about whom.
\end{dlist}

\subsubsection{The Menu}
\index{Menu}

Roll $3D6$ -- the \glspl{pc} can order any of these meals.

\newcommand\menuItem[3][(\arabic{r12} \glspl{cp})]{%
  \randomdozen%
  \randomthree%
  \randomfourB%
  \ifodd\value{enumi}
    \randomthreeC%
    \randomfour%
  \fi
  \item
  \textbf{#2:}
  #3
  #1
}

\begin{dlist}
  \menuItem{Griffin-wing}{freshly killed this morning, after the griffin tried to fly away with a gnomish patron.}
  \menuItem{Mystery-stew}{why are you hesitating?
  It goes rotten quick, so get eating!
  \footnote{Chitincrawler `meat' (webbing as sauce!)}
  }
    \ifnum\value{r4}<2
      \newcommand\morningSoup{uproot}
    \else
      \newcommand\morningSoup{marching_mushroom}
    \fi
    \menuItem{Sunrise Soup}{the chef found a new plant this morning, and he's already learning how to cook it!
    \footnote{In fact this is \nameref{\morningSoup}, see \autopageref{\morningSoup}, for the effects.}
    }
  \menuItem{Deer}{thank the man in black, sitting in the corner.}
  \ifodd\value{r4}
    \menuItem{Dwarf-beard}{actually just a type of seaweed, left as payment by a local trader; but it tastes just like the real thing!}
  \else
    \menuItem{Eye-Spy}{made with actual woodspy.}
  \fi
  \ifodd\value{r3}
    \menuItem[(0 \glspl{cp})]{Get bent}{the barkeep's in a foul mood, because they need a day off.}
  \else
    \menuItem[]{\ldots and bugger-all-else}{a few barrels turned out to be rotten, and now someone's stolen an entire pot of soup.
    The menu will be limited for the day.}
  \fi
\end{dlist}
\null

\subsection{Lost Cities}
\label{lostCities}
\index{Lost Cities}

\widePic[t]{Nelness/city}

People call them `\gls{yonder}'s house', as a reminder that most people who enter never return, but remain with \gls{yonder}.
People also call them `land \glspl{deep}', because these spaces act according to their own rules; they do not have the usual wandering monsters, but some predators always take residence in them, along with the fiend who ruined the city.

Before the \glspl{pc} enter, you should make some rolls to find out the nature of the town.
What creatures live there?
Which towers are still standing?
Dice to find out which tall towers still stand in the city, in steps \vrefrange{lostDwellers}{lostTowers}.

Once the \glspl{pc} enter, they can attempt to look for valuables, quietly, in steps \vrefrange{lostWhispers}{lostForaging}.

\begin{exampletext}
  The \gls{gm} knows the troupe are headed to the lost city, where an ogre stays with his goblin horde.

  First, she rolls up the inhabitants -- `\dicef{3} \dicef{3} \dicef{6}' indicates two city-dwellers: chitincrawlers and a coven of $1D3+1$ demiliches.
  One more roll shows that means a coven of 3 demiliches have made a little secluded spot for themselves in the lost city, unbothered by the goblins.
  Perhaps they use the chitincrawlers as a source of \glspl{ingredient}?

  Next, she rolls $4D6$ to find the towers, and finds `\dicef{2} \dicef{2} \dicef{6} \dicef{4}' -- that means that this lost city still has 3 tall towers standing.

  As the troupe enter, she describes the scene to the players.

  \begin{boxtext}
    The city seemed massive before, but here at the rotten gate, you can only see doorways and windows, mostly lying open, like black eyes.
    Thorny bushes with little multicoloured berries grow all down the street ahead, where the Sunlight strikes brightest.
    Dropping around them suggest deer frequently come here to feed.
  \end{boxtext}

  The troupe moves quietly towards the location their map claims the old guild halls once stood, by the market.
  But before hunting for loot, they roll \roll{Dexterity}{Stealth} (\tn[8]) to stay silent.
  The group roll succeeds, so they move quietly, then roll \roll{Intelligence}{Vigilance} (with a +3 Bonus for the map) at \tn[12].

  The first roll fails, so the \gls{gm} mentions that they can cover more ground by splitting up if they want.
  But \pgls{interval} has already passed while they hunted, and the Sun is beginning to set.

  The troupe is now fairly deep into the lost city, so they can wait in the dark, or navigate back out again, replacing the usual city encounters with the wandering monsters outside.
  They decide to stop foraging, and hide in an abandoned house, then wait to see what morning brings.

  During the night, the \gls{gm} must keep track of what the dwellers in the lost city get up to, so she describes the sounds of demiliches casting Death magic on chitincrawlers, so they can use their bodies to make \glspl{boon}.

\end{exampletext}

\mapentry{Finding the City}

It may seem difficult to lose a city, but when grass grows over the roads, and the wooden \glspl{village} crumble, cities can become almost invisible.
If the troupe only know about a city's location through legend, they will have to roll \roll{Intelligence}{Wyldcrafting} at \tn[14] to locate the city.

Cities which have fallen more recently often still have a visible road, and will not require any roll to find.

\mapentry{Wandering Dwellers}
\label{lostDwellers}

Each city has some population of the servants of the local fiend.
If a lich lives in the city, it has a lot of ghouls; if an ogre king lives here, it has goblins.

Additionally, lost cities attract a few other creatures which end up taking residence.

Roll $3D6$ -- each number which comes up determines one type of city-dweller.
Ignore any repeats.

\begin{dlist}
  \item
  \textbf{\Pgls{seeker} of the \gls{templeOfCuriosity}} has enlisted $1D6+4$ fully armoured \glspl{sunGuard} as part of a reconnaissance mission for \pgls{warden}.
  A camp sits nearby, with two more \glspl{sunGuard}, six donkeys, and 20 days' rations.

  The \gls{seeker} won't like anyone interfering with his business, so if he encounters the troupe, roll a morale check; on a pass, he tries to kill them.
  \item\label{lostOoze}
  \textbf{Oozes} of all types slide across the streets, which seem strangely clean, and free of debris or foliage.
  \item
  \textbf{Chitincrawlers} hide in every abode.
  Verdant berries of every colour have encouraged deer into the area, but every shadow gleams with thick webbing.

  $1D6+8$ live here in total.
  \item
  \textbf{Griffins} look down from every tower.
  The high towers make perfect perches to surveille the area, and the degrading wood helps to make nests.

  Each tower has 2 griffins.
  \item
  \textbf{Woodspies} have not only made this place their home, they have taken to imitating the items in the area, such as stools, piles of books, or chests.

  $1D6+4$ live here in total.
  \item\label{lostDemilich}
  \textbf{Demilich} covens help these undead sorcerers study with their own kind.
  However, their lack of basic empathy makes them dangerous to each other -- none of them really trust the others, so they share information slowly, always hinting that they have more to teach while masking their true abilities.
  Whenever one is wounded while another is not, they both attack each other (the first has spotted the right moment to slay a potential enemy, and the second knows what the first is thinking, as they all operate in the same way).

  Roll $1D3 + 1$ to determine the number of demiliches.
\end{dlist}

\mapentry{Tall Towers in the Labyrinth}
\label{lostTowers}

Wooden rooves rot, leaving bear stone walls across much of the fallen city.
Some walls collapse, opening new passages, while trees grow and thorny bushes grow to block doors and streets.
A twisted, mossy, labyrinth forms.

Despite the rot and chaos, some buildings still stand tall.
Once \pgls{pc} climbs to the top of a tower, two things happen:


\begin{itemize}
  \item
  The excellent perspective from the tower grants a +2 Bonus to foraging, as they spot good places to investigate (such as places which look like guild houses, or less-damaged areas of the city).

  This only works while the character can see, so it probably only works during daylight.
  \item
  Any undead in the city can clearly see the character, and will begin moving towards them.
  \item
  There is a 1-in-6 chance of seeing the dwellers of the city interact with each other (start lowest to highest).

  If the city has a curious \gls{seeker} and some acidic oozes, then the character may see the \gls{seeker} running away from the oozes.
  Or if the city has chitincrawlers and griffins, they may see a griffin caught in a web.
\end{itemize}

Climbing these buildings gifts a wide perspective of the city, which grants a +2~Bonus to the \glspl{pc}' foraging rolls.

Roll $4D6$ -- each number which comes up indicates a tower in the lost city.

\begin{dlist}
  \item
  This old \gls{templeOfPoison} had lavish beds up at the top, though the only current inhabitants are dead spiders and rat-droppings.

  The kitchen has a basement, filled with sealed clay pots.
  The wooden ladder has rotted, and will break once anyone of \gls{weight} 6 or more steps onto it.

  Anyone investigating the basement should make an \roll{Intelligence}{Vigilance} roll (\tn[10]).
  Success means they have found a secret stash of 20~\glspl{sp}, hidden in a honey pot.
  Failure means that a clay pot has gone rotten, and explodes; the character has contracted Corpse Hands (\vpageref{Corpse Hands}), and they will start to feel the effects after 2~\glspl{interval}.
  \item
  This tower has one wall remaining -- a \roll{Speed}{Athletics} check (\tn[8]) allows the climber to see a little in all four directions.
  \item
  This tower once held the \gls{sunGuard}, but now has a single servant of the local fiend (usually goblins or ghouls).
  If the \glspl{pc} do not take it by surprise, it will shout for aid (the undead can shout to other undead silently).
  \item
  This tower stands tall enough to see all around.
  The bones of dead humans fill the stairway.

  Any movement which disturbs the bones will send them falling down the stairs with a clank and a thud.
  \Glspl{pc} must roll \roll{Dexterity}{Stealth} (\tn[11]) to move past the bones without issue, or else waste \pgls{interval} moving everything down by hand (in this case they roll \roll{Dexterity}{Crafts}, \tn[7]).
  \item
  The old citadel still stands\ldots mostly.
  Parts of the original \gls{warden}'s home have rotten flooring, so moving through it requires a \roll{Wits}{Crafts} roll (\tn[8]) to notice weak areas of flooring, and the best way to walk.
  Failure indicate that the character falls two~\glspl{step}%
  \exRef{core}{Core Rules}{falling}
  and makes an awful sound.
  \begin{enumerate}
    \item
    On the second story, characters will encounter the city dweller with the lowest number.
    \item
    If anyone gets to third story (which requires three rolls), then they will find a cupboard with \lootBig.
  \end{enumerate}
  \item
  This tower holds an empty library, once part of \pgls{templeOfCuriosity}.
  It has no books, and little stability left.

  Upon entering, \glspl{pc} can make a \roll{Wits}{Crafts} roll to understand if this tower will hold their own weight (the rolls tells the \gls{pc} nothing about others).
  Anyone with \pgls{weight} above 5 prompts the structure to collapse, but only once they have reached the top.

  If the tower begins to collapse, anyone inside can run down (\roll{Speed}{Athletics}, \tn[9]), but failure indicates the building has collapsed upon them, inflicting $4D6$ Damage.

  Or the \gls{pc} can elect to jump, and try to grab something nearby (\roll{Speed}{Athletics}, \tn[12]), but failure indicates that they will suffer a nasty fall of $2D6$ Damage.
\end{dlist}

\bigLine

\mapentry{Creeping in the Cold Labyrinth}
\label{lostWhispers}

When the \glspl{pc} want to move through a lost city, they roll \roll{Dexterity}{Stealth} to remain quiet.
The \gls{tn} is normally 8, but often easier.

\begin{itemize}
  \item
  -1 if moving at night.
  \item
  -1 in the rain.
  \item
  -1 in a storm.
  \item
  -2 if the dwellers are busy with something.
  \item
  -4 if hiding indoors.
  \item
  +5 if lighting a fire.
\end{itemize}

The undead are a separate matter.
Even if the troupe move around silently, any active undead will spot them, unless they have some way of hiding from the sight of the dead.%
\footnote{See \autopageref{undead_senses}.}

\paragraph{When the roll fails,}
\label{lostChase}

Then they have an immediate encounter, in this order:

\begin{enumerate}
  \item
  The dwellers with the lowest number.

  \textit{For example, a city with oozes and griffins would mean that if the \glspl{pc} make a noise, then an acidic ooze wanders out of a house, and heads towards them.
  Then on the next failed check to creep around, a griffin would attack.}
  \item
  $2D6 + 8$ of the fiend's servants encounter them, and attack.

  \textit{Usually this means a lot of goblins, or a lot of ghouls.}
  \item
  The fiend itself has awakened, and begins planning an assault.
  They will probably not approach the \glspl{pc}, but will find a vantage point to see the \glspl{pc} from, and co\"ordinate attacks.
\end{enumerate}

\paragraph{Running away,}
requires a \roll{Speed}{Stealth} roll, at a \gls{tn} equal to the enemy's \roll{Speed}{Vigilance}.

Of course, characters may not want to run away, but fighting prompts noise, which means another encounter, and so on, until the city's fiend awakens.

\mapentry{Foraging Quietly}
\label{lostForaging}

Each time the \glspl{pc} forage, they roll \roll{Intelligence}{Vigilance} (\tn[12]) to figure out where the best loot lies.
If they succeed, roll $2D6$ -- the first die determines the place, and the second determines the prize!

\begin{itemize}
  \item
  Once they have a prize, cross it off the list.
  \item
  If they get the same number again, they will find the second prize with that number.
  \item
  The third time they roll a number, they find nothing -- lost cities only hold so many treasures.
  \item
  If the \glspl{pc} split up, each one can perform individual rolls, so they will loot far more efficiently, but run more chance of one getting caught alone.

\end{itemize}

\subsubsection{Climbing the Walls}

\Glspl{pc} might feel a temptation to climb the outer walls to get a proper perspective of the city, but doing so leaves them visible to every dweller within.
Move to `the Chase' immediately, but at \tn[12] to hide.

\end{multicols}

\foragingChart


\chapter[The Cartographers]{Cartography}
\label{civilization}

Fantasy stories have two special characters -- magic, and the landscape.
The audience discover both, just as they learn about the other characters.

Magic typically remains a mysterious character, but the landscape should always feel open.
As \pgls{gm}, your character is the landscape, and making a map is your character creation.

\section{Your Map}
\label{yourMap}

Welcome to \gls{fenestra}.
Let's look around.

\begin{multicols}{2}

\mapentry{Features of the Land}

Take a pencil, a blank sheet of paper, and four dice.

\begin{enumerate}
\item
  Imagine a circle spanning your page.
\item
  Draw 6 points along the circle, evenly spaced.
\item
  Label each point with numbers 1-6.
  All points go in a circular order, so after 6 comes 1 (until you roll a higher number).
\item
  Roll $1D6$ and draw mountains around the point with the number you rolled.
\item
  Draw hills outside of the mountains, stretching to the nearest two points.
\item
  Take the opposite point -- there lies the water.
  Roll $1D6$ to find out the type of watery area:
  \begin{dlist}
  \item
    Swamp.
  \item
    Lake.
  \item
    Large lake, which also covers the centre of the map.
    Add $1D3$ islands.
  \item
    Sea, extending to the edge of the map.
    Add $1D6-1$ islands.
  \item
    Sea, which extends to the centre of the map, then out to the edge of the map.
    Add $1D6$ islands.
  \item
    Vast sea. It stretches from this point, to the next numbered-point, and then out to the edge of the map.
    Add $1D6+1$ islands to your water.
  \end{dlist}
\end{enumerate}

\noindent
Forests cover all lands beyond the mountain, but do not draw them. Just
know that they cover everything.

Leave room to describe up to 12 points somewhere on your page.

\mapentry{Rivers}

Rivers lay the foundations of civilization.
They race from mountains to the sea, joining together with other rivers and becoming stronger as more join together.

Roll $4D6$, and draw a river flowing through each of those numbers, from the mountains to the water.
Ignore repeats.

\begin{itemize}
  \item
  Try starting with two or three rivers close together, then joining them before reaching their numbered location.
  \item
  Rivers going to different locations should always begin fairly far apart.
  \item
  Rivers never split apart, they only join.
  \item
  If the mountains received a river, place a lake there, then let it run towards the sea as usual.
\end{itemize}

\mapentry{Towns}
\index{Towns}

\begin{enumerate}
  \item
  Place a town on each point which has a river running through it.
  \begin{itemize}
  \item
    Draw a circle inside a circle to indicate the town.
  \item
    Towns on mountains indicate dwarves, in which case, place their symbol by the town: `\Dw'.
  \end{itemize}
  \item
  Draw a wavy flat-land area around each town.
  This indicates a civilized area, with guard towers and traps, where humans can walk outside their walls safely.
  \item
  For each town, add $1D6 \times 2$ \glspl{village} around it.
  \begin{itemize}
    \item
    Draw each one as a small circle.
    \item
    Most \glspl{village} sit next to water or rivers.
    \item
    \Glspl{village} love sitting on islands.
  \end{itemize}
  \item
  Draw a dotted-line road from each \gls{village} to a nearby \gls{village} its town.
  \begin{itemize}
    \item
    Where the roads cross water, draw a small bridge.
  \end{itemize}
\end{enumerate}

Leave the \glspl{village} alone for now.
Once the troupe enter one, you can discover its features in `\nameref{villageFeatures}' (\autopageref{villageFeatures}).

\paragraph{Town Character}
\label{mapCharacter}
\index{Towns}

To discover a town's most famous feature, roll $1D6$, and add a number equal to the total \glspl{village} surrounding it.
For example, if it has 4 \glspl{village}, roll $1D6+4$.

\begin{enumerate}
\setcounter{enumi}{2}
  \item
  Newly built -- every day, stone and wood comes in to build a little more each day.
  Occasionally, parts of the wall collapse and archers flock to protect it.
  \item
  Trackers
  \item
  Deadly and magical plants
  \item
  Ale, and the Guild has tight control over the area (\autopageref{god:Poison})
  \item
  White rock (actually limestone)
  \item
  \ifodd\value{r4}
    Wood-carvers
  \else
    Metallurgists
  \fi
  \item
 Bards
  \item
  Madness (neighbouring witches are to blame)
  \item
  Witchcraft
  \item
  \ifodd\value{r4}
  Friendly with the next town -- the locals go back and forth all the
    time.
  \else
    Cartographers
  \fi
  \item
  Statues of heroes and gods litter the roads around.
  \item
  A grand \gls{court} where the \gls{warden} finds beasts guilty, and has them fight for entertainment.
  \item
  Bath houses
  \item
  War brews with the next town.
  Soldiers will march soon\ldots

  Each one has 1D3 military outposts guarding against the other.
  \item
  `Demi-human suburbs', where all manner of other races live.
  \item
  The grandest library in the land.
\end{enumerate}

\paragraph{They call it\ldots{}}

Select a name by combining what everyone knows it for, with the type of
land.
Write the town's name next to it, and underline it.

\begin{itemize}
  \item
  People know a town in the hills for its library. That's why they call
  it `Page Valley'.
  \item
  A town by the sea, famed for Witchery.
  They call it `Hexwave'.
  \item
  `Bards' + `Islands' = `Lyrisles'
  \item
  \ifodd\value{r4}
    `Metallurgists' + `forest' = `Iron Basin' (perhaps a dried-up lake sits nearby)
  \else
    `Wood-carvers' + `forest' = `Taming Woods'%
  \fi
  \footnote{If you think these names sound stupid, you should look up the meaning of your hometown's name.}
\end{itemize}

Try some word-association with the town's theme or location.

\paragraph{It is Ruled by\ldots{}}
\index{Towns!Rulers}

For each town, roll $1D6$ to determine its ruler.
Write down the information on a key, referencing each point by its number.

\begin{dlist}
  \item
  Multiple \underline{warring factions} -- roll twice more, with +1 to the second roll.
  \item
  A \underline{Warlord}, amassing an army to expand their power.
  Shipments of weapons come from the road to other lands.
  \item
  An \underline{Opulent noble}, who demands 20\% tithes from all who enter.
  The local oddities annoy and fascinate them equally (from step \ref{mapOddities}).
  \item
  A new \underline{Sorcerer king} with no experience or business ruling.
  He once killed a fiend, how hard could balls and arranged marriages be?
  (roll $1D6$ to find which type of fiend, on step \ref{fiendTypes})
  \item
  A \underline{Spoilt noble} who has never seen a peasant.
  \item
  The \underline{servant of the next fiend} you roll (see step \ref{fiendTypes}).
  All who oppose them disappear on the road.
  \item
  A \underline{Crime-Lord}, with a history of theft, now risen to fortune\ldots but not nobility.

\end{dlist}

\mapentry{Lonely Roads}
\index{Roads}

\begin{enumerate}
  \item
  Draw a road from every town, to every other town.
  These roads wind through any \glspl{village} along the way, but never through the outer points (1--6).
  \begin{itemize}
    \item
    Where the roads cross water, draw a small bridge.
    \item
    If the route enters water, draw a \glspl{village} port at the border.
  \end{itemize}
  \item
  Find a free space for the scale, and draw a line which shows `30 miles'.
  This line should be as long as the distance from an outer point, to the centre point.
  \item
  On your last road place a Lonely Tavern in the middle (see \autopageref{lonelyTaverns}).
  Mark it as point `7'.
  \item
  Every road over 10 miles long should have \pgls{broch}, or at least \pgls{bothy}.
  People rely on these towers and little stone houses to keep themselves and their horses safe at night.
  Only the \gls{guard} and the insane ever sleep outside.
\end{enumerate}

\mapentry{Oddities}
\label{mapOddities}
\index{Oddities}

\begin{enumerate}
  \item
  Mark the centre of your map as point `8'.
  \item
  At every point without a town or Tavern, place a hidden element (roll $1D6$ below).
\end{enumerate}

\begin{multicols}{2}
\begin{dlist}\raggedright
\item
  1D3 Elven glades, nestled close (\El)
\item
  Goblin warren in an abandoned mine. (\N)

  Draw a road leading half-way to it, from the nearest road.
  This road now lies abandoned, as the forest slowly eats it.
\item
  $1D6$ Gnomish warrens, arranged in a perfect geometric alignment (\Gn)
  \columnbreak
\item
  Witch commune (\Hu)
\item
  $1D6$ Gnoll tribes (\Nl)
\item
  Lost city (\D) (\autopageref{lostCities})

  If you roll a lost city somewhere without water, draw a new river coming from outside the map, going through the lost city, then meeting with the closest water as it winds downhill.
\end{dlist}
\end{multicols}

\mapentry{Fiends in the Forest}
\label{mapFiends}
\index{Fiends}

Dangerous people, and stranger creatures, live in the forests, controlling large territories.

\begin{enumerate}
\item
  Roll $3D6$: each point which comes up has one fiend between it and the
  next point.
\item
  Place the fiend on your map with the next number available, starting with `9'.
\item
  For each fiend, roll three dice, to determine its type, wishes, and
  ability.
\end{enumerate}

\paragraph{Who are They?}
\label{fiendTypes}

Roll $1D6$ to identify the fiend, then add this number to the fiend's next two rolls.

\begin{multicols}{2}
\begin{dlist}
\item
  Bandit troupe (\autopageref{bandit})
\item
  Lich (\autopageref{lich})
\item
  Dragon (\autopageref{dragon})
\item
  Dryad (\autopageref{dryad})
\item
  Ogre king (\autopageref{ogre})
\item
  Hag (\autopageref{hag})
\end{dlist}
\end{multicols}

Write the fiend's name on the map, and underline it.

\paragraph{The Fiend Has}

($1D6$ plus fiend's number)

\begin{enumerate}
\setcounter{enumi}{1}
\item
  gained the sympathies of some local villagers, who will help them.
\item
  many servants across the land -- roll a $D6 + 1$ four times, and place a hidden camp in each number which comes up.
  Each one has a dozen soldiers.
  Mark the camps with an `X'.
\item
  made a temporary alliance with the occupant of the previous point (i.e. the fiend's point-number minus 1).
\item
  imprisoned a gnomish alchemist, and can force him to cast spells.
\item
  complete plate armour!
\item
  a powerful magical item.
\item
  myriad tunnels underground: currently empty, mostly. But nobody knows
  the full extent of them, only that myriad openings exist, most of
  which have been covered by a shallow layer of topsoil.
\item
  a small fortress on an island. No one can approach safely, or without
  being noticed.

  (draw a lake, if necessary)
\item
  a powerful magical item.
\item
  cast a spell which forces those who venture close to forget what they
  came for.
\item
  a garden which grows all manner of magical and deadly plants.
\end{enumerate}

\paragraph{The Fiend Wants}
\label{fiendDesires}

(1D6 plus fiend's number)

\begin{enumerate}
\setcounter{enumi}{1}
\item
  to return to society, without giving up their gold.
\item
  to cement an alliance with those on the previously-numbered point.
\item
  something beautiful to look at.
\item
  standard medical equipment.
\item
  to kill those on the previously-numbered point.
\item
  standard supplies from \glspl{village}.
  It's so hard to find someone who delivers!
\item
  to destroy the nearby oddities (descend numbers until reaching an oddity from step \ref{mapOddities}).
\item
  a massive pot, three bags of thyme, a gnomish cook, and four fresh men.
\item
  to acquire magical ingredients.
\item
  to watch birds in peace (do not make noise on the road).
\item
  a child to raise as her own.
\end{enumerate}

\mapentry{The Wider World}

The area needs some relation to the outside world -- at least a name, and a route out.

\begin{enumerate}
  \item
  Combine the fiend with the highest number, with the local water-type.
  For example:
  \begin{itemize}
  \item
    `Hag' + `lake' = `Haglake'
  \item
    `Dryad' + `islands' = `Drylands'
  \item
    `Lich' + `sea' = `Portlich' (stress on the first syllable!)
  \end{itemize}

  \item
  Roll $2D6$; each point you roll has a route leading off the map, to lands uncharted.
  Draw the routes with a dotted line.
  If they do not lead towards human settlements, no problem.
\end{enumerate}

\mapentry{Depth}

Your map has everything it needs, but you could always add more.

\begin{itemize}
  \item
  Place a settlement on any uninhabited islands, and connect them to the nearest settlement.
  Is it inhabited by humans, or something else?

  Draw sea-routes to anywhere people might easily sail.
  \item
  Draw contour-lines around any lakes or sea.
  \item
  Think about what the local populations say about the local oddities from step \ref{mapOddities} -- do they know what lives there?
  \item
  What gossip would travellers hear on the road about local fiends and oddities?
  \item
  Fill in the missing names.
  Human rulers typically take their town's name as their last name, so someone in charge of a town called `Palemarsh' might be called  `\Gls{warden} Cartpike Palemarsh', or similar.
  \item
  What are the roads' names?
  Do you have any room left to write them down?
  \item
  Pull out a pen and start adding depth and shading to the map.
  \item
  Find a small section of your map, covering only two points, and make a new map based on it to give to your players.
  \item
  Follow a road leading off your map.
  Where does it go?

  Start a new map, and connect them by the roads which venture past the page.
  Repeat until you charter the world.
\end{itemize}

\bigLine

\subsubsection{Player Connections}

How are the \glspl{pc} connected to this land?
Make a numbered list of possible past events, with a default and non-default race for each entry.

\begin{itemize}
  \item
  If a human rolls a point where humans live, or a gnoll rolls a point where gnolls live, then they come from that place -- perhaps from a town, or \glspl{village} surrounding it.
  \item
  Otherwise, they may have a connection to a nearby fiend, living between the points.
  Perhaps the fiend destroyed their home and family.
  Perhaps they suffered an attack on the road which still haunts their dreams.
  \item
  Non-humans rolling a human town might have worked with whatever the town is famous for (step \vref{mapCharacter}) -- a gnome might have worked as a tracker, or come to the city to listen to its famous bards.
  Or perhaps a gnoll helped keep the great beasts in a town's \gls{court}.
  \item
  Humans rolling an elven settlement might have been raised there, or lived there for some time after fleeing the law.
\end{itemize}

When the players create their \glspl{pc}, have each one roll on your chart to determine their connection to the land.

\needspace{8em}
Your chart might look something like this:
\begin{enumerate}
  \item
  \underline{Horseshoe Valley}
  \begin{description}
    \item[\textbf{Humans:}] you grew up in a little \gls{village} by Horseshoe Valley, and want to seek your fortune in the wider world.
    \item[\textbf{Others:}] you came as a trader to purchase iron goods, but soon grew scared of travelling the road alone.
    Life in the \gls{guard} now seems safer than travelling with all those goods.
  \end{description}
  \item
  \underline{Elven Meadows}
  \begin{description}
    \item[\textbf{Elves:}]
    you come from the elven villages.
    While the elders gave their blessings, they made you swear never to bring outsiders back home.
    \item[\textbf{Others:}]
    the local hag -- `Thingizard' -- killed your family and tribe.
    You alone escaped, and now dream impotently of revenge.
  \end{description}
  \item
  And so on\ldots
\end{enumerate}

\end{multicols}

\section{Strange Places}

\begin{multicols}{2}

\subsection{Lonely Taverns}
\label{lonelyTaverns}
\index{Lonely Tavern}

These taverns exist on long stretches of road, far from any town, and
charge high prices for a drink. They must live off traders passing
through, and survive whatever the forest brings out. Normal people don't
stay for long. Those who stay a while often have problems with the local
law, as these places often make their own laws. Barkeeps punish any
robbery close to the tavern harshly, but don't often care about what
people do around the towns.

\mapentry{The Barkeep}

Roll $1D6$ to find this season's barkeep (they change all the time):

\begin{dlist}
  \item
  A veteran of the \gls{guard}, with a hundred war-stories. Of course
  when he tells them, nobody can get a drink, so don't ask!
  \item
  Someone from one of the oddities (step \ref{mapOddities}) -- perhaps a gnome; or someone descended from a nearby lost city, who speaks about plans to find it, and return.
  \item
  An outlander from a land so far away, nobody has ever heard of it.
  Every story she tells sounds made-up, but the strange accent shows she really does come from somewhere distant.
  \item
  A powerful mage who swore an oath never to use magic again.
  He won't say why.
  \item
  A collective -- you stay as long as you like, earn your keep, then go
  when you please. Sometimes in the colder Seasons, the place just lies
  barren.
  \item
  A dwarf who records all he can.
  The patrons say he works as a spy for someone, but they disagree about whom.
\end{dlist}

\mapentry{The Menu}
\index{Menu}

Roll $3D6$ -- the \glspl{pc} can order any of these meals.

\newcommand\menuItem[3][(\arabic{r12} \glspl{cp})]{%
  \randomdozen%
  \randomthree%
  \randomfourB%
  \ifodd\value{enumi}
    \randomthreeC%
    \randomfour%
  \fi
  \item
  \textbf{#2:}
  #3
  #1
}

\begin{dlist}
  \menuItem{Griffin-wing}{freshly killed this morning, after the griffin tried to fly away with a gnomish patron.}
  \menuItem{Mystery-stew}{why are you hesitating?
  It goes rotten quick, so get eating!
  \footnote{Chitincrawler `meat' (webbing as sauce!)}
  }
    \ifnum\value{r4}<2
      \newcommand\morningSoup{uproot}
    \else
      \newcommand\morningSoup{marching_mushroom}
    \fi
    \menuItem{Sunrise Soup}{the chef found a new plant this morning, and he's already learning how to cook it!
    \footnote{In fact this is \nameref{\morningSoup}, see \autopageref{\morningSoup}, for the effects.}
    }
  \menuItem{Deer}{thank the man in black, sitting in the corner.}
  \ifodd\value{r4}
    \menuItem{Dwarf-beard}{actually just a type of seaweed, left as payment by a local trader; but it tastes just like the real thing!}
  \else
    \menuItem{Eye-Spy}{made with actual woodspy.}
  \fi
  \ifodd\value{r3}
    \menuItem[(0 \glspl{cp})]{Get bent}{the barkeep's in a foul mood, because they need a day off.}
  \else
    \menuItem[]{\ldots and bugger-all-else}{a few barrels turned out to be rotten, and now someone's stolen an entire pot of soup.
    The menu will be limited for the day.}
  \fi
\end{dlist}
\null

\subsection{Lost Cities}
\label{lostCities}
\index{Lost Cities}

\widePic[t]{Nelness/city}

When calamities destroy a city, some leave quickly, and beasts eat the rest.
The beasts often remain long-term, even for generations, which creates a tantalizing trap for the greedy and curious.
Some treasures always remain in lost cities -- coinage, statues, weapons, and secrets.
Every city has secrets.

People call them \gls{yonder}'s house, as a reminder that most people who enter never return, but remain with \gls{yonder}.
People call them `land \glspl{deep}', to remind the young how deadly an apparently empty city becomes.
Strange creatures always take residence in them.

Before the \glspl{pc} enter, roll dice to make the town with steps \vrefrange{lostCataclysm}{lostTowers}.
Once the \glspl{pc} enter, two rolls occur each \gls{interval}.

\begin{enumerate}
  \item
  You roll to see when dweller interactions occur next.
  During this time, the \glspl{pc} can sneak about more easily.
  (\nameref{lostCries})
  \item
  The \glspl{pc} then try to move quietly through the city, without drawing the attention of those who dwell there.
  (\nameref{lostWhispers})
  \begin{itemize}
    \item
    If the \glspl{pc} fail to sneak, the chase is on!
    (\nameref{lostChase})
    \item
    If the \glspl{pc} go unnoticed, they can attempt to forage.
    (\nameref{lostForaging})
  \end{itemize}
  Most \glspl{pc} will struggle with foraging, but they can get better at it by climbing the towers from step \ref{lostTowers}, or splitting up and making individual rolls.
\end{enumerate}

\mapentry{The Cataclysm}
\label{lostCataclysm}

Everyone fled the city long ago.
The neighbouring towns don't always remember exactly where the city lies, but they always remember the tale of how it fell.

\begin{dlist}
  \item
  They burnt the wrong witch, and her enraged sister came to the city.
  She made plants grow between every brick, and pulled down hailstones in the shape of carrots.
  They killed her in the end, but the walls lay cracked beyond repair, and the central citadel had fallen.
  \item
  A fire started.
  As people banded together to walk outside and escape the smoke, hungry residents of the forest began to watch them, and pick them off, bit by bit.
  Some stayed in the nearby \glspl{village}, but they could not keep everyone inside, so the rest tried to take boats or walk away.

  Many who remained died of smoke inhalation.
  The rest found that walls without guards do little good, as the fire died out, and beasts began to creep in.

  By the time some inhabitants dared to return to the blackened rocks of their city, the forest had claimed the town as its own.
  Then the city's new dwellers grew hungry, and one day the \glspl{village} disappeared.
  \item
  A dragon came, full of hatred and hunger.
  They say it still sleeps somewhere in the city, but who knows?
  \item
  Underground tunnels opened, and little goblins popped up.
  Others followed soon, and one day, not long after, a horde ascended from the \gls{deep} to feast upon the city.
  They've long-since left, but long tunnels remain, going down to the \gls{deep}.
  \item
  Foolishness and bad luck lead to an unlikely series of calamities.
  The local \glspl{warden} began a war with another civilization, and lost.
  With fewer soldiers about, people suffered more casualties from the forest.
  Less food from the \glspl{village} lead to theft, and the \glspl{warden} demanded executions.

  When the city's fighting-beasts escaped their cages, and fled from the arena through the streets, people decided they could take no more, and many left.
  With a reduced population, limited food, and walls too long to properly guard, the city collapsed in on itself, bit by bit.
  \item
  \ifodd\value{r4}
    Local witches had warned not to expand the \glspl{village} -- the nearby \glspl{village} had strange ideas, and too many developed sorcery of some kind, when the secrets were whispered to them in dreams.

    No great even happened here -- the population simply shrank more than it grew, with a thousand little catastrophes.

    Some say spirits haunt the place.
    Others warn not to drink the river water.
    Whatever the truth, that land is cursed.
  \else
    A flood wiped away half the \glspl{village} and collapsed the town's up-river wall.
    Survivors crowded around rooftops to the see the architect hanged, as the streets had all turned to rivers.

    By nightfall, woodspies were picking people off their houses like a casual snack.
    Soon, the remainder left.
    The local \glspl{warden} planned to rebuild the outer wall, but could never raise the money to start rebuilding.
  \fi
\end{dlist}

\mapentry{Wandering Dwellers}

Roll $3D6$ -- each number which comes up determines one type of city-dweller.
Ignore any repeats.

\begin{dlist}
  \item\label{lostOoze}
  \textbf{Oozes} of all types slide across the streets, which seem strangely clean, and free of debris or foliage.
  \item
  \textbf{Chitincrawlers} hide in every abode.
  Verdant berries of every colour have encouraged various deer into the area, but every shadow gleams with thick webbing.

  $1D6+8$ live here here in total.
  \item
  \textbf{Griffins} look down from every tower.
  The high towers make perfect perches to surveille the area, and the degrading wood helps to make nests.

  $1D6+4$ live here in total.
  \item
  \textbf{Crazed Witches} occasionally enter these cities, looking for spell components.

  Roll $1D6$ to find their number.
  \item\label{lostDemilich}
  \textbf{Demilich} covens help these undead sorcerers study with their own kind.
  However, their lack of basic empathy makes them dangerous to each other -- none of them really trust the others, so they share information slowly, always hinting that they have more to teach while masking their true abilities.

  Roll $1D3 + 1$ to determine the number of demiliches.
  \item
  \textbf{Dragon} eggs make for powerful magical items, so many dragons like nesting in strange and dangerous areas.
  Of course, only the largest of buildings, such as town halls, or theatres, can house such a massive creature.
\end{dlist}

\mapentry{Dweller Relations}
Each dweller in a lost city has a relationship with the others, depending on their relative numbers.
(For example, oozes are \ref{lostOoze}, and the demilich is \ref{lostDemilich}, so their relation is `\ref{lostDemilich} minus \ref{lostOoze}'.)

Those with a higher number control or prey on those with a lower number.

\begin{enumerate}
  \item
  Dwellers within \arabic{enumi} step of each other become allies if sentient, and otherwise ignore each other.
  \item
  A dweller numbered \arabic{enumi} greater than another becomes aggressive, and the two begin to fight.
  \item
  Dwellers \arabic{enumi} ahead of another kill the lower form, and use the bodies to feed, or cast spells.
  \item
  Dwellers \arabic{enumi} ahead of another use magic or threats to control the lesser dweller.
  \item
  Any dweller \arabic{enumi} ahead of another will ignore them entirely.
\end{enumerate}

Draw a line, triangle, or square, to show the relations of the dwellers in the lost city.

\cityRelations

\mapentry{Tall Towers in the Labyrinth}
\label{lostTowers}

Wooden rooves rot, leaving bear stone walls across much of the fallen city.
Some walls collapse, opening new passages, while trees grow and thorny bushes grow to block doors and streets.
A twisted, mossy, labyrinth forms.

Despite the rot and chaos, some buildings still stand tall.
Climbing these buildings gifts a wide perspective of the city, which grants a bonus to the \glspl{pc} foraging rolls.

Roll $4D6$: one for each tower.

\begin{dlist}
  \item
  This tower has fallen to the ground, leaving nothing worth climbing.
  \item
  This tower has one wall remaining -- a \roll{Speed}{Athletics} check (\tn[8]) allows the climber to see a little in all four directions.
  \begin{itemize}
    \item
    During daylight, the perspective adds +1 to all foraging rolls.
  \end{itemize}
  \item
  This tower has a single candle, and the light is clearly visible at night.
  If anyone enters, any sentient city-dwellers in the area will see them blocking the light when they enter the room, and investigate.
  (If there are many, pick the lowest numbered dweller)
  \begin{itemize}
    \item
    If no sentient life exists in the area, the candle has been left by a traveller who has left a note stating which creatures he has seen.
    \item
    All further towers on the roll of a \arabic{dlist} create an encounter with the dwellers with the lowest number.
    \item
    During daylight, the perspective adds +1 to all foraging rolls.
  \end{itemize}
  \item
  This tower stands tall enough to see all around.
  The bones of dead humans fill the stairway.

  Any movement which disturbs the bones will send them falling down the stairs with a clank and a thud.
  \Glspl{pc} must roll \roll{Dexterity}{Stealth} (\tn[11]) to move past the bones without issue, or else waste \pgls{interval} moving everything down by hand (in this case they roll \roll{Dexterity}{Crafts}, \tn[7]).
  \begin{itemize}
    \item
    During daylight, the perspective adds +2 to all foraging rolls.
  \end{itemize}
  \item
  The tallest of towers allow anyone inside to see all around.
  They can receive a complete list of towers in the area.
  \begin{itemize}
    \item
    During daylight, the perspective adds +2 to all foraging rolls.
    \item
    The first time \pgls{pc} climbs to this height, they see two dwellers interacting.
  \end{itemize}
  \item
  The tallest towers often have the least stability.
  The \gls{pc} who climbs this far will receive the same benefits as above, but if their \gls{weight} is above 5, the structure begins collapsing.

  They can run down (\roll{Speed}{Athletics}, \tn[9]), but failure indicates the building has collapsed upon them, inflicting $4D6$ Damage.

  Or the \gls{pc} can elect to jump, and try to grab something nearby (\roll{Speed}{Athletics}, \tn[12]), but failure indicates that they will suffer a nasty fall of $2D6$ Damage.
\end{dlist}

\bigLine

\subsubsection{Cries in the Night}
\label{lostCries}

When dwellers interact, they often make noise.
Witches casting spells upon local oozes, or dragons ordering minions about can't be done quietly; although the \glspl{pc} will probably not understand what the noises mean.

Every $1D6$ \glspl{interval}, the \glspl{pc} hear an interaction.

\subsubsection{Whispers in the Cold Labyrinth}
\label{lostWhispers}

When the \glspl{pc} want to move through a lost city, they test \roll{Dexterity}{Stealth} to remain quiet.
The \gls{tn} is normally 8, but often easier.

\begin{itemize}
  \item
  -1 if moving at night.
  \item
  -1 in the rain.
  \item
  -1 in a storm.
  \item
  -2 if the dwellers are busy with something.
  \item
  -4 if hiding indoors.
  \item
  +5 if lighting a fire.
\end{itemize}

\subsubsection{The Chase}
\label{lostChase}

Any time the \glspl{pc} accidentally make a noise, they can try to gingerly run and hide.

\begin{itemize}
  \item
  They can try to escape without making more noise with another \roll{Speed}{Stealth} check, this time at \tn[10].
  \item
  The dweller with the lowest number arrives the first time the \glspl{pc} fail a check, then the next dweller on the list, and so on.
  \item
  Once all dwellers have been alerted, the first two arrive to investigate the noise at the same time.
  From then on, all dwellers come to investigate at every noise.
  \item
  Up to $1D3$ dwellers emerge at a time.
\end{itemize}

\subsubsection{Foraging Quietly}
\label{lostForaging}

Each time the \glspl{pc} forage, they roll \roll{Intelligence}{Vigilance} (\tn[12]) to figure out where the best loot lies.
If they succeed, roll $2D6$ -- the first die determines the place, and the second determines the prize!

\begin{itemize}
  \item
  Once they have a prize, cross it off the list.
  \item
  If they get the same number again, they will find the second prize with that number.
  \item
  The third time they roll a number, they find nothing -- lost cities only hold so many treasures.
  \item
  If the \glspl{pc} split up, each one can perform individual rolls, so they will loot far more efficiently, but run more chance of one getting caught alone.

\end{itemize}

\subsubsection{Climbing the Walls}

\Glspl{pc} might feel a temptation to climb the outer walls to get a proper perspective of the city, but doing so leaves them visible to every dweller within.
Move to `the Chase' immediately, but at \tn[12] to hide.

\end{multicols}

\foragingChart


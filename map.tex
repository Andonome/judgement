\chapter[The Cartographers]{Cartography}
\label{civilization}

Welcome to \gls{fenestra}.
Let's look around.

\section{Rivers \& Roads}
\label{yourMap}

\begin{multicols}{2}

\noindent
Fantasy stories have two special characters -- magic, and the landscape.
The audience discover both, just as they learn about the other characters.

Magic typically remains a mysterious character, but the landscape should always reveal itself quickly.
As \pgls{gm}, your character is the landscape, and making a map is your character creation.

\mapentry{The Pentacle}
provides the map's layout.
You need a pencil, blade, and dice.
Remove the example map on the left to start.

Roll $1D6$ for each of the 10 points, and add the point's number to the dice roll to find the entry for that point on your map.

This should feel a little sloppy.
You might just roll $10D6$ all at once, and then pick them in left-to-right order.
The map can hold all the information you will roll, as long as you have very fine handwriting -- otherwise, a space at the bottom of the page, or a new page to keep track of each point might work better.

\begin{multicols}{2}

\begin{enumerate}
  \raggedright
  \setcounter{enumi}{1}
  \item
  (\Hu)~Humans
  (\autopageref{humanPoint})
  \item
  (D)~Dragon!
  (\autopageref{dragonPoint})
  \item
  (B\Hu)~Bandits
  (\autopageref{banditsPoint})
  \item
  (\Hu)~Humans
  \item
  (\Hu)~Humans
  \item
  (\Hu)~Humans
  \item
  (\El)~Elves
  (\autopageref{elvesPoint})
  \item
  (\Dw)~Dwarves
  (\autopageref{dwarvenPoint})
  \item
  (\Gn)~Gnomes
  (\autopageref{gnomePoint})
  \item
  (D\El)~Dryad
  (\autopageref{dryadPoint})
  \item
  (\N)~Ogre
  (\autopageref{ogrePoint})
  \item
  (\Nl)~Gnolls
  (\autopageref{gnollPoint})
  \item
  (H)~Hag
  (\autopageref{hagPoint})
  \item
  (\D)~Lich
  (\autopageref{lichPoint})
  \item
  (\Hu)~Humans
\end{enumerate}

\end{multicols}

\begin{exampletext}
  For example, if your first die reads `\dicef{4}', then the result is $4+1 = 5$, and you can write `humans', or just `\Hu' on the map.

  If the next roll is `\dicef{6}', then you would write `elves' or `\El' next to point 2 on the map.
\end{exampletext}

Each dot has four neighbours.
The highest neighbour is the neighbour with the highest number, and the lowest neighbour, is the neighbour with the lowest number.

\mapentry{Rivers}
come from mountains, then flow to the sea.
If your map has dwarves or a dragon, then it has mountains.
Use the steps below to make rivers.

If the map has no dragons nor dwarves, then it has no mountains.
Use the steps \vpageref{mapSea}.

\begin{enumerate}
  \item
  Draw a mountain range around each point which has dwarves or a dragon.
  \item
  If two mountains share a neighbour, then that point must be a valley.
  Draw a few hills along the valley.
  \item
  At each mountain, draw a river to each of the four neighbours, unless that neighbour has a mountain.

  Rivers always begin from separate mountain-locations.
  They might later join together, but rivers never split apart.

  Draw your rivers with wavy, lazy, lines.
  \item
  Each river flows down hill, towards the lowest neighbour.
  They avoid mountains and avoid doubling-back to a point they've already been at.
  \item
  Rivers begin as a single line, but when two rivers join, they flow faster and wider.

  Draw the wide river with a \emph{thick} line.
  \item
  If three or more rivers join together, they form a canyon which floods over the stormy seasons, and recedes during the cold seasons.

  Draw the canyon with a parallel~line.
  \item
  If four or more rivers join together, they form a lake, which exits towards the lowest neighbour as another thick line (and counts as two rivers).
\end{enumerate}

\paragraph{Without mountains,}
\label{mapSea}
small rivers make their way to the sea.

\begin{enumerate}
  \item
  Points 1 and 2 become islands in a sea.
  \item
  The coast covers points 5, 6, and 7, then wanders off the map.
  \item
  Some well-springs open up on point 10, then join together to form a river.
  The river goes down through points 4, and 3, then goes towards point 2 and enters the sea.
\end{enumerate}

\paragraph{If the highest point has no water,}
it becomes a swamp.

\mapentry{Roads}
\index{Roads}

At each human settlement, draw $1D6$ \glspl{broch}.
They form a protective barrier around rivers, so that \glspl{village} can sit between them.
They sit around 3 to 10 miles apart, and tend to form defensive triangles.

\begin{enumerate}
  \item
  If a settlement has 3 or more \glspl{broch}, place a town between them, on a river.
  \item
  Draw a road from each \gls{broch} to the nearest river.
  \item
  Connect each human settlement to \textit{every} other settlement by a long road.
  \begin{itemize}
  	\item
    If a road crosses a small river, draw a small bridge.
    If it crosses a larger river, drawn with two lines, draw a larger bridge.
    If a canyon blocks a road, then the road must go around instead.
  \end{itemize}
  \item
  The human settlement with the highest number has a road leading out of the map, towards the closest border.
  \label{roadOut}
  \item
  If a human settlement has dwarvish neighbours, extend an existing road to the dwarvish neighbour with the highest number.
  \item
  Label some roads with their length in miles.
  \begin{itemize}
    \item
    The short distance between points 6 and 7 is about 20 miles.
    \item
    The medium distance between points 1 and 6 is about 30 miles.
    \item
    The long distance between points 1 and 2 is about 50 miles.
  \end{itemize}
  \item
  If any road stretches for 50 miles or more, place a Lonely Tavern half-way along that road (details \vpageref{lonelyTaverns}).
  \item
  The road between points takes a long time to walk.
  Put \pgls{bothy} on the road for every 20 miles of distance.
\end{enumerate}


\bigLine

\paragraph{Do not draw trees on the map,}
just know that they cover everything.
Drawing a few trees or a little forest suggests that everywhere else has open plains, but open areas are rare in \gls{fenestra}.
A twilight of towering trees covers almost everything which people have not cut down.

\paragraph{The map's name}
should come from a distinctive feature.
A good starting point is the most prolific feature of the map.
For example, a map with a swamp and three elvish settlements ranges might receive the name `Faebog', or one with two dragons and a lot of lakes might receive `Lochscale'.

\end{multicols}

\section{Zooming In}
\label{mapCharacter}

\begin{multicols}{2}

\noindent
Each point on the map needs more detail -- but only once the \glspl{pc} arrive.
The \gls{campaign} begins on the lowest human settlement.%
\footnote{If there are no human settlements on your map, you have found a rare place in \glsentrytext{fenestra}.
The troupe are likely lost, and will need to return to civilization, or significantly adjust their career expectations!}
Create a few details with the random lists \vpageref{humanPoint}.

Whenever you find out a new location's details, note them down as a key to the main map, and check if you can add any features to the main map.
Alternatively, you can start a smaller map of this area, to give yourself more space.

\subsection{Civilization}

The civilized lands have people of all types.
Elves, humans, gnolls, dwarves, or gnomes -- any place with a population that either sustains itself, or trades with others, counts as `civilization'.

\subsubsection[Dwarven Settlements]{\Dw\ Dwarven Settlements}
\label{dwarvenPoint}
begin with 100 dwarves, and a mushroom garden to sustain them.
Roll $3D6$, then combine each pair of 2 dice to find each unique feature.
Whenever you roll 8 or more, add 100 dwarves to the population.

\begin{exampletext}
  For example, on the roll `\dicef{2} \dicef{4} \dicef{3}', you would use the results for:

  \begin{itemize}
    \item
    $2+4 = 6$
    \item
    $4+3 = 7$
    \item
    $3+2 = 5$
  \end{itemize}

  The unique results are `5', `6', and `7'.

  \labelledDiceTrio{2}{4}{3}{Magma Stream}{Tin Seam}{Polite Passage}
\end{exampletext}

\begin{enumerate}
  \stepcounter{enumi}
  \item
  A king rules this population.
  Dwarves consider this bad luck, but the king promises to have a daughter and let her take his place in the next generation.%
  \exRef{stories}{Stories}{dwarven_structure}
  \item
  Gold seams run through the mountain, allowing the dwarves to make precious items, and to trade with overland peoples.
  \item
  A Hall of Records stands at the base of this mountain, guarded by heavy doors.
  It works a little like a library, but only contains factual accounts of the things dwarves take interest in, such as facts about rocks, the lineage of queens, descriptions of how to farm on mountains, and techniques to measure altitude.

  Each book has a minimum of three seals, pressed a dwarven record-keeper, who has verified the authenticity of every statement in the book personally.
  \item
  The `Polite Passage', is the name dwarves use for the long, thin, bridges they create, with only room for a single dwarf to walk at a time.
  Settlements have one for entering, and another for exiting.

  Larger rooms around the mountain's exterior store wide equipment, such as wagons.
  \item
  Magma streams in the mountain's heart flow eternally, and mix with underground rivers, producing a great plume of steam at the top.
  Some tunnels contain steam-traps, where a metal grate will ring the `dinner bell', and allow hot steam to flow in to cook the intruding predator.
  \item
  A tin seam allows dwarves to mine, and trade with nearby settlements.
  \item
  In the dark \gls{deep} below the mountain, extensive underground tunnels, filled with fungi, moss, living oozes, skein, and umber hulks.

  With enough traps laid out for the monsters, this creates a kind of rough garden, or `hunting ground' for the dwarves to gather additional food.
  It also provides dyes, which dwarves use to turn their beards and hair green, yellow, or blue, depending on their wealth and gender.\index[Dye]
  \item
  $1D6$ farms extend outside the mountains, across the fertile lands at the base.
  The dwarves create them with high stone walls, and polish each brick until it has nothing to hold onto, and no way to climb up.

  Inside, may allow as much space as any \gls{village} to grow vegetables for the mountain.
  \item
  An elevated barley garden around a third of the way up the mountain, provides food for the settlement.
  Nobody can see the gardens from below, as it grows on steps carved into the mountain.

  A series of rock-bells will sound when anything steps on them, but do not ring in response to the wind.
  \item
  An iron seam allows dwarves to create quality weapons and armour.
  If they have no road out, then they only make rare, wholesale trades.
  \item
  Half way up the mountain, shepherds take sheep and goats on long walks, then put them safely away at night, to sleep in stone rooms.
\end{enumerate}

\subsubsection[Elven Grounds]{\El\ Elven Grounds}
\label{elvesPoint}
centre themselves around a couple of powerful spellcasters who provide safety and often food.
Roll $3D6$, then combine each pair of 2 dice to find each unique feature of the local elven territory.
Take the highest roll, and multiply it by 5 to find the population.

\begin{exampletext}
  For example, a roll of `\dicef{4} \dicef{4} \dicef{6}' would show a powerful enchanter lives here, who tames griffins for the other elves to ride
  (discard duplicate rolls).
  Another elder has constructed a massive tree-dwelling, with rooms inside and paths between the various trees.

  In total, ($\dicef{6}\dicef{4}\times 5 =$) 50 elves live here.

  \labelledDiceTrio{4}{4}{6}{Enchanter}{Tree-house}{Enchanter}

\end{exampletext}

\begin{enumerate}
  \stepcounter{enumi}
  \item
  An elder, focussed on the Force Sphere, often casts portal spells around the many bushes and briers in the area, warping space.
  Anyone entering may accidentally emerge at the other side of the elvish lands.
  \roll{Wits}{Wyldcrafting} to notice, \tn[7] in the day, or \tn[12] at night.

  (Name fragment: `\textit{ando}')
  \item
  Expert archers leave their marks on every tree.
  They always collect every arrow, but the marks remain.

  (Name fragment: `\textit{pilin}')
  \item
  An elder, focussed on the Life Sphere, often casts spells around the perimeter to make anything there shrink.
  Replace the first two standard encounters with a miniaturized version (e.g. miniaturized chitincrawlers or bears).
  Elves know how to avoid the effects, and never suffer from them.

  (Name fragment: `\textit{pitya}')
  \item
  An elder, focussed on the Light Sphere, often casts illusions of elves.
  Anyone entering will find $2D6$ dancing apparitions scattered around the area, but not responding to them.

  (Name fragment: `\textit{pirne}')
  \item
  An elder with the Life Sphere creates enchanted gardens around the area, but always puts them in a new location each time (so the soil never becomes depleted).

  This elder has taken to shape-shifting, and will soon leave the other elves (or the elves will leave them), to wander free as a dryad.

  (Name fragment: `\textit{lelya}')
  \item
  Useful plants grow all over the region.
  Each of the three dice you have rolled indicates one common plant in the area:
  \randomPlants
  Find the plants \vpageref{plants}.

  (Name fragment: `\textit{tarwa}')
  \item
  An elder, focussed on the Mind Sphere, often casts spell around the perimeter, which confuse anyone who enters.
  They lose a portion of the day, wander aimlessly, and emerge somewhere else, without any memory of what they have done.

  The elder occasionally enchants griffins in the area, allowing a few elves to fly on their backs.

  (Name fragment: `\textit{vanwa}')
  \item
  A peat-bog furnishes the elves here with iron, allowing them to craft iron weapons.

  (Name fragment: `\textit{norna}')
  \item
  Slow spells from the Life Sphere have crafted massive tree-houses around the area.
  When the lights are on, anyone can see them.
  When the lights go out, they appear like normal trees.

  (Name fragment: `\textit{alda}')
  \item
  Expert artisans live in this community.
  Local clay deposits allow them to craft excellent pottery, including ceramic armour.
  If they have dwarven neighbours, they also trade for gold and silver, and make fine jewellery.

  (Name fragment: `\textit{namba}')
  \item
  Xenophobia has gripped the elves here hard.
  They will kill anyone who enters on sight, and everyone in the land knows it.

  The elves will destroy any gnomish roads leading to this location half-way along.
  The elves will only meet outsiders at this half-way point.

  (Name fragment: `\textit{tol}')
\end{enumerate}

\paragraph{The settlement's name}
comes from combining the name fragments in each feature.
Combine two `a' sounds wherever possible.

\begin{boxtable}[YcYcY]
  vanwa &   &      & = & Vanwa \\
  pitya & + & ando & = & Pityando \\
  vanwa & + & alda & = & Vanwalda \\
  lelya & + & alda + namba & = & Lelyalda Namba \\
\end{boxtable}

\subsubsection[Gnoll Grounds]{\Nl\ Gnoll Grounds}
\label{gnollPoint}

Gnolls typically wander back and forth around a territory.
They will use roads when they see them, but never like to rely on them.

The tribe has $2D6 \times 5$ members, and the same number of hunting dogs.
Roll 3 dice and check every combination of 2 to find the characteristics the gnolls who wander this ground.

\begin{enumerate}
  \stepcounter{enumi}
  \item
  The tribe have taken in a bear-cub, and raised it as one of their own.
  It hunts with them, and understands when to be quiet, and when to alert people to danger.
  \item
  This tribe suffer from isolation.
  Where most tribes have members who learn the neighbouring languages, but these gnolls only know how to trade with explicit allies, and still cannot speak their languages properly.
  \item
  The lowest neighbour with a fiend has killed a few of these gnolls, and they want revenge.
  \item
  The highest neighbour with civilization (i.e. any spot without a fiend) pays well for the gnolls' hunting dogs, so many of the tribe members have items from them, along with a general feeling of kinship.

  For example, if the highest neighbouring point near them has elves, then they will have elven jewellery, and plenty of elven songs (though the thick accent may not make this obvious).
  \item
  The tribe have figured out how to use local plants as hair-dye, and they have decided that this is the best thing ever!
  \index[Dye]
  \item
  A powerful druid advises the tribe.
  \item
  These gnolls have some of the best hunting dogs around, and plenty of them.
  Double the number of hunting dogs, so they have twice as many dogs as tribe-members.
  \item
  The gnolls know all the local plants, including where to find marching mushrooms and screechmoss (\autopageref{marching_mushroom}).
  \item
  The tribe keep a herd of aurochs, and wander constantly in order to let them graze.
  They have no fences, but they can still stop the aurochs wandering away -- each one of their dogs knows to stop them escaping.
  \item
  The tribe keep a small herd of sheep.
  \item
  Everyone has contracted \nameref{Torpid Flesh}.
  They are hairless, and look half-dead
  (see \autopageref{Torpid Flesh}).
\end{enumerate}

\subsubsection[Gnomish Warrens]{\Gn\ Gnomish Warrens}
\label{gnomePoint}
interact in subtle ways with their four neighbouring points.
To find out about this gnomish settlement, you should finish every point around it first.
Every point which might benefit the gnomes results in another warren, with some new specialization.

When multiple warrens exist on a single point, they each connect to a nexus cavern below,%
\footnote{Or as the gnomes call it, a `gnexus' (pronounced `nexus').}
which allows the gnomes to trade.

\begin{itemize}
  \item
  If they have gnomish neighbours, they enlarge the natural tunnels beneath them to make an underground road.
  The ups and downs mean the path is twice as long below ground as above.
  It is also not without danger, but underground roads have far fewer creatures than above-ground.
  (add another warren)
  \item
  If this point has neighbours on a mountain, the gnomes have crafted an underground stream boat-ride, which goes from the mountains, to their warren.
  (add another warren)
  \item
  If they have a road nearby, they construct a series of underground bothies, each 6 miles apart, to reach the road.  They can walk above ground, but always have somewhere to rest along the way. (add another warren)
  \item
  If a lake lies within the area, they construct a grotto to lay traps for fish and eels.
  These traps catch so much that nobody else has much luck fishing. (add another warren)
  \item
  If gnolls live nearby, they trade for hunting dogs, and occasionally ride them when going out. (add another warren)
  \item
  If a dryad lives nearby, they create irritating songs which get stuck in your head, and sing them constantly.  Dryads hate this kind of thing.
  \item
  If an ogre lives nearby, they create pit-traps with spikes at the bottom.
  When an ogre steps in one, they cannot walk straight for a week.
  If a goblin steps in one, it usually dies.
  (\roll{Wits}{Wyldcrafting} \tn[12] to notice, $1D6+1$ Damage for failure)
  \item
  If a hag lives nearby, the gnomes grow wheat, and use an underground mill to make flour, and finally, to bake cakes.  Cakes and a little flattery always works on old crones.
  \item
  If elves live nearby, they arrange for a magical dual (just for sport), every \gls{Atya}.

  The gnomes also add a road between them and the elves.
  Elves don't appreciate roads, but the road serves as a useful warning that if anyone attacks the gnomes, they may attack the elves next.
\end{itemize}

\paragraph{Politics in the Warrens}

Each warren governs itself a little differently.

The number of gnomes in a given warren equals $2D6\times 10$.
The first die indicates the political structure this warren has adopted.

\begin{dlist}
  \item
  Do-opoloy (you decide how something works by making it work)
  \item
  Thaumocratic (alchemists rule)
  \item
  Anarcho-syndicalist (which consists of debating how anarcho-syndicalism works)
  \item
  Technocratic (whoever understands most about a subject decides how it's done)
  \item
  Direct Democracy (dinner takes about 3 hours each night)
  \item
  Indirect democracy (people vote once a week, documentation is crucial)
\end{dlist}

\begin{exampletext}
  For example, this area has 2 warrens.
  \begin{description}
    \item[\dicef{2}\dicef{4}]
    indicates that this warren has 30 gnomes, where a couple of alchemists make the decisions.
    \item[\dicef{12}]
    indicates that this warren has 60 gnomes, who vote for leaders, who then make decisions.
  \end{description}
  
\end{exampletext}

\subsubsection[Human Settlements]{\Hu\ Human Settlements}
\label{humanPoint}
\index{Towns}
have a load of \glspl{village}, often surrounding a central town.

\begin{enumerate}
  \item
  Place two \glspl{village} by each \gls{broch}.
  \begin{dlist}
    \item
    2 \glspl{village} and \pgls{broch} forming a triangle.
    The \gls{broch} stands away from the road, protecting the \glspl{village} from whatever might emerge from the \gls{edge}.

    (civilization rating: 2)
    \item
    4 \glspl{village} with 2 \glspl{broch}.
    (civilization rating: 4)
    \item
    6 \glspl{village} form a circle around a small town.
    3~\glspl{broch} sit opposite the road, to protect the \glspl{village} from the \gls{edge}.

    One of the \glspl{village} began as \pgls{broch} and still has that tall tower.
    \Pgls{warden} stays there, ruling over this tiny kingdom.

    (civilization rating: 6)
    \item
    8 \glspl{village} surround a town, with room to safely grow crops and raise cattle outside its walls.
    4~\glspl{broch} help to harden the outer barrier of settlements which protects the town.

    The local \gls{warden} has plenty of profits coming in, and can afford his own \glspl{sunGuard}, to ensure everyone pays their taxes, so that the \gls{warden} can pay for the \gls{guard}.
   
   (civilization rating: 8)
    \item
    10 \glspl{village} and 5~\glspl{broch} keep the town safe.
    A healthy flow of traders ensures the \glspl{village} by the roads stay relatively clear -- anything which wants to attack will try to stalk those traders before attacking the outer \glspl{village}.

    A couple of the outer \glspl{village} have their own \glspl{warden}.

    (civilization rating: 10)
    \item
    12 \glspl{village} and 6~\glspl{broch} surround a massive town.
    Inner hamlets allow farmers to raise beasts in peace.
    The network of roads has plenty of hand-crafted signs, directing travellers to wherever they want to go.

    The town supports a minor army of \glspl{sunGuard}, and the area exports plenty of food.
    Each \gls{village} has \pgls{warden} associated with it, though the \glspl{warden} stay in the inner hamlets.

    (civilization rating: 12)
  \end{dlist}
  \item
  Each \gls{village} and \gls{broch} has a small radius of flattened land, without trees, stretching around a hundred \glspl{step} from the wall.
  \item
  Once you have some \glspl{village} and \glspl{broch} down, draw roads between them all.
  \item
  If the \glspl{pc} approach \pgls{village}, check its features on \vpageref{villageFeatures}.
\end{enumerate}

\paragraph{To discover a settlement's most famous feature,}
take the number of \glspl{village} and add $1D6$.
For example, if a settlement has 4~\glspl{village}, roll $1D6+4$.

\begin{enumerate}
\setcounter{enumi}{2}
  \item
  Newly built -- every day, stone and wood comes in to build a little more each day.
  Occasionally, parts of a wall collapse and archers flock to protect it.
  \item
  Trackers
  \item
  Deadly and magical plants.
  Roll $2D6$ and apply both results.

  \randomPlants

  (Details \vpageref{plants})
  \item
  Ale.
  The \gls{templeOfPoison} has tight control over the area (\autopageref{god:Poison})
  \item
  White rock (actually limestone)
  \item
  \ifodd\value{r4}
    Wood-carvers
  \else
    Metallurgists
  \fi
  \item
  Bards
  \item
  Madness (neighbouring \glspl{witch} are to blame)
  \item
  \Glspl{witch}
  \item
  Cartographers
  \item
  Statues of heroes litter the roads around.
  \item
  A grand \gls{court} where the town \gls{warden} finds beasts guilty, and has them fight for entertainment.
  \item
  Bath houses.
  \item
  `Demi-human suburbs', where all manner of other races live.
  \item
  A fight about marriages and road taxes has caused an argument between the town \gls{warden} and the highest neighbour.
  The \gls{sunGuard} will march to war soon\ldots
  \item
  The grandest \gls{templeOfCuriosity} known, with books (and soaps) of every size and type.
\end{enumerate}

\paragraph{They call it\ldots{}}

Select a name for the settlement by combining what everyone knows it for, with the type of land.
Write the settlement's name next to it, and underline it.

Any towns might receive the same name, or something related.

\begin{itemize}
  \item
  People know a town in the hills for its library. That's why they call
  it `Page Valley'.
  \item
  A town by the sea, famed for Witchery.
  They call it `Hexwave'.
  \item
  `Bards' + `Islands' = `Lyrisles'
  \item
  \ifodd\value{r4}
    `Metallurgists' + `forest' = `Iron Basin' (perhaps a dried-up lake sits nearby)
  \else
    `Wood-carvers' + `forest' = `Taming Woods'%
  \fi
  \footnote{If you think these names sound stupid, you should look up the meaning of your hometown's name.}
\end{itemize}

Try some word-association with the town's theme or location.

There are more random tables for towns \vpageref{encTownWarden}, for when the troupe try to enter.

\subsection{Fiends}
\label{mapFiends}
\index{Fiends}

\subsubsection{Bandits}
\label{banditsPoint}
\index{Dragons}

\Pgls{village} left alone, with broken roads and the threat of starvation, has only one option -- banditry.
\Gls{fenestra}'s thick forests create a natural barrier around any \gls{village} which wants to remain hidden.

One road leads out of their settlement and joins with the nearest main road in secret; it splits into many roads at the last moment, so none of them look well-trodden.
The bandits cover their tracks with foliage once they enter the main road, and only leave their private road at night.

A total of 200 villagers live in the commune, but only 50 regularly exit to perform robberies.
See \autopageref{bandit} for details on the fiend.

Roll $3D6$, and apply every result.

\begin{dlist}
  \item
  Ten of the bandit has a suit of plate armour.
  \item
  Fences among the \gls{village} with the lowest number help the bandits sell their wares.
  Without this aid, they must journey to settlements themselves, or make deals with people from other lands, accessible only through the long road out of the map.
  (see step \vref{roadOut})
  \item
  The bandits have brokered a temporary alliance with whatever fiend has the highest number.
  If there are no other fiends on this map, the alliance is with a fiend in another land -- accessible by the long road.
  \item
  An imprisoned gnomish alchemist performs tricks and makes \glspl{talisman} for the bandits.
  \item
  They hold an old \gls{broch}, abandoned by the \gls{guard}.
  It watches the private bandit road, and always has $1D6$ bandits watching, with crossbows.

  At night, the \gls{broch} can send signals back to the bandit \gls{village}.
  \item
  \Pgls{doula} leads the bandits from the rear, arming them with \glspl{talisman} and information.
  %\item
  % A fake \gls{village}, upon a rocky area which leaves no footprints.
  % Only a dozen bandits stay inside, but it looks sufficiently lived-in that anyone could think they had found the real bandit \gls{village}, and not look any further.
\end{dlist}

\subsubsection{Dragon Coves}
\label{dragonPoint}
\index{Dragons}

Dragons may live in deep caverns, shallow caverns, or sometimes just lay about in a Sunny patch of forest, without any roof.
In any case, they like to live close to mountains -- surrounding trees can make flight difficult, while having a tall crag to leap from helps them take flight.

Their destructive habits and association with the sky make people think they are divine.
Of course, in \gls{fenestra}, this is not a good thing.

See \autopageref{dragon} for details on the fiend.

Roll $3D6$, and apply every result.

\begin{dlist}
  \item
  An abandoned dwarvish settlement.
  Roll up a dwarven settlement, then replace those dwarves with a dragon (\vpageref{dwarvenPoint}).
  \item
  Complete plate armour, made by gnomes under threat of death.
  \iftoggle{core}{(See the core rules, \autopageref{bandingArmour}.)}{}
  \item
  Knowledge of the local languages.
  Dragons without this know only elvish (it doesn't change much).
  \item
  A glass-smelting workshop (everyone needs a hobby).
  Glass statues litter the dragon's lair.
  \item
  A portal to another realm, filled with glass flowers which function as every type of \gls{ingredient}.

  The suffocating heat in that strange desert inflicts 3~\glspl{ep} every \gls{interval}.
  \item
  Three eggs, which will hatch next season.
  Once that happens, the dragon will need to venture out to feed her newborn.
\end{dlist}

\subsubsection{Dryad}
\label{dryadPoint}
\index{Dryads}

Dryads concern themselves with plants and animals more than people\ldots but they also consider most people to be animals.

See \autopageref{dryad} for details on the fiend.
Roll $3D6$, and apply every result.

\begin{dlist}
  \item
  A dozen ex-farmers, now transformed into strange mutant creatures.
  \item
  A maze of venomous thorns envelopes the entire area.
  Touching them inflicts \glspl{ep}.
  \item
  The dryad makes \glspl{talisman}, just for fun, and leaves them as gifts for travellers who behave politely.
  Other \glspl{talisman} find their way to less polite travellers.

  Sensible people leave all of them alone; one can never tell what a dryad considers good behaviour, or what they consider a `reward'.
  \item
  Deep caverns exit in the centre of the dryad's lair.
  The dryad sometimes journeys down, to bring up umber hulks and oozes, and observe them, then release them into the wild.

  These caverns go down into the \gls{deep}, and eventually connect to each other point which has caves.
  \item
  $1D3$ basilisks live with, and love, the dryad.
  They will protect the dryad with their lives.
  (find basilisks on \autopageref{basilisk})
  \item
  $1D6$ griffins, used as protectors and steeds.
  (find griffins on \autopageref{griffin})
\end{dlist}

\subsubsection{Hag Cottage}
\label{hagPoint}
\index{Hags}

Here, an old lady lives alone in a hut.
Of course, in \gls{fenestra}, only one type of old lady lives alone.
See \autopageref{hag} for details on the fiend.

Roll $3D6$, and apply every result.

\begin{dlist}
  \item
  The hag has a dozen griffins circling her hut at all times.
  Each one will kill anyone who gets close to the hut, but will not attack them before they approach.
  \item
  The hag keeps 4 chitincrawlers as pets.
  They spin webs across every window and tree.
  (find chitincrawlers on \autopageref{chitincrawler})
  \item
  Mouthdiggers create pot-holes around the entire area, and ensure nobody attacks from below.
  (find mouthdiggers on \autopageref{mouthdigger})
  \item
  A garden of carrots and \nameref{screechMoss} encircles the cottage.
  It screams the moment someone steps on it.
  (find \nameref{screechMoss} on \autopageref{screechMoss})
  \item
  Spells have ravaged the soil in all directions.
  The trees have no leaves, the grass looks half-yellowed, and no animals inhabit the area except some small insects.

  The hag now has to take long walks to grow food, which continuously degrades more and more of the landscape.
  \item
  An assortment of potion bottles line her shelves.
  Each one has a poorly-written label, or the wrong label.
\end{dlist}

\subsubsection{Ogre}
\label{ogrePoint}
\index{Ogres}
\index{Goblins}

The ogre has $2D6 \times 5$ goblins, ready to raid all the local \glspl{village}.

See \autopageref{ogre} for details on the fiend.
Roll $3D6$, and apply every result.

\begin{dlist}
  \item
  The goblin population grew until they ate an entire \gls{village}, and then consumed every \gls{village} in this area.
  They laid siege to the city here, and then managed to enter by using a catapult and primitive parachutes.

  The city was half-burnt, and remains inhabited by the goblin horde, and their king.
  Everyone remembers this area, and takes it as a warning to deal with small problems before they become big problems.

  Check the details for this lost city \vpageref{lostCities}.
  \item
  Long caverns below provide the goblins a mushroom farm.
  They extend to a cold region, and eventually into the \gls{deep}.
  \item
  The ogre has a set of complete plate armour, created by a blacksmith he still holds prisoner in an abandoned \gls{broch}.
  \item
  The goblins have acquired 10 crossbows and 100 quarrels, and have started to figure out how to use them.
  \item
  The ogre king has a pack of $3D6$ hobgoblins.
  \item
  A travelling dryad, who found the goblins rude, erected a living hedge-maze, as vengeance for their bad manners.
  It changes a little each season, so the route out changes with it.
  And every cold season, the hedge maze vanishes, to reveal a horde of starving goblins.
\end{dlist}

\subsubsection{Lich}
\label{lichPoint}
\index{Liches}

Every lich has a set of caverns to stay.
Finding a lich in the cold, twisting \gls{deep} makes a near-impossible challenge, as they change location at random.

See \autopageref{lich} for details on the fiend.
Roll $3D6$, and apply every result.

\begin{dlist}
  \item
  The lich waited for centuries, building an army of ghouls.
  When a nearby human town weakened, the lich attacked.

  He remains in the town to this day.
  The local \glspl{village} have rotted, but 3~\glspl{broch} remain standing.
  Create a lost city for the lich, on \vpageref{lostCities}.
  \item
  An old \gls{broch}, lost to the \gls{guard}, now used as an alchemical workshop.
  \item
  The lich keeps a crew of $3D6$ ghasts, ready to follow orders.
  \item
  A cabal of ten necromancers-in-training visit with bodies to feed the lich's army, in exchange for tutelage.
  They frequent local towns, and pick up the dead with a cart.
  \item
  The lich has complete plate armour for himself and any steed.
  \item
  When a basilisk attacked, he killed it, turning it into an undead steed.
  It remains inside the city, waiting for the lich to summon it.
\end{dlist}

\end{multicols}

\section{Strange Places}

\begin{multicols}{2}

\subsection{Lonely Taverns}
\label{lonelyTaverns}
\index{Lonely Taverns}

These taverns exist on long stretches of road, far from any town, and
charge high prices for a drink. They must live off traders passing
through, and survive whatever the forest brings out.

Normal people don't stay for long.
Those who stay a while often have problems with the local
law, as these places often make their own laws.
Barkeeps punish any robbery close to the tavern harshly, but don't often care about what people do around the towns.
This makes these taverns a safe intermediate location where anyone can talk in peace.

Of course, bandits won't announce themselves as such when speaking with \glspl{guard}, but then the \glspl{guard} often won't announce their employment either, no matter how obvious that sword on their back makes them.

\subsubsection{The Barkeep}

Roll $1D6$ to find this season's barkeep (they change all the time):

\begin{dlist}
  \item
  A veteran of the \gls{guard}, with a hundred war-stories. Of course
  when he tells them, nobody can get a drink, so don't ask!
  \item
  Someone from point 4 on the map, hiding here with a bounty on their head for thieving from \pgls{warden}.
  \item
  An outlander from a land so far away, nobody has ever heard of it.
  Every story she tells sounds made-up, but the strange accent shows she really does come from somewhere distant.
  \item
  A powerful \gls{witch} who swore an oath never to use magic again.
  He won't say why.
  \item
  A collective -- you stay as long as you like, earn your keep, then go
  when you please. Sometimes in the colder Seasons, the place just lies
  barren.
  \item
  A dwarf who records all he can.
  The patrons say he works as a spy for someone, but they disagree about whom.
\end{dlist}

\subsubsection{The Menu}
\index{Menus|see {Lonely Tavern}}

Roll $3D6$ -- the \glspl{pc} can order any of these meals.

\newcommand\menuItem[3][(\arabic{r12} \glspl{cp})]{%
  \randomdozen%
  \randomthree%
  \randomfourB%
  \ifodd\value{enumi}
    \randomthreeC%
    \randomfour%
  \fi
  \item
  \textbf{#2:}
  #3
  #1
}

\begin{dlist}
  \menuItem{Griffin-wing}{freshly killed this morning, after the griffin tried to fly away with a gnomish patron.}
  \menuItem{Mystery-stew}{why are you hesitating?
  It goes rotten quick, so get eating!
  \footnote{Chitincrawler `meat' (webbing as sauce!)}
  }
    \ifnum\value{r4}<2
      \newcommand\morningSoup{uproot}
    \else
      \newcommand\morningSoup{marching_mushroom}
    \fi
    \menuItem{Sunrise Soup}{the chef found a new plant this morning, and he's already learning how to cook it!
    \footnote{In fact this is \nameref{\morningSoup}, see \autopageref{\morningSoup}, for the effects.}
    }
  \menuItem{Deer}{thank the man in green, sitting in the corner -- he caught it this morning.}
  \ifodd\value{r4}
    \menuItem{Dwarf-beard}{actually just a type of seaweed, left as payment by a local trader; but it tastes just like the real thing!}
  \else
    \menuItem{Eye-Spy}{made with actual woodspy.}
  \fi
  \ifodd\value{r3}
    \menuItem[(0 \glspl{cp})]{Get bent}{the barkeep's in a foul mood, because they need a day off.}
  \else
    \menuItem[]{\ldots and bugger-all-else}{a few barrels turned out to be rotten, and now someone's stolen an entire pot of soup.
    The menu will be limited for the day.}
  \fi
\end{dlist}
\null

\subsection{Lost Cities}
\label{lostCities}
\index{Lost Cities}

\widePic[t]{Nelness/city}

People call them `\gls{yonder}'s house', as a reminder that most people who enter never return, but remain with \gls{yonder}.
People also call them `land \glspl{deep}', because these spaces act according to their own rules; they do not have the usual wandering monsters, but some predators always take residence in them, along with the fiend who ruined the city.

Before the \glspl{pc} enter, you should make some rolls to find out the nature of the town.
What creatures live there?
Which towers are still standing?
Roll dice to find out which tall towers still stand in the city, in steps \vrefrange{lostDwellers}{lostTowers}.

Once the \glspl{pc} enter, they can attempt to look for valuables, quietly, in steps \vrefrange{lostWhispers}{lostForaging}.

Entering a lost city will go something like this:

\begin{exampletext}
  The \gls{gm} knows the troupe are headed to the lost city, where an ogre stays with his goblin horde.

  First, she rolls up the inhabitants.
  `\dicef{3} \dicef{3} \dicef{6}' indicates two city-dwellers: chitincrawlers and a coven of $1D3+1$ demiliches.
  One more roll shows that means a coven of 3 demiliches have made a little secluded spot for themselves in the lost city, unbothered by the goblins.
  Perhaps they use the chitincrawlers as a source of \glspl{ingredient}?

  Next, she rolls $4D6$ to find the towers, and finds `\dicef{2} \dicef{2} \dicef{6} \dicef{4}' -- that means that this lost city still has 3 tall towers standing
  (towers let the \glspl{pc} take in an overview of the city).

  As the troupe enter, she describes the scene to the players.

  \begin{boxtext}
    The city seemed massive from outside, but here at the rusted gate, you can only see doorways and windows lying open like black eyes.
    Thorny bushes with little multicoloured berries grow all down the street ahead, where the Sunlight strikes brightest.
    Dropping around them suggest deer frequently come here to feed.
  \end{boxtext}

  The troupe moves quietly towards the location their map claims the old guild halls once stood, by the market.
  But before hunting for loot, they roll \roll{Dexterity}{Stealth} (\tn[8]) to stay silent.
  The group roll succeeds, so they move quietly, then roll \roll{Intelligence}{Vigilance} (with a +3 Bonus for the map) at \tn[12].

  The first roll fails, so the \gls{gm} mentions that they can cover more ground by splitting up if they want.
  But \pgls{interval} has already passed while they hunted, and the Sun is beginning to set.

  The troupe is now deep into the lost city, so they can wait in the dark, or navigate back out again, replacing the usual city encounters with the wandering monsters outside.
  They decide to stop foraging, and hide in an abandoned house, then wait to see what morning brings.

  During the night, the \gls{gm} must keep track of what the dwellers in the lost city get up to.
  Since the city has demiliches and chitincrawlers, it makes sense that the demiliches would want to use the chitincrawlers' bodies as \glspl{ingredient} for \gls{witchcraft}.
  The \gls{gm} describes the sounds of demiliches casting Death magic on chitincrawlers, so they can use their bodies to make \glspl{boon}.

  They probably don't make much sound -- shuffling, grappling, and swooshing cloaks as the necromancers gesticulate to cast spells.

\end{exampletext}

\mapentry{Finding the City}

It may seem difficult to lose a city, but when grass grows over the roads, and the wooden \glspl{village} crumble, cities can become almost invisible.
If the troupe only know about a city's location through legend, they will have to roll \roll{Intelligence}{Wyldcrafting} at \tn[14] to locate the city.

Cities which have fallen more recently often still have a visible road, and will not require any roll to find.

\mapentry{Wandering Dwellers}
\label{lostDwellers}

Each city has some population of the servants of the local fiend.
If a lich lives in the city, it has a lot of ghouls; if an ogre king lives here, it has goblins.

Additionally, lost cities attract other creatures which end up taking residence.

Roll $3D6$ -- each number which comes up determines one type of city-dweller.
Ignore any repeats.

\begin{dlist}
  \item
  \textbf{\Pgls{seeker} of the \gls{templeOfCuriosity}} has enlisted $1D6+4$ fully armoured \glspl{sunGuard} as part of a reconnaissance mission for \pgls{warden}.
  A camp sits nearby, with two more \glspl{sunGuard}, six donkeys, and 20 days' rations.

  The \gls{seeker} won't like anyone interfering with his business, so if he encounters the troupe, roll a Morale check (see \vpageref{morale}); on a pass, he tries to kill them.
  \item\label{lostOoze}
  \textbf{Oozes} of all types slide across the streets, which seem strangely clean, and free of debris or foliage.
  \item
  \textbf{Chitincrawlers} hide in every abode.
  Verdant berries of every colour have encouraged deer into the area, but every shadow gleams with thick webbing.

  $1D6+8$ live here in total.
  \item
  \textbf{Griffins} look down from every tower.
  The high towers make perfect perches to surveille the area, and the degrading wood helps to make nests.

  Each tower has 2 griffins at the top.
  \item
  \textbf{Woodspies} have not only made this place their home, they have taken to imitating the items in the area, such as stools, piles of books, or chests.

  $1D6+4$ live here in total.
  \item\label{lostDemilich}
  \textbf{Demilich} covens help these undead sorcerers study with their own kind.
  However, their lack of basic empathy makes them dangerous to each other -- none of them really trust the others, so they share information slowly, always hinting that they have more to teach while masking their true abilities.
  Whenever one is wounded while another is not, they both attack each other (the first has spotted the right moment to slay a potential enemy, and the second knows what the first is thinking, as they all operate by the same unempathic principles).

  Roll $1D3 + 1$ to determine the number of demiliches.
\end{dlist}

\mapentry{Tall Towers in the Labyrinth}
\label{lostTowers}

Wooden rooves rot, leaving bear stone walls across much of the fallen city.
Some walls collapse, opening new passages, while trees grow and thorny bushes grow to block doors and streets.
A twisted, mossy, labyrinth forms.

Despite the rot and chaos, some buildings still stand tall.
Once \pgls{pc} climbs to the top of a tower, two things happen:


\begin{itemize}
  \item
  The tower's vast perspective grants a +2 Bonus to foraging, as they spot good places to investigate (such as places which look like guild houses, or less-damaged areas of the city).

  This only works while the character can see, so it probably only works during daylight.
  \item
  Any undead in the city can clearly see the character, and will begin moving towards them.
  \item
  There is a 1-in-6 chance of seeing the dwellers of the city interact with each other (start lowest to highest).

  If the city has a curious \gls{seeker} and some acidic oozes, then the character may see the \gls{seeker} running away from the oozes.
  Or if the city has chitincrawlers and griffins, they may see a griffin caught in a web.
\end{itemize}

Climbing these buildings gifts a wide perspective of the city, which grants a +2~Bonus to the \glspl{pc}' foraging rolls.

Roll $4D6$ -- each number which comes up indicates a tower in the lost city.

\begin{dlist}
  \item
  This old \gls{templeOfPoison} had lavish beds up at the top, though the only current inhabitants are dead spiders and rat-droppings.

  The kitchen has a basement, filled with sealed clay pots.
  The wooden ladder has rotted, and will break once anyone of \gls{weight} 6 or more steps onto it.

  Anyone investigating the basement should make an \roll{Intelligence}{Vigilance} roll (\tn[10]).
  Success means they have found a secret stash of 20~\glspl{sp}, hidden in a honey pot.
  Failure means that a clay pot has gone rotten, and explodes; the character has contracted Corpse Hands (\vpageref{Corpse Hands}), and they will start to feel the effects after 2~\glspl{interval}.
  \item
  This tower has one wall remaining -- a \roll{Speed}{Athletics} check (\tn[8]) allows the climber to see a little in all four directions.
  \item
  This tower once held the \gls{sunGuard}, but now has a single servant of the local fiend (usually goblins or ghouls).
  If the \glspl{pc} do not take it by surprise, it will shout for aid (the undead can shout to other undead silently).
  \item
  This tower stands tall enough to see all around.
  The bones of dead humans fill the stairway.

  Any movement which disturbs the bones will send them falling down the stairs with a clank and a thud.
  \Glspl{pc} must roll \roll{Dexterity}{Stealth} (\tn[11]) to move past the bones without issue, or else waste \pgls{interval} moving everything down by hand (in this case they roll \roll{Dexterity}{Crafts}, \tn[7]).
  \item
  The old citadel still stands, despite decay.
  Parts of the original \gls{warden}'s home have rotten flooring, so moving through it requires a \roll{Wits}{Crafts} roll (\tn[8]) to notice weak areas of flooring, and the best way to walk.
  Failure indicate that the character falls two~\glspl{step}%
  \exRef{core}{Core Rules}{falling}
  and makes an awful sound.
  \begin{enumerate}
    \item
    On the second story, characters will encounter the city dweller with the lowest number.
    \item
    If anyone gets to third story (which requires three rolls), then they will find a cupboard with \lootBig.
  \end{enumerate}
  \item
  This tower holds an empty library, once part of \pgls{templeOfCuriosity}.
  It has no books, and little stability left.

  Upon entering, \glspl{pc} can make a \roll{Wits}{Crafts} roll to understand if this tower will hold their own weight (the rolls tells the \gls{pc} nothing about others).
  Anyone with \pgls{weight} above 5 prompts the structure to collapse, but only once they have reached the top.

  If the tower begins to collapse, anyone inside can run down (\roll{Speed}{Athletics}, \tn[9]), but failure indicates the building has collapsed upon them, inflicting $4D6$ Damage.

  Or the \gls{pc} can elect to jump, and try to grab something nearby (\roll{Speed}{Athletics}, \tn[12]), but failure indicates that they will suffer a nasty fall of $2D6$ Damage.
\end{dlist}

\bigLine

\mapentry{Creeping in the Cold Labyrinth}
\label{lostWhispers}

When the \glspl{pc} want to move through a lost city, they roll \roll{Dexterity}{Stealth} to remain quiet.
The \gls{tn} is normally 8, but often easier.

\begin{itemize}
  \item
  -1 if moving at night.
  \item
  -1 in the rain.
  \item
  -1 in a storm.
  \item
  -2 if the dwellers are busy with something.
  \item
  -4 if hiding indoors.
  \item
  +5 if lighting a fire.
\end{itemize}

The undead are a separate matter.
Even if the troupe move around silently, any active undead will spot them, unless they have some way of hiding from the sight of the dead.%
\footnote{See \autopageref{undead_senses}.}

\paragraph{When the roll fails,}
\label{lostChase}

Then they have an immediate encounter, in this order:

\null
\begin{enumerate}
  \item
  The dwellers with the lowest number.

  \textit{For example, a city with oozes and griffins would mean that if the \glspl{pc} make a noise, then an acidic ooze wanders out of a house, and heads towards them.
  Then on the next failed check to creep around, a griffin would attack.}
  \item
  $2D6 + 8$ of the fiend's servants encounter them, and attack.

  \textit{Usually this means a lot of goblins, or a lot of ghouls.}
  \item
  The fiend itself has awakened, and begins planning an assault.
  They will probably not approach the \glspl{pc}, but will find a vantage point to see the \glspl{pc} from, and co\"ordinate attacks.
\end{enumerate}

\paragraph{Running away,}
requires a \roll{Speed}{Stealth} roll, at a \gls{tn} equal to the enemy's \roll{Speed}{Vigilance}.

\paragraph{Trying to fight}
means making more noise.
Every $1D6$ rounds, another encounter occurs.

\mapentry{Foraging Quietly}
\label{lostForaging}

Each time the \glspl{pc} forage, they roll \roll{Intelligence}{Vigilance} (\tn[12]) to figure out where the best loot lies.
If they succeed, roll $2D6$ -- the first die determines the place, and the second determines the prize!

\begin{itemize}
  \item
  Once they have a prize, cross it off the list.
  \item
  If they get the same number again, they will find the second prize with that number.
  \item
  The third time they roll a number, they find nothing -- lost cities only hold so many treasures.
  \item
  If the \glspl{pc} split up, each one can perform individual rolls, so they will loot far more efficiently, but run more chance of one getting caught alone.

\end{itemize}

\subsubsection{Climbing the Walls}

\Glspl{pc} might feel a temptation to climb the outer walls to get a proper perspective of the city, but doing so leaves them visible to every dweller within.

They gain a +2 Bonus to foraging rolls, but one of the dwellers with the highest number moves (or flies) to attack them.

\end{multicols}

\foragingChart

\section{Depth}

\begin{multicols}{2}

\subsubsection{Finishing the Map}

\noindent
Once you have each area detailed, you have a complete map, but you can always add more details.

\begin{itemize}
  \item
  Draw contour-lines inside any lakes or sea.
  \item
  What are the roads' names?
  \item
  Do you have any points with civilization, but without a road connecting them to anywhere?
  What would the neighbouring points think about people they occasionally see, but cannot reach by road?
  What rumours would the troupe hear about those places?

  Both fiends and civilized locations can spark rumours when nobody sees them.
  \item
  Select two neighbouring points on your map, and make a new map based on them.
  Add detail, put names on everything, and maybe add some misleading elements.

  If you want to put forests on this map, make it an abstract shape with a clear border, and make sure it goes right up to the edges of the map.
  Everyone in \gls{fenestra} knows they inhabit a tiny island of civilization, walled in by a towering, green, sea.
  \item
  Follow a road leading off your map.
  Where does it go?

  Start a new map, and connect them by the roads which venture past the page.
  Repeat until you charter the world.
\end{itemize}

\bigLine

\subsubsection{Player Connections}

How are the \glspl{pc} connected to this land?
Make a numbered list of possible past events, with a default and non-default race for each entry.

\begin{itemize}
  \item
  If a human rolls a point where humans live, or a gnoll rolls a point where gnolls live, then they come from that place -- perhaps from a town, or \glspl{village} surrounding it.
  \item
  Otherwise, they may have a connection to a nearby fiend, living between the points.
  Perhaps the fiend destroyed their home and family.
  Perhaps they suffered an attack on the road which still haunts their dreams.
  \item
  Non-humans rolling a human town might have worked with whatever the town is famous for (step \vref{mapCharacter}) -- a gnome might have worked as a tracker, or come to the city to listen to its famous bards.
  Or perhaps a gnoll helped keep the great beasts in a town's \gls{court}.
  \item
  Humans rolling an elven settlement might have been raised there, or lived there for some time after fleeing the law.
\end{itemize}

When the players create their \glspl{pc}, have each one roll on your chart to determine their connection to the land.

\needspace{8em}
Your chart might look something like this:
\begin{enumerate}
  \item
  \underline{Horseshoe Valley}
  \begin{description}
    \item[\textbf{Humans:}] you grew up in a little \gls{village} by Horseshoe Valley, and want to seek your fortune in the wider world.
    \item[\textbf{Others:}] you came as a trader to purchase iron goods, but soon grew scared of travelling the road alone.
    Life in the \gls{guard} now seems safer than travelling with all those goods.
  \end{description}
  \item
  \underline{Elven Meadows}
  \begin{description}
    \item[\textbf{Elves:}]
    you come from the elven villages.
    While the elders gave their blessings, they made you swear never to bring outsiders back home.
    \item[\textbf{Others:}]
    the local hag -- `Thingizard' -- killed your family and tribe.
    You alone escaped, and now dream impotently of revenge.
  \end{description}
  \item
  And so on\ldots
\end{enumerate}

\end{multicols}


\chapter[The Cartographers]{Cartography}
\label{civilization}

Welcome to \gls{fenestra}.
Let's look around.

Fantasy stories have two special characters -- magic, and the landscape.
The audience discover both, just as they learn about the other characters.

Magic typically remains a mysterious character, but the landscape should always feel open.
As \pgls{gm}, your character is the landscape, and making a map is your character creation.

\section{Your Map}
\label{yourMap}

\begin{multicols}{2}

\mapentry{The Pentacle}

Tear out an example map on the left for the basic map.
You'll need a pencil and dice.

Each dot has four neighbours.
The highest neighbour is the neighbour with the highest number, and the lowest neighbour, is the neighbour with the lowest number.

At each point, roll $1D6$ and add the number at the point, then add the entry to the map.

\begin{multicols}{2}
\index{Fiends}

\begin{enumerate}
  \setcounter{enumi}{1}
  \item
  Humans
  \item
  Bandits
  \item
  Dragon
  \item
  Humans
  \item
  Humans
  \item
  Humans
  \item
  Elves
  \item
  Dwarves
  \item
  Gnomes
  \item
  Dryad
  \item
  Ogre
  \item
  Gnolls
  \item
  Hag
  \item
  Lich
  \item
  Humans
\end{enumerate}

\end{multicols}

\subsubsection{Rivers}

\paragraph{If your map has dwarves,}
then they live in mountains.

\begin{enumerate}
  \item
  Draw a mountain range around the point with the dwarves and then draw a river from the mountains to each of this point's four neighbours.

  River always begin from separate locations.
  They might later join together, but rivers never split apart.
  \item
  At each point with a river, extend the river to the lowest neighbour.
  \item
  Rivers begin as a single line, but when they join, they flow faster and wider -- draw them with two parallel lines.
  \item
  If three or more rivers join together, they form a lake, which exits towards the lowest neighbour.
  It still has the strength of many rivers, so continue the exit river with two lines.
\end{enumerate}

\paragraph{If your map has no dwarves,}
then it has no mountains.

\begin{enumerate}
  \item
  Points 1 and 2 become islands in a sea.
  \item
  The coast covers points 5, 6, and 7, then wanders off the map.
  \item
  Some well-springs open up on point 10, then join together to form a river.
  The river goes down through points 4, and 3, then goes towards point 2 and enters the sea.
\end{enumerate}

\paragraph{If your map has any points without water,}
then the highest of them becomes a swamp.

\subsubsection{Roads}
\index{Roads}

\begin{enumerate}
  \item
  Each human settlement has a road its two nearest neighbours (ties are broken by going to the highest point).
  Draw the roads with dotted lines, and avoid passing through any other points along with way.
  \item
  If a road crosses a small river, draw a small bridge.
  If it crosses a larger river, drawn with two lines, draw a larger bridge.
  \item
  The human settlement with the highest number has a road leading out of the map, towards the closest border.
  This road stays as far away as it can from other settlements.
  \label{roadOut}
  \item
  If a human settlement has dwarvish neighbours, connect them with a road.
  \item
  The road between points take a long time to walk.
  Put \pgls{bothy} on the road for travellers to rest, and then another for each point the road passes.
\end{enumerate}

Label a few roads with the miles in length.

\begin{itemize}
  \item
  The short distance between points 6 and 7 is about 20 miles.
  \item
  The medium distance between points 1 and 6 is about 30 miles.
  \item
  The long distance between points 1 and 2 is about 50 miles.
\end{itemize}

\paragraph{The Name}

Name the entire area.
A good starting point is the most prolific feature of the map.
For example, a map with a swamp and three elvish settlements ranges might receive the name `Faebog'.

You have completed your general map, but each point can receive a lot more detail.
You should select a few connected points you might use, and fully detail just those point.

\end{multicols}

\section{Zooming In}
\label{mapCharacter}

\begin{multicols}{2}

\noindent
Each point on the map needs more detail once the \glspl{pc} arrive.
Select your campaign's starting point (almost certainly a human settlement) and find out the details.
You might fill in a neighbouring point, if you think the \glspl{pc} could journey there within the first session.

Whenever you find out a new location's details, note them down as a key to the main map, and check if you can add any features to the main map.

\subsection{Dwarven Settlements}

A population of 100 dwarves lives here, with a mushroom garden.
Roll $2D6$ to find their special features.
Whenever you roll 8 or more, add 100 dwarves to the population and roll again.

\begin{enumerate}
  \stepcounter{enumi}
  \item
  This population is ruled by a king.
  Dwarves consider to be bad luck, but the king promises to have a daughter and let her take his place in the next generation.
  \item
  Gold seams run through the mountain, allowing the dwarves to make precious items, and to trade with various races outside.
  \item
  A Hall of Records stands at the base of this mountain, guarded by heavy doors.
  It works a little like a library, but only contains factual accounts of the things dwarves take interest in, such as facts about rocks, the lineage of queens, descriptions of how to farm on mountains, and techniques to measure altitude.

  Each book has a minimum of three seals, pressed a dwarven record-keeper, who has verified the authenticity of every statement in the book personally.
  \item
  The `Polite Passage', is the name dwarves use for the long, thin, bridges they create, with only room for a single dwarf to walk at a time.
  Settlements have one for entering, and another for exiting.

  Larger rooms around the mountain's exterior store wide equipment, such as wagons.
  \item
  Magma streams in the mountain's heart flow eternally, and mix with underground rivers, producing a great plume of steam at the top.
  Various tunnels contain steam-traps, where a metal grate will ring the `dinner bell', and allow hot steam to flow in to cook the intruding predator.
  \item
  A tin seam allows dwarves to mine, and trade with nearby settlements.
  \item
  In the dark \gls{deep} below the mountain, extensive underground tunnels, filled with fungi, moss, living oozes, olms, and umber hulks.

  With enough traps laid out for the monsters, this creates a kind of rough garden, or `hunting ground' for the dwarves to gather additional food.
  \item
  A few farms extend outside the mountains, across the fertile lands at the base.
  The dwarves create them with high stone walls, and polish each brick until it has nothing to hold onto, and no way to climb up.

  Inside, may allow as much space as any \gls{village} to grow vegetables for the mountain.
  \item
  An elevated barley garden around a third of the way up the mountain, provides food for the settlement.
  Nobody can see the gardens from below, as it grows on steps carved into the mountain.

  A series of rock-bells will sound when anything steps on them, but do not ring in response to the wind.
  \item
  An iron seam allows dwarves to create excellent weapons and armour.
  If they have no road out, then they only make rare, wholesale trades.
  \item
  Half way up the mountain, shepherds take sheep and goats on long walks, then put them safely away at night, to sleep in stone rooms.
\end{enumerate}

\subsection{Elven Grounds}

A group of $2D6 \times 10$ elves lives here.

Roll $3D6$, then combine each pair of 2 dice to find each unique feature of the local elven territory.

\begin{exampletext}
  For example, on the roll `\dicef{2} \dicef{4} \dicef{2}', you would use the results for:

  \begin{itemize}
    \item
    $2+4 = 6$
    \item
    $4+2 = 6$
    \item
    $2+2 = 4$
  \end{itemize}

  The unique results are `6', and `4'.
\end{exampletext}

\begin{enumerate}
  \stepcounter{enumi}
  \item
  An elder, focussed on the Force Sphere, often casts portal spells around the many bushes and briers in the area, warping space.
  Anyone entering may accidentally emerge at the other side of the elvish lands.
  \roll{Wits}{Wyldcrafting} to notice, \tn[7] in the day, or \tn[12] at night.
  \item
  Expert archers leave their marks on every tree.
  They always collect every arrow, but the marks remain.
  \item
  An elder, focussed on the Life Sphere, often casts spells around the perimeter to make anything there shrink.
  Replace the first two standard encounters with a miniaturized version (e.g. miniaturized chitincrawlers or bears).
  Elves know how to avoid the effects, and never suffer from them.
  \item
  An elder, focussed on the Light Sphere, often casts illusions of elves.
  Anyone entering will find $2D6$ dancing apparitions scattered around the area, but not responding to them.
  \item
  An elder with the Life Sphere creates enchanted gardens around the area, but always puts them in a new location each time (so the soil never becomes depleted).
  \item
  Useful plants grow all over the region.
  Each of the three dice you have rolled indicate one common plant in the area:
  \begin{dlist}
    \item
    Bedleaves
    \item
    Bloodwood
    \item
    Dryad's Kiss
    \item
    Mage Oak
    \item
    Seekmist
    \item
    Whistling Cane
  \end{dlist}
  (find the plans no \vpageref{plants})
  \item
  An elder, focussed on the Mind Sphere, often casts spell around the perimeter, which confuse anyone who enters.
  They lose a portion of the day, wander aimlessly, and emerge somewhere else, without any memory of what they have done.

  The elder occasionally enchants griffins in the area, allowing a few elves to fly on their backs.
  \item
  A peat-bog furnishes the elves here with iron, allowing them to craft iron weapons.
  \item
  Slow spells from the Life Sphere have crafted massive tree-houses around the area.
  When the lights are on, anyone can see them.
  When the lights go out, they appear like normal trees.
  \item
  Expert artisans live in this community.
  Local clay deposits allow them to craft excellent pottery, including ceramic armour.
  If they have dwarven neighbours, they also trade for gold and silver, and make fine jewellery.
  \item
  Xenophobia has gripped the elves here hard.
  They will kill anyone who enters on sight, and everyone in the land knows it.
\end{enumerate}

\subsection{Gnoll Grounds}

Gnolls typically wander back and forth around a territory.
They will use roads when they see them, but never like to rely on them.

The tribe has $2D6 \times 5$ members, and the same number of hunting dogs.
Roll $2D6$ twice to find a couple of characteristics for the gnolls who wander this ground.

\begin{enumerate}
  \item
  The tribe have taken in a bear-cub, and raised it as one of their own.
  It hunts with them, and understands when to be quiet, and when to alert people to danger.
  \item
  This tribe suffer from isolation.
  Where most tribes have a few members who learn the neighbouring languages, these gnolls only know how to trade with explicit allies, and still cannot speak their languages properly.
  \item
  The lowest neighbour with a fiend has killed a few of these gnolls, and they want revenge.
  \item
  The highest neighbour with civilization (i.e. any spot without a fiend) pays well for the gnolls' hunting dogs, so many of the tribe members have items from them, along with a general feeling of kinship.

  For example, if the highest neighbouring point near them has elves, then they will have elven jewellery, and plenty of elven songs (though the thick accent may not make this obvious).
  \item
  The lowest neighbour with civilization has started a fight with this tribe, and they want revenge.
  Currently, the tribe wants to identify which humans make decisions, and once they know, they will begin the assassinations.

  Their main barrier is that humans change clothes and how they smell regularly, making them hard to re-identify.
  \item
  The tribe have figured out how to use local plants as hair-dye, and they have decided that this is the best thing ever!
  \item
  A powerful druid advises the tribe.
  \item
  These gnolls have some of the best hunting dogs around, and plenty of them.
  Double the number of hunting dogs, so they have twice as many dogs as tribe-members.
  \item
  The gnolls know all the local plants, including where to find marching mushrooms and screechmoss (\autopageref{marching_mushroom}).
  \item
  The tribe keep a herd of aurochs, and wander constantly in order to let them graze.
  They have no fences, but they can still stop the aurochs wandering away -- each one of their dogs knows to stop them escaping.
  \item
  The tribe keep a small herd of sheep.
  \item
  Everyone has contracted \nameref{Torpid Flesh}.
  They are hairless, and look half-dead
  (see \autopageref{Torpid Flesh}).
\end{enumerate}

\subsection{Gnomish Warrens}

Gnomish warrens interact in subtle ways with every point around them.
To find out about a gnomish warren, you should finish every point around it first.
Every point which might benefit the gnomes results in another warren, with some new specialization.

When multiple warrens exist on a single point, they each connect to a nexus cavern below, which allows the gnomes to trade.

\begin{enumerate}
  \item
  If they have gnomish or dwarvish neighbours, they enlarge the natural tunnels beneath them to make an underground road. The ups and downs mean the path is twice as long below ground as above.  It is also not without danger, but underground roads have far fewer creatures than above-ground. (add another warren)
  \item
  If this point has a river and dwarvish neighbours, the gnomes craft an underground stream boat-ride, which goes from the dwarven stronghold, to their warren. (add another warren)
  \item
  If they have a road nearby, they construct a series of underground bothies, each 6 miles apart, to reach the road.  They can walk above ground, but always have somewhere to rest along the way. (add another warren)
  \item
  If a lake lies in the area, they construct a grotto to lay traps for the creatures there.  These traps catch so much that nobody else has much luck fishing. (add another warren)
  \item
  If gnolls live nearby, they trade for hunting dogs, and occasionally ride them when going out. (add another warren)
  \item
  If a dryad lives nearby, they create irritating songs which get stuck in your head, and sing them constantly.  Dryads hate this kind of thing.
  \item
  If an ogre lives nearby, they create pit-traps with spikes at the bottom.  When an ogre steps in one, they cannot walk straight for a week.  If a goblin steps in one, it usually dies.
  \item
  If a hag lives nearby, the gnomes grow wheat, and use an underground mill to make flour, and finally, to bake cakes.  Cakes and a little flattery always works on old crones.
  \item
  If elves live nearby, they arrange for a magical dual (just for sport), every \gls{Atya}.

  The gnomes also add a road between them and the elves.
  Elves don't appreciate roads, but the road serves as a useful warning that if anyone attacks the gnomes, the elves may fall soon after.
\end{enumerate}

\subsubsection{Politics in the Warrens}

Each warren governs itself a little differently.

The number of gnomes in a given warren equals $2D6\times 10$.
The first die indicates the political structure this warren has adopted.

\begin{dlist}
  \item
  Do-opoloy (you decide how something works by making it work)
  \item
  Thaumocratic (alchemists rule)
  \item
  Anarcho-syndicalist (this mostly consists in debating how anarcho-syndicalism works)
  \item
  Technocratic (whoever understands most about a subject decides how it's done)
  \item
  Direct Democracy (dinner takes about 3 hours each night)
  \item
  Indirect democracy (people vote once a week, documentation is crucial)
\end{dlist}


\subsection{Human Settlements}
\index{Towns}

\begin{enumerate}
  \item
  Start with $1D6$ to determine the size of the settlement.
  \begin{dlist}
    \item
    2 \glspl{village} and \pgls{broch} forming a triangle.
    The \gls{broch} stands away from the road, protecting the \glspl{village} from whatever might emerge from the \gls{edge}.

    (civilization rating: 10)
    \item
    4 \glspl{village} with 2 \glspl{broch}.
    (civilization rating: 12)
    \item
    6 \glspl{village} form a circle around a small town.
    3~\glspl{broch} sit opposite the road, to protect the \glspl{village} from the \gls{edge}.

    One of the \glspl{village} began as \pgls{broch} and still has that tall tower.
    \Pgls{warden} stays there, ruling over this tiny kingdom.

    (civilization rating: 14)
    \item
    8 \glspl{village} surround a town, with room to safely grow crops and raise cattle outside its walls.
    4~\glspl{broch} help to harden the outer barrier of settlements which protects the town.

    The local \gls{warden} has plenty of profits coming in, and can afford his own \glspl{sunGuard}, to ensure everyone pays their taxes, so that the \gls{warden} can pay for the \gls{guard}.
   
   (civilization rating: 16)
    \item
    10 \glspl{village} and 5~\glspl{broch} keep the town safe.
    A healthy flow of traders ensures the \glspl{village} by the roads stay relatively clear -- anything which wants to attack will try to stalk those traders before attacking the outer \glspl{village}.

    A couple of the outer \glspl{village} have their own \glspl{warden}.

    (civilization rating: 18)
    \item
    12 \glspl{village} and 6~\glspl{broch} surround a massive town.
    Various hamlets allow farmers to raise beasts in peace.
    The network of roads has plenty of hand-crafted signs, telling people how to get to various areas.

    The town supports a minor army of \glspl{sunGuard}, and the area exports plenty of food.
    Each \gls{village} has \pgls{warden} associated with it, though the \glspl{warden} stay in the inner hamlets.

    (civilization rating: 20)
  \end{dlist}
  \item
  Each \gls{village} and \gls{broch} has a small radius of flattened land, without trees, stretching up to a mile from its walls.
  \item
  Once you have some \glspl{village} and \glspl{broch} down, draw roads between them all.
\end{enumerate}

To discover a settlement's most famous feature, take the number of \glspl{village} and add $1D6$.
For example, if a settlement has 4~\glspl{village}, roll $1D6+4$.

\begin{enumerate}
\setcounter{enumi}{2}
  \item
  Newly built -- every day, stone and wood comes in to build a little more each day.
  Occasionally, parts of the wall collapse and archers flock to protect it.
  \item
  Trackers
  \item
  Deadly and magical plants
  \item
  Ale.
  The \gls{templeOfPoison} has tight control over the area (\autopageref{god:Poison})
  \item
  White rock (actually limestone)
  \item
  \ifodd\value{r4}
    Wood-carvers
  \else
    Metallurgists
  \fi
  \item
 Bards
  \item
  Madness (neighbouring witches are to blame)
  \item
  \Glspl{doula}
  \item
  \ifodd\value{r4}
  Friendly with the next town -- the locals go back and forth all the
    time.
  \else
    Cartographers
  \fi
  \item
  Statues of heroes litter the roads around.
  \item
  A grand \gls{court} where the town \gls{warden} finds beasts guilty, and has them fight for entertainment.
  \item
  Bath houses
  \item
  `Demi-human suburbs', where all manner of other races live.
  \item
  A fight about marriages and road taxes has caused an argument between the town \gls{warden} and the nearest settlement.
  The \gls{sunGuard} will march to war soon\ldots
  \item
  The grandest library in the land.
\end{enumerate}

\paragraph{They call it\ldots{}}

Select a name for the settlement by combining what everyone knows it for, with the type of land.
Write the settlement's name next to it, and underline it.

Any towns might receive the same name, or something related.

\begin{itemize}
  \item
  People know a town in the hills for its library. That's why they call
  it `Page Valley'.
  \item
  A town by the sea, famed for Witchery.
  They call it `Hexwave'.
  \item
  `Bards' + `Islands' = `Lyrisles'
  \item
  \ifodd\value{r4}
    `Metallurgists' + `forest' = `Iron Basin' (perhaps a dried-up lake sits nearby)
  \else
    `Wood-carvers' + `forest' = `Taming Woods'%
  \fi
  \footnote{If you think these names sound stupid, you should look up the meaning of your hometown's name.}
\end{itemize}

Try some word-association with the town's theme or location.

\paragraph{It is Ruled by\ldots{}}
\index{Towns!Rulers}

For each town, roll $1D6$ to determine its ruler.
Write down the information on a key, referencing each point by its number.

\begin{dlist}
  \item
  Multiple \underline{warring factions} -- roll twice more, with +1 to the second roll.
  \item
  A \underline{Warlord}, amassing an army to expand their power.
  Shipments of weapons come from the road to other lands
  (see \vref{roadOut}).
  \item
  An \underline{Opulent noble}, who demands 20\% tithes from all who enter.
  Any neighbouring fiends annoy and fascinate them equally.
  \item
  A new \underline{Sorcerer king} with no experience or business ruling.
  He once killed a fiend, how hard could ruling \pgls{court} and arranged marriages be?

  The sorcerer has killed the same type of fiend as the fiend with the highest number on the map.
  \item
  A \underline{Spoilt noble} who has never seen a farmer in their life.
  \item
  The \underline{servant of the next fiend} on the map (counted by point number).
  All who oppose them disappear on the road.
  \item
  A \underline{Crime-Lord}, with a history of theft, now risen to fortune\ldots but not nobility.
\end{dlist}

\mapentry{Depth}

Your map has everything it needs, but you could always add more.

\begin{itemize}
  \item
  Draw contour-lines around any lakes or sea.
  \item
  Think about what the local populations say about the local oddities from step \ref{mapOddities} -- do they know what lives there?
  \item
  What gossip would travellers hear on the road about local fiends and oddities?
  \item
  Fill in the missing names.
  Human rulers typically take their town's name as their last name, so someone in charge of a town called `Palemarsh' might be called  `\Gls{warden} Cartpike Palemarsh', or similar.
  \item
  What are the roads' names?
  Do you have any room left to write them down?
  \item
  Pull out a pen and start adding depth and shading to the map.
  \item
  Find a small section of your map, covering only two points, and make a new map based on it to give to your players.
  \item
  Follow a road leading off your map.
  Where does it go?

  Start a new map, and connect them by the roads which venture past the page.
  Repeat until you charter the world.
\end{itemize}

\bigLine

\subsubsection{Player Connections}

How are the \glspl{pc} connected to this land?
Make a numbered list of possible past events, with a default and non-default race for each entry.

\begin{itemize}
  \item
  If a human rolls a point where humans live, or a gnoll rolls a point where gnolls live, then they come from that place -- perhaps from a town, or \glspl{village} surrounding it.
  \item
  Otherwise, they may have a connection to a nearby fiend, living between the points.
  Perhaps the fiend destroyed their home and family.
  Perhaps they suffered an attack on the road which still haunts their dreams.
  \item
  Non-humans rolling a human town might have worked with whatever the town is famous for (step \vref{mapCharacter}) -- a gnome might have worked as a tracker, or come to the city to listen to its famous bards.
  Or perhaps a gnoll helped keep the great beasts in a town's \gls{court}.
  \item
  Humans rolling an elven settlement might have been raised there, or lived there for some time after fleeing the law.
\end{itemize}

When the players create their \glspl{pc}, have each one roll on your chart to determine their connection to the land.

\needspace{8em}
Your chart might look something like this:
\begin{enumerate}
  \item
  \underline{Horseshoe Valley}
  \begin{description}
    \item[\textbf{Humans:}] you grew up in a little \gls{village} by Horseshoe Valley, and want to seek your fortune in the wider world.
    \item[\textbf{Others:}] you came as a trader to purchase iron goods, but soon grew scared of travelling the road alone.
    Life in the \gls{guard} now seems safer than travelling with all those goods.
  \end{description}
  \item
  \underline{Elven Meadows}
  \begin{description}
    \item[\textbf{Elves:}]
    you come from the elven villages.
    While the elders gave their blessings, they made you swear never to bring outsiders back home.
    \item[\textbf{Others:}]
    the local hag -- `Thingizard' -- killed your family and tribe.
    You alone escaped, and now dream impotently of revenge.
  \end{description}
  \item
  And so on\ldots
\end{enumerate}

\end{multicols}

\section{Strange Places}

\begin{multicols}{2}

\subsection{Lonely Taverns}
\label{lonelyTaverns}
\index{Lonely Tavern}

These taverns exist on long stretches of road, far from any town, and
charge high prices for a drink. They must live off traders passing
through, and survive whatever the forest brings out. Normal people don't
stay for long. Those who stay a while often have problems with the local
law, as these places often make their own laws. Barkeeps punish any
robbery close to the tavern harshly, but don't often care about what
people do around the towns.

\mapentry{The Barkeep}

Roll $1D6$ to find this season's barkeep (they change all the time):

\begin{dlist}
  \item
  A veteran of the \gls{guard}, with a hundred war-stories. Of course
  when he tells them, nobody can get a drink, so don't ask!
  \item
  Someone from one of the oddities (step \ref{mapOddities}) -- perhaps a gnome; or someone descended from a nearby lost city, who speaks about plans to find it, and return.
  \item
  An outlander from a land so far away, nobody has ever heard of it.
  Every story she tells sounds made-up, but the strange accent shows she really does come from somewhere distant.
  \item
  A powerful mage who swore an oath never to use magic again.
  He won't say why.
  \item
  A collective -- you stay as long as you like, earn your keep, then go
  when you please. Sometimes in the colder Seasons, the place just lies
  barren.
  \item
  A dwarf who records all he can.
  The patrons say he works as a spy for someone, but they disagree about whom.
\end{dlist}

\mapentry{The Menu}
\index{Menu}

Roll $3D6$ -- the \glspl{pc} can order any of these meals.

\newcommand\menuItem[3][(\arabic{r12} \glspl{cp})]{%
  \randomdozen%
  \randomthree%
  \randomfourB%
  \ifodd\value{enumi}
    \randomthreeC%
    \randomfour%
  \fi
  \item
  \textbf{#2:}
  #3
  #1
}

\begin{dlist}
  \menuItem{Griffin-wing}{freshly killed this morning, after the griffin tried to fly away with a gnomish patron.}
  \menuItem{Mystery-stew}{why are you hesitating?
  It goes rotten quick, so get eating!
  \footnote{Chitincrawler `meat' (webbing as sauce!)}
  }
    \ifnum\value{r4}<2
      \newcommand\morningSoup{uproot}
    \else
      \newcommand\morningSoup{marching_mushroom}
    \fi
    \menuItem{Sunrise Soup}{the chef found a new plant this morning, and he's already learning how to cook it!
    \footnote{In fact this is \nameref{\morningSoup}, see \autopageref{\morningSoup}, for the effects.}
    }
  \menuItem{Deer}{thank the man in black, sitting in the corner.}
  \ifodd\value{r4}
    \menuItem{Dwarf-beard}{actually just a type of seaweed, left as payment by a local trader; but it tastes just like the real thing!}
  \else
    \menuItem{Eye-Spy}{made with actual woodspy.}
  \fi
  \ifodd\value{r3}
    \menuItem[(0 \glspl{cp})]{Get bent}{the barkeep's in a foul mood, because they need a day off.}
  \else
    \menuItem[]{\ldots and bugger-all-else}{a few barrels turned out to be rotten, and now someone's stolen an entire pot of soup.
    The menu will be limited for the day.}
  \fi
\end{dlist}
\null

\subsection{Lost Cities}
\label{lostCities}
\index{Lost Cities}

\widePic[t]{Nelness/city}

When calamities destroy a city, some leave quickly, and beasts eat the rest.
The beasts often remain long-term, even for generations, which creates a tantalizing trap for the greedy and curious.
Some treasures always remain in lost cities -- coinage, statues, weapons, and secrets.
Every city has secrets.

People call them \gls{yonder}'s house, as a reminder that most people who enter never return, but remain with \gls{yonder}.
People call them `land \glspl{deep}', to remind the young how deadly an apparently empty city becomes.
Strange creatures always take residence in them.

Before the \glspl{pc} enter, roll dice to make the town with steps \vrefrange{lostCataclysm}{lostTowers}.
Once the \glspl{pc} enter, two rolls occur each \gls{interval}.

\begin{enumerate}
  \item
  You roll to see when dweller interactions occur next.
  During this time, the \glspl{pc} can sneak about more easily.
  (\nameref{lostCries})
  \item
  The \glspl{pc} then try to move quietly through the city, without drawing the attention of those who dwell there.
  (\nameref{lostWhispers})
  \begin{itemize}
    \item
    If the \glspl{pc} fail to sneak, the chase is on!
    (\nameref{lostChase})
    \item
    If the \glspl{pc} go unnoticed, they can attempt to forage.
    (\nameref{lostForaging})
  \end{itemize}
  Most \glspl{pc} will struggle with foraging, but they can get better at it by climbing the towers from step \ref{lostTowers}, or splitting up and making individual rolls.
\end{enumerate}

\mapentry{The Cataclysm}
\label{lostCataclysm}

Everyone fled the city long ago.
The neighbouring towns don't always remember exactly where the city lies, but they always remember the tale of how it fell.

\begin{dlist}
  \item
  They burnt the wrong witch, and her enraged sister came to the city.
  She made plants grow between every brick, and pulled down hailstones in the shape of carrots.
  They killed her in the end, but the walls lay cracked beyond repair, and the central citadel had fallen.
  \item
  A fire started.
  As people banded together to walk outside and escape the smoke, hungry residents of the forest began to watch them, and pick them off, bit by bit.
  Some stayed in the nearby \glspl{village}, but they could not keep everyone inside, so the rest tried to take boats or walk away.

  Many who remained died of smoke inhalation.
  The rest found that walls without guards do little good, as the fire died out, and beasts began to creep in.

  By the time some inhabitants dared to return to the blackened rocks of their city, the forest had claimed the town as its own.
  Then the city's new dwellers grew hungry, and one day the \glspl{village} disappeared.
  \item
  A dragon came, full of hatred and hunger.
  They say it still sleeps somewhere in the city, but who knows?
  \item
  Underground tunnels opened, and little goblins popped up.
  Others followed soon, and one day, not long after, a horde ascended from the \gls{deep} to feast upon the city.
  They've long-since left, but long tunnels remain, going down to the \gls{deep}.
  \item
  Foolishness and bad luck lead to an unlikely series of calamities.
  The local \glspl{warden} began a war with another civilization, and lost.
  With fewer soldiers about, people suffered more casualties from the forest.
  Less food from the \glspl{village} lead to theft, and the \glspl{warden} demanded executions.

  When the city's fighting-beasts escaped their cages, and fled from the arena through the streets, people decided they could take no more, and many left.
  With a reduced population, limited food, and walls too long to properly guard, the city collapsed in on itself, bit by bit.
  \item
  \ifodd\value{r4}
    Local witches had warned not to expand the \glspl{village} -- the nearby \glspl{village} had strange ideas, and too many developed sorcery of some kind, when the secrets were whispered to them in dreams.

    No great even happened here -- the population simply shrank more than it grew, with a thousand little catastrophes.

    Some say spirits haunt the place.
    Others warn not to drink the river water.
    Whatever the truth, that land is cursed.
  \else
    A flood wiped away half the \glspl{village} and collapsed the town's up-river wall.
    Survivors crowded around rooftops to the see the architect hanged, as the streets had all turned to rivers.

    By nightfall, woodspies were picking people off their houses like a casual snack.
    Soon, the remainder left.
    The local \glspl{warden} planned to rebuild the outer wall, but could never raise the money to start rebuilding.
  \fi
\end{dlist}

\mapentry{Wandering Dwellers}

Roll $3D6$ -- each number which comes up determines one type of city-dweller.
Ignore any repeats.

\begin{dlist}
  \item\label{lostOoze}
  \textbf{Oozes} of all types slide across the streets, which seem strangely clean, and free of debris or foliage.
  \item
  \textbf{Chitincrawlers} hide in every abode.
  Verdant berries of every colour have encouraged deer into the area, but every shadow gleams with thick webbing.

  $1D6+8$ live here in total.
  \item
  \textbf{Griffins} look down from every tower.
  The high towers make perfect perches to surveille the area, and the degrading wood helps to make nests.

  $1D6+4$ live here in total.
  \item
  \textbf{Crazed Witches} occasionally enter these cities, looking for spell components.

  Roll $1D6$ to find their number.
  \item\label{lostDemilich}
  \textbf{Demilich} covens help these undead sorcerers study with their own kind.
  However, their lack of basic empathy makes them dangerous to each other -- none of them really trust the others, so they share information slowly, always hinting that they have more to teach while masking their true abilities.

  Roll $1D3 + 1$ to determine the number of demiliches.
  \item
  \textbf{Dragon} eggs make for powerful magical items, so many dragons like nesting in strange and dangerous areas.
  Of course, only the largest of buildings, such as town halls, or theatres, can house such a massive creature.
\end{dlist}

\mapentry{Dweller Relations}
Each dweller in a lost city has a relationship with the others, depending on their relative numbers.
(For example, oozes are \ref{lostOoze}, and the demilich is \ref{lostDemilich}, so their relation is `\ref{lostDemilich} minus \ref{lostOoze}'.)

Those with a higher number control or prey on those with a lower number.

\begin{enumerate}
  \item
  Dwellers within \arabic{enumi} step of each other become allies if sentient, and otherwise ignore each other.
  \item
  A dweller numbered \arabic{enumi} greater than another becomes aggressive, and the two begin to fight.
  \item
  Dwellers \arabic{enumi} ahead of another kill the lower form, and use the bodies to feed, or cast spells.
  \item
  Dwellers \arabic{enumi} ahead of another use magic or threats to control the lesser dweller.
  \item
  Any dweller \arabic{enumi} ahead of another will ignore them entirely.
\end{enumerate}

Draw a line, triangle, or square, to show the relations of the dwellers in the lost city.

\cityRelations

\mapentry{Tall Towers in the Labyrinth}
\label{lostTowers}

Wooden rooves rot, leaving bear stone walls across much of the fallen city.
Some walls collapse, opening new passages, while trees grow and thorny bushes grow to block doors and streets.
A twisted, mossy, labyrinth forms.

Despite the rot and chaos, some buildings still stand tall.
Climbing these buildings gifts a wide perspective of the city, which grants a bonus to the \glspl{pc} foraging rolls.

Roll $4D6$: one for each tower.

\begin{dlist}
  \item
  This tower has fallen to the ground, leaving nothing worth climbing.
  \item
  This tower has one wall remaining -- a \roll{Speed}{Athletics} check (\tn[8]) allows the climber to see a little in all four directions.
  \begin{itemize}
    \item
    During daylight, the perspective adds +1 to all foraging rolls.
  \end{itemize}
  \item
  This tower has a single candle, and the light shines brightly at night.
  If anyone enters, any sentient city-dwellers in the area will see them blocking the light when they enter the room, and investigate.
  (If there are many, pick the lowest numbered dweller)
  \begin{itemize}
    \item
    If no sentient life exists in the area, the candle has been left by a traveller who has left a note stating which creatures he has seen.
    \item
    All further towers on the roll of a \arabic{dlist} create an encounter with the dwellers with the lowest number.
    \item
    During daylight, the perspective adds +1 to all foraging rolls.
  \end{itemize}
  \item
  This tower stands tall enough to see all around.
  The bones of dead humans fill the stairway.

  Any movement which disturbs the bones will send them falling down the stairs with a clank and a thud.
  \Glspl{pc} must roll \roll{Dexterity}{Stealth} (\tn[11]) to move past the bones without issue, or else waste \pgls{interval} moving everything down by hand (in this case they roll \roll{Dexterity}{Crafts}, \tn[7]).
  \begin{itemize}
    \item
    During daylight, the perspective adds +2 to all foraging rolls.
  \end{itemize}
  \item
  The tallest of towers allow anyone inside to see all around.
  They can receive a complete list of towers in the area.
  \begin{itemize}
    \item
    During daylight, the perspective adds +2 to all foraging rolls.
    \item
    The first time \pgls{pc} climbs to this height, they see two dwellers interacting.
  \end{itemize}
  \item
  The tallest towers often have the least stability.
  The \gls{pc} who climbs this far will receive the same benefits as above, but if their \gls{weight} is above 5, the structure begins collapsing.

  They can run down (\roll{Speed}{Athletics}, \tn[9]), but failure indicates the building has collapsed upon them, inflicting $4D6$ Damage.

  Or the \gls{pc} can elect to jump, and try to grab something nearby (\roll{Speed}{Athletics}, \tn[12]), but failure indicates that they will suffer a nasty fall of $2D6$ Damage.
\end{dlist}

\bigLine

\subsubsection{Cries in the Night}
\label{lostCries}

When dwellers interact, they often make noise.
Witches casting spells upon local oozes, or dragons ordering minions about can't be done quietly; although the \glspl{pc} will probably not understand what the noises mean.

Every $1D6$ \glspl{interval}, the \glspl{pc} hear an interaction.

\subsubsection{Whispers in the Cold Labyrinth}
\label{lostWhispers}

When the \glspl{pc} want to move through a lost city, they test \roll{Dexterity}{Stealth} to remain quiet.
The \gls{tn} is normally 8, but often easier.

\begin{itemize}
  \item
  -1 if moving at night.
  \item
  -1 in the rain.
  \item
  -1 in a storm.
  \item
  -2 if the dwellers are busy with something.
  \item
  -4 if hiding indoors.
  \item
  +5 if lighting a fire.
\end{itemize}

\subsubsection{The Chase}
\label{lostChase}

Any time the \glspl{pc} accidentally make a noise, they can try to gingerly run and hide.

\begin{itemize}
  \item
  They can try to escape without making more noise with another \roll{Speed}{Stealth} check, this time at \tn[10].
  \item
  The dweller with the lowest number arrives the first time the \glspl{pc} fail a check, then the next dweller on the list, and so on.
  \item
  Once all dwellers have been alerted, the first two arrive to investigate the noise at the same time.
  From then on, all dwellers come to investigate at every noise.
  \item
  Up to $1D3$ dwellers emerge at a time.
\end{itemize}

\subsubsection{Foraging Quietly}
\label{lostForaging}

Each time the \glspl{pc} forage, they roll \roll{Intelligence}{Vigilance} (\tn[12]) to figure out where the best loot lies.
If they succeed, roll $2D6$ -- the first die determines the place, and the second determines the prize!

\begin{itemize}
  \item
  Once they have a prize, cross it off the list.
  \item
  If they get the same number again, they will find the second prize with that number.
  \item
  The third time they roll a number, they find nothing -- lost cities only hold so many treasures.
  \item
  If the \glspl{pc} split up, each one can perform individual rolls, so they will loot far more efficiently, but run more chance of one getting caught alone.

\end{itemize}

\subsubsection{Climbing the Walls}

\Glspl{pc} might feel a temptation to climb the outer walls to get a proper perspective of the city, but doing so leaves them visible to every dweller within.
Move to `the Chase' immediately, but at \tn[12] to hide.

\end{multicols}

\foragingChart


\chapter{Constructed Minds}

What we think of as `mind' arrives covered in so many irrelevant layers, obscuring its essence.
An unfiltered mind would seem essentially alien.
Standard minds feed, fight, flee, and fuck.
They stop to rest, then start to move, depending on the Sun.
Their desires change, depending on thing inside them, such as their stomach, and the minds around them understand what has changed, despite not seeing inside their stomach.

But minds made by magic -- \glspl{artefact} and the undead -- have none life's filth, so their purity makes them strange.

\section{The Essence of Mind}

\begin{multicols}{2}
\renewcommand\npcsymbol{\D}

\subsection[The undead are deaf and blind, but see by the light of living souls]{Senses}
\label{undead_senses}
\label{artefact_senses}

Undead sensory organs do not work.
Their eyes do not see, and their ears do not hear.
They are so completely deaf most will forget about the existence of sound.
However, they can see a hidden light -- normally blocked by the walls of the flesh -- the light which comes from the mind itself; a light which every soul emanates at all times.

When no living soul is near, the undead walk blindly, and must feel the world by noticing how it presses back against their bodies.
They can faintly make out the little lights of animals, and perceive other undead similarly.

When a living person approaches, unfiltered minds see them like a beacon so bright that looking at it hurts.
Only this light allows the undead and \glspl{artefact} to see properly.

If an undead creature knows how to read, it can only read by the light of a soul; their own light isn't bright enough for the task.
Some sentient undead have a store of living humans, so they can place one in a cage within their workshop in order to see.

Without skin, magical constructions cannot feel any more than the movement of their bodies.
They remain unaware of heat or texture.

\subsection{Communication}
\label{dead_communication}

\Gls{artefact} don't speak, and the undead lose their ability to communicate rapidly, as parts rot and they forget how to breathe.
But naked souls have a natural means of communication.
Powerful or insistent thoughts will alter the light that their own mind gives off, forming wave-like patterns, or blotches.
Any two minds -- whether \gls{artefact} or undead -- which spend enough time with each other will quickly learn to communicate, if they have any interest in each other.

These thought-patterns come by instinct, not as an arrangement of symbols to construct a language.
In theory, this forms a kind of natural language of the universe.
But in practice, the thoughts and feelings of these uncivilized minds cannot communicate any more than an observer has known themselves, which is always a narrow subset of the thinker's thoughts.

\end{multicols}

\section[The Undead]{The Undead \D}
\index{Undead}
\label{undead}
\begin{multicols}{2}
\subsection{Properties}

\subsubsection{Awakening}
starts by learning how their body works.
They begin with a -2 Dexterity Penalty, which slowly changes to a -2 Speed Penalty as they learn how to walk while \textit{rigor mortis} sets in.

\subsubsection{Undead Bodies}
feel no pain and suffer little from scrapes and bruises.
As a result, they automatically have a \gls{dr} of 2 which stacks with armour in the usual way.%
\exRef{core}{Core Rules}{bandingArmour}
This `armour' has \gls{covering}~5, as attacks which rip out large sections of muscle, or damage the heart, still debilitate them.

However, they cannot move forever.
Each \gls{interval} of activity wears out the body if they have not fed.
Every ten \glspl{interval}, they permanently lose 1~\gls{hp} to rot.

Undead created with a \textit{Preservation} spell can remain whole forever if they resist moving.
The rest rot at the standard rate of any corpse, depending on the climate.

\subsubsection{Hunger}
pulls the undead towards the light of living souls by instinct.
Their mechanical bodies feed from life as it spills out, just by being near.
They might look like they feed from flesh, as they chomp through a neck or face, but that's just the fastest way to `crack the walnut'.

Undead spellcasters cannot absorb \glsentrylongpl{mp} from the air, so they rely on feeding in this way to recharge.
They gain 6~\gls{mp}, plus all \glspl{mp} a spellcaster has unspent.

Undead which cannot cast spells still feel this additional energy, and will always approach spellcasters before others; at least while they have \glspl{mp} unspent.

\subsubsection{Aggression}
comes naturally to the undead, from their hunger and from their complete inability to fear.
Each begins with a minimum Brawl Skill at +2.

\subsection{Specimens}


\newBeast[\D]{Ghouls}% Name
  {ghoul}% label
  {a mindless corpse, possessed by a hungry spirit}% description
\index{Undead!Human}
are the bread and butter of necromancers.
When they first rise from the dead, they stumble clumsily, and chase after any living humanoid.
With nobody nearby, some just stand there, and most wander aimlessly.
Within hours, the undead get used to their form but their bodies seize up from their dead state, and they start to give off the standard stench of death.

The given example is a standard human ghoul.
To make ghouls of other races, simply change the default Strength bonus -- dwarves have Strength 0, gnomes have Strength -2.

\ghoul[\npc{\D}{Ghoul}]

\paragraph{Tactics:}
Ghouls attack in swarms and almost always grapple their targets as a first attack in order to claw and bite in later attacks.
Where most creatures wouldn't grapple people mid-battle, because it makes them vulnerable, these non-sentient undead have no sense of self-preservation.
They grab, bite, and the other dead have an easier time assaulting the target after that.%
\iftoggle{core}{%
  \footnote{See the core rules, page \pageref{grappling}, for grappling rules.}
}{}

\paragraph{Tactics:}
They say humans are `endurance hunters' -- that they cannot outrun any animals, but they can keep walking for days, while most animals tire.
Deer may outrun a human for a day or two, but eventually, they run out of energy, and lay down.
Ghouls work in the same way.

\begin{exampletext}
  Dead, stiff bodies lumber towards a group of travellers, who quickly flee.
  For ten minutes, the travellers sprint, then land; exhausted.
  They stand and walk for another hour, to make sure they have enough distance.
  They have heard nothing in all this time.
  They have seen nothing but darkness.
  But they never understand that they only made space between themselves and the dead while sprinting.

  Once the dead emerge again, they sprint again, but this time a little slower, and they rest a little longer.
  But the dead keep walking, at a steady pace\ldots
\end{exampletext}

\newBeast[\D]{Ghasts}% Name
  {ghast}% label
  {people who died while semi-undead; parts of the mind remain, but they only think about drinking murder}% description
\index{Undead!Human Ghast}
are often masterworks created by a necromancer as a personal body-guard or the lead warrior in an army.

\paragraph{Ecology:} They are fully sentient, but often too intent upon feeding on humanoid souls to do much except obsess over murder.
Occasionally, one will escape and control its urges enough to find a secluded spot and simply try to exist.
These creatures often end up haunting local crypts, mines or other forgotten areas.

\paragraph{Encounters:}

\begin{itemize}

  \item
  Two ghasts hide among a pack of standard ghouls.
  They wander slowly, at the back of the pack, then suddenly unsheathe their swords and attack ferociously.
  \item
  A ghast kills travellers, then lays down with the dead, pretending to be one of their number.
  When the troupe arrive and examine the dead, it jumps the weakest member while they're off-guard.
  \item
  Using its ability to see the living at all times, a ghast stalks the troupe from a long distance.
  It waits until they enter combat with some other force, and jumps in as soon as they become injured.

\end{itemize}

\ghast[\npc{\D\E}{Ghast}]

Ghasts under the control of a necromancer can plan admirably, and often pretend to be yet another one of the dead.

Independent ghasts tend to stalk prey from afar, waiting for something else to wound or tire them.

\newBeast[\D]{Demiliches}% Name
  {demilich}% label
  {a spellcaster who has become undead}% description
\index{Undead!Demilich}
are the worst scum -- they understand how the undead make people suffer, and decide to become part of the problem.
Demiliches are the first stages of a lich, as they ascend to become a machine of pure power, gathering undead forces, powerful spells, and a formidable lair.

\paragraph{Ecology:} While in theory these creatures can live anywhere, most live in secluded areas.
Deep caves, horrible deserts, or icy mountains provide excellent spots for the dead because they are such difficult areas for the living.
Snowy wastelands, filled with frozen corpses, ready to walk again once called, can provide the perfect location for an ice palace.

\paragraph{Encounters:}

\begin{itemize}
  \item
  The demilich covers a dozen corpses in pitch and sends them into \pgls{village}.
  Once the houses start to burn, it plans to send ghasts to pick off fleeing villagers.
  The dead move easily through the smoke, but living creatures suffer 2 \glspl{ep} each round they wander between houses, and 4 when inside the houses.
  \item
  A demilich wanders with its horde, and spots the characters a mile away.
  It begins by cursing them, draining them of their \glsentrylongpl{fp}.
  If they flee, it wanders after them, keeping its small enclave of ghouls with it at all times.

\end{itemize}

\paragraph{Tactics:}
Step 1: Gather corpses to make ghouls.
Step 2: Use ghouls to gather more powerful corpses.
Step 3: Repeat.

When Demiliches arise, they require somewhere as a base of operations.
Too far from civilization is bad as they cannot gather the corpses necessary to raise an army.
Too close is also bad as they cannot hide.

Often they turn the first corpses they find into ghasts -- powerful, sentient, undead.
They occasionally go on missions for more powerful corpses, such as strong humans, bears, ogres, or horses.

\demilich

\end{multicols}

\needspace{20\baselineskip}
\section[\Glsfmtplural{artefact}]{\Glsfmtplural{artefact}~\E}

\begin{multicols}{2}

\subsection{Properties}

\subsubsection{Awakening}
begins when someone creates \pgls{talisman}.
Every spell starts with a clear intention, and \glspl{talisman} are spells.
\Pgls{talisman} might wait for the right word to open a gateway; or look out for a man holding a torch, so they can expend their energy on putting it out.
And then they perish, like a spent bee.

But some \glspl{talisman} never succeed, so they just keep looking about for their one thing -- the gateway, or the man, or healing \pgls{sunGuard} soldier.
And given enough time, some start to theorize, then learn.
At this point, they begin to learn magical Spheres.

\Pgls{talisman} which waits to open a gateway underground always learns Earth and Fire, so it can learn Force.
Some even skip straight to the Force Sphere, without any idea that it has constituent parts.
So now the thing has broader plans and flexible tools.
Now it can absorb mana from the wind.
Now it has become \pgls{artefact}.

\subsubsection{The Goal}
which creates \pgls{talisman} remains in the \gls{artefact}, and grows stronger as it becomes more intelligent.
They will seek to maximize it, to repeat it, perfect it, and multiply it eternally.

\begin{itemize}
  \item
  \Pgls{talisman} created to help someone hide with a Light spell will become \pgls{artefact} with the Fire and Earth Spheres.
  These work well for burning someone's body, and pulling the corpse underground.
  Now \textit{nobody} will find them.
  Mission success!
  \item
  \Pgls{talisman} made to curse people will become \pgls{artefact} that wants to see the curse take hold.
  So when someone picks it up, it may resist cursing them, and consider how to have itself placed in view of \pgls{village} or town, so it can begin casting wide curses on everyone it sees, every day.
  \item
  \Pgls{talisman} made to keep a family safe may become \pgls{artefact} that wants to maximize the members of the family, encouraging as many marriages and births as it can.
  Alternatively, it may seek to kill all family members but one; because looking after one person is a lot less work than looking after dozens.
\end{itemize}

\subsection{Specimens}

\artefact[2]{Floating Discus}% Name
  {A wide, bronze-inlaid, feasting-bowl, 1~\gls{step} wide.}% Body
  {2}% Intelligence
  {0}% Wits
  {3}% Charisma
  {to carry master's items downhill}% Mission
  {Migrating Butterfly}% Base Spell
  {
    \setcounter{Fire}{2}
    \setcounter{Earth}{2}
    \setcounter{Water}{2}
    \setcounter{Vigilance}{2}
  }% Spheres

A weak and lazy \gls{doula} decided she could make better money as a grave-robber, and joined a bandit group to go grave-robbing in an abandoned dwarven settlement.
She created an ornate disk, with bronze and wood, to float away with the most expensive prizes she could find, taking them to her home, downhill.

The lich which had killed the dwarves also killed her, so she never got to use it.
The lich grew tired of hearing it babble about taking items, but it continued listening, and picked up some ideas about magic.

Once picked up, it will decide that whoever holds it must be `the master', and that it must take their items `down hill'.
After this, it waits until it is somewhere high, then use the Force sphere to make itself float, and wait for the master to put items in it (if anyone else puts items inside, it turns upside down).
Once it has as many items as it can carry, it floats downhill as far as it can, dumps the items, and then returns.

Once the job starts, it returns and continues, employing every spell it has to take as many items as possible `down hill'.

\showStdSpells

\artefact[0]{Little Vial Pendant}% Name
  {A phial of golden flecks, suspended in a thick blue liquid}% Body
  {0}% Intelligence
  {0}% Wits
  {3}% Charisma
  {to carry master's items downhill}% Mission
  {Atrophy}% Base Spell
  {
    \setcounter{Earth}{2}
    \setcounter{Water}{2}
    \setcounter{Caving}{3}
  }% Spheres

A mine prospector struggled squeezing through small openings, so he asked \pgls{doula} to make this item for him.
He never got to drink the potion to make him small, so it stayed, unused, and unloved, around a goblin's neck.

After it grew tired of goblins passing it about, it learned how to make them smaller.
It has decided that minified goblins are the best thing in the world, and will do anything to find more goblins, and make them as small as possible.

It currently does not know about gnomes, but will change its plans if it ever finds out about them.

\showStdSpells

\artefact[1]{Striker}% Name
  {A fist-sized wooden statue of a man with eyes closed, hands forward, palms out}% Body
  {2}% Intelligence
  {2}% Wits
  {-1}% Charisma
  {To put out a forest fire}% Mission
  {Party Kill}% Base Spell
  {
    \setcounter{Fire}{3}
    \setcounter{Survival}{1}
  }% Spheres

\Pgls{seeker} continuously worried about creatures sneaking up on him in the forest while his troupe made camp.
He created this little figurine, with a single, easy task -- put the fire out once a monster comes.

Monsters have come and gone quite a lot since then.
The first time, the \gls{seeker} did not have a fire lit, so Striker did nothing, but intends to make up for every beast it has ever seen by putting out a massive fire.

The grand plan is simple:

\begin{enumerate}
  \item
  Start a forest-fire.
  \item
  Put the forest-fire out.
\end{enumerate}

\showStdSpells

\end{multicols}

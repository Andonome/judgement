\newlength{\mapcircle}

\setlength{\mapcircle}{.32\textwidth}

\newcommand\mapSkeleton{

  \centering

  \vspace{1\mapcircle}

  \begin{tikzpicture}
    \Large
    \node[draw, color = background, minimum size = .9\linewidth, regular polygon, regular polygon sides = 5] (a) {};
    \foreach \x in {1,2,...,5} 
        \fill (a.corner \x) circle [radius=2pt];
    \draw (a.corner 1) node[above] {1};
    \draw (a.corner 2) node[left] {5};
    \draw (a.corner 3) node[below] {4};
    \draw (a.corner 4) node[below] {3};
    \draw (a.corner 5) node[right] {2};
    \draw [color = background, name path = seg1] (a.corner 1)--(a.corner 3);
    \draw [color = background, name path = seg2] (a.corner 3)--(a.corner 5);
    \draw [color = background, name path = seg3] (a.corner 5)--(a.corner 2);
    \draw [color = background, name path = seg4] (a.corner 2)--(a.corner 4);
    \draw [color = background, name path = seg5] (a.corner 4)--(a.corner 1);

    \path [name intersections = {of = seg1 and seg3, by = inter1}];
    \fill (inter1) circle [radius = 2pt];
    \draw (inter1) node[below] {6};

    \path [name intersections = {of = seg3 and seg5, by = inter2}];
    \fill (inter2) circle [radius = 2pt];
    \draw (inter2) node[below] {7};

    \path [name intersections = {of = seg2 and seg5, by = inter3}];
    \fill (inter3) circle [radius = 2pt];
    \draw (inter3) node[left] {8};

    \path [name intersections = {of = seg2 and seg4, by = inter4}];
    \fill (inter4) circle [radius = 2pt];
    \draw (inter4) node[below] {9};

    \path [name intersections = {of = seg1 and seg4, by = inter5}];
    \fill (inter5) circle [radius = 2pt];
    \draw (inter5) node[right] {10};

  \end{tikzpicture}

  \vspace{\mapcircle}

}

\newcommand\encRef[1]{%
  \nameref{#1} (\autopageref{#1})%
}

\newcommand\foragingChart{
  \begin{nametable}[c|L|L|L]{Foraging}

    \textbf{Roll} & \textbf{Place} & \textbf{Prize 1} & \textbf{Prize 2} {\footnotesize(\glspl{talisman} \vpageref{talismanIndex})}\\\hline

    \dicef{1} &
      Under a hidden floorboard -- broken patches show something underneath &
      \lootJewellery &
      \lootTalisman \\

    \dicef{2} &
      Hidden room, behind an old bookcase &
      \lootJewellery &
      \lootJewellery, and \lootBig \\

    \dicef{3} &
      In empty home, inside a dark and empty doorway &
      \lootJewellery, and \lootJewellery &
      \lootJewellery, and \lootBig \\

    \dicef{4} &
      Inside a largely preserved house.
      The family died peacefully, somehow\ldots &
      \lootMedium{} and, surprise, \gls{woodspy}! (\autopageref{woodspy}) &
      \lootJewellery \\

    \dicef{5} &
      Behind a fully stone door, clearly used as a safe.
    Opening it requires Intelligence + Crafts (\gls{tn} 12 to do so quietly, otherwise, \gls{tn} 7) & \lootJewellery & \lootTalisman \\

    \dicef{6} &
      Lying under a pile of human bones &
      \lootBig &
      \lootTalisman \\

  \end{nametable}
}

\newcommand\encSummary{

  \begin{tikzpicture}[thick]

  \node (A) at (0,0){\Large\dicef{6}};

  \node (B) at (4,0){\Large\dicef{1}};

  \node (C) at (2,2.5){\Large\dicef{3}};

  \draw[background] (A) to[bend left] node[right]{\large\outline{= 9}}(C);

  \draw[background] (C) to[bend left] node[right,align=center]{\large\outline{= 4}}(B);

  %\draw (B) to[bend left]node[below]{\outline{7}} (A);

  \draw (A)+(0,-2em) node[below,text width=6em]{Time to next encounter};

  \draw (B.east)+(0,-2em) node[below,text width=6em]{Number \\ appearing};

  \end{tikzpicture}

}

\newcommand\labelledDiceTrio[6]{
  \vspace{1em}
  \setcounter{diceNo}{#1}
  \addtocounter{diceNo}{#2}
  \setcounter{diceNo2}{#2}
  \setcounter{age}{#3}
  \addtocounter{diceNo2}{#3}
  \addtocounter{age}{#1}
  \begin{tikzpicture}[thick]

  \node[color = foreground] (Left) at (2,2.5){(\arabic{age}) \Large\sffamily #4};

  \node[color = foreground] (Right) at (4,0){(\arabic{diceNo2}) \Large\sffamily #5};

  \node[color = foreground] (Top) at (0,0){(\arabic{diceNo}) \Large\sffamily #6};

  \draw[background] (Top) to[bend left] node[left]{\large \dicef{#1}}(Left);

  \draw[background] (Left) to[bend left] node[right]{\large \dicef{#3}}(Right);

  \draw[background] (Right) to[bend left] node[below]{\large \dicef{#2}} (Top);

  \end{tikzpicture}
}

\newcommand\diceTrio[6]{
  \vspace{1em}
  \setcounter{diceNo}{#1}
  \setcounter{diceNo2}{#2}
  \setcounter{age}{#3}
  \begin{tikzpicture}[thick]

  \node (Left) at (2,2.5){\Large\sffamily #4};

  \node (Right) at (4,0){\Large\sffamily #5};

  \node (Top) at (0,0){\Large\sffamily #6};

  \draw[background] (Top) to[bend left] node[left]{\large \dicef{#1}}(Left);

  \draw[background] (Left) to[bend left] node[right]{\large \dicef{#3}}(Right);

  \draw[background] (Right) to[bend left] node[below]{\large \dicef{#2}} (Top);

  \end{tikzpicture}
}

\newcommand\showBandits{
  \item
  \setcounter{gold}{\value{diceNo}}
  \multiply\value{gold} by 2
  \addtocounter{gold}{4}
  \arabic{gold} Bandits
}

\newcommand\encBandits{
  \begin{multicols}{2}
  \begin{dlist}
    \showBandits
    \showBandits
    \showBandits
    \showBandits
    \showBandits
    \showBandits
  \end{dlist}
  \end{multicols}
}

\newcommand\encTraders{
  \label{traderWares}
  \begin{multicols}{2}
  \begin{dlist}
    \item
    \ifnum\value{temperature}>0
      Fresh fruits (\arabic{r2}0~\glspl{cp} per day's worth, but they go rotten after 2 days)%
    \else
      Smoked meats (\arabic{r4}0~\glspl{cp} per day's worth)%
    \fi
    \randomize
    \item
    Bows (\arabic{r12}~\glspl{sp})
    and arrows (\arabic{r12}0~\glspl{cp})
    \randomize
    \item
    \Glspl{torch} (\arabic{r12}~\glspl{cp})
    \randomize
    \columnbreak
    \item
    Salted meats (\arabic{r12}~\glspl{cp} per day's worth)
    \randomize%
    \item
    Rope (\arabic{r12}0~\glspl{cp})
    \randomize%
    \item
    Caged beast (use the creature from the encounter number)
  \end{dlist}
  \end{multicols}
}

\newcommand\stepDown[1]{
  \ifnum\value{#1}>2
    \addtocounter{#1}{-1}
    \ifnum\value{#1}<13
        \arabic{#1}
    \fi
  \fi
}

\newcommand\encLine{
  \stepDown{track} & \stepDown{diceNo} & \stepDown{diceNo2} &  \stepDown{encnum} & \stepDown{enc} &
}

\newenvironment{encChart}[1]%
  {
    \setcounter{track}{13}
    \setcounter{diceNo}{15}
    \setcounter{diceNo2}{17}
    \setcounter{encnum}{19}
    \setcounter{enc}{21}% Encounter charts display creatures based entirely off this counter.

    \rowcolors{2}{}{gray!10}
    \vspace{2em}
    \tabularx{\linewidth}{c|c|c|c|c|L}
      \hline
      \hline
      \textbf{Hot} & \textbf{Warm} & \textbf{Mild} & \textbf{Cold} & \textbf{Frigid} & \textbf{#1} \\
      \hline
  }%
  {
      \hline
    \endtabularx
    \vspace{2em}
  }%

\newcommand\bigWeatherList{
  \ifcase\value{enc}\or%
  \or%
    Hurricane (4~\glspl{mp})%
  \or%
    Biting winds (4~\glspl{mp})%
  \or%
    Mist%
  \or%
    \iftoggle{genExamples}{
      Overcast (or cold snap during \gls{cTwo})
    }{
      \ifnum\value{temperature}=0%
        Cold snap%
      \else%
        Overcast sky%
      \fi%
    }
  \or%
    Dead calm (2~\glspl{mp})%
  \or%
    Mild breeze%
  \or%
    \iftoggle{genExamples}{
      Clear skies (or heat wave during \gls{cFive})
    }{
      \ifnum\value{temperature}=3%
        Heat wave%
      \else%
        Clear skies
      \fi%
    }
  \or%
    \iftoggle{genExamples}{
      Light rain (or snow over \gls{cTwo})
    }{
      Light
      \ifnum\value{temperature}=0%
        snow%
      \else%
        rain%
      \fi%
    }
  \or%
    Thunder%
  \or%
    Lightning storm (4~\glspl{mp})%
  \or%
    \iftoggle{genExamples}{
      Torrential rain (or heavy snow during \gls{cTwo})
    }{
      \ifnum\value{temperature}=0 Heavy snow%
      \else%
        Torrential rain%
      \fi%
    }
  \else%
    Error
  \fi
}

\newcommand\bigBeastList{
  \ifcase\value{enc}\relax%
  \or%
    Hibernating \iftoggle{genExamples}{creature ($1D6+7$)}{%
    \ifodd\value{r3}%
      \gls{crawler}%
    \else%
      \gls{woodspy}%
    \fi%
    }%
    \setcounter{encnum}{0}%
  \or%
    $1D3$ Goats%
    \setcounter{encnum}{0}%
  \or%
    $1D6+6$ Wolves\iftoggle{genExamples}{ (\autopageref{wolf})}{}
    \setcounter{encnum}{2}%
  \or%
    $1D6\times 20$ Aurochs\iftoggle{genExamples}{ (\autopageref{auroch})}{}
    \setcounter{encnum}{1}%
  \or%
    Boar\iftoggle{genExamples}{ (\autopageref{boar})}{}
    \setcounter{encnum}{1}%
  \or%
    $1D6$ \Glspl{griffin}\iftoggle{genExamples}{ (\autopageref{griffin})}{}
    \setcounter{encnum}{0}%
  \or%
    \Gls{digger}\iftoggle{genExamples}{ (\autopageref{mouthdigger})}{}
    \setcounter{encnum}{\value{r4}}%
  \or%
    \Gls{woodspy}\iftoggle{genExamples}{ (\autopageref{woodspy})}{}
    \stepcounter{r4}%
  \or%
    Bear\iftoggle{genExamples}{ (\autopageref{bear})}{}
    \setcounter{encnum}{3}%
  \or%
    \Gls{crawler}\iftoggle{genExamples}{ (\autopageref{chitincrawler})}{}
  \or%
    \Gls{basilisk}\iftoggle{genExamples}{ (\autopageref{basilisk})}{}
    \setcounter{encnum}{\value{temperature}}%
  \or%
    Stirges\iftoggle{genExamples}{ (\autopageref{stirge})}{}
    \setcounter{encnum}{0}%
  \or%
    Bluepins\iftoggle{genExamples}{ (\autopageref{bluepin})}{}
    \setcounter{encnum}{1}%
  \or%
    Strange plant\iftoggle{genExamples}{ (\autopageref{plants})}{}
    \setcounter{encnum}{0}%
  \or%
    \Gls{crawler} swarm\iftoggle{genExamples}{ (\autopageref{crawlerSwarm})}{}
    \setcounter{encnum}{1}%
  \or%
    Weird fungus\iftoggle{genExamples}{ (\autopageref{fungi})}{}
    \setcounter{encnum}{0}%
  \or%
    $3D6$ Deer
    \setcounter{encnum}{0}%
  \or%
    \Gls{hulk}\iftoggle{genExamples}{ (\autopageref{umber_hulk})}{}
    \setcounter{encnum}{1}%
  \else%
    Dragons!%
  \fi%
}

\newcommand\encCivilization[1][\value{encnum}]{%
  \ifcase#1\relax%
  \or%
    On the \gls{lonelyRoad}
  \or%
    By \pgls{broch}%
  \or%
    Outer \glspl{village}%
  \or%
    Inner \glspl{village}%
  \or%
    By a town%
  \else%
    In a town%
  \fi%
}

\newcommand\weatherChart{
  \label{encWeather}
  \setcounter{enc}{13}
  \begin{nametable}{Weather}
    \Repeat{11}{%
      \addtocounter{enc}{-1}%
      \arabic{enc} & \bigWeatherList \\
    }
  \end{nametable}
}

%%%%% Full Encounters %%%%%

\newcommand\encTownEvents{
  \index{Towns!Events}
  \index{Events!Towns}
  \label{townEvents}
  \setcounter{diceNo2}{6}
  \setcounter{diceNo}{6}
  \begin{wideTable}[c|c|LL]{Town Events}
  \hline
  \textbf{Day} & \textbf{Night} & \textbf{Market} & \textbf{Tavern} \\
  \hline
  \Large\dicef{6} & &
              \Pgls{doula} offers to sell \glspl{boon} for 3 \glspl{sp} each. &
              Dead: nobody there.
              \\
  \Large\dicef{5} & &
              \Pgls{doula} wants to purchase \glspl{ingredient}, for 50 \glspl{cp} each. &
              Mostly dead: a single barman, and an alcoholic.
              \\
  \Large\dicef{4} & \Large\dicef{6} &
              A trader stands outside a workshop, screaming that a cart fixer produces shoddy goods (`I almost died, out in the forest, when a wheel came off').  Moments later, the cart fixer attacks. &
              Two drinkers and two traders, discussing local prices.
              \\
  \Large\dicef{3} & \Large\dicef{5} &
              The \gls{templeOfCuriosity} have a famed Storyteller visiting, and everyone wants to know where dragons come from, in his latest tale of mystery and magic. &
              A dozen patrons, complaining about local crime, and hoping to see the \gls{court}.
              \\
  \Large\dicef{2} & \Large\dicef{4} &
              The workers at the bakers guild are rebelling, and have staged a walkout.  Within a day, the Guild leaders will slip poison into the drinks of the organizers. &
              Myriad customers, all in for a quick drink. The bar empties soon after.
              \\
  \Large\dicef{1} & \Large\dicef{3} & A merchant is selling \gls{basilisk} flesh which he found on the side of the road.  When guards ask if he took the \gls{basilisk} recently killed by the \gls{guard}, he says `no', and that his cousin sold it to him. &
              A slap-fight breaks out between two \glspl{seeker}, as one won't tell the other the answer to a riddle.
  \\
            & \Large\dicef{2} &
                        A shady fence hangs around the market, whistling a happy tune.  He left all his goods in a nearby, seedy tavern's basement. &
                        Nine confused patrons sit in a corner, surrounded by a wedding party of hundreds.
                        \\
            & \Large\dicef{1} &
                        Four thugs, waiting to rob anyone who looks weak. &
                        The barkeep is raving at people to fetch him \pgls{mixer} for \pgls{elixir} for his illness (Mindflash Syndrome, \autopageref{Mindflash Syndrome}), but none will come since he insulted \pgls{helper} last year.
                        \\
  \hline
  \end{wideTable}
}

\newcommand\bothyEvents{
  \index{Events!\Glspl{bothy}}
  \begin{dlist}
  \item
  \ifodd\value{r4}
  All full!
  Seriously, we have a dozen horses and two dozen people in here.
  You're the \gls{guard} -- sleep outside!
  \else
  Help!
  I can't nail this yurt down properly.
  It's reinforced with bronze wiring to keep the horses safe at night, but if I can't secure it to the ground then something will just pull it up and eat them.

  The horses won't fit inside the \gls{broch}, so let's start now, before the Sun sets!
  \fi
  \item
  Nice and comfy \gls{bothy}, with two piles of logs.
  Only one of them is \pgls{woodspy}\ldots
  \item
  Two \gls{guard} deserters have camped here the last week, begging food from passing traders.
  They have ignored the last orders received from \pgls{jotter} (roll \vpageref{ngMissions}) and will soon turn to banditry.
  Both have the rank of `Fodder', and will lie about their situation, saying \pgls{jotter} asked them to guard the \gls{bothy}.
  \item
  This \gls{bothy} lies empty.
  The roof could do with some repairs, and the door's seen better days.
  \item
  The last group never left any fire wood.
  They took the firewood axe as well.
  \item
  A few traders have settled in.
  They will spend the night trying to sell their wares to the troupe (find their wares \vpageref{traderWares}).
  \end{dlist}
}

\newcommand\encVillageEvent{
  \begin{dlistDouble}
    \ifodd\value{r4}
      \item
      A child steals from one of the characters.
      \item
      A local archer shot down \pgls{griffin}. It now sits in the central square, on a spit-roast.
      \item
      Children have spotted \pgls{woodspy}, camouflaging itself into a tree.
      A dozen have gathered to throw rocks at it from a `safe' distance.
    \else
      \item
      A scout has destroyed \pgls{crawler}'s nest and found an interesting item -- one \lootTalisman\ (\vpageref{talismanIndex}).
      \item
      \Gls{basilisk}-stench assaults the troupe's faces.
      Someone sunk a lucky arrow into \pgls{basilisk}'s eye a month ago, but without the proper equipment, the \gls{village} could only pull a little meat off it before the corpse spoiled.
      Farmers pulled chunks off for a while, but traders refuse to help take it away with them.
      At present, the entire \gls{village} has just resigned itself to stinking until the problem goes away itself.
      \item
      The villagers hold a funeral for a fallen soldier. Everyone is drunk.
    \fi
    \item
    A few \glspl{digger} have burrowed under the \gls{village}.
    You can hear them at night.
    The farmers discuss plans to root them out, but nobody has anything solid.
    \item
    \ifodd\value{r3}
      Villagers are constructing a wooden cage to safely take water from the river, so \glspl{woodspy} do not drown them, and eat them.
    \else
      Villagers gather to burn away or cut down all the foliage and trees they can, making space for their archers to see.
    \fi
    \item
    Rumours about the strange things happening on the highest neighbour (cannibalism, incest, and eating dwarven beards).
    \item
    A child wants to join the \glspl{pc}' troupe, his family disapproves but they will keep trying to join somehow.
    \item
    A guard has saved $1D6 \times 30$~\gls{sp}, and now hides in the \gls{village}, having abandoned his duties.
    \item
    \Pgls{griffin} has picked up a child, and flown into the forest.
    A dozen farmers prepare to go after it, but they move too slowly.
    \roll{Speed}{Athletics}, \tn[11] to get to the child on time.
    \item
    A troupe of \gls{village} elders, armed with spears, return to report they have seen \pgls{fiend} in the forest (check \autoref{fiends} to see what the \gls{fiend} was up to).
    \item
    A wall has broken, and $2D6$ pigs run loose in the forest.
  \end{dlistDouble}
}

\newcommand\encVillageFeatures{
  \begin{dlistDouble}
    \item
    Verdant \nameref{screechMoss} grows along the old, outer walls (find it \vpageref{screechMoss}).
    \item
    A massive moat, wider than anyone could hope to jump, flows round the outskirts. 
    \item
    Long, thin, copper spikes crown the outer wall.
    Most are fully green, but some chromatic, shining patches show where a creature's body has fallen to them.
    \item
    Tiny wire fences surround every field around the outskirts.
    The bells and chimes make a gentle sound in the wind, and a far louder noise if anything mounts those fences.
    \item
    A watermill sits just outside of the \gls{village}, at the edge of the farmland (or windmill, if the \gls{village} does not have a river).
    The miller lives inside, and the first story is made from stone.
    \item
    \Pgls{broch} sits at the outer wall.
    The \gls{guard} are expected to enter there, rather than the \gls{village} proper.
    \item
    \ifnum\value{temperature}=0
      Bear traps surround the perimeter, hidden by grass, bush, and cabbages.
      The locals know where every single one lies.
      Spot them with \roll{Wits}{Vigilance} (\tn[10]), with $1D6$ Damage for failure.
      \item
      The fields are covered in meat-on-sticks, and the meats are covered in poison.
      Eating the poison should inflict 8 \glspl{ep} to local predators.
      Dead rats and foxes litter the area, indicating that the plan has some problems.
    \else
      $1D3$ tripple-bows sit along the wall (giant crossbows on wheels).
      They inflict $3D6$ Damage, but inflict a -2 penalty to aim.
    \item
      Walnuts grow around the \gls{village}'s perimeter, as they keep other trees at bay.
      Nobody would be stupid enough to approach the perimeter and eat them, unless starving.
    \fi
    \item
    A little house outside the wall, built into the ground, suggests \pgls{doula} lives here.
    \item
    The outer wall has four corridor-entrances.
    Each one ends in a metal gate, which provides a good view of the people inside, and each stretch of wood is littered with murder-holes, ready for archers and spearmen to slay anything which crawls into a corridor.
    \item
    A tall, central house shows \pgls{warden} lives here.
  \end{dlistDouble}
}

\newcommand\ngMissions{
  \label{ngMissions}
  \begin{enumerate}
    \item
    \Pgls{ranger} says to
      \begin{dlist}
        \item
        cover the shit-pit at the side of the building -- it's the bit outside, under the hole at the top.
        Just take everything to the little garden at the side.
        \item
        shine his armour.
        \item
        take his wine delivery up to his room, at the top of the \gls{broch}.
        \item
        clean the \gls{broch}, top to bottom.
        \item
        give a complete recount of the area you come from.
        They need the information for `data related purposes'.
        \item
        check the perimiter -- just run round everywhere that's not forest.
      \end{dlist}
    Don't dawdle, or I'll add 10~\glspl{sp} to your debt.
    \item
    The \gls{jotter} orders you to
    \begin{dlist}
      \item
      help the piper!
      She has a cough, so tonight you can sing to call the forest out to fight.
      \item
      shoo those \glspl{griffin} off the roof.
      They're making a nest, but nobody can land an arrow on them all the way up there.
      Take this broom and climb up top, there's a good lad!
      \item
      scrape the dreameater moss off the walls.
      It's grown so high it's near the top, and it keeps giving the \gls{broch} nightmares (find details \vpageref{dreameater_moss}).
      \item
      stand watch over the hole in the wall overnight.
      Bricks should be here in a few days, so just don't take your eyes off it till then.
      \item
      go pick some berries in the forest.
      They're about a mile North, next to a tree with lots of branches, by a river (or was it a bush?).
      \item
      go chop wood.
      Here's an axe.
    \end{dlist}
    \textit{\ldots or you'll sleep outside the \gls{broch} tonight.}
    \item
    The \gls{jotter} needs you to help \pgls{village}
    \begin{dlist}
      \item
      by watching over its sheep.
      Something nasty has taken ten from inside a barred door, and nobody knows how.
      \item
      by widening their road.
      Spend the next week going along the road, cutting, burning, and pushing back the forest.
      The \gls{village} will feed you.
      \item
      \ifnum\value{temperature}=0
        deal with its undead problem, but \emph{don't} come back here a dead, moaning failure, alright!?.
      \else
        deal with its \gls{basilisk} problem.
        Bonus points for finding the nest.
      \fi
      \item
      clear the perimeter -- they need someone to raise and chop all around the edges.
      \item
      \ifodd\value{r4}
        with a particularly large \gls{griffin}.
        They say it picked up a pregnant ewe!
      \else
        lay five traps around its border.
        They have \pgls{weight} of 10 each, so you can take this old donkey and cart to help.
      \fi
      \item
      to defeat the local bandits.
      If the bandits find out about the \glspl{pc}, they will simply leave for a season.
    \end{dlist}
    \textit{And if you fuck this one up, don't bother coming back -- you can find a new \gls{broch}.}
    \item
    The \gls{jotter} tells you to transport a
    \begin{dlist}
      \item
      child to wherever they're from
      (something attacked its caravan, so he fled).
      \item
      convict to the \gls{court}
      (might be \pgls{witch}, so best gag them).
      \item
      guild leader to town
      (bandits killed the rest of their entourage).
      \item
      wounded ranger to the nearest \gls{templeOfSickness}.
      \item
      \gls{talisman} to the nearest \gls{templeOfCuriosity}
      (select one \vpageref{talismanIndex}).
      \item
      secret document and drop it at the meeting place (roll $1D6$ for the point on the map).
    \end{dlist}
    This one's serious.
    \textit{If you screw this up, I'll see you in the \gls{court}, and make sure you swing.}
    \item
    The \gls{jotter} says to
    \label{riderMissions}
    \begin{dlist}
      \item
      lay these three traps around a nearby \gls{village}.
      They weight a tonne, so take these six donkeys and just try to work something out.
      \item
      deliver a message to a local \gls{fiend}.
      You do not have to approach them, but you \emph{do} need evidence that they received the message.
      \label{fiendLetter}
      \item
      buy some food from \pgls{village} and get back here by morning.
      Seriously -- everyone's hungry.
      Get going.
      NOW!
      \item
      find out what happened to the \glspl{guard} who disappeared on another mission, and bring back their stuff if they're dead.
      \label{findDead}
      \item
      clear a road blocked by some creature (use the highest possible monster roll for the place and season).
      \item
      hunt down a half-elven \gls{witch} who leads a crew of bandits.
    \end{dlist}
    \item
    A travelling tradesman wants you to
      \begin{dlist}
        \item
        come guard his caravan on a journey to the next town.
        He'll pay you 30~\glspl{sp}.
        \item
        take this crossbow -- it should keep you safe.
        \item
        \ifodd\value{r4b}
          tell him about what the \gls{guard} are up to.
          He will join you on the road and pay 5 \glspl{sp} per person for the `news'.
        \else
          join her going to the next town, and vouch that the bag of 300 \glspl{sp} belongs to the \gls{guard} and not her, so she doesn't have to pay a tax on it.
        \fi
        \item
        tell her about any interesting things you might have found on the road\ldots
        anything you maybe haven't reported to the \gls{jotter}\ldots
        maybe something you need help getting rid of\ldots?
        \item
        deliver something to a local \gls{fiend}.
        The tradesman has just what it wants.
        She can pay a total of 60~\glspl{sp}.

        \textit{Nevermind `why', \gls{yonder} take ye!}
        \item
        break a friend out of jail in a nearby town.
        He had no idea those items were stolen!
      \end{dlist}
    \textit{Tell nobody -- you could get in a lot of trouble for moonlighting!}
    (re-roll to find your orders)
  \item
  \Pgls{thane} requests you seek out the
    \begin{dlist}
      \item
      \gls{guard} deserter, and prosecute them in the local \gls{court}.
      \item
      \gls{fiend} with the lowest number and find out what you can (but don't get too close).
      \item
      elven fence, who sells stolen items to traders from far-away lands.
      \item
      native speaker of Gnomish, because someone needs a book translated.
      \item
      \gls{doula}, famed for her healing abilities, to help the local \gls{warden}.
      She doesn't want to help for some reason.
      \item
      local \gls{fiend}, and find everything they can about it.
    \end{dlist}
  \item
  \Pgls{builder} orders you to capture a
    \begin{dlist}
      \item
      bard, known for spreading completely unverified rumours about the \glspl{warden}.
      \item
      \gls{witch}, who keeps casting strange and deadly enchantments on local plants.
      \item
      place where mana wells up from the ground. It lies somewhere in the forest, or so the bards sing.
      \item
      creature for the arena; the \glspl{warden} want to see some good sport.
      \item
      ray of full moonlight.
      The \gls{doula} need it for something, but won't explain what they mean.
      \item
      thief in the act. He robs \glspl{warden} at night, but nobody has seen his face.
    \end{dlist}
  \item
  The \gls{overseer} needs you to travel beyond the \gls{edge} and
    \begin{dlist}
    \item
    find where those goblins come from.
    \item
    confirm what you can about this map -- it makes some strange claims.
    \item
    bring back a feast of \gls{basilisk} eggs -- the \gls{warden} wants to impress a lady.
    \item
    recover the golden wand from the lost city.
    We hear it lies there, but if we heard wrong, return with evidence that there is no lost city there.
    \item
    confirm the \gls{doula}'s claims about the great site for starting a new \gls{village}.
    \item
    deliver this message to the local elves, or gnomes, or whatever they are.
    \end{dlist}
  \item
  Take the day off -- you've earned it!
  \end{enumerate}
}

\newcommand\missionComplications{
  \label{missionComplications}
  \begin{enumerate}
    \item
    The \gls{jotter} has assigned a Cutter to stand watch over everything you do.
    They won't help, they only make sardonic comments.
    \item
    Take this Wanderer from the \gls{paperGuild} with you to chronicle the mission.
    \label{takeAWander}
    And \emph{don't} let him get hurt.
    \item
    Rival \glspl{guard} work on a related mission.
    Roll the second $D6$ again, to find their quarry.
    Both items are related.

    For example, if you roll a `\ref{riderMissions}' then \ref{fiendLetter}, the \glspl{pc} would have to deliver a letter to \pgls{fiend}.
      If the rivals rolled `\ref{findDead}', they would need to find some presumed-deceased \glspl{guard}, and may upset the \gls{fiend} in the process.
      Perhaps both missions involve targets who travel together, or perhaps one mission's success leads to the other's failure.
    \item
    There are two of them!

    If the troupe need to transport a wounded ranger, they hear about another wounded ranger en route, and have to rescue the other as well (after finding them).
    If the troupe need to find \pgls{doula}, then the \gls{doula} tells them they first have to help find her sister.
    \item
    Roll two more complications!
    If you roll `\ref{doubleTrouble}' again, treat it as rolling `\ref{takeAWander}'.
    \label{doubleTrouble}
    \item
    The troupe leader has developed a nasty disease (\vpageref{diseases}).
    You'll have to visit town's \gls{templeOfSickness} for a cure.
    \item
    A distant \gls{warden} hopes for their failure, and has sent his men to stop them.
    \item
    A nearby \gls{fiend} does not want the mission to succeed.
    \item
    The \gls{templeOfPoison} has a vested interest in the mission failing, and will influence all the groups they can to stop the troupe succeeding.
    If the troupe have an encounter with an intelligent creature (such as a bandit or trader) there is a 50\% chance they work as a spy for the \gls{templeOfPoison}.
  \end{enumerate}
}

\newcommand\brochDerths{
  \begin{dlist}
    \item
    A piper.
    Looks like you'll have to play the pots and sing to the beasts!
    \item
    Food.
    There's only \arabic{r4} days' worth left, and they're not for you lot!
    \item
    Weapons.
    We have someone guarding the kitchen knives, so go find a stick in the forest if you need something to protect yourself.
    \item
    Arrows.
    Looks like you'll have to stand outside and do the job by hand!
    \item
    \Glspl{torch}.
    But don't worry; if you're blind, they're blind, so it kinda evens out.
    \item
    \Glspl{guard}.
    Nobody's here but you and a grumpy \gls{jotter}.
  \end{dlist}
}

\newcommand\newBeast[4][\A]{
  \subsubsection[#2 (#1): #4.]{#1~#2}
  \label{#3}
  \index{#2 (#1)}
}

\newcommand\exampleGoblinEnc{
  \setcounter{enc}{15}
  \setcounter{diceNo}{13}
  \vspace{2em}
  \rowcolors{2}{}{gray!10}
  \noindent
  \begin{boxtable}[c|L]
    \hline
    \hline
    \textbf{Roll} & \textbf{Inner Hamlets} \\
    \hline
    \Repeat{6}{
      \addtocounter{diceNo}{-1}
      \addtocounter{enc}{-1}
      \arabic{diceNo} & \bigBeastList \\
    }
    \Repeat{5}{
      \addtocounter{diceNo}{-1}
      \addtocounter{enc}{-1}
      \arabic{diceNo} & \multiply\value{diceNo} by 4 \arabic{diceNo} goblins! \\
    }
    \hline
  \end{boxtable}
}

\newcommand\hitLocation{
  \begin{boxtable}[XXXX]
    \textbf{Damage} & \textbf{Location} & \textbf{Slash} & \textbf{Bash} \\
    \hline
    \dicef{1} & ear & cut & bruised \\
    \dicef{2} & cheek & slashed & bashed \\
    \dicef{3} & belly & spliced & smashed \\
    \dicef{4} & ribs & pierced & shattered \\
    \dicef{5} & arm & mutilated & crippled \\
    \dicef{6} & thigh & gashed open & cracked \\
    \dicef{7} & jaw & thwacked & chiseled \\
    \dicef{8} & shin & segmented & splintered \\
    \dicef{9} & skull & bisected & demolished \\
    \dicef{10} & heart & opened & interrupted \\
  \end{boxtable}
  }

\newcommand\showEnc[1][\gls{A}]{%
  \renewcommand\npcsymbol{#1}%
  \needspace{3em}%
  \begin{center}%
  \textcolor{background}{\glssymbol{sylf}}\qquad\textbf{Encounters}\qquad\textcolor{background}{#1}
  \end{center}
}

\newcommand\showWhenLow{\ifnum\value{encnum}<7\dicef{\value{encnum}}\fi}

\newcommand\funkyPlant[4][]{
  \stepcounter{encnum}%
  \ifglsentryexists{#2}{%
    \subsubsection[\Glsfmtname{#2}]{\showWhenLow~\Glsfmtname{#2}}
    \glsentrydesc{#2}.
    \label{#2}
    \glsadd{#2}
  }{%
    \subsubsection[#2]{\showWhenLow~#2}
    \index{#2~\Pl}%
    \label{#3}
  }
  #4
  \ifstrempty{#1}{\par}{%
    \index[mana]{#1!\glsentryname{#2}~\Pl}
    \par
    \textbf{\Gls{ingredient}}: #1
  }%
}

\newcommand\disease[5][]{
  \stepcounter{encnum}%
  \subsubsection[#2]{\showWhenLow~#2}
  \index{#2 \glssymbol{eldren}}
  \label{#2}
  #3
  \par
  \textbf{Cure}: #4 (\tn[#5])
}

\newcommand\rotRates{
  \sidebox[20]{
    \begin{boxtable}
      Scorching  & 1 day \\
      Hot  & $1D3$ days \\
      Mild & $1D6$ days \\
      Cold & $1D6+3$ days \\
    \end{boxtable}
  }
}

\newcommand\showRealTimeWeek{%
  \stepcounter{r4b}%
  \ifnum\value{r4b}>4%
    \addtocounter{r4b}{-4}%
    \stepcounter{track}%
  \fi%
  \ifnum\value{track}>12%
    \addtocounter{track}{-12}%
  \fi%
  \trackMonth, week~\arabic{r4b}%
}


\newcommand\parallelCalendar{
  \setcounter{track}{\month}
  \setcounter{age}{\day}
  \begin{boxtable}[X|Y|c]

    \textbf{Games Night} & \textbf{\Glsentrytext{cycle}} & \textbf{Climate} \\
    \hline
    \hline
    \Repeat{6}{
      \ifnum\value{age}=1%
        \setcounter{age}{16}%
      \else%
        \setcounter{age}{1}%
        \stepcounter{track}%
        \ifnum\value{track}>12%
          \addtocounter{track}{-12}%
        \fi%
      \fi%
      \setCycle{\value{track}}{\value{age}}
      \trackMonth~\arabic{age} &
      \showCycle &
      \showTemperature
      \\
    }
    \ldots &
    \ldots &
    \ldots
    \addtocounter{track}{3}
    \setcounter{age}{2}
    \\
    \Repeat{3}{
      \ifnum\value{age}=1%
        \setcounter{age}{16}%
      \else%
        \setcounter{age}{1}%
        \stepcounter{track}%
        \ifnum\value{track}>12%
          \addtocounter{track}{-12}%
        \fi%
      \fi%
      \setCycle{\value{track}}{\value{age}}
      \trackMonth~\arabic{age} &
      \showCycle &
      \showTemperature
      \\
    }
    \ldots &
    \ldots &
    \ldots
    \addtocounter{track}{3}
    \setcounter{age}{2}
    \\
    \Repeat{2}{
      \ifnum\value{age}=1%
        \setcounter{age}{16}%
      \else%
        \setcounter{age}{1}%
        \stepcounter{track}%
        \ifnum\value{track}>12%
          \addtocounter{track}{-12}%
        \fi%
      \fi%
      \setCycle{\value{track}}{\value{age}}
      \trackMonth~\arabic{age} &
      \showCycle &
      \showTemperature
      \\
    }
  \end{boxtable}
}

\newcommand\allEncounterTables{
  \setcounter{enc}{19}
  \setcounter{diceNo}{7}
  \begin{boxtable}[L|r|Lc]
    \Repeat{6}{%
      \addtocounter{enc}{-1}%
      \addtocounter{diceNo}{-1}
      \dicef{\value{diceNo}}~\encCivilization[\value{diceNo}] &
      \arabic{enc} & \bigBeastList &
        \randomize\showInterval{\value{encnum}}
      \\
    }
    \hline
    \Repeat{11}{%
      \addtocounter{enc}{-1}%
      \bigWeatherList &
      \arabic{enc} & \bigBeastList &
        \randomize\showInterval{\value{encnum}}
      \\
    }
    \iftoggle{genExamples}{
      \hline
      \addtocounter{enc}{-1}%
      &
      \arabic{enc} & \bigBeastList &
      \randomize\showInterval{\value{encnum}}
      \\
    }{}
  \end{boxtable}
}

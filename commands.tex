\newcommand\encRef[1]{%
  \nameref{#1} (\autopageref{#1})%
}

\newcommand{\moralechart}{
  \begin{nametable}{Morale Chart}
    \textbf{Bonus} & \textbf{Situation} \\\hline

    +4 & Monsters outnumber characters 3:1. \\

    +2 & Monsters outnumber characters 2:1. \\

    +2 & Character's top Strength Bonus is lower than the monster's.  \\

    +2 & Monster is desperate or hungry.  \\

    -2 & Character's top Strength Bonus is higher than the monster's.  \\

    -2 & Characters outnumber the monsters. \\

    -2 & Monster is wounded. \\

    -1 & Characters have displayed awesome magical abilities. \\

      \end{nametable}
}

\newcommand\foragingChart{
  \begin{nametable}[c|p{.4\textwidth}|L|L]{Foraging}

    \textbf{Roll} & \textbf{Place} & \textbf{Prize 1} & \textbf{Prize 2}\\\hline

    \dicef{1} &
      Under a hidden floorboard -- broken patches show something underneath &
      \lootJewellery &
      \lootMagic \\

    \dicef{2} &
      Hidden room, behind an old bookcase &
      \lootJewellery &
      \lootJewellery, and \lootBig \\

    \dicef{3} &
      In empty home, inside a dark and empty doorway &
      \lootJewellery, and \lootJewellery &
      \lootJewellery, and \lootBig \\

    \dicef{4} &
      Inside a largely preserved house.
      The family died peacefully, somehow\ldots &
      \lootMedium{} and, surprise, woodspy! (\autopageref{woodspy}) &
      \lootJewellery \\

    \dicef{5} &
      Behind a fully stone door, clearly used as a safe.
    Opening it requires Intelligence + Crafts (\gls{tn} 12 to do so quietly, otherwise, \gls{tn} 7) & \lootJewellery & \lootMagic  \\

    \dicef{6} &
      Lying under a pile of human bones &
      \lootBig &
      \lootMagic \\

  \end{nametable}
}

% This is an array of encounters.


\newcommand\showBandits{
  \item
  \setcounter{gold}{\value{diceNo}}
  \multiply\value{gold} by 2
  \addtocounter{gold}{4}
  \arabic{gold} \ifodd\value{diceNo}
    Brigands
  \else
    Bandits
  \fi
}

\newcommand\encBandits{
  \begin{multicols}{2}
  \begin{dlist}
    \showBandits
    \showBandits
    \showBandits
    \showBandits
    \showBandits
    \showBandits
  \end{dlist}
  \end{multicols}
}

\newcommand\encTraders{
  \begin{multicols}{2}
  \begin{dlist}
    \item
    Fresh fruits (\arabic{r2}0~\glspl{cp} per day's worth)
    \randomdozen
    \item
    Bows (\arabic{r12}~\glspl{sp})
    and arrows (\arabic{r12}0~\glspl{cp})
    \randomdozen
    \item
    Torch pitch for (\arabic{r12}~\glspl{cp} per hour's worth)
    \randomdozen
    \item
    Salted meats (\arabic{r12}~\glspl{cp} per day's worth)
    \randomdozen
    \item
    Rope (\arabic{r12}0~\glspl{cp})
    \randomdozen
    \item
    Caged beast (use the creature from the encounter number)
  \end{dlist}
  \end{multicols}
}

\newcommand\encMons{
  \ifcase\value{enc}\relax%
  \or%
    \encRef{ghast}
  \or%
    \encRef{wolf}
  \or%
    \encRef{dryad}
  \or%
    \encRef{griffin}
  \or%
    \encRef{mouthdigger}
  \or%
    \encRef{woodspy}
  \or%
    \encRef{bear}
  \or%
    \encRef{chitincrawler}
  \or%
    \encRef{basilisk}
  \else%
    \encRef{umber_hulk}
  \fi%
  \stepcounter{enc}%
}

\newcommand\encSight{%
  \ifcase\value{track}\relax%
  \or%
    Hibernating \encRef{chitincrawler}
  \or%
    Deer
  \or%
    \encRef{auroch}
  \or%
    \encRef{boar}
  \or%
    \encRef{mage_oak}
  \or%
    \encRef{uproot}
  \or%
    \encRef{bedleaves}
  \or%
    \encRef{marching_mushroom}
  \else%
    \encRef{seekers}
  \fi%
  \stepcounter{track}%
}

\newcommand\trackMonth{%
  \ifcase\value{season}\or%
    January\or February\or March\or April\or May\or June\or%
    July\or August\or September\or October\or November\else December\fi%
}

\newcommand\showSeasonLine[1]{
  \setcounter{season}{#1}
  \trackMonth &
  \textbf{\showSeason} &
  \showTemperature &
  \showAstroEvents
  \\
}

\newcommand\encMonsters{
  \setcounter{enc}{1}
  \begin{boxtable}[cccL]
    \textbf{Cold} & \textbf{Mild} & \textbf{Hot}  & \textbf{Encounter} \\
    \hline
    \dicef{1} &           &           & \encMons                       \\
    \dicef{2} &           &           & \encMons                       \\
    \dicef{3} & \dicef{1} &           & \encMons                       \\
    \dicef{4} & \dicef{2} & \dicef{1} & \encMons                       \\
    \dicef{5} & \dicef{3} & \dicef{2} & \encMons                       \\
    \dicef{6} & \dicef{4} & \dicef{3} & \encMons                       \\
              & \dicef{5} & \dicef{4} & \encMons                       \\
              & \dicef{6} & \dicef{5} & \encMons                       \\
              &           & \dicef{6} & \encMons                       \\

  \end{boxtable}
}

\newcommand\stepDown[1]{
  \ifnum\value{#1}>1
    \addtocounter{#1}{-1}
    \ifnum\value{#1}<13
        \arabic{#1}
    \fi
  \fi
}

\newcommand\encLine{
  \stepDown{diceNo} & \stepDown{diceNo2} & \stepDown{enc} & 
}

\newenvironment{encChart}[1]%
  {
    \setcounter{diceNo}{13}
    \setcounter{diceNo2}{15}
    \setcounter{enc}{17}

    \rowcolors{2}{}{gray!10}
    \vspace{2em}
    \tabularx{\linewidth}{c|c|c|L}
      \hline
      \hline
      \textbf{Warm} & \textbf{Mild} & \textbf{Cold} & \textbf{#1} \\
      \hline
  }%
  {
      \hline
    \endtabularx
    \vspace{2em}
  }%

\newcommand\bigWeatherList{
  \ifcase\value{enc}\or%
    Extreme cold snap
  \or%
    Hail
  \or%
    Snow
  \or%
    Biting winds
  \or%
    Mild breeze
  \or%
    Dead calm
  \or%
    Mist
  \or%
    Overcast sky
  \or%
    Brewing storm
  \or%
    Lightning
  \or%
    Thunder
  \or%
    Clear skies
  \or%
    Light showers
  \or%
    Hurricane
  \or%
    Heatwave
  \or%
    Floods
  \else%
    Error
  \fi
}

\newcommand\bigBeastList{
  \ifcase\value{enc}\or%
    Roaming Ghast
  \or%
    $1D6$ Wandering Ghouls
  \or%
    Dryad
  \or%
    $1D6\times 2$ Elves on a quest
  \or%
    $1D6\times 3$ Deer
  \or%
    $1D6\times 20$ Aurochs
  \or%
    Warthog
  \or%
    Bear
  \or%
    $1D6+6$ Wolves
  \or%
    $1D3$ Griffins
  \or%
    Woodspy
  \or%
    Mouthdigger
  \or%
    Chitincrawler
  \or%
    Basilisk
  \or%
    Swarming stirges
  \or%
    Umber Hulk
  \else%
    Error! \LaTeX\ has broken, the beasts are loose!
  \fi%
}


%%%%% Full Encounters %%%%%

\newcommand\encVillageEvent{
  \begin{enumerate}
    \setcounter{enumi}{1}
    \ifodd\value{r4}
      \item
      A child steals from one of the characters.
      \item
      A local archer shot down a griffin. It now sits in the central square, on a spit-roast.
      \item
      Children have spotted a woodspy, camouflaging itself into a tree.
      A dozen have gathered to throw rocks at it from a `safe' distance.
    \else
      \item
      A scout has destroyed a chitincrawler's nest and found a \lootMagic.
      \item
      Basilisk-stench assaults the troupe's faces.
      Someone sunk a lucky arrow into a basilisk's eye a month ago, but without the proper equipment, the village could only pull a little meat off it before the corpse spoiled.
      Villagers pulled chunks off for a while, but traders refuse to help take it away with them.
      At present, the entire village has just resigned itself to stinking until the problem goes away itself.
      \item
      The villagers hold a funeral for a fallen soldier. Everyone is drunk.
    \fi
    \item
    Villagers are constructing a wooden cage to safely take water from the river, so woodspies do not drown them, and eat them.
    \item
    Villagers gather to burn away or cut down all the foliage and trees they can, making space for their archers to see.
    \item
    Rumours of local oddity (see \autopageref{mapOddities}).
    \item
    A boy wants to join the characters band, his family disapproves but he will try everything he can.
    \item
    A guard has saved $1D6 \times 30$~\gls{sp}, and now hides in the village, having abandoned his duties.
    \item
    A griffin has picked up a child, and flown into the forest.
    A dozen villagers prepare to go after it, but they move too slowly.
    \roll{Speed}{Athletics}, \tn[11] to get to the child on time.
    \item
    A troupe of village elders, armed with spears, return to report they have seen the local fiend in the forest (from \autopageref{mapFiends}).
    \item
    A wall has broken, and $2D6$ pigs run loose in the forest.
  \end{enumerate}
}

\newcommand\ngMissions{
  \begin{enumerate}
    \item
    The Jotter says to
      \begin{dlist}
        \item
        clean out the shit-pits; his first, his horse's second, then everyone else's.
        \item
        shine his armour.
        \item
        take his wine delivery up to his room, at the top of the tower.
        \item
        clean the place, top to bottom.
        \item
        give a complete recount of the area you come from.
        They need the information for `data related purposes'.
        \item
        check the perimiter -- just run round everywhere that's not forest.
      \end{dlist}
    And ignore the `complications', because nobody cares about this job.
    \item
    Help a village
    \begin{dlist}
      \item
      by watching over its sheep.
      Something nasty has taken ten already.
      \item
      sleeping better.
      They've lost three men this month, and can't afford any more.
      But \emph{we} can.
      \item
      deal with its basilisk problem.
      Bonus points for finding the nest.
      \item
      clear the perrimiter -- they need someone to raise and chop all around the edges.
      \item
      particularly large griffin, known for taking off with sheep.
      \item
      defeat the local brigands.
      If the brigands find out about the \glspl{pc}, they will simply leave for a season.
    \end{dlist}
    \item
    Help transport a
    \begin{multicols}{2}
    \begin{dlist}
      \item
      child
      \item
      convict
      \item
      guild leader
      \item
      wounded ranger
      \item
      magical item
      \item
      secret document
    \end{dlist}
    \end{multicols}
    \ldots to the next town.
    \item
    A Night Rider needs you to
    \label{riderMissions}
    \begin{dlist}
      \item
      lay these three traps around a nearby village.
      They weight a tonne, so take these six donkeys.
      \item
      deliver a message to a local fiend.
      You do not have to approach them, but you \emph{do} need evidence that they received the message.
      \label{fiendLetter}
      \item
      track down and kill a team of local bandits.
      \item
      tell a distant village to resume paying their taxes to their warden, and start causing trouble if they refuse.
      \label{taxesMission}
      \item
      clear a road blocked by some creature (use the highest possible monster roll for the place and season).
      \item
      hunt down a half-elven sorcerer who leads a crew of bandits.
    \end{dlist}
    \item
    A travelling tradesman wants you to
      \begin{dlist}
        \item
        come guard his caravan on a journey to the next town.
        He'll pay you 30~\glspl{sp}.
        \item
        take this shortsword -- it should keep you safe.
        \item
        tell him about what the \gls{guard} are up to.
        He will join you on the road and pay 5 \glspl{sp} per person for the `news'.
        \item
        join her going to the next town, and vouch that the bag of 300 \glspl{sp} belongs to the \gls{guard} and not her, so she doesn't have to pay a tax on it.
        \item
        help deliver something to a local fiend.
        The tradesman has just what it wants
      (as per \autopageref{fiendDesires}).
        She can pay a total of 60~\glspl{sp}.
        \item
        break a friend out of jail in a nearby town.
      \end{dlist}
    Tell nobody -- you could get in a lot of trouble for moonlighting
    (re-roll to find the orders).
  \item
  Seek out the
    \begin{dlist}
      \item
      \gls{guard} deserter, and return with their head.
      \item
      local oddity and report back once you know where they live
      (see \nameref{mapOddities}, \autopageref{mapOddities}).
      \item
      elven fence, who sells stolen items to traders from far-away lands.
      \item
      native speaker of Gnomish, because someone needs a book translated.
      \item
      doula, famed for her healing abilities, to help the local warden.
      \item
      local fiend, and find out what it wants
      (as per `\nameref{fiendDesires}', \autopageref{fiendDesires}).
    \end{dlist}
  \item
  Capture a
    \begin{dlist}
      \item
      bard, known for spreading completely unverified rumours about the wardens.
      \item
      witch, who keeps creating strange and deadly enchantments on local plants.
      \item
      place where mana wells up from the ground. It lies somewhere in the forest, or so the bards sing.
      \item
      creature for the arena; the wardens want to see some good sport.
      \item
      ray of full moonlight.
      The doula need it for something, but won't explain what they mean.
      \item
      thief in the act. He robs wardens at night, but nobody has seen his face.
    \end{dlist}
  \item
  Travel beyond the \gls{edge} and
    \begin{dlist}
    \item
    find where those goblins come from.
    \item
    confirm what you can about this map -- it makes some strange claims.
    \item
    bring back a feast of basilisk eggs -- the warden wants to impress a lady.
    \item
    recover the golden wand from the lost city.
    We hear it lies there, but if we heard wrong, return with evidence that there is no lost city there.
    \item
    confirm the doula's claims about the great site for starting a new village.
    \item
    deliver this message to the local elves, or gnomes, or whatever they are.
    \end{dlist}
  \item
  Take the day off -- you've earned it!
  \end{enumerate}
}

\newcommand\missionComplications{
  \begin{enumerate}
    \item
    The Jotter has assigned a Cutter to stand watch over everything you do.
    They won't help, they only make sardonic comments.
    \item
    Take this Wanderer from the Guild of Knowledge with you to chronicle the mission (see \autopageref{knowledgeWanderer} for Wanderers).
    And \emph{don't} let him get hurt.
    \item
    There are two of them!
    \begin{exampletext}
      If the troupe need to transport a wounded ranger, they hear about another wounded ranger en route, and have to rescue the other as well (after finding them).
      If the troupe need to find a doula, then the doula tells them they first have to help find her sister.
    \end{exampletext}
    \item
    Roll twice more!
    \item
    The troupe leader has developed a nasty disease (see \autopageref{diseases}).
    \begin{dlist}
      \item
      Breathrot
      \item
      Corpse Hands
      \item
      Mindflash Syndrome
      \item
      Guardbane
      \item
      Spychoke
      \item
      Torpid Flesh
    \end{dlist}
    \item
    Rival \glspl{guard} work on a related mission.
    Roll the second $D6$ again, to find their quarry.
    Both items are related.

    \begin{exampletext}
      For example, if you roll a \ref{riderMissions} then \ref{fiendLetter}, the \glspl{pc} would have to deliver a letter to a fiend.
      If the rivals rolled a \ref{taxesMission}, they would have to threaten a village into paying its taxes, and these two missions would interfere with each other somehow.

      Perhaps both missions involve targets who travel together, or perhaps one mission's success leads to the other's failure.
    \end{exampletext}
    \item
    A distant warden hopes for their failure, and has sent his men to stop them.
    \item
    A nearby fiend does not want the mission to succeed.
    \item
    The Temple of Poison%
    \footnote{See \autopageref{god:Abderian}.}
    has a vested interest in the mission failing, and will influence all the groups they can to stop the troupe succeeding.
    If the troupe have an encounter with an intelligent creature (such as a bandit or trader) there is a 50\% chance they work as a spy for the Temple of Poison.
  \end{enumerate}
}

\newcommand\hitLocation{
  \begin{boxtable}[XXXX]
    \textbf{Damage} & \textbf{Location} & \textbf{Slash} & \textbf{Bash} \\
    \hline
    \dicef{1} & ear & cut & bruised \\
    \dicef{2} & cheek & slashed & bashed \\
    \dicef{3} & belly & spliced & smashed \\
    \dicef{4} & ribs & pierced & shattered \\
    \dicef{5} & arm & mutilated & crippled \\
    \dicef{6} & thigh & gashed open & cracked \\
    \dicef{7} & jaw & thwacked & chiseled \\
    \dicef{8} & shin & segmented & splintered \\
    \dicef{9} & skull & bisected & demolished \\
    \dicef{10} & heart & opened & interrupted \\
  \end{boxtable}
  }


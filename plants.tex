\chapter{Strange Growths}
\label{plants}

\begin{multicols}{2}

Exotic, dangerous, and occasionally useful flora of Fenestra.

\subsubsection{Uproot}
\label{uproot}

Eating the root (after preparation) gives you +1 Strength for an interval.
Each interval, the player rolls 1D3.
On a `1', the effect stops, on any other number the character loses 1 HP, and the effect continues.

\subsubsection{Downroot}
\label{downroot}

Eating the root gives +2 HP after a night's rest, and -1 to all Attributes for the next three days.

\subsubsection{Seekers}
\label{seekers}

Seekers sit passively, filling up with dust, waiting for a seed.
The seeds are sticky, and float on the wind until they latch onto an animal (or traveller).
One the carrier comes within range - poof!
The dust spreads everywhere, and the seeds have been properly joined.
The dust causes nausea, and sometimes hallucinations.

\subsubsection{Bedleaves}
\label{bedleaves}

These massive plants appear a little like gunnera, but larger.
Anyone sufficiently small can rest on them overnight and heal an additional Fatigue point.

\subsubsection{Thorny Thickets}
\label{thorny_thickets}

Just regular thorny bushes, but with combat conditions.
Moving through thickets is an absolute nightmare.

\subsubsection{Screeching Moss}
\label{screeching_moss}

It squeaks when stepped on.
They tend to form a symbiotic relationship with predators, who always respond to the squeaks by investigating the source, or laying traps around the thickets.
The moss then feeds off the blood of any victims.
Nobles love to encourage screeching moss to grow around their estates, to protect against unwanted intruders.

\subsubsection{Dreameater Moss}
\label{dreameater_moss}

Created by a sorcerer, this moss joins the dreams of anyone who sleeps on it.
The sorcerer still lives in those dreams, and wanders through them, eating the minds of anyone they find.
It provides a powerful but dangerous way to communicate across long distances.

\subsubsection{Arch Weed}
\label{arch_weed}

Two puffs helps the tired sleep, and the awake think sharper and better.
But it makes the wounded itch and itch.
Set fire to a whole bunch and hope the wind doesn't change direction - then it's a real weapon of war.

\subsubsection{Mage Oak}
\label{mage_oak}

Roots of this tree tap into a mana lake deep underground. It restores 3 mana for every interval spent near it.
The tree will also repeat every spell cast near it, with random targets.

\subsubsection{Fly Flowers}
\label{fly_flowers}

Named after swarms of flies that surround them at all times, these vibrant flowers secrete sticky, pungent mucus that attracts swarms 
of insects. The stench is really hard to wash off, and anyone who comes in contact with this mucus will attract large numbers of 
insects at least for the rest of the day.

\subsubsection{Marching Mushroom}
\label{marching_mushroom}

Chewing on this tough mushroom relieves one of fatigue, but slows both their body and mind. Ignore Fatigue penalties for a day, but 
with -1 penalty to Dexterity, Speed, Intelligence and Wits.

\subsubsection{Molted Basilisk}
\label{molted_basilisk}

Type of lichen that grows on the backs of basilisks, it also absorbs and retains the beasts overwhelming stench. When it grows large 
enough, small clumps of it fall off. A rare find, and also a warning that a basilisk is nearby.

\subsubsection{Dryad's Kiss}
\label{dryads_kiss}

Although edible, this mushroom makes you very gullible. Gain -2 penalty to all Deceit and Empathy checks

\subsubsection{Glowshroom}
\label{glowshroom}

These subterranean fungi give off a soft, faint light, but only in complete darkness. Dwarves sometimes use them instead of torches, 
even though the light is dimmer.

\subsubsection{Bloodwood}
\label{bloodwood}

Just the scent of sap extracted from the Bloodwood tree makes your blood rush. Refined, this sap is considered a potent aphrodisiac, 
and many a noble past their prime is willing to pay well for it. Unfortunately, the sap is secreted only during the harshest winters, 
and tends to attract and irritate local wildlife, making procuring it a risky endeavour.

\subsubsection{Whistling Cane}
\label{whistlingCane}

These swamp plants play an eerie tune on windy days. Inexperienced travellers might stray from their path, into the dangerous bog, 
looking for the source of the sound. They can be crafted into musical instruments, but they're never popular, since their sound is 
always sorrowful.

\end{multicols}

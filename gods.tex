\chapter{Guilded Temples}
\label{cosmology}

\newcommand\guildRank[1]{\item[#1]\index{#1 (\npcsymbol)|textbf}}

\begin{multicols}{2}

\label{godsOfDeath}

\noindent
As any idiot can plainly see, the gods do not love humanity.
They treat people like children playing with insects.
And when a god finally kills someone, they claim that human soul for their realm.
Death by cold means joining \gls{sable}'s frozen realm, and death by violence allows \gls{wrecan} to pull the soul into her realm of eternal violence.

So when humanity looks to the divine, they do not look for salvation, but for danger.
Temples exists to protect civilization from the wrath of their god.

The temples are guilds with a divinely mandated monopoly.
Even a `guild hall' which is nothing more than an old lady's house still offers divine guidance (or rather, guidance on avoiding divine interference).
The \gls{templeOfHate} encourages reconciliations, the \gls{templeOfFrost} offers warmth, and the \gls{guard} work for the \gls{templeOfBeasts}, where they kill the \gls{sylf}'s children.

\newcommand\guild[7]{
  \renewcommand\npcsymbol{\glssymbol{#1}}
  \vspace{2em}
  \needspace{16em}
  \subsection[The Temple of #2]{\glssymbol{#1}~\gls{#1}~\glssymbol{#1} \\ \& \\ the Temple of #2}
  \index{Gods}
  \label{god:#2}

  \begin{exampletext}
  \noindent
  #3
  \end{exampletext}

  \noindent
  \begin{minipage}{\linewidth}
  \begin{description}
  \item[Domain:] #4

  \item[Defence:] #5

  \item[Members:] #6

  \item[Activities:] #7

  \end{description}
  \end{minipage}

  \subsubsection{#6}
}

\guild{abderian}% Name
  {Poison}% Death
  {
    Led doors, covered in ivy, open to usher in \pgls{warden}, still dressed in the finery he wore for feasts in life.

    ``I told my daughter to bury me in my dinner jacket.
    I always knew she was the good sort'', he smiled.

    A tall and impossibly thin woman balances a wine glass between two fingers and hands it to him.

    ``So what's for dinner in this noble realm?'', but the smile turns to a stare half way through a swig of wine, as he scans the room, and notes all the filthy beggars staring at their glasses, without drinking.

    {\sffamily ``We have some nobility''}, she says thoughtfully.
    {\sffamily ``We even have kings, from the olden days''}, pointing to a skin-clad skeleton, trying to lift a crown from the floor.

    The \gls{warden} chokes and coughs, and notes the specks of blood on his dinner jacket.

    ``Is this wine poisoned?!''

    {\sffamily ``All wine is poison.
    We just drink it stronger here.''}
  }% God
  {poisons, rot, mould, fungus, ale, feasts}% Domain
  {salt, distillation, rendering, vinegar, and wine}% Defence
  {\Glspl{server}}% Watchers
  {
    Brewing, and baking.
  }% Guild activities
work in \glspl{warden}' kitchens, and every town has a bakery near their slums.
They provide bread for the hungry, and beer for the sober.
Despite this, nobody likes the guild.

The \glspl{warden} complain that they insist on a monopoly on cooking \emph{in every \gls{warden}'s home}, and \glspl{warden} who refuse end up poisoned sooner rather than later.
But the \gls{wheatGuild} are quick to note that nobody with an official \gls{wheatGuild} cook has ever succumbed to poison, so this only shows that everyone should employ one.

\widePic{Irina/old_age}

\subsubsection{The Structure}

\begin{description}
  \item[\Glspl{baker}]
  \glsdesc{baker}.
  \item[\Glspl{tender}]
  \glsdesc{tender}.
  \item[\Glspl{cook}]
  \glsdesc{cook}.
  \item[\Glspl{curator}]
  \glsdesc{curator}.
  They also test or switch their bottles to make sure they don't leave any shoes for someone else to step into, as nobody from the streets can imagine a better position than being a curator.
  \item[\Glspl{landlord}]
  \glsdesc{landlord}.
  \item[\Glspl{chef}]
  \glsdesc{chef}.
\end{description}

\null
\subsubsection{The Taverns}
are temples, which purify food with heat, and water with fermentation.
Not every tavern holds membership, but the bigger the town, the more keen the \gls{wheatGuild} will feel on establishing themselves in the area.

\guild{eldren}% Name
  {Sickness}% Death
  {
    Approaching the doors, the sick may find a woman with her left arm and leg missing, and an older man with downs syndrome, ushering them into large, open halls, full of beds where people chat.

    Moving inside, a long hallways presents doors, where less comfortable residents try to die peacefully.
    Unfortunately the temple forbids alcohol, as they do not want to risk a single soul to death by poison.
    The gods have their own laws and technicalities.

    Those who reach the end unharmed can let their soul fall out peacefully.
    Death by old age guarantees a place with the peaceful dead, in \gls{eldren}'s realm.
  }% God
  {pensions, disability, family, and ancestry}% Domain
  {none}% Defence
  {\Glspl{helper}}% Watchers
  {
    Caring for the old, sick and infirm.
  }% Guild activities
work together to overcome their problems, and heal the sick.
Anyone with a long-term illness may join.
Parents with disabled children try to have their children work as \pgls{helper}, rather than joining the \gls{guard}.

Most live their lives inside a city, not far from their temple.
A rare few travel around neighbouring \glspl{village}, checking for anyone who cannot handle the hard physical labour in the outer \glspl{village}.

Where other temples focus on finances, followers of \gls{eldren} serve their communities well, and always receive respect from people.
They say that killing a member of the \gls{templeOfSickness} brings a curse which lasts for ten generations.

\paragraph{Funding}
comes from people who give a little to the temple throughout their lives, in the hopes of one day dying in it.
These payments form something like a pension, as those who donate well always receive the first available beds in the \gls{healersGuild}.

As a result, this temple may have more wealth than all the others, or may not.
They don't keep open books, but rumours abound of hidden, underground vaults, which store rubies the size of a fist, elven necklaces with magical abilities, and orbs which summon a thousand dragons to serve the user.
The least likely element in these stories is the \gls{healersGuild} building somewhere with a lot of stairs.

\guild{nulla}% Name
  {Misgenesis}% Death
  {
  \ifnum\value{temperature}=0\index{Nulla}\fi
  Some things die before they begin.
  Forgotten hopes, poems the writer never starts, and people who could not be born as their parents never met.
  Not a single soul resides in \hphantom{Nulla}'s realm, as her victims never existed.

  Speaking her name attracts her attention, so it soon became a curse, then a rarity, and eventually unbecame altogether.
  Only the \gls{doula} remember, as they pass her name from one to another, by entering each others' dreams, then writing the name on sand, one glyph at a time.
  }% God
  {regret, failure to thrive, the road untaken}% Domain
  {hope}% Defence
  {\Glspl{doula}}% Watchers
  {
    Birthing, lambing, planting, marriages, business mergers, novel journeys, new settlements and death.
  }% Guild activities
live alone in \glspl{village}, forming groups only to train apprentices.
Everyone who wants to start something new goes to the \glspl{doula} -- whether starting a business venture, asking someone to marry them, or just travelling between settlements, people request the \gls{doula}'s blessings.
They frequent towns less often, as the demand on their attention rises rapidly once people know \pgls{doula} is available.
Despite the high demand for their attention, they never charge much -- everyone knows they should, by rights, give their blessings in return for a scarf, or a little meat.
If people see them charging much more, they generally accuse the \gls{doula} in question of going sour, and begin to refer to her as `\pgls{witch}'.

People can spot \pgls{doula} by the pouches of \glspl{ingredient} by her side, a scowl on her face and a fierce sense of direction and purpose.
They never wander, they only march.

\subsubsection{The Structure}
remains a matter of opinion and leverage.
Almost no \glspl{doula} practice \gls{witchcraft}, but those who do won't display it, so those who don't can hint that they can.
When one casts a soothsaying spell, she tells others of the spell's results.
And almost nobody can really track how effective a blessing has been.
But even without evidence, everyone needs safety from \hphantom{Nulla}, every \gls{doula} remains in demand.

\begin{description}
  \guildRank{Apprentices}
  serve for as long as it takes to pick up the basics -- sometimes until their mentor has died (which means they get to keep the house).
  \guildRank{Doula}
  care for the young, assist with births, and wish people well on new ventures; and whether or not their blessings truly change people's fate; or their predictions really see the future, nobody can tell.
  \guildRank{Wayfinders}
  have sufficient magical abilities that they walk fearlessly, deep into the forest.
  Overseers in the \gls{guard} pay them handsomely (but rarely in direct coinage) to seek out areas in the forest fit to built \pgls{village}.

  While the \gls{doula} never like to reveal their rank, seeing an old woman walk out of the green darkness, where even the \glspl{guard} fear to enter, with twigs clogging up her hair, quickly gives the game away.
\end{description}

\subsubsection{Temples with Tea}
The \gls{doula} tend to live away from others, in case of mishaps with their \gls{witchcraft}.
They have no grand halls, just shoddy old houses, or occasionally a potion shop in a large city.

\guild{paik}% Name
  {Justice}% Death
  {
    As you enter the pit, the executioner leans in to whisper, ``just don't make a fuss, give them a good show, okay?''.

    You haven't done anything wrong, but that doesn't matter now.
    The noose swings above the pit.
    Two more doors sit around the stone edge.
    Some of the crowd wave to you through the bars.
    Your voice pushes up the pit, over the chattering jury.

    ``My lord.
    The man accusing me of theft was in fact looting the bodies when I arrived.
    I have the location where I believe he buried the items on the road.
    I saw him dig there, shortly after.''

    {\sffamily ``Peeping Tom, eh?
    What \gls{witchcraft} lets you stare at people from the dark of the forest?''}

    The crowd cheers as the jester begins climbing the cage-dome above the pit, in a mocking skulk.

    {\sffamily ``Do you stand next to the beasts when you watch people, or prefer sitting on your own in the dark like this?''}

    The jester's lips purse as he fondles his crotch while staring cross-eyed at a woman in the crowd.

    The \gls{warden} tells the jester to be quiet and declares that the first character witness has arrived.
    The guards open a side door, and a woman walks in.
    You recognize the waitress, and she begins recounting a night you were in her bar, two years ago\ldots
    The crowd murmurs in titillated shock.
  }% God
  {law, taxation, punishments, and oaths}% Domain
  {obedience}% Defence
  {\Glspl{keeper}}% Watchers
  {
    Arresting and sentencing criminals, and witnessing oaths.
  }% Guild activities
guard the people from themselves.
The footsoldiers track down and arrest criminals, and take them to prisons for sentencing by the \glspl{warden}.

\Glspl{warden} form the only temple which actively kills, though they remind people that society must have laws, and that they attempt to limit deaths by Justice to an absolute minimum.
To minimize deaths, they typically hand criminals over to the \gls{guard}.
Sentencing a soul to the forest technically means that \gls{paik} did not take them to his afterlife, so the \glspl{warden} have successfully saved someone from him, and therefore done their job well.

\subsubsection{The Structure}
is bithright, from \gls{warden} to \gls{sunGuard}, everyone has a proper place, except the jester.
The jester symbolizes anarchy, and the mobs which can arise without authoritative leadership.

\begin{description}
  \item[Jesters]
  are not formally ranked in any way, but they make the accusations in court, so the local populace fears them without exception.
  See `\nameref{guildJester}', below.
  \item[\glsfmtplural{sunGuard}]
  work in towns and cities.
  Anyone sufficiently fit, overbearing, and loyal to the \glspl{warden} can become a member.
  \guildRank{Prefects}
  administer the guards in a station, and travel along roads when needed to protect the \gls{warden}.
  \guildRank{Captains}
  administer a whole city.
  \guildRank{Seneschals}
  keep track of a lord's resources, take tax money, record legal rulings, and keep the treasure safe.
  Every \gls{warden} needs a trusted seneschal to operate, and in many ways the seneschal holds a more important position than any \gls{warden}, in terms of the land's stability.
  \item[\Gls{village} \Glspl{warden}]
  have power proportional to the number of \glspl{village} they control.
  Unfortunately, these \glspl{village} can cost more than they generate, as \glspl{warden} must pay for any \gls{guard} visits to \pgls{village}.
  \item[Town \Glspl{warden}]
  make ties with \glspl{warden} from other settlements, and have long Philosophical conversations on the nature of justice, or tell stories about the funniest sentences they've passed on the common people.
  \item[City \Glspl{warden}]
  hold more power than any other human.
  If they become agitated, they can raise armies, and wage war.
\end{description}

\pic{Irina/justice}

\subsubsection{The \Glsfmttext{court}}
\index{Pit of Justice}
\label{pitOfJustice}
is deep, so many spectators can look down at the accused.
The \gls{warden} sits at the top, with a little space for advisors or family.

A single noose hangs above the pit, and locked doors sit around the edges.
Sometimes the \gls{warden} has additional accusers behind them, waiting to pop out and give the crowd some real `wow-factor'.
Other times, the \gls{warden} has \glspl{crawler} or \glspl{woodspy} trapped in the side-rooms, so they can declare a surprise trial-by-combat.

\index{Theatre of Oaths}
City-folk love seeing the `theatre of oaths'.%
\footnote{See how to conduct a trial \vpageref{courtVerdicts}.}

\paragraph[the Jester]{The Jester}
\label{guildJester}%
always screeches accusations and demand ridiculous punishments.
\Glspl{warden} can never agree with jesters publicly -- they would look foolish -- so they must provide a harsher or lighter penalty than the jester.
This leads jesters to propose the harshest and most ridiculous of punishments, to limit the \gls{warden}'s decree (``\emph{pull off his eyelids my lord!}'', ``\emph{put her children in the forest, and let her run about finding them all night!}'').
The \gls{warden} can then appear as a magnanimous arbiter, seeing the best in everyone, and reducing the jester's sentence to something more moderate.

Unfortunately, the jester's punishments often `anchor' the court's final judgement before it starts.
Nobody really knows the cost for kicking someone's pig in the head, or for lying about the quality of a cartwheel, so when the first suggestion is `20~\glsentrylongpl{gp}', the rest of the trial focusses on gold, where it might have been silver.

Jesters often enjoy parading about the streets in their ridiculous get-up.
They may look like ridiculous little people, with the bells and red hat, but nobody would dare make fun of a jester.
Anyone who laughs at him in the wrong way, or fails to laugh in the right way, may see him again from below one day, in the pit.

\guild{sable}% Name
  {Frost}% Death
  {
    He lies comfortably naked with white, glistening snow, pulled over his pale chest like a blanket.
    He looks inviting, like you could curl up next to him and feel just as comfortable.
    He sleepily scratches his black nose with his black fingertips, and lets out a tiny, frosty, {\sffamily sigh}.
  }% God
  {snow, ice, dead houses, icicles, and biting wind}% Domain
  {clothing, fire, preservation, and savings}% Defence
  {\Glspl{weaver}}% Watchers
  {
    Spinning thread and tales, and baking long-life cakes.
  }% Guild activities
ward against frost by making and mending clothes.
Every town has a wing of the \gls{weaversGuild}.
When someone asks for their clothes to for mending, the weavers ask the waiter a dozen questions each.
Half the traders in town stop at the weavers before the market, half to hear news of their competition, and half for a hole in their hat.

\subsubsection{The Structure}
must be tight, or the fabric falls apart.
Beautiful patterns can put one ahead in the temple, or at least gives an excuse to stop working and start talking for a little while.

The group decides on internal payments week to week.
By talking with each other, the guild keeps prices high, and makes a killing.

\subsubsection{The \Glsfmttext{templeOfFrost}}
has a hearth in the centre, heating everyone equally.
The room is abuzz with at least a dozen people, young and old, mostly women.
At least half work on looms at any given time.
Others rest their fingers, while townsfolk drop off a log of wood in return for some embers -- starting a fire with the embers of the \gls{templeOfFrost} brings good luck throughout \gls{cTwo}.

\widePic[t]{Irina/forest}

Some \glspl{templeOfFrost} also store riches for people, as a fail-safe against the cold times.
Their valuables remain somewhere safe, while they receive a note attesting their ownership of the item.
Instead of transporting items (which risks losing them), people simply pass along the note, and have \pgls{notary} record the transaction.

\guild{sylf}% Name
  {Beasts}% Death
  {%
    A painfully pregnant belly, and a thorax bloated with eggs; the mother of \gls{fenestra}'s predators overflows with life.
    She meanders around the world, laying monstrosities and feeding them with anything she finds on the world's surface.

    She holds the forms, abilities, and hunger of every predator in the world.
    And when full \glspl{village} disappear in a night, humans wonder if \gls{sylf} herself came to eat them.
  }% God
  {the deep forest, darkness, barbarism, and anything which feeds on humanity}% Domain
  {dogs, bows, and keeping watch all night}% Defence
  {\Glspl{guard}}% Watchers
  {
    Protecting traders, manufacturing bows, and burning the forest.
  }% Guild activities
informally, include anyone who watches for beasts at night.
However, the \gls{guard} proper, where people have ranks and duties, typically come from society's unwanted -- they come from towns and cities, where \glspl{warden} send them to the forest.
The \gls{guard} answer the call of \glspl{village} struggling to survive, build new settlements, and perform missions beyond the \gls{edge}.

Despite high death-tolls, notable `types' reliably survive the hardships.
Firstly, anyone from a rich family can often buy their child into a promotion good enough to keep them safe.
Secondly, the fastest of the \gls{guard} can survive almost any danger, as long as they don't take the lead often, and stay alert when someone makes them.
Almost every beast in the forests will stop attacking once it has a body to feed on, so above all, survival depends on sprinting faster than one's companions.

\subsubsection{The Structure}
is tall, so the heads can stay far from the fodder, who are unfit for civilization.
Lower ranking \glspl{guard} -- \glspl{fodder} and \glspl{gDigger} -- should not enters civilized areas; they should live in \glspl{broch} by the \gls{edge}, protecting civilization.
Others are `broadly understood' to guard the \gls{edge}, but may enter civilized areas any time their missions don't pull them away.

\begin{description}
  \item[\Glspl{fodder}]
  \label{fodder}%
  do as anyone tells them, until the forest eats them, or they gain proper rank.
  None may enter a settlement of any kind.
  They sleep outside, at the foot of remote towers.
  \index{Towns!Entry}

  \item[\Glspl{gDigger}]
  have joined a group past the \gls{edge} and survived.
  This encourages others to treat them like real people, but does not earn them the right to return to civilization without special permission.

  \item[\Glspl{soldier}]
  make a kill which was witnessed by someone of higher rank than them.
  They may enter any settlement if they have a solid reason.

  Despite their name, they do not need to use a bow, but most do.

  \item[\Glspl{cutter}]
  have journeyed past the \gls{edge} and returned with a trophy of some kind.

  They build new settlements, protect caravans, and occasionally try to sneak off to relax in a town somewhere, without anyone noticing.

  \item[\Glspl{ranger}]
  \label{ranger}
  have at least three heads from the forest -- beasts preferred, but bandits will do.
  They protect fast caravans, deliver urgent messages between settlements, and provide fast reinforcements to \glspl{village} around the \gls{edge} dealing with \glspl{basilisk}.

  \item[\Glspl{jotter}]
  \label{jotter}%
  need only show their literacy, and some basic organizational skills.
  They can go straight from Cutters, as long as the local \gls{guard} have need of a writer at the time.

  \Glsentrylongpl{pc} should never achieve this rank.

  \item[\Glspl{thane}]
  \label{thanes}
  must return with captured beasts in order to gain this rank.
  They track down deserters, burn nests, shoot sleeping creatures in the dark, and often travel through the forest with such stealth and foresight that can move alone.
  
  \item[\Glspl{builder}]
  organize the lower ranks to build \glspl{broch}.

  \item[\Glspl{overseer}]
  plan new settlements, organize funds, ensure new recruits have weapons (or don't, if weapons become too expensive), and otherwise do as they please.
  When they grow bored of the job, they turn \pgls{broch} into \pgls{village} and become \pgls{warden}.

\end{description}

\noindent
Would-be \glspl{guard} begin their journey far from any civilization, in \pgls{broch}.
This leaves them with no possibility to protest their position.

Anyone can give the \glspl{fodder} any order, at any time.
Eventually, they become miserable enough to request a weapon, and join a band of others, in the hopes of gaining some rank by travelling beyond the \gls{edge}, and possibly slaughtering a beast.

\widePic{Irina/fury}

\paragraph{Funding}
comes from \glspl{warden}, who pay the guard to keep settlements safe.

The \gls{warden} generally think of the \gls{guard} as a protection racket -- stop payment, and you could lose \pgls{village}.
The \gls{guard} think of the \glspl{warden} as middle-men, since they receive their funds from the surrounding \glspl{village}.

\subsubsection{\Glsfmtplural{broch}}
guard \glspl{village} from the forest, by drawing monsters towards them with light and sound.
Over the day, \gls{guard} fodder cut down trees, and lift logs to the top for a beacon.
At dusk, they light the beacon, and bagpipes provide the noise.
The drone echoes around the whole \gls{broch} as archers lean from windows or balconies, ready to take down whatever comes out.

The \gls{guard} use these towers to protect \glspl{village} for five miles in any direction, by drawing monsters towards themselves with light and sound.
Fodder pull trees down, and haul them up to the top to prepare a beacon for the dusk, then bagpipes start to drone.
Any instrument is welcome, but bagpipes are the loudest.%
\footnote{Technically the loudest instrument is a cannon, but blackpowder has not seen general adoption, and anyone using it would probably dislodge the building's walls.}

When building \pgls{broch}, one should start with a quarry, a stream, and a view of two \glspl{village}, but overseers will accept two out of three.
\Glspl{village} often have towers in their centre, or just beside; most are willing to let the \gls{guard} take the stone and rebuild the \gls{broch} nearby, since this helps keep the predators away.

% image of tower?
% Bailey in the west.
% 3: Beacon of fire.  A griffin flies overhead to the bailey.
% 2: Archer on balcony.  Fodder goes up the stairs carrying wood.
% 1: Stairs up.  Jotter doing paperwork.  Chitincrawler climbing the wall.
% G: Sun Guard arrive with noose, jester at their side.
% U: Digger waiting under a bush
% -1: Digger babies, with NG skeleton.

\subsubsection{\Glsfmtplural{bothy}}
dot the road, every ten miles.
Without \pgls{broch} nearby to summon the beasts of the forest, these long roads crawl with predators; so without the \glspl{bothy}, nobody could move between settlements safely.


\guild{wrecan}% Name
  {Hate}% Death
  {When men fight, when hunger threatens and temper flairs, she comes for souls.
  Within her realm, everyone must pick a side, and every side gains new enemies.
  She watches over wars with contempt for both sides, then claims the souls of the dead.}% God
  {banditry, bigotry, blood feuds, and war}% Domain
  {meditation, camaraderie, reconciliation, diplomacy, and quality armour}% Defence
  {\Glspl{armourer}}% Watchers
  {Creating armour, treaties, and reconciling local disagreements}% Guild activities
work with leather or metal in the \gls{armourHall}, but that's not usually how they start.
Families with bad-tempered, narcissistic, and self-righteous children send them to the \gls{templeOfHate} to learn self-control, and possibly inner-peace.
Within the temple, they learn that aggressively striking iron only makes mangled iron -- to really forge something, a metallurgist must strike solidly, reliably, and with precision.

Apprentices must negotiate their own departure by showing they have learnt control, and by selling at least a few pieces (while giving the guild its cut).
Anyone who becomes cynical and contemptuous of the long process finds themself stuck crafting for years.

Those with the grit to stay become master craftsmen.

\subsubsection{The Structure}

\begin{description}
  \guildRank{Apprentices}
  work, work, work, and work some more.
  \guildRank{Crafters}
  form the entire guild.
  Masters produce good quality armour, and sell it for a high price, but even the least skilled among the armourers earn well enough.
  \guildRank{Chamberlains}
  keep the books, and keep an eye on supply lines.
  The guild always needs more metal.
  \item[\Glspl{proctor}]
  settle disputes, and often have to travel.
  Service usually lasts for a year or two before returning to proper work.
  It doesn't pay well (or at all, sometimes), but at least the guild can continue selling while both sides live.

  While nobody enjoys this job, many gain a reputation for their mediation skills, and find themselves forced into fulfilling the demand, either by plaintiffs, or the rest of the \gls{templeOfHate}.
\end{description}

\noindent
People value the mediation offered by the \gls{templeOfHate} for two reasons.
Firstly, they tend to make impartial rulings.
Secondly, nobody wants to visit the \gls{templeOfJustice} to receive an official ruling.

Common people often make an agreement to accept whatever the local armourer says, and give him a small payment as thanks for listening (both sides pay equally at all times).
\Glspl{warden} pay more, as they often require the armourers to travel to distant towns in order to help them calm a rival (or lull them into a false sense of security).

Despite their calling, they never profit from war.
People need enough armour just to survive the forests -- if they start killing each other as well, entire \glspl{village} could disappear, leaving them with less demand overall.

Many charge the \gls{templeOfHate} with tackling \glspl{fiend} in the forest, especially bandits.
However, the \gls{templeOfHate}'s official position is that the \gls{templeOfBeasts} should handle anything and anyone which lives beyond the \gls{edge}.
\index{Fiends@\glsentryname{fiend}!Temple Responsibility}

\subsubsection{Clanging Halls}
make irritating noise for several streets around, as people work on metal day and night.
The constant noise of hammers, metal, and shrieks from accidents forces everyone to yell instead of talk.
The temple of hate allows people to practice inner calm, but never hands out peace freely.

\guild{yonder}% Name
  {Curiosity}% Death
  {
    A feint bell rings in the distance.

    A child stands at the edge of the dark of the forest, looking amazed.
    He points intently.

    {\sffamily``You need to see this!''}

    The child pries into the dark foliage and disappears, but you can still faintly hear his bell.
  }% God
  {light, candles, torches, paper, cartography, soap, and secrets}% Domain
  {knowledge}% Defence
  {\Glspl{scribe}}% Watchers
  {
    The lower shop concerns itself with fat, used to make candles, soap, and perfumes.
    The upper shop keeps, copies, and sells books.
  }% Guild activities

\widePic[t]{Irina/curiosity}

When people wander into a dark cave, wondering what lies inside, or jump across a chasm, just to see if they can, then fall and die, they go to the realm of curiosity.%
\footnote{If this seems suspiciously indeterminate, then the reader has stumbled into one of \gls{fenestra}'s theological debates.
Someone who leaves and never returns may well have ended their life in the mandibles of some strange creature, but people will call this `death by curiosity' (rather than by beast) if they had no business going where they went.
Every disappearance in the world looks identical to the outside observer, so people label the disappearances by what the missing person wanted to before disappearing, rather than hunting for a body which could be anywhere, and likely consists of some bones.

Similarly, nobody in \gls{fenestra} dies from starvation, since they can pick food from the forest, except of course during \gls{cTwo}.
If someone wanders into \pgls{village} hungry, and shaking from the cold, then later dies, the villagers would say he died from the cold, rather than starvation.}

Parents teach their children about the dangers from an early age.
Any child who asks what lies in curiosity's realm gets a swift smack across the head, and learns to stop asking those questions.
And if that doesn't work, they can join the \gls{paperGuild}.
No need to search in the dangerous dark when you can read about what's there, or so the theory goes.
But the scribes, like the helpers, cannot help themselves, and too many end up investigating something they should have left alone.

\subsubsection{The Structure}

\begin{description}
  \item[\Glspl{chandler}]
  \glsdesc{chandler}.
  \guildRank{Scribblers}
  copy books.
  Anyone can begin as a scribbler, if they answer a riddle (and everyone may receive only one riddle per year, to be answered on the spot or never).%
  \footnote{This system is rigged. The real test is whether someone can cheat tactfully.}
  \guildRank{Scribes}
  have copied over 100 books, and most simply continue.
  \guildRank{Wanderers}
  \label{knowledgeWanderer}%
  begin by identifying the rank of other guild members in the area.
  With no uniforms, and indoor-focussed interests, many have to resort to very clever letters.
  Before setting out, they must display optimized route planning, and encryption skills.

  Once officiated (by letter), they ferry messages for anyone who pays the guild, and occasionally create riddles for the chandlers.
  \item[\Glspl{cartographer}]
  \glsdesc{cartographer}.
  \item[\Glspl{seeker}]
  must prove themselves by taking on three missions, delivered by letter from \gls{secretLibrary}.
  Two missions lead to already-known places, and one leads to uncharted territory.

  Once \gls{secretLibrary} confirms all reports, the guild member becomes an official \gls{seeker}, able to request any knowledge from any library the guild controls.
  \item[\Glspl{librarian}]
  \glsdesc{librarian}.
  \item[\Glspl{philosopher}]
  \glsdesc{philosopher}.
  \ifnum\value{cycle}=2
  \item[The Last Librarian]
  sits in \gls{secretLibrary}.
  Others may enter to work, but they have no idea about the significance of the institution.

  \Gls{secretLibrary} is filled with light at all times.
  \fi
\end{description}

\subsubsection{Scented Libraries}
\index{Libraries}
give cities light.
Every one has the same basic structure.
Downstairs for candles, perfumes, and soaps; upstairs for the written material.
They keep and make maps, write myriad letters daily, and occasionally attempt to unravel the web of lies in historical documents.%
\footnote{`\nameref{warOfLies}' is covered on \vpageref{warOfLies}.}
So far, all they know for sure, is that people once believed that \gls{fenestra} had a king.

\end{multicols}


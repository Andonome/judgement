\chapter{Encounters}

\begin{boxtable}[l|XXXXX|c]
&
  \textbf{Village}  &
  \textbf{Village, Edge}  &
  \textbf{Village, Edge, Forest}  &
  \textbf{Edge, Forest}  &
  \textbf{Forest}  &
  \\\hline
\textbf{Hot} &
  Flood    &
  Heatwave       &
  Bandits  &
  Flood  &
  Peace &
  \textbf{4} \\
\textbf{Hot, Mild} &
  Peace &
  Traders &
  Rain  &
  Vegetation  &
  Cold Blooded Monster  &
  \textbf{3}  \\
\textbf{Hot, Mild, Cold}  &
  Peace     &
  Broken path    &
  Storm  &
  Peace  &
  Warm Blooded Monster  &
  \textbf{2}  \\
Mild, Cold
  &
  Peace     &
  Bandits        &
  Peace &
  Warm Blooded Monster  &
  Peace  &
  \textbf{1}  \\
Cold &
  Traders  &
  Icy road       &
  Snowfall  &
  Peace  &
  Hibernating monster  &
  \textbf{0}  \\\hline
&
  \textbf{0}  &
  \textbf{1}     &
  \textbf{2}  &
  \textbf{3}  &
  \textbf{4}  & \\

\end{boxtable}

\begin{multicols}{2}

\subsubsection*{Roll 2D3.}

\begin{enumerate}
  \item
  The `left D3' determines the row.
  \begin{itemize}
    \item
    Add -1 in the safety of nestled villages.
    \item
    Add +1 in the deep forests, beyond the last road.
  \end{itemize}
  \item
  The `right D3' determines the column.
  \begin{itemize}
    \item
    Add -1 in the cold seasons.
    \item
    Add +1 in the hot seasons.
  \end{itemize}
  \item
  Add the encounters from every row and column.
  
\end{enumerate}

\paragraph{Example:}
Out at the \gls{edge}, the \gls{gm} rolls and finds the result of \textit{3, 2}'.
Rolling a `3' (on the left-hand die) highlights the top row, and the `2' highlights the middle column.

\begin{boxtable}[XXX|c]
  \textbf{Village, Edge}  &
  \textbf{Village, Edge, Forest}  &
  \textbf{Edge, Forest} 
  \\\hline
  \underline{Peace}  &
  \underline{Rain}  &
  \underline{Vegetation}  &
  \textbf{3}  \\
  Broken path &
  \underline{Storm} &
  Peace &
  \textbf{2} \\
  Bandits &
  \underline{Peace} &
  Warm Blooded Monster &
  \textbf{1}  \\\hline
  \textbf{1}  &
  \textbf{2}  &
  \textbf{3}  \\
\end{boxtable}

The left-hand die has struck across `Peace', `Rain', and `Vegetation', while the right-hand die adds `Storm', and another `Peace'.

\begin{exampletext}
  The journey remains peaceful for several days, but clouds gather darkness overhead.
  It breaks suddenly, at night, and the clouds pour out the threats they made over the last week.
\end{exampletext}

The \gls{gm} rolls 1D6 to find out what kind of weird vegetation will come into play, then rolls 2D6 to see how many days the troupe will enjoy their peace.

The storm also gets a 1D6 roll, just to see how bad it is.

\subsubsection{Peace}

Every time you roll peace, add another 1D6 days of peace before the encounter hits.
The party may arrive at their destination unharmed, with all encounters discarded.

\subsubsection{Floods}

Heat often brings flooding, especially around rivers.
The \gls{tn} to overcome problems with flooding equals $7 + 1D6$.
Floods can destroy bridges, roads, and even houses, although most people who live in areas prone to flooding will build their houses somewhere high, or put them on stilts.

\subsubsection{Heatwaves}

The forests present many dangers, but they at least provide protection against one of the most insidious.
Heat begins by stripping people of their senses, and soon burns their skin.

Mechanically, heatwaves inflict 1D6 Fatige on anyone in the area, although the right clothing and preparation (Intelligence + Wyldcrafting) will half the amount taken.

Anyone foolish enough to wear heavy armour in a heatwave will always take the full brunt of the Fatigue Points, plus one per Weight Rating of the armour.

\subsubsection{Bandits \& Brigands}

Once farmers cannot sustain themselves any more, many turn to banditry.

The number appearing is equal to $1D6\times 2 + 4$.

If the left and right-hand dice both point to the bandits, then the party have encountered \emph{brigands} -- soldiers turned to crime.
Brigands plan better, have excellent training, and the best weapons money can buy.


\subsubsection{Traders \& Nomads}

Local traders wander back and forth, pedalling what they have to places that do not have that thing.
They form the blood of civilization.
Unfortunately, they have a very short life expectancy, given all the beasts and bandits on the road.
As a result, most wander the road armed, wearing armour outside of the warmer seasons, and often with a crossbow at their side.

























\end{multicols}

\chapter{The Oracles}
\label{encounters}

This chapter has random encounter tables, random \glspl{village} event lists, and plenty more, for all the things that threaten to stump \pgls{gm} mid-way through a session.
But first, we're gonna get weird.
Please bear with me.

\section{Astronomy}

\label{astronomy}
\label{seasons}
\index{Seasons}
\index{Astronomy}
\index{Time Parallel}

\begin{multicols}{2}

\subsection{The \Glsfmtplural{cycle}}

\begin{exampletext}
  The \gls{ainumar} shines bright in the night sky, and appears a little larger than our moon does to us.
  A raging storm moves across its face, which people can see about half of the year.
  They believe that the gods live there, inside the moving eye of the storm.

  \Gls{fenestra} swings around the \gls{ainumar}, coming so close in that it could almost kiss the gods, and then hurtles back to sit in empty space, far away from the Sun.
  \index{Seasons}
\end{exampletext}

\Gls{fenestra} changes constantly as it circles the \gls{ainumar}, and the \gls{ainumar} circles the Sun.
The current season determines the events in the sky, which \glspl{monster} stalk the endless forest, and which plants are in bloom.

Every sixty days \gls{fenestra} completes \pgls{cycle} around the \gls{ainumar}, before \pgls{storm} descends, bringing earthquakes, lightning, and tidal waves.
The \glspl{storm} last around a day, and mark a change in weather.
\index{Weather}

Each year has six \glspl{cycle}.

\subsubsection{\Glsfmttext{cOne}}
\glsentrydesc{cOne}.

\subsubsection{\Glsfmttext{cTwo}}
\glsentrydesc{cTwo}.

\subsubsection{\Glsfmttext{cThree}}
\glsentrydesc{cThree}.

\subsubsection{\Glsfmttext{cFour}}
\glsentrydesc{cFour}.

\subsubsection{\Glsfmttext{cFive}}
\glsentrydesc{cFive}.

\subsubsection{\Glsfmttext{cSix}}
\glsentrydesc{cSix}.

\setcounter{track}{\month}
\setcounter{age}{1}
\boxPair[t]{
  \parallelCalendar
}{
  \orrery[1/8, 1/23, 2/8, 2/23, 3/8, 3/23]
}

\setCycle{\month}{\day}

\subsection{Cosmological Events}
\label{weather}

The changing sky brings danger, but also power, as many events boost people's spell \glspl{sphere} -- even those who had none before.
These bonuses stack with \glspl{boon}, so someone with the right know-how can (and do) stand under the unnatural darkness of an eclipse, holding the ground-up remains of a stirge queen, with auroch hooves boiled to melting point, at the top of a mountain,%
\footnote{High level spells often have such a long casting range that casters must climb high to see their target.}
screeching about opening a portal from a distant \gls{village} to an underground realm.

Since most people don't know any spells, they have little use for the extra \gls{sphere}, but this marks a good time to learn.

Since mana comes on the wind, people can regenerate more or fewer \glspl{mp} depending on the weather.

\toggletrue{genExamples}
\weatherChart

\subsubsection{\Glsfmtplural{coldSnap}}
\index{Cold Snap \glsentrytext{R}}
\index[mana]{Fate!Cold Snaps \glsentrytext{R}}
\glsentrydesc{coldSnap}, including those who cannot normally cast spells.

\subsubsection{\Glsfmtplural{eclipse}}
\index{Eclipses \glsentrytext{R}}
\index[mana]{Air!Eclipses \glsentrytext{R}}
\glsentrydesc{eclipse}.

\subsubsection{\Glsfmtplural{heatwave}}
\index[mana]{Fire!Heatwaves \glsentrytext{R}}
\glsentrydesc{heatwave}.

\subsubsection{\Glsfmtplural{snow}}
\glsentrydesc{snow}.
In the deep forests, this makes little difference, but an open road which once granted five miles of easy travel per \gls{interval} would only allow three miles once covered in frost.

\subsection{\Glsfmtplural{storm}}

Every sixty days, \pgls{storm} comes, bringing calamity.

\subsubsection{\Glsfmtplural{earthquake}}
\index[mana]{Earth!Quakes \glsentrytext{R}}
\glsentrydesc{earthquake}.

\subsubsection{\Glsfmtplural{flood}}
\index[mana]{Water!Floods \glsentrytext{R}}
\glsentrydesc{flood}.

\subsubsection{\Glsfmtplural{hurricane}}
\glsentrydesc{hurricane}.

\subsubsection{\Glsfmtplural{landslide}}
\glsentrydesc{landslide}.

With enough foot-fall, the penalty reduces to nothing, as others stamp the snow down, reducing it to frosty mud.

%! Lightning!

\subsection{Parallel Time}
\index{Parallel Time}

\Gls{fenestra}'s time keeps moving, just as ours does, but faster.
For every Mundane month, \gls{fenestra} experiences two sixty-day \glspl{cycle}.
\setcounter{track}{\month}
\setCycle{\month}{\day}
While we have \trackMonth, \gls{fenestra} has `\showCycle' -- a \showTemperature\ season.

Every three Mundane months, \gls{fenestra} begins a new year; so on January, April, July, and October the 1st, a new \gls{cycle} starts with \gls{cOne}.
Four days pass there for each of ours, so when we hit the second day of the month, \gls{fenestra} hits day 8 of \pgls{cycle}.
Then on the 15th of each Mundane month, the \gls{cycle} hits day 60, which means \pgls{storm} descends, and a new \gls{cycle} begins.

During a game, the \gls{gm} tracks time as usual, skipping over large chunks as they please.
But each week should not track more than half \pgls{cycle} (i.e. thirty days in \gls{fenestra}).
Games which begin with a new \gls{cycle} will begin with \pgls{storm}, and games which cover the full thirty days until \pgls{cycle}'s end will end with \pgls{storm}.

And if your face has contorted with disapproval and you wonder about the point of keeping every \gls{downtime} consistent, read `\nameref{realtimeScheduling}' \vpageref{realtimeScheduling}.

\end{multicols}

\section{Things on the Road}
\label{randomEncounters}

\begin{multicols}{2}

\subsection{Creatures}

Which creatures are active depends on the \gls{cycle}'s temperature.

\toggletrue{genExamples}
\setcounter{enc}{19}
\begin{boxtable}[rXc]
  \Repeat{18}{%
    \addtocounter{enc}{-1}%
    \arabic{enc} & \bigBeastList &
      \randomize\showInterval{\value{encnum}}
    \\
  }
\end{boxtable}

\setcounter{temperature}{0}
\paragraph{During \showTemperature\ \gls{cTwo}}
encounters use the first $1D6$ to find the number of days until the encounter, the second $1D6$ for the result, and the last to determine the number.%
\footnote{Rolling to see `how many days until an encounter' may seem strange to anyone used to the standard method of `roll to check for an encounter \emph{now}.
However, with a five-day journey, rolling a die five times in a row becomes a little tedious.
This method lets the \gls{gm} make one roll for a large chunk of a journey.}

Rolling \dicef{4}~\dicef{2}~\dicef{1} would means `after 4 days, the \glspl{pc} encounter a goat during the morning'.

\stepcounter{temperature}
\paragraph{The \showTemperature\ \glspl{cycle}}
uses the lowest of the first two dice to determine how many days until the encounter occurs.
Add the next two dice to determine the encounter.

Rolling \dicef{2}~\dicef{1}~\dicef{3}~\dicef{4} would means `after 1 day, the \glspl{pc} encounter a herd of 80 aurochs during the afternoon'.

\stepcounter{temperature}
\paragraph{The \showTemperature\ \glspl{cycle}}
use use the lowest of the first three dice to determine how many days until the encounter occurs.
Add the last three dice to determine the encounter.

Rolling \dicef{2}~\dicef{2}~\dicef{3}~\dicef{5} would means `after 2 days, the \glspl{pc} encounter \pgls{crawler}'.

\stepcounter{temperature}
\paragraph{During \showTemperature\ \gls{cFive}}
use use the lowest of all four dice to determine how many days until the encounter occurs, and the highest three of the four dice to find the encounter.
The last die determines any quantities.

Rolling \dicef{3}~\dicef{3}~\dicef{1}~\dicef{5} would means `after 1 day, the \glspl{pc} encounter \pgls{basilisk}..

\boxPair{
  \subsection*{Fast Rolls}

  Nobody likes waiting while \pgls{gm} makes multiple dice rolls.
  Generate a full encounter instantly by casting four dice like a fist of rune-stones; read their left-to-right order, and interpret the last two as weather.

  \begin{minipage}{.2\linewidth}
    \dicef{6}

    \qquad\dicef{1}

    \hspace{3pt}\dicef{2}{\par\hspace{-2em}\dicef{3}}
  \end{minipage}
  \begin{minipage}{.7\linewidth}
    \vspace{.5em}
    Even if the dice-roll looks like a complete mess, every roll has a visible left-to-right order.
    Arranging this example roll, we would get \dicef{3}~\dicef{6}~\dicef{2}~\dicef{1}.
  \end{minipage}

  \begin{exampletext}

    One \showTemperature~\gls{cOne}, the \glspl{pc} emerge from \pgls{broch}, and head towards \pgls{village}.

    \dicef{1}~\dicef{4}~\dicef{2}~\dicef{5}: A mild breeze blows.
    They meet a trader on the road, selling salted meats, but \pgls{jotter} already gave them some rations, so they continue towards the \gls{village}.
    Work is difficult, and takes three days.

    \dicef{4}~\dicef{3}~\dicef{3}~\dicef{1}: The day they head back, a thick mist rises.
    \Pgls{griffin} spots them, and begins its mimicry, singing a child's song from the forest (see \vpageref{griffin} for notes on their mimicry).

    Once \gls{cTwo} hits, their next mission involves going past the \gls{edge}.
    \dicef{2}~\dicef{3}~\dicef{3}~\dicef{6}: After two days of peace, snow falls.
    As the \glspl{pc} make camp, nine wolves take the opportunity to steal their food.

  \end{exampletext}
}{
  \toggletrue{genExamples}
  \allEncounterTables
}

\makeAutoRule{morale}{Morale Checks}{When a creature's \glsfmtlongpl{hp} fall below $2D6$, it flees}
\index{Morale}
give the \gls{gm} an impartial way to rule when antagonists flee from combat.
At the start of an encounter, roll $2D6$ and leave the result on the table.
If the creature has more \glspl{hp} than the roll, it attacks; otherwise, it watches quietly.

\sidebox{
  \begin{boxtable}
    \textbf{Roll} & \textbf{\glsfmtlongpl{hp}} \\
    \hline
    \twoDice{5} & 10 (\textit{ATTACK!}) \\
    \twoDice{5} &  8 (\textit{ATTACK!}) \\
    \twoDice{5} &  5 (\textit{wait\ldots}) \\
    \twoDice{8} &  4 (\textit{RUN!}) \\
    \twoDice{8} &  2 (\textit{RUN!}) \\
  \end{boxtable}
}

\Glspl{crawler} often have 10~\glspl{hp}, so they often attack people the moment they spot them.
\Glspl{griffin} have fewer \glspl{hp} than \glspl{crawler}, so they are more likely to stalk the party, or simply leave.
And goblins might have only 5 or 6~\glspl{hp} -- they know how small they are, and prefer to spend time laying traps, or stealing equipment, rather than engaging directly.
Every \gls{hp} a creature loses could prompt it to turn and flee, so sometimes only wounded creatures will flee, while others keep fighting.

On a tie, the creature pauses to reassess the situation.
A single \gls{crawler} eyeing up a party might display its claws, then observe their reactions; if the troupe do nothing to change the situation (such as casting a spell, or making a loud noise) then the creature rolls again.

\Glspl{pc} do not take Morale checks -- the players decide when it's time to run away by the look of the situation.
Usually a good time is when all the \glspl{fp} have run out.
\index{Henchmen}%
However, \glspl{npc} travelling with the \glspl{pc} still take morale checks as usual.%
\footnote{The \glsentrytext{gm} may also wish to cut all Morale checks for any \glspl{npc} with remaining \glsentrytext{fp}.}%

\makeAutoRule{noise}{Beyond the \Glsfmttext{edge}}{Each noise or light prompts a 1 in 6 chance of an encounter roll}
every noise or light becomes a beacon for creatures all around.
If anyone shouts, makes a fire, or otherwise draws attention to themselves, roll $1D6$; on the roll of a `1', roll on the encounter table immediately.

Deer and auroch flee when they hear a sound, but predators will immediately hunt for the source of the sound, and arrive after \pgls{interval}.

\subsection{The Civilization Rating}
\label{civilizationRating}
\index{Civilization Rating}

More civilized areas have fewer creatures and more traders.
If the first die is equal or below the Civilization Rating, replace the usual creature encounter with traders.
In this way, anyone approaching a town will slowly find the natural encounters being eaten away by civilization.

The time until the encounter looks at a number of dice equal to the Civilization Rating, and selects the lowest; so a Civilization Rating of `2' means the lowest of the first 2 dice are used to see how many days until the encounter occurs.

\setcounter{encnum}{1}
\begin{dlist}
  \item
  \encCivilization\ indicates a caravan with \arabic{dlist} useful item.
  \stepcounter{encnum}

  \item
  \encCivilization\ indicates a caravan with \arabic{dlist} useful item.
  \stepcounter{encnum}

  \item
  \encCivilization\ indicates a caravan with \arabic{dlist} useful item.
  \stepcounter{encnum}

  \item
  \encCivilization\ indicates a caravan with \arabic{dlist} useful item.
  \stepcounter{encnum}

  \item
  \encCivilization\ indicates a caravan with \arabic{dlist} useful item.
  \stepcounter{encnum}

  \item
  \encCivilization means no possible monster encounters, and \glspl{pc} can use the local, massive, markets.
\end{dlist}

\subsubsection{Items for Sale}
\label{traders}
use the remaining dice to determine what the traders have which might interest the \glspl{pc}.
Traders always shout this information ahead of them, to entice people to buy their goods.

\encTraders

Each of your encounter dice also represent an interesting item the traders have.
For example, rolling \dicef{8}\dicef{3} would indicate that the \glspl{pc} can buy some rope and torches.

\index{Selling Beasts}
Traders also purchase live \glspl{monster} for a number of \glspl{sp} equal to their \gls{cr}, or half that number if the body is well prepared.


\end{multicols}

\setCycle{\month}{\day}

\section{Missions at the \Glsfmttext{broch}}
\label{NGmissions}
\index{Adventure Generator}
\index{Missions}

\begin{multicols}{2}

\noindent
When players start a game, their characters will be \glspl{fodder}, with menial, pointless chores.
These menial chores do three things: first, they let players know their characters' place in society; second, they let new players get used to resolving actions; and third, they set the right tone for the pace.
In just one roll the characters complete a `mission' to shine armour, then they might help \pgls{village} within fifteen minutes' of Mundane time.

Once \pgls{pc} or two has risen in rank, the missions will take more time, skill, and effort.
These higher-level missions have two purposes: first, they push the \glspl{pc} into the wider-world where \glspl{sq} force them into further action; and second, so you have something for them to do when you don't know what to do.

So when the game feels full, ignore the missions; the \gls{jotter} doesn't need anything right now.
But the moment the game feels like it has no direction, come back and roll up a new mission.

\paragraph{The troupe leader}
takes responsibility for the troupe's failure, whether or not anyone actually listens to them.
At the start of each mission, \pgls{jotter} will appoint the character with the highest rank (ties are broken by \roll{Charisma}{Deceit}).

\paragraph{Successful missions}
mean the \gls{jotter} will promote the lowest-ranking member of the troupe, or possibly two, if they did well.%
\footnote{Many of the \glsentrytext{guard} actively avoid gaining any kind of rank, as it only leads to more difficult duties.}
\index{Promotion}

\null
\begin{itemize}
  \item
  Roll $1D3$ for \glspl{fodder}.
  \item
  Roll $1D6$ for \glspl{gDigger}.
  \item
  Roll $1D6+1$ for \glspl{soldier}.
  \item
  Roll $1D6+2$ for \glspl{cutter}.
  \item
  Roll $1D6+3$ for \glspl{ranger}.
\end{itemize}

Then add a complication with the same bonus as before (\autopageref{missionComplications}).

\subsubsection{Goal}
\index{Missions}

\null
\ngMissions

\subsubsection{Complications}

Add a bonus for rank, as before.

\missionComplications

\subsection{At the \Glsfmttext{broch}}

\Glspl{broch} often run short on supplies, especially when they don't have many \glspl{village} nearby.
Once the \glspl{pc} arrive at \pgls{broch}, make a roll to see what it lacks.

\null
\brochDerths

If an area suffers from poverty and death, it always affects supplies to the \glspl{broch}.
In this case, you might roll two dice at each \gls{broch} to see what's missing, and increase this roll until prosperity returns.

\end{multicols}


\section{Emergent Situations}
\label{roadEncounters}

\begin{multicols}{2}

\noindent
The \glspl{pc} will find more trouble than monsters as they travel.
Whenever a journey ends, a situation begins.

\begin{itemize}
  \item
  \Glspl{bothy}: \vpageref{bothyEvents}.
  \item
  \Glspl{village}: \vpageref{villageEvents}.
  \item
  \Glspl{lonelyTavern}: \vpageref{lonelyTaverns}.
  \item
  Town: \vpageref{townEvents}.
  \item
  \Glspl{court}: \vpageref{courtVerdicts}.
\end{itemize}

\subsubsection{\Glsfmtplural{bothy}}
\label{bothyEvents}
\glsentrydesc{bothy}.

Each \gls{bothy} begins when someone dies on the road, and others leave rocks there to pay their respect.
If enough people gather rocks, it becomes a cenotaph, and takes the name of the fallen.
If more rocks follow, the cenotaph becomes \pgls{bothy}.

Anyone using \pgls{bothy} must gather and leave as much firewood as they found when they entered, except the \gls{guard}, who must leave even more, fix anything which needs fixing.
Nobody will trust someone known to leave \pgls{bothy} in a bad state, but if \glspl{guard} leave them in a bad way, they will face questions in the \gls{court}.

Roll to check \pgls{bothy}'s circumstances upon arrival:

\bothyEvents

%!
\needspace{18em}
\subsubsection{\Glsfmtplural{village}}
\label{villageEvents}

\encVillageEvent

\subsection{Lonely Taverns}
\label{lonelyTaverns}
\index{Lonely Taverns}

These taverns exist on long stretches of road, far from any town, and
charge high prices for a drink. They must live off traders passing
through, and survive whatever the forest brings out.

Normal people don't stay for long.
Those who stay a while often have problems with the local
law, as these places often make their own laws.
Barkeeps punish any robbery close to the tavern harshly, but don't often care about what people do around the towns.
This makes these taverns a safe intermediate location where anyone can talk in peace.

Of course, bandits won't announce themselves as such when speaking with \glspl{guard}, but then the \glspl{guard} often won't announce their employment either, no matter how obvious that sword on their back makes them.

\subsubsection{The Barkeep}

Roll $1D6$ to find this \gls{cycle}'s barkeep (they change all the time):

\begin{dlist}
  \item
  A veteran of the \gls{guard}, with a hundred war-stories. Of course
  when he tells them, nobody can get a drink, so don't ask!
  \item
  Someone from point 4 on the map, hiding here with a bounty on their head for thieving from \pgls{warden}.
  \item
  An outlander from a land so far away, nobody has ever heard of it.
  Every story she tells sounds made-up, but the strange accent shows she really does come from somewhere distant.
  \item
  A powerful \gls{witch} who swore an oath never to use magic again.
  He won't say why.
  \item
  A collective -- you stay as long as you like, earn your keep, then go
  when you please. Sometimes in the colder Seasons, the place just lies
  barren.
  \item
  A dwarf who records all he can.
  The patrons say he works as a spy for someone, but they disagree about whom.
\end{dlist}

\subsubsection{The Menu}
\index{Menus|see {Lonely Tavern}}

Roll $3D6$ -- the \glspl{pc} can order any of these meals.

\newcommand\menuItem[3][(\arabic{r12} \glspl{cp})]{%
  \item
  \textbf{#2:}
  #3
  \randomize%
  #1
}

\begin{dlist}
  \menuItem{Griffin-wing}{freshly killed this morning, after the \gls{griffin} tried to fly away with a gnomish patron.}
  \menuItem{Mystery-stew}{why are you hesitating?
  It goes rotten quick, so get eating!
  \footnote{\Glsfmttext{crawler} `meat' (webbing as sauce!)}
  }
    \menuItem{Sunrise Soup}{the chef found a new plant this morning, and he's already learning how to cook it!
    \footnote{In fact this is \nameref{uproot}, see \autopageref{uproot}, for the effects.}
    }
  \menuItem{Deer}{thank the man in green, sitting in the corner -- he caught it this morning.}
  \ifodd\value{r4}
    \menuItem{Dwarf-beard}{actually just a type of seaweed, left as payment by a local trader; but it tastes just like the real thing!}
  \else
    \menuItem{Eye-Spy}{made with actual \gls{woodspy}.}
  \fi
  \ifodd\value{r3}
    \menuItem[(0 \glspl{cp})]{Get bent}{the barkeep's in a foul mood, because they need a day off.}
  \else
    \menuItem[]{\ldots and bugger-all-else}{a few barrels turned out to be rotten, and now someone's stolen an entire pot of soup.
    The menu will be limited for the day.}
  \fi
\end{dlist}

\subsubsection{The Patrons}
include $2D6$ \gls{guard} \glspl{gDigger}.
Roll $3D6$ for the rest, and accept the unique results.

\begin{dlist}
  \item
  An elf who doesn't speak the \gls{tradeTongue}.
  \item
  \Pgls{mixer} from the \gls{templeOfSickness}, on a mission to recruit from the \glspl{village}.
  \item
  \Pgls{doula}, carrying a map of one point (roll $1D6+4$), and preparing to make another.
  \item
  A piper who really wants to practice, and keeps justifying why they should be allowed to make loud noises.
  \item
  A caravan of $1D6$ traders (roll their wares \vpageref{traders}).
  \item
  \Pgls{cartographer}, here to collect information for the \gls{templeOfCuriosity}.
\end{dlist}

\subsection{Trouble at the \Glsfmttext{court}}
\label{courtVerdicts}

\Gls{fenestra}'s legal system can be unpredictable, but must never be dull.
When looking for a legal verdict, in the \gls{court} the accuser and accused make speeches, and make a resisted roll of \roll{Charisma}{Empathy}.%
\footnote{See \autopageref{pitOfJustice} for the building's structure.}

%\null
\begin{itemize}
  \item
  The accuser always gets +1.
  \item
  Either side gets +1 for good evidence.
  \item
  Either side gets +1 for entertaining evidence.
  \item
  Membership in the \pgls{templeOfJustice} grants +3 to the roll.
  \item
  Knowing the \gls{warden} grants +1 to the roll.
  \item
  Either side gets -3 for insulting the \gls{warden}.
  \item
  If the jester likes one side, they gain +1.
  \item
  If the jester hates one side, they gain -1.
\end{itemize}

The jester is fickle, the crowds love a twist, and the \gls{warden}'s patience grows thin.

\begin{description}
  \item[A successful Accusation]
  means a roll for punishment.
  \item[A tie]
  means both characters may be in trouble: roll on the mistrial results, \vpageref{mistrials}.
  \item[A failed accusation]
  means there is no punishment for the accused.
  However, if the accuser is not a high-standing member of a guild, they will receive punishment for wasting everyone's time.
  Roll for the accuser's punishment!
\end{description}

\begin{dlist}
  \item
  \ifodd\value{temperature}
    The accused must pay a fine of $1D6\times 1D6\times 1D6\times 10$~\glspl{sp}.
    Every day unpaid raises the fine by 10\%.
    If they owe more than 2,500~\glspl{sp}, they must go to prison until they repay all debts.

    (interest reduces to 5\% per day while in prison)
  \else
    The accused becomes the jester (the jester was getting old anyway).
  \fi
  \item
    The accused becomes the executioner -- the job is `for life!' (and apparently the crowd find this hilarious).
  \item
    The Jester demands the accused become a `\gls{basilisk} bather', and wash the stench off \pgls{basilisk} as penance for their filthy deeds.
    To everyone's shock, the \gls{warden} agrees!
  \item
  The accused seems like capable person\ldots Capable of violence!
  Time to join the \gls{guard} as fodder.%
  \footnote{See \autopageref{fodder}. Anyone already in the \gls{guard} becomes demoted, but has no other punishment.}
  \item
  Death by irony!
  The accused will be killed by the very thing they inflicted on others.
  \item
  Death by hanging!
  The accused must die!

  (Four \glspl{sunGuard} enter, carrying swords and rope)
\end{dlist}

\index{Mistrial}
\paragraph{Mistrials}
mean that both accuser and accused must pay the price.
After all, just because the accuser has wasted everyone's time, doesn't mean the defendant doesn't look like the `wrong type', and if the accused makes their own accusations (as most do) then justice may demand that both suffer.

\label{mistrials}
\begin{dlist}
  \item
  Boring!
  We're here for justice, not a lecture.
  Both accuser and accused go to prison for $1D6$ days, then the trial repeats.
  \item
  Time to make up and be friends, during\ldots

  Trial!

  ~by!

  BEAST!

  Let in the \ifodd\value{r4} \glspl{crawler}\else \glspl{griffin}\fi!

  (everyone gets a shortsword, the \gls{warden} declares any survivors innocent)
  \item
  Another accuser appears, to accuse the accuser of an even worse crime (this nullifies the first crime, as everyone forgets about it).
  Restart the trial!
  \item
  Accusation-switch!
  Accuser becomes accused as a new witness comes to light.
  (Restart the show and roll for the other side).
  \item
  $1D6$~\glspl{sp} fine each, for wasting time.
  \item
  The \gls{warden} feels lenient -- everyone goes free.
\end{dlist}

\end{multicols}

\encTownEvents


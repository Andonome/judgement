\chapter{Encounters at the Crossroads}
\index{Seasons}
\label{encounters}

\Gls{fenestra}'s institutions, structure, politics, religion, and \emph{food} all revolve around the constant problems presented by monstrous creatures wandering around the world.
These encounters do not occur every hour of every day; just as deer walk through dark forests for a lifetime, so people can also visit a nearby \gls{village} without certain death.
However, they will experience a \textit{reasonable danger} of death if they walk carefully, and worse if they wander carelessly.

\section{Things on the Road}

\begin{multicols}{2}

\subsection{Daily Encounters}

\begin{speechtext}
  Everyone travels together, in heavy wagons, and bows ready for action.
  But once you arrive back home, the traders finish and want to move on by morning, which doesn't leave much time to say `hello' to everyone.
  So you decide to leave with the next trading caravan, but a days later and nobody's arrived.
  Nobody tells you to go, but when food's running low you just kinda \emph{feel} it.

  The little places by the towns don't see much danger, so if you're not too far from a big place, you might just walk to the nearest road and see if you can join a caravan heading back to town.
  Of course the bow's at the ready, and a spear if you can stand the extra weight.

  When you spot a creature, you can try to stare it down.
  If that doesn't work, give it a nasty sting and most go away.
  Gotta be careful close to the cold though -- once the beasts feel the frost coming on they get \emph{fearless}.

  Once the cold settles in, the forest calms down.
  Most of the big nasty ones sleep under the snow until it melts.
\end{speechtext}

\boxPair[t]{
  \begin{encChart}{Weather}
    \Repeat{16}{
      \encLine \bigWeatherList \\
    }
  \end{encChart}
}{
  \begin{encChart}{Forest Encounter}
    \Repeat{16}{
      \encLine \bigBeastList \\
    }
  \end{encChart}
}

\begin{figure*}[b!]

\begin{multicols}{3}
\small
%%% Ex table 1
\setcounter{enc}{15}
\setcounter{diceNo}{13}
\vspace{2em}
\rowcolors{2}{}{gray!10}
\noindent
\begin{boxtable}[c|L]
  \hline
  \hline
  \textbf{Roll} & \textbf{Inner Hamlets} \\
  \hline
  \Repeat{2}{
    \addtocounter{diceNo}{-1}
    \addtocounter{enc}{-1}
    \arabic{diceNo} & \bigBeastList \\
  }
  \hline
  \addtocounter{diceNo}{-1}
  \addtocounter{enc}{-1}
  \arabic{diceNo} & Bandits! \\
  \hline
  \Repeat{8}{
    \addtocounter{diceNo}{-1}
    \addtocounter{enc}{-1}
    \arabic{diceNo} & \arabic{diceNo} wagons \\
  }
  \hline
\end{boxtable}

%%% Ex table 2
\setcounter{enc}{15}
\setcounter{diceNo}{13}
\vspace{2em}
\rowcolors{2}{}{gray!10}
\noindent
\begin{boxtable}[c|L]
  \hline
  \hline
  \textbf{Roll} & \textbf{The Lonely Road} \\
  \hline
  \Repeat{6}{
    \addtocounter{diceNo}{-1}
    \addtocounter{enc}{-1}
    \arabic{diceNo} & \bigBeastList \\
  }
  \hline
  \addtocounter{diceNo}{-1}
  \addtocounter{enc}{-1}
  \arabic{diceNo} & Bandits! \\
  \hline
  \Repeat{4}{
    \addtocounter{diceNo}{-1}
    \addtocounter{enc}{-1}
    \arabic{diceNo} & \arabic{diceNo} wagons \\
  }
  \hline
\end{boxtable}

%%% Ex table 3
\setcounter{enc}{15}
\setcounter{diceNo}{13}
\vspace{2em}
\rowcolors{2}{}{gray!10}
\noindent
\begin{boxtable}[c|L]
  \hline
  \hline
  \textbf{Roll} & \textbf{Outer \glspl{village}} \\
  \hline
  \Repeat{9}{
    \addtocounter{diceNo}{-1}
    \addtocounter{enc}{-1}
    \arabic{diceNo} & \bigBeastList \\
  }
  \hline
  \addtocounter{diceNo}{-1}
  \addtocounter{enc}{-1}
  \arabic{diceNo} & Bandits! \\
  \hline
  \Repeat{1}{
    \addtocounter{diceNo}{-1}
    \addtocounter{enc}{-1}
    \arabic{diceNo} & \arabic{diceNo} wagons \\
  }
  \hline
\end{boxtable}

\end{multicols}

\end{figure*}

\noindent
Encounters adjust, depending on the season, time of day, and how far the troupe have travelled from civilization.

\begin{description}
  \item
  Roll 3 dice read the next encounter.
  \begin{description}
    \item[Die 1]
    shows how many \glspl{interval} until the encounter.
    \item[Die 2 + die 3]
    tell you the weather.
    \item[Die 1 + die 2]
    tell you what creature emerges.
    \item[Die 3]
    may tell you the number of creatures appearing.
  \end{description}
  \item[During the day]
  staying close to settlements means a low encounter roll replaces the normal predators with traders.

  However, it can also mean a greater chance of bandits.

  \item[At night]
  roll as normal, with -1 to the encounter roll and weather rolls.
  The troupe will never find traders -- they do not travel at night.
  \item[In the deep forest]
  Every noise prompts a chance for an encounter.
\end{description}

\end{multicols}

\pagebreak

\begin{multicols}{2}

\begin{exampletext}
  For example, the \gls{gm} rolls `\dicef{6} \dicef{3} \dicef{4}'.

  On the first reading, this says `after 6~\glspl{interval}, a storm is brewing'.
  (\dicef{6}~:~\underline{\dicef{3} + \dicef{4}})

  On the second reading, wolves arrive, numbering 12.
  (\underline{\dicef{6} + \dicef{3}}~:~\dicef{4})

\diceTrio{6}{3}{4}{}{12 Wolves}{Brewing storm}

\end{exampletext}

\subsubsection{The Civilization Rating}
\label{civilizationRating}
\index{Civilization Rating}
Around the large towns, almost nothing gets through the protective layers of \glspl{broch} and outer \glspl{village}.
Of course, any person can get through, but wild predators just move towards the first noises they hear, and then receive a nasty sting from archers.
After enough predators die or leave an area, it becomes more reliably safe, at least during the daylight.

The `Civilization Rating' equals the number of \glspl{village} within a few miles.
So on the roads near a small settlement with 2~\glspl{village}, the civilization rating is `2', which means you should replace any encounters with a roll of 2 or less with traders.
Larger settlements, with 10~\glspl{village} in the area, mean you can replace any encounter~roll of `10' or less with traders.

You can think of the Civilization Rating as a line which moves down, slowly altering the encounter tables.
Past the end of the last road, the forest replaces the all traders with beasts.

\subsection{Traders \& Nomads}

While moving, everyone meets other travellers head-on, moving in the opposite direction from them.
They briefly swap news of the road behind, occasionally make a trade, then move on.

\encTraders

Each of your encounter dice also represent an interesting item the traders have.
For example, rolling \dicef{8}\dicef{3} would indicate that the \glspl{pc} can buy some rope and pitch.

\subsubsection{Noises in the Forest}

In the dark forests, where towering trees dampen the Sunlight, any out-of-place noise, light, or smell, can excite and draw in the predators which live there.
Whenever the troupe draw attention to themselves, roll $1D6$.
On the roll of a `1', roll for a random encounter.

You can pull this off quickly with $3D6$, where only the first die indicates the encounter, where the others indicate \emph{which} (if any).

\subsection{Weather}
\label{weather}
Cosmological and meteorological events can boost people's abilities to cast spells -- even those who had none before.
These bonuses stack with \glspl{ingredient}, so someone with the right know-how can (and do) stand under the unnatural darkness of an eclipse, holding the ground-up remains of a stirge queen, with auroch hooves boiled to melting point, at the top of a mountain, screeching about opening a portal from a distant \gls{village} to an underground realm.%
\footnote{High level spells sometimes need to target something beyond the horizon.
As a result, spellcasters often have to traipse up mountains in order to cast their biggest spells.}

\subsubsection{Cold Snap}
\index{Cold Snap}
\index[mana]{Fate!Cold Snaps}

Cold snaps drain the life from characters and wildlife in a blink.
Until the weather warms up again, all actions outdoors inflict an additional \gls{ep} or more, depending on clothing.
Someone journeying outdoors effectively naked will receive 4 \glspl{ep}, per \gls{interval}.

The mass disturbances in the air, mixed with the darkness, creates a kind of magic in the air, which grants +1 to everyone's Fate Sphere, including those who cannot normally cast spells.

\subsubsection{Earthquakes}
\index{Earthquakes}
\index{Quakes}
\index[mana]{Earth!Quakes}

When the ground shakes, \gls{village} walls often fall, houses can topple inward, and gnomish warrens have their planning tested from the foundations to the rooves.
Despite the shaking, most structures remain standing.
Architects in \gls{fenestra} design castles to remain standing, despite the quakes, and dwarvish settlements at the edge of the \gls{deep} often feel nothing, as underground caverns don't shake during quakes as badly as the surface does.

When quakes alter rock, everyone with the tiniest understanding of magic finds themselves able to speak to stone, at least a little, as rocks and ice start to wake up.
This time grants a +1 bonus to everyone's Earth Skill, for the short duration of the quake.

The average person makes little use of this time -- they have more pressing concerns given the sudden lack of Sunlight.
But casters often plan grand rituals to coincide with an eclipse, as the extra powers allow them to perform wide-ranging, potent spells.

\subsubsection{Eclipses}
\index{Eclipses}
\index[mana]{Air!Eclipses}

When \gls{fenestra} passes behind the \gls{ainumar}, the Sun remains covered for hours.
For this entire \gls{interval}, the winds becomes more pliable, and anyone with a voice to speak gains +1 to their Air Skill.

\subsubsection{Floods}
\index{Floods}
\index[mana]{Water!Floods}

Floods can often damage infrastructure worse than earthquakes.
They rot food, and degrade the foundations of houses in subtle ways, which only become apparent years later.
While high fortifications can remain untouched, and underground dwelling runs the risk of water pouring in from above, driving everything inside up into the Sunlight.

The predators of \gls{fenestra} seem to have an instinct for floods, and will camp outside any homely holes in their territories.

Travel during a flood poses serious problems -- any affected area will half the travellers' rate of movement.

While floods occur, everyone gains a +1 Bonus to their Water Skill, including those who began with 0.

\subsubsection{Heatwaves}
\index{Heatwaves}
\index[mana]{Fire!Heatwaves}

At the height of the warm seasons, heatwaves arrive and make \gls{fenestra} deadly, especially for the \gls{guard}.
All travel in direct Sunlight will have the travellers endure an additional +2 \glspl{ep} when wearing loose-fitting clothing, or +4 when wearing inappropriate clothes.
Armour of any kind does not count as `appropriate', so a heatwave can force any warrior to remove their armour.

During a heatwave, everyone gains a +1 bonus to their Fire Skill.

\subsubsection{Snow}
\index{Snow}

Snow halts travel.
Travelling over snow-covered lands reduces the standard travel by a third, then a half if snow falls again.
So when the troupe wants to travel 8 miles over road, they take the usual 4 \glspl{ep}, but only cover 6 miles after the snow falls.
Once more falls, they would only cover 4 miles.




\end{multicols}

\setcounter{diceNo}{13}
\setcounter{diceNo2}{15}
\setcounter{enc}{17}

\begin{wideTable}[c|c|c|LLL]{Encounters Across Biomes}
  \hline
  \hline
  \textbf{Warm} & \textbf{Mild} & \textbf{Cold} & \textbf{Forest} & \textbf{Lakeside} & \textbf{Mountains} \\
  \hline
  \encLine \bigBeastList & \encLakeside & \encMountains \\
  \encLine \bigBeastList & \encLakeside & \encMountains \\
  \encLine \bigBeastList & \encLakeside & \encMountains \\
  \encLine \bigBeastList & \encLakeside & \encMountains \\
  \encLine \bigBeastList & \encLakeside & \encMountains \\
  \encLine \bigBeastList & \encLakeside & \encMountains \\
  \encLine \bigBeastList & \encLakeside & \encMountains \\
  \encLine \bigBeastList & \encLakeside & \encMountains \\
  \encLine \bigBeastList & \encLakeside & \encMountains \\
  \encLine \bigBeastList & \encLakeside & \encMountains \\
  \encLine \bigBeastList & \encLakeside & \encMountains \\
  \encLine \bigBeastList & \encLakeside & \encMountains \\
  \encLine \bigBeastList & \encLakeside & \encMountains \\
  \encLine \bigBeastList & \encLakeside & \encMountains \\
  \encLine \bigBeastList & \encLakeside & \encMountains \\
  \encLine \bigBeastList & \encLakeside & \encMountains \\
  \hline
\end{wideTable}


\section{Morale}
\label{morale}
\index{Morale}
\begin{multicols}{2}

\noindent
When combat seems likely, roll $2D6$ and leave the result on the table.
If the creature has more \glspl{hp} than the roll, it attacks; otherwise, it flees.

\sidebox{
  \begin{boxtable}
    \textbf{Roll} & \textbf{\glsfmtlongpl{hp}} \\
    \hline
    \twoDice{5} & 10 (\textit{ATTACK!}) \\
    \twoDice{5} &  8 (\textit{ATTACK!}) \\
    \twoDice{5} &  5 (\textit{wait\ldots}) \\
    \twoDice{8} &  4 (\textit{RUN!}) \\
    \twoDice{8} &  2 (\textit{RUN!}) \\
  \end{boxtable}
}

Chitincrawlers often have 10~\glspl{hp}, so they often attack people the moment they spot them.
Griffins have fewer \glspl{hp} than chitincrawlers, so they attack less often.
And goblins might have only 5 or 6~\glspl{hp} -- they know how small they are, and prefer to spend time laying traps, or stealing equipment, rather than engaging directly.
Every \gls{hp} a creature loses could prompt it to turn and flee, so sometimes only wounded creatures will flee, while others keep fighting.

On a tie, the creature pauses to reassess the situation.
A single chitincrawler eyeing up a party might display its claws, then observe their reactions; if the troupe do nothing to change the situation (such as casting a spell, or making a loud noise) then the creature rolls again.

\subsubsection{Decisions before Dice}

The dice only exist to provide a neutral result, not a sensible result.
Creatures always fight when backed into a corner, or starving.
Creatures never fight when they will probably die.%
\footnote{Exceptions include particularly dim creatures, who don't know what's good for them.
See chitincrawlers, \vpageref{chitin:tactics}.}

Use your best judgement about whether or not something, or someone, attacks the troupe, before rolling.
A trio of griffins will not attack a loud group of one hundred humanoids, all moving together, but will definitely swoop for a small child who has wandered away from the group.

\Glspl{npc} work exactly the same as \gls{fenestra}'s predators, but display their reactions differently.
\Glspl{npc} who fail a Morale check may act friendly and harmless, if they think the \glspl{pc} will not attack them.

\Glspl{pc} do not take Morale checks -- the players decide when it's time to run away by the look of the situation.
Usually a good time is when all the \glspl{fp} have run out.
\index{Henchmen}%
However, \gls{npc} travelling with the \glspl{pc} still take morale checks as usual.%
\footnote{The \glsentrytext{gm} may also wish to cut all Morale checks for any \glspl{npc} with remaining \glsentrytext{fp}.}%

\end{multicols}

\section{Happenings \& Situations}

\begin{multicols}{2}

\subsection{On the Road}

\subsubsection{\Glsfmttext{bothy} Events}

\Glspl{bothy} \glsdesc{bothy}.

Each \gls{bothy} begins when someone dies on the road, and others leave rocks there to pay their respect.
If enough people gather rocks, it becomes a cenotaph, and takes the name of the fallen.
If more rocks follow, the cenotaph becomes \pgls{bothy}.

Anyone using \glspl{bothy} must gather and leave as much firewood as they found when they entered, except the \gls{guard}, who must leave even more, and generally maintain these small houses.
Nobody will trust someone known to leave \pgls{bothy} in a bad state.

Roll to check \pgls{bothy}'s circumstances upon arrival:

\bothyEvents

\subsubsection{\Glsfmttext{broch} Events}

\Glspl{broch} \glsdesc{broch}.

This slaughterhouse of predators keeps nearby \glspl{village} safe, although it also teaches predators not to approach large, noisy towers.

Roll to check \pgls{broch}'s circumstances upon arrival:

\brochEvents

\subsection{Civilization}

\subsubsection{\Glsfmttext{village} Features}
\label{villageFeatures}

Each \gls{village} has its own oddities, and means of protection.
Once the troupe reach \pgls{village}, roll $3D6$ and add the first two dice to find a feature, then add the second two to find another (ignore all duplicates).
Finally, check what's happening at the \gls{village} today with the chart \vpageref{villageEvents}.

Note down the \gls{village}'s feature, and give it a name based on the feature, or its location.
\Pgls{village} with \nameref{screeching_moss} nestled in hills could receive the name `Mossdale', or `Screechvale'.

\encVillageFeatures

\subsubsection{\Glsfmttext{village} Events}
\label{villageEvents}

When the troupe stay in \pgls{village}, keep rolling for encounters in \glspl{village}, as usual.
Sometimes \pgls{village} receives traders, at other times a beast arrives.
As usual, beasts may attack, or simply watch, and wait for someone to leave.

If the troupe have already visited this \gls{village}, you won't need to roll to find its features, but you can roll $2D6$ again to find something new that's happening.

\encVillageEvent

\labelledDiceTrio{2}{3}{1}{Copper Spikes}{Moat}{Roast Griffin}

For example, once the troupe reach \pgls{village}, the \gls{gm} might roll $3D6$ and bind \pgls{village} with copper spikes around the wall, and a moat.

\subsubsection{Town Features}
\index{Towns!Rulers}

Roll $1D6$ to find the town's \gls{warden}.

\begin{dlist}
  \item
  Multiple \underline{warring factions} -- roll twice more, with +1 to the second roll.
  \item
  A \underline{Warlord}, recently out of the \gls{guard}, amassing an army to expand their power.
  Shipments of weapons come from the road to other lands
  (see \vref{roadOut}).
  \item
  A \underline{greedy \gls{warden}}, who demands 10\% tithes from all who enter.
  Any neighbouring fiends annoy and fascinate them equally.
  \item
  A new \underline{Sorcerer king} with no experience or business ruling.
  He once killed a fiend, how hard could ruling \pgls{court} and arranged marriages be?

  The sorcerer has killed the same type of fiend as the highest fiend on the map (if none, it was a dragon).
  \item
  A \underline{Spoilt noble} who has never seen a farmer in their life.
  \item
  A \underline{servant of the next fiend} on the map (counted by point number).
  All who oppose this \gls{warden} disappear on the road.
  \item
  A \underline{Crime-Lord}, with a history of theft, now risen to fortune\ldots but not nobility.
\end{dlist}


\subsubsection{Town Events}

The \gls{guard} should not enter towns.
No taverns, nor markets.
But they always find an excuse, or unguarded entrance.

\end{multicols}

\encTownEvents

\section{Missions}
\label{NGmissions}
\index{Adventure Generator}
\index{Missions}

\begin{multicols}{2}

\noindent
\Gls{guard} recruits can expect harsh duties.
The character with the highest rank will be asked to lead the party (ties are broken by \roll{Charisma}{Deceit}), and give a bonus to the first die-roll.

\begin{itemize}
  \item
  Roll $1D6$ for Fodder or Diggers
  \index{Diggers (rank)}%
  \item
  Roll $1D6+1$ for Archers
  \index{Archers (rank)}%
  \item
  Roll $1D6+2$ for Cutters and Riders
  \index{Cutters (rank)}%
  \index{Night Riders (rank)}%
  \item
  Roll $1D6+3$ for Rangers
  \index{Rangers (rank)}%
\end{itemize}

Then add a complication with the same bonus as before (\autopageref{missionComplications}).

\subsubsection{Goal}
\index{Missions}

\ngMissions

\subsubsection{Complications}

Add a bonus for rank, as before.

\missionComplications

\end{multicols}

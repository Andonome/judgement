\chapter[Encounters at the Crossroads]{Encounters}
\index{Random Encounters}
\index{Encounters}
\label{encounters}

\section{Things on the Road}

\begin{multicols}{2}

All encounters combine multiple elements together.

\begin{enumerate}
  \item
  Start with the current Season

  \begin{exampletext}
    Let's assume a warm Season.
  \end{exampletext}
  \item
  Select the location: villages, edge, or forest

  \begin{exampletext}
    And assume the forest for now.
  \end{exampletext}
  \item
  Roll $1D3$ to find the row, and add those three encounters

  \begin{exampletext}
    So `2' would mean `rain + peace + monster'.
  \end{exampletext}

  \item
  Roll $1D3$ and add the remaining two encounters on that column.

  \begin{exampletext}
    Rolling `1' means adding `hot spell + storm'.
  \end{exampletext}

  \item
  Combine all five encounters into a narrative.

  \begin{exampletext}
    Both `Peace' and `Monster' have a chart.
    Rolling `Aurochs' and `Chitincrawler' means we have all the elements we need for a full encounter.
  \end{exampletext}

\end{enumerate}

\begin{exampletext}
  \begin{speechtext}
    A storm gathers along the hot road.
    Take 4 \glspl{fatigue}, or more if wearing armour.

    Is anyone wearing armour?
  \end{speechtext}

  \ldots here the \glspl{pc} can decide how badly they want that protective covering.

  \begin{speechtext}
    Around the next corner, a horrid mess of webbing blocks the path.
    Behind you, a great arachnid leaps out, galloping like a horse!''

    Thunder roars from the sky\ldots and the forest as the storm breaks, and rain begins to fall.
  \end{speechtext}

  The `thunder from the forest', comes from a herd of aurochs.
  Freaked out by the storm, a bovine stampede heads towards the troupe.
  On the second round of combat, they arrive, interrupting whatever has happened.

\end{exampletext}

\end{multicols}

\bigLine

\begin{multicols}{2}

\subsection{Hot Seasons}

\noindent
Heat transforms the land around the villages.
Without the great forest canopies, the Sun scorches the earth, providing a protective ring of open land around villages.
Once people can see creatures coming from a long way off, archers have a much easier time.

The hot seasons melt mountaintops, which floods plains and forests below.
Swelling rivers destroy bridges and any houses built too close to the banks.
Stone-walled cities often create miniature lakes or run-off areas for the river to expand into.

\setcounter{enc}{0}
\setcounter{track}{0}

\begin{nametable}[l|YYY]{Villages}
  & \dicef[\Large]{1} & \dicef[\Large]{2} & \dicef[\Large]{3} \\
  \hline
  \setcounter{enc}{0}
  \setcounter{track}{0}
  \dicef[\Large]{1} \Repeat{3}{& \encBigList \stepcounter{track} } \\
  \stepcounter{enc}
  \setcounter{track}{0}
  \dicef[\Large]{2} \Repeat{3}{& \encBigList \stepcounter{track} } \\
  \stepcounter{enc}
  \setcounter{track}{0}
  \dicef[\Large]{3} \Repeat{3}{& \encBigList \stepcounter{track} } \\
\end{nametable}

\begin{nametable}[l|YYY]{Edge}
  & \dicef[\Large]{1} & \dicef[\Large]{2} & \dicef[\Large]{3} \\
  \hline
  \setcounter{enc}{0}
  \setcounter{track}{1}
  \dicef[\Large]{1} \Repeat{3}{& \encBigList \stepcounter{track} } \\
  \stepcounter{enc}
  \setcounter{track}{1}
  \dicef[\Large]{2} \Repeat{3}{& \encBigList \stepcounter{track} } \\
  \stepcounter{enc}
  \setcounter{track}{1}
  \dicef[\Large]{3} \Repeat{3}{& \encBigList \stepcounter{track} } \\
\end{nametable}

\begin{nametable}[l|YYY]{Forest}
  & \dicef[\Large]{1} & \dicef[\Large]{2} & \dicef[\Large]{3} \\
  \hline
  \setcounter{enc}{0}
  \setcounter{track}{2}
  \dicef[\Large]{1} \Repeat{3}{& \encBigList \stepcounter{track} } \\
  \stepcounter{enc}
  \setcounter{track}{2}
  \dicef[\Large]{2} \Repeat{3}{& \encBigList \stepcounter{track} } \\
  \stepcounter{enc}
  \setcounter{track}{2}
  \dicef[\Large]{3} \Repeat{3}{& \encBigList \stepcounter{track} } \\
\end{nametable}

\end{multicols}

\bigLine

\begin{multicols}{2}

\subsection{Mild Seasons}

Mild Seasons outnumber the others, but also bring heavy storms.
During peaceful months, people wander and trade, farm and plan.

When storms come, strong winds and earthquakes can tear apart bridges and houses.
Stone castles, if properly made, can withstand earthquakes, and anyone living far enough underground often won't notice as the heavy earth around them protects.
But anyone on the land should leave their house quickly.

\begin{nametable}[l|YYY]{Villages}
  & \dicef[\Large]{1} & \dicef[\Large]{2} & \dicef[\Large]{3} \\
  \hline
  \setcounter{enc}{1}
  \setcounter{track}{0}
  \dicef[\Large]{1} \Repeat{3}{& \encBigList \stepcounter{track} } \\
  \stepcounter{enc}
  \setcounter{track}{0}
  \dicef[\Large]{2} \Repeat{3}{& \encBigList \stepcounter{track} } \\
  \stepcounter{enc}
  \setcounter{track}{0}
  \dicef[\Large]{3} \Repeat{3}{& \encBigList \stepcounter{track} } \\
\end{nametable}

\begin{nametable}[l|YYY]{Edge}
  & \dicef[\Large]{1} & \dicef[\Large]{2} & \dicef[\Large]{3} \\
  \hline
  \setcounter{enc}{1}
  \setcounter{track}{1}
  \dicef[\Large]{1} \Repeat{3}{& \encBigList \stepcounter{track} } \\
  \stepcounter{enc}
  \setcounter{track}{1}
  \dicef[\Large]{2} \Repeat{3}{& \encBigList \stepcounter{track} } \\
  \stepcounter{enc}
  \setcounter{track}{1}
  \dicef[\Large]{3} \Repeat{3}{& \encBigList \stepcounter{track} } \\
\end{nametable}

\begin{nametable}[l|YYY]{Forest}
  & \dicef[\Large]{1} & \dicef[\Large]{2} & \dicef[\Large]{3} \\
  \hline
  \setcounter{enc}{1}
  \setcounter{track}{2}
  \dicef[\Large]{1} \Repeat{3}{& \encBigList \stepcounter{track} } \\
  \stepcounter{enc}
  \setcounter{track}{2}
  \dicef[\Large]{2} \Repeat{3}{& \encBigList \stepcounter{track} } \\
  \stepcounter{enc}
  \setcounter{track}{2}
  \dicef[\Large]{3} \Repeat{3}{& \encBigList \stepcounter{track} } \\
\end{nametable}
\end{multicols}

\bigLine
\begin{multicols}{2}

\subsection{Cold Seasons}

Snow covers the land so deeply that people only know where the road lies by the lack of trees.
Protective clothing becomes a necessity.

Despite this, people can travel the lonely roads safer than before, as two of Fenestra's most dangerous creatures -- the chitincrawler and basilisk -- must hibernate when cold.
Wolves and bears 

\begin{nametable}[l|YYY]{Villages}
  & \dicef[\Large]{1} & \dicef[\Large]{2} & \dicef[\Large]{3} \\
  \hline
  \setcounter{enc}{2}
  \setcounter{track}{0}
  \dicef[\Large]{1} \Repeat{3}{& \encBigList \stepcounter{track} } \\
  \stepcounter{enc}
  \setcounter{track}{0}
  \dicef[\Large]{2} \Repeat{3}{& \encBigList \stepcounter{track} } \\
  \stepcounter{enc}
  \setcounter{track}{0}
  \dicef[\Large]{3} \Repeat{3}{& \encBigList \stepcounter{track} } \\
\end{nametable}

\begin{nametable}[l|YYY]{Edge}
  & \dicef[\Large]{1} & \dicef[\Large]{2} & \dicef[\Large]{3} \\
  \hline
  \setcounter{enc}{2}
  \setcounter{track}{1}
  \dicef[\Large]{1} \Repeat{3}{& \encBigList \stepcounter{track} } \\
  \stepcounter{enc}
  \setcounter{track}{1}
  \dicef[\Large]{2} \Repeat{3}{& \encBigList \stepcounter{track} } \\
  \stepcounter{enc}
  \setcounter{track}{1}
  \dicef[\Large]{3} \Repeat{3}{& \encBigList \stepcounter{track} } \\
\end{nametable}

\begin{nametable}[l|YYY]{Forest}
  & \dicef[\Large]{1} & \dicef[\Large]{2} & \dicef[\Large]{3} \\
  \hline
  \setcounter{enc}{2}
  \setcounter{track}{2}
  \dicef[\Large]{1} \Repeat{3}{& \encBigList \stepcounter{track} } \\
  \stepcounter{enc}
  \setcounter{track}{2}
  \dicef[\Large]{2} \Repeat{3}{& \encBigList \stepcounter{track} } \\
  \stepcounter{enc}
  \setcounter{track}{2}
  \dicef[\Large]{3} \Repeat{3}{& \encBigList \stepcounter{track} } \\
\end{nametable}

\end{multicols}

\bigLine

\begin{multicols}{2}

\subsection{Peace}

Every time you roll peace, add another 1D6 days of peace before the encounter hits.
The party may arrive at their destination unharmed, with all encounters discarded.

\newcommand\encRef[1]{%
  \nameref{#1} (\autopageref{#1})%
}

\noindent
\begin{boxtable}[cccL]
  \textbf{Cold} & \textbf{Mild} & \textbf{Hot}  & \textbf{Encounter} \\
  \hline
  \dicef{1} &           &           & Hibernating \encRef{chitincrawler}  \\
  \dicef{2} & \dicef{1} &           & Deer               \\
  \dicef{3} & \dicef{2} & \dicef{1} & \encRef{auroch}    \\
  \dicef{4} & \dicef{3} & \dicef{2} & \encRef{boar}      \\
  \dicef{5} & \dicef{4} & \dicef{3} & \encRef{mage_oak}  \\
  \dicef{6} & \dicef{5} & \dicef{4} & \encRef{uproot}    \\
            & \dicef{6} & \dicef{5} & \encRef{bedleaves} \\
            &           & \dicef{6} & \encRef{seekers}   \\
\end{boxtable}

\subsection{Bandits \& Brigands}

Once farmers cannot sustain themselves any more, many turn to banditry.

When soldiers cannot or will not sustain themselves on their duties, they become brigands.
Brigands plan better, have excellent training, and the best weapons money can buy.

The number appearing is equal to $1D6\times 2 + 4$.
Rolling an odd number means brigands, rather than bandits.

\newcommand\showBandits{
  \item
  \setcounter{gold}{\value{list}}
  \multiply\value{gold} by 2
  \addtocounter{gold}{4}
  \arabic{gold} \ifodd\value{list}
    Brigands
  \else
    Bandits
  \fi
}

\begin{dlist}
  \showBandits
  \showBandits
  \showBandits
  \showBandits
  \showBandits
  \showBandits
\end{dlist}

\subsection{Traders \& Nomads}

Local traders wander back and forth, pedalling what they have to places that do not have that thing.
They form the blood of civilization.
Unfortunately, they have a very short life expectancy, given all the beasts and bandits on the road.
As a result, most wander the road armed, wearing armour outside of the warmer seasons, and often with a crossbow at their side.

\subsection{Monsters}

Add -2 during cold Seasons, and +2 during hot Seasons.

\begin{boxtable}[cccL]
  \textbf{Cold} & \textbf{Mild} & \textbf{Hot}  & \textbf{Encounter} \\
  \hline
  \dicef{1} &           &           & \encRef{ghast}                                                     \\
  \dicef{2} &           &           & \encRef{wolf}                                                      \\
  \dicef{3} & \dicef{1} &           & \encRef{dryad}                                                     \\
  \dicef{4} & \dicef{2} & \dicef{1} & \encRef{griffin}                                                   \\
  \dicef{5} & \dicef{3} & \dicef{2} & \encRef{bear}                                                      \\
  \dicef{6} & \dicef{4} & \dicef{3} & \encRef{woodspy}                                                   \\
  \dicef{5} & \dicef{5} & \dicef{4} & \encRef{mouthdigger}                                               \\
            & \dicef{6} & \dicef{5} & \encRef{chitincrawler}                                             \\
            & \dicef{6} & \dicef{6} & \encRef{basilisk}                                                  \\
            &           & \dicef{6} & \encRef{umber_hulk}                                                \\

\end{boxtable}

\subsection{Complications}

\subsubsection{Floods}

Heat often brings flooding, especially around rivers.
The \gls{tn} to overcome problems with flooding equals $7 + 1D6$.
Floods can destroy bridges, roads, and even houses, although most people who live in areas prone to flooding will build their houses somewhere high, or put them on stilts.

\subsubsection{Hot Spell}

The forests present many dangers, but they at least provide protection against one of the most insidious.
Heat begins by stripping people of their senses, and soon burns their skin.

Mechanically, heatwaves inflict 1D6 Fatige on anyone in the area, although the right clothing and preparation (Intelligence + Wyldcrafting) will half the amount taken.

Anyone foolish enough to wear heavy armour in a heatwave will always take the full brunt of the Fatigue Points, plus one per Weight Rating of the armour.


\end{multicols}

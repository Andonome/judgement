\chapter{Encounters at the Crossroads}
\index{Random Encounters}
\index{Encounters}
\index{Seasons}
\label{encounters}

You don't have to roll random encounters to be a \gls{gm}, but it helps.

When the troupe move to the next town, and you don't want to `time travel', and just say `okay, next week, you arrive', but don't have anything planned, then you can throw in a random encounter or two.
Many of these encounters won't involve any combat -- even those with an actual monster can result in seeing something strange happening, before the troupe decide to move on.
Many encounters include weather-based events, which give a sense of time passing, and drain a couple of \glspl{fatigue} from the troupe before they move on.

Below that, the random mission generator won't replace a full adventure, but it provide you with a starting-point to get a few ideas.

\section{Things on the Road}

\begin{multicols}{2}

\noindent
When it rains, it pours; and when creatures attack, the \glspl{pc} will often find them among interesting plants, or else bandits might attack by weakening a bridge while the \glspl{pc} cross it.

\paragraph{Villages}
encounters show what happens in the safe area around towns, where sufficient traps, \glspl{guard} and surrounding walled settlements stop encroaching creatures.

\paragraph{Edge}
encounters are the most common -- these are the walled villages, the roads, and the lost taverns on them.

\paragraph{Forests}
have no bandits, because they have nobody to rob.
In warmer seasons they have plenty of food for the picking, and plenty of monsters too.

All encounters combine multiple elements together.

\begin{enumerate}
  \item
  Start with the current Season.%
  \footnote{For more on seasons, see \autopageref{seasons}.}

  \begin{exampletext}
    Let's assume a hot Season.
  \end{exampletext}
  \item
  Select the location: villages, edge, or forest.

  \begin{exampletext}
    And assume the forest for now.
  \end{exampletext}
  \item
  Roll $1D3$ to find the row, and add those three encounters.

  \begin{exampletext}
    So `2' would mean
    \setcounter{enc}{1}%
    \setcounter{track}{2}%
    `\encBigList'\stepcounter{track},
    `\encBigList'\stepcounter{track}, and
    `\encBigList'.
  \end{exampletext}

  \item
  Roll $1D3$ and add the remaining two encounters on that column.

  \begin{exampletext}
    Rolling `2' means adding
    \setcounter{enc}{0}%
    \setcounter{track}{3}%
    that `\encBigList'\stepcounter{enc}\stepcounter{enc}, and
    `\encBigList'
    (we already have the result between them).
  \end{exampletext}

  \item
  Roll to see what monsters and sightings the \glspl{pc} come across, and combine all five encounters into a single encounter.

\end{enumerate}

We combine \underline{every} result into a single encounter.

\begin{nametable}[l|Y||Y||Y]{Forest}
  & \dicef[\Large]{1} & \dicef[\Large]{2}~$\Downarrow$ & \dicef[\Large]{3} \\
  \hline
  \setcounter{enc}{0}
  \setcounter{track}{2}
  \dicef[\Large]{1} & \encBigList \stepcounter{track}  & \textbf{\encBigList} \stepcounter{track} & \encBigList \stepcounter{track} \\
  \hline
  \hline
  \stepcounter{enc}
  \setcounter{track}{2}
  \dicef[\Large]{2}$\Rightarrow$ \Repeat{3}{& \textbf{\encBigList} \stepcounter{track} } \\
  \hline
  \hline
  \stepcounter{enc}
  \setcounter{track}{2}
  \dicef[\Large]{3} & \encBigList \stepcounter{track}  & \textbf{\encBigList} \stepcounter{track} & \encBigList \stepcounter{track} \\
\end{nametable}

The combinations have no formula.
You simply have to draw everything together, in the first way that comes to mind.

\begin{speechtext}
  The rain drizzles from every leaf in the forest, making a constant racket.
  Pulling back your hood to look around, you see a plant with gigantic leaves -- the perfect spot to shield from the rain.
  Unfortunately a boar seems to have taken the place already.
  She rests, unaware that you stand only ten steps away.

  What do you do?
\end{speechtext}

  \ldots\ldots

\begin{speechtext}
  You have your bow out, but before loosing it, long, dark-brown legs wrap around the tree above the boar.
  Each leg seems longer than horse, but grips the tree silently.
\end{speechtext}

Every combination holds a lot of options.

\begin{description}
  \item[Chitincrawler + Traders]
  might imply an empty caravan, with a chitincrawler eating one horse, while the others remain strapped to their carriage, and panicking.

  The traders have fled up the road without their wares.
  \item[Griffins + Chitincrawler]
  \ifcase\value{season}\relax\or
  could imply being stalked by some griffins, and soon after the \glspl{pc} spot an obvious chitincrawler web-trap above them.
  With some cunning, they could set one danger against the other\ldots
  \or
  can involve one monster attacking the other.
  In this case, a griffin, entangled in a chitincrawler web.

  The party need do nothing -- encounters don't require combat.
  Soon after, they notice a lone griffin, guarding a nest.
  \or
  might reference the fact that both have eggs.
  A chitincrawler with an engorged thorax, full-to-bursting with eggs, makes for tempting prey.
  The party observe two griffins stalking it in the distance, just as it stalks the party.

  If the party simply manage to put enough distance between them and the engorged arachnid, the griffins will finish her off.
  \else%
  means each stalk the party.
  They notice the griffins circling above, beyond arrow-range, waiting for the best time to attack.
  Once the chitincrawler encounters has ended, the griffins reassess how wounded the party appear\ldots
  \fi
  \item[Bandits + Snow]
  suggests a clever trap, where bandits hide in the snow with bows, waiting to fire on anyone who looks like they have something nice.
\end{description}

\end{multicols}

\begin{multicols}{2}

\subsection{Hot Seasons}

\noindent
Heat transforms the land around the villages.
Without the great forest canopies, the Sun scorches the earth, providing a protective ring of open land around villages.
Once people can see creatures coming from a long way off, archers have a much easier time.

The hot seasons melt mountaintops, which floods plains and forests below.
Swelling rivers destroy bridges and any houses built too close to the banks.
Stone-walled cities often create miniature lakes or run-off areas for the river to expand into.

\setcounter{enc}{0}
\setcounter{track}{0}

\encWarmVillages

\encWarmEdge

\encWarmForest

\begin{multicols}{2}
\subsubsection*{Monsters}

\setcounter{enc}{4}
\begin{dlist}
  \Repeat{6}{
    \item
    \encMons
  }
\end{dlist}

\subsubsection*{Sightings}

\setcounter{track}{3}
\begin{dlist}
  \Repeat{6}{
    \item
    \encSight
  }
\end{dlist}

\end{multicols}

\subsubsection{Floods}

Heat often brings flooding, especially around rivers.
Anyone doing anything (or doing nothing) by a river must make some kind of check at \tn[8] to ensure the flood does not sweep away their plans.
Floods can destroy bridges, roads, and even houses, although most people who live in areas prone to flooding will build their houses somewhere high, or put them on stilts.

\subsubsection{Hot Spell}

The forests present many dangers, but they at least provide protection against one of the most insidious.
Heat begins by stripping people of their senses, and soon burns their skin.

Mechanically, heatwaves inflict 4 \glspl{fatigue} on anyone in the area, although the right clothing and preparation (\roll{Intelligence}{Wyldcrafting}) will half the amount taken.

Anyone foolish enough to wear heavy armour in a heatwave will always take the full brunt of the \glspl{fatigue}, plus one per \gls{weight} of the armour.

\end{multicols}

\bigLine

\begin{multicols}{2}

\subsection{Mild Seasons}

Mild Seasons outnumber the others, but also bring heavy storms.
During peaceful months, people wander and trade, farm and plan.

When storms come, strong winds and earthquakes can tear apart bridges and houses.
Stone castles, if properly made, can withstand earthquakes, and anyone living far enough underground often won't notice as the heavy earth around them protects.
But anyone on the land should leave their house quickly.

\encMildVillages

\encMildEdge

\encMildForest

\begin{multicols}{2}
\subsubsection*{Monsters}
\label{monsterEncounters}

\setcounter{enc}{3}
\begin{dlist}
  \Repeat{6}{
    \item
    \encMons
  }
\end{dlist}

\subsubsection*{Sightings}

\setcounter{track}{2}
\begin{dlist}
  \Repeat{6}{
    \item
    \encSight
  }
\end{dlist}

\end{multicols}

\subsubsection{Storms}
Those who live in walled villages always fear a storm, because the archers cannot land their arrows on encroaching beasts until they approach close.
Traders don't fare much better.

\end{multicols}

\bigLine
\begin{multicols}{2}

\subsection{Cold Seasons}

Snow covers the land so deeply that people only know where the road lies by the lack of trees.
Protective clothing becomes a necessity.

Despite this, people can travel the lonely roads safer than before, as two of Fenestra's most dangerous creatures -- the chitincrawler and basilisk -- must hibernate when cold.

\encColdVillages

\encColdEdge

\encColdForest

\subsubsection*{Monsters}

\begin{multicols}{2}
\setcounter{enc}{1}
\begin{dlist}
  \Repeat{6}{
    \item
    \encMons
  }
\end{dlist}

\subsubsection*{Sightings}

\setcounter{track}{1}
\begin{dlist}
  \Repeat{6}{
    \item
    \encSight
  }
\end{dlist}

\end{multicols}

\end{multicols}

\bigLine

\begin{multicols}{2}

\subsection{Bandits \& Brigands}

Once farmers cannot sustain themselves any more, many turn to banditry.

When soldiers cannot or will not sustain themselves on their duties, they become brigands.
Brigands plan better, have excellent training, and the best weapons money can buy.

The number appearing is equal to $1D6\times 2 + 4$.
Rolling an odd number means brigands, rather than bandits.


\encBandits

\subsection{Traders \& Nomads}

Local traders wander back and forth, pedalling what they have to places that do not have that thing.
They form the blood of civilization.
Unfortunately, they have a very short life expectancy, given all the beasts and bandits on the road.
As a result, most wander the road armed, wearing armour outside of the warmer seasons, and often with a crossbow at their side.

Traders often have important knowledge of the way ahead, since they will almost invariably come towards the troupe, rather than from behind them.%
\footnote{People only come behind you on the road if you walk very slowly, or if they're going very fast.}
If a path ahead if blocked by a tree, or a bridge looks like it's about to give out, they'll know it.
They may also warn of bandits ahead, especially if they've just been robbed!

Traders carry all manner of goods.
Roll $3D6$ and add all results.

\encTraders

\end{multicols}

\section{Happenings \& Situations}

\begin{multicols}{2}

\subsubsection{Village Events}
\index{Villages!Events}

Roll $2D6$ for a random village event.

\encVillageEvent

\end{multicols}

\section{Missions}
\index{Adventure Generator}
\index{Missions Generator}

\begin{multicols}{2}

\noindent
\Glspl{guard} recruits can expect harsh duties.
The character with the highest rank will be asked to lead the party (ties are broken by Charisma + Deceit), and give a bonus to the first die-roll.

\begin{itemize}
  \item
  Roll $1D6$ for Fodder
  \item
  Roll $1D6+1$ for Archers
  \item
  Roll $1D6+2$ for Cutters
  \item
  Roll $1D6+3$ for Rangers
\end{itemize}

Then add a complication with the same bonus as before (\autopageref{missionComplications}).

\subsubsection{Missions}
\index{Missions}

\ngMissions

\subsubsection{Complications}
\label{missionComplications}

Add a bonus for rank, as before.

\missionComplications

\end{multicols}

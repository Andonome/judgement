\chapter{Encounters at the Crossroads}
\index{Random Encounters}
\index{Encounters}
\index{Seasons}
\label{encounters}

You don't have to roll random encounters to be a \gls{gm}, but it helps.

When the troupe move to the next town, and you don't want to `time travel', and just say `okay, next week, you arrive', but don't have anything planned, then you can throw in a random encounter or two.
Many of these encounters won't involve any combat -- even those with an actual monster can result in seeing something strange happening, before the troupe decide to move on.
Many encounters include weather-based events, which give a sense of time passing, and drain a couple of \glspl{fatigue} from the troupe before they move on.

Below that, the random mission generator won't replace a full adventure, but it provide you with a starting-point to get a few ideas.

\section{Things on the Road}

\begin{multicols}{2}

\begin{encChart}{Forest}
  \encLine Umber Hulk \\
  \encLine Swarming stirges \\
  \encLine Basilisk \\
  \encLine Chitincrawler \\
  \encLine Mouthdigger \\
  \encLine Woodspy \\
  \encLine $1D3$ Griffins \\
  \encLine $1D6+6$ Wolves \\
  \encLine Bear \\
  \encLine Warthog \\
  \encLine $1D6\times 20$ Aurochs \\
  \encLine $1D6\times 3$ Deer \\
  \encLine $1D6\times 2$ Elves on a quest \\
  \encLine Dryad \\
  \encLine $1D6$ Wandering Ghouls \\
  \encLine Roaming Ghast \\
\end{encChart}

\begin{itemize}
  \item
  Each day, roll $3D6$ to check for encounters.
  \begin{itemize}
    \item
    Die 1 and 2 tell you the encounter.
    \item
    Die 3 may tell you the number appearing.
    \item
    Die 2 and 3 tell you the weather.
  \end{itemize}
  \item
  Each night, roll again, with -1 to the encounter rolls.
  \item
  Once you roll an encounter, mark it off -- it does not repeat this session.
\end{itemize}

\begin{encChart}{Weather}
  \encLine Floods \\
  \encLine Heatwave \\
  \encLine Hurricane \\
  \encLine Light showers \\
  \encLine Clear skies \\
  \encLine Thunder \\
  \encLine Lightning \\
  \encLine Brewing storm \\
  \encLine Overcast \\
  \encLine Mist \\
  \encLine Dead calm \\
  \encLine Mild breeze \\
  \encLine Biting winds \\
  \encLine Snow \\
  \encLine Hail \\
  \encLine Extreme cold snap \\
\end{encChart}

\begin{itemize}
  \item
  {\Large\epsdice{6} + \epsdice{3} + \epsdice[black]{4}}
  \item
  $5 + 3 = 8$, so $1D3$ griffins approach.
  \item
  The third die result is cut in half to produce a D3.
  The result is `2', so 2 griffins approach.
  \item
  $3 + 4 = 7$, so a storm is brewing.
\end{itemize}

\clearpage

\begin{encChart}{Civilized Villages}
  \encLine Umber Hulk \\
  \encLine Swarming stirges \\
  \encLine Basilisk \\
  \encLine Chitincrawler \\
  \encLine Mouthdigger \\
  \encLine Woodspy \\
  \encLine $1D3$ Griffins \\
  \hline
  \encLine \textit{Bandits!} \\
  \hline
  \encLine \arabic{diceNo} Traders caravans. \\
  \encLine \arabic{diceNo} Traders caravans. \\
  \encLine \arabic{diceNo} Traders caravans. \\
  \encLine \arabic{diceNo} Traders caravans. \\
  \encLine \arabic{diceNo} Traders caravans. \\
  \encLine \arabic{diceNo} Traders caravans. \\
  \encLine \arabic{diceNo} Traders caravans. \\
  \encLine \arabic{diceNo} Traders caravans. \\
\end{encChart}

\begin{itemize}
  \item
  In more civilized areas, fewer creatures of the forest dare to set foot.
  \begin{itemize}
    \item
    On the roll of a 9, the \glspl{pc} encounter brigands.
    \item
    Rolling below 9 means a harmless encounter with a trader, going the other way from the players, and happy to share news.
    \item
    Rolling above 9 means a normal encounter with a forest creature.
  \end{itemize}
\end{itemize}

\columnbreak

\begin{encChart}{Outer Edges}
  \encLine Umber Hulk \\
  \encLine Swarming stirges \\
  \encLine Basilisk \\
  \encLine Chitincrawler \\
  \encLine Mouthdigger \\
  \encLine Woodspy \\
  \encLine $1D3$ Griffins \\
  \encLine $1D6+6$ Wolves \\
  \encLine Bear \\
  \encLine Warthog \\
  \encLine $1D6\times 20$ Aurochs \\
  \hline
  \encLine \textit{Bandits!} \\
  \hline
  \encLine \arabic{diceNo} Traders caravans. \\
  \encLine \arabic{diceNo} Traders caravans. \\
  \encLine \arabic{diceNo} Traders caravans. \\
  \encLine \arabic{diceNo} Traders caravans. \\
\end{encChart}

\begin{itemize}
  \item
  This number is the `Bandit Line', and it decreases as characters move away from civilization.
  \begin{itemize}
    \item
    On the lonely roads to the outer walled towns, the bandit line may drop to 6, then 3, until not trace of humanity remains.
    \item
    The lower that Line of Civilization goes, the fewer wild creatures one sees during the day.
  \end{itemize}
  \item
  By night, all the beasts return, as humans cower in the dark.
\end{itemize}

\clearpage

\subsection{Bandits \& Brigands}

Once farmers cannot sustain themselves any more, many turn to banditry.

When soldiers cannot or will not sustain themselves on their duties, they become brigands.
Brigands plan better, have excellent training, and the best weapons money can buy.

Roll $1D6$ to check the number and type appearing.%
\footnote{The number appearing is equal to $1D6\times 2 + 4$.
Rolling an odd number means brigands, rather than bandits.}

\encBandits

\subsection{Traders \& Nomads}

Local traders wander back and forth, pedalling what they have to places that do not have that thing.
They form the blood of civilization.
Unfortunately, they have a very short life expectancy, given all the beasts and bandits on the road.
As a result, most wander the road armed, wearing armour outside of the warmer seasons, and often with a crossbow at their side.

Traders often have important knowledge of the way ahead, since they will almost invariably come towards the troupe, rather than from behind them.%
\footnote{People only come behind you on the road if you walk very slowly, or if they're going very fast.}
If a path ahead if blocked by a tree, or a bridge looks like it's about to give out, they'll know it.
They may also warn of bandits ahead, especially if they've just been robbed!

Roll $4D6$ when a caravan approaches (or fewer dice, on a smaller road) -- each one represents a single wagon, carrying goods.
Any duplicate results are ignored, so if you roll three 1's, then that wagon travels alone.

\encTraders

\end{multicols}

\section{Morale}
\label{morale}
\index{Morale}
\begin{multicols}{2}

\noindent
An antagonist's morale decides when a fight starts, and when it ends.

Predators don't like fair fights -- they want to pounce, subdue, and \emph{feed}.
Most predators will back off once hurt, and much of the time, they won't approach in the first place.


\begin{itemize}
  \item
  If a predator's actions seem unclear, have them roll \roll{Charisma}{Combat} (or any other martial Skill) against \gls{tn} 7.
  \item
  Beasts, who have no Charisma Attribute, must simply roll Brawl.
  \item
  The morale roll has some adjustments (see the chart).
  Most modifications grant +2 to the check when the \glspl{pc} seem weak, and -2 when they seem strong.
  \item
  A failed roll for a mindless animal might mean a chitincrawler waits and watches the party, but does not approach.

  \item
  On a tie, the enemy threatens the \glspl{pc} but takes their lead -- if the \glspl{pc} run, they attack, if the \glspl{pc} stay, they back off.
  \item
  Success means `I think I can take them', but the story does not end there.
  Once you roll the dice, keep them there.
  The bandits who felt fearless because they outnumbered the \glspl{pc} may reconsider once a few die, and they no longer outnumber the \glspl{pc}.
  \item
  Enemies only roll a single check for the group, but everyone keeps their own score.
  \item
  The \gls{gm} should keep the \glspl{npc} roll hidden.
\end{itemize}

\begin{exampletext}
  The Witch coven approach, shouting their demands arrogantly.
  The \gls{gm} has rolled only `6' for their Morale check, but most of the witches present have at least one magical sphere at +2 (a martial Skill), so they have a total of `8', and think they can take the troupe on in a fight.

  Unfortunately, a few of the apprentices' highest sphere is only at level 1, meaning they rolled a `7' -- a tie.
  And if those apprentices receive a single point of Damage, the wound will reduce them to a score of `5', and they will run, which will mean the troupe outnumber the witches, and the rest will flee on the next round.
\end{exampletext}

You can use a single roll for an entire combat -- the \gls{gm} simply keeps that roll hidden.
If the enemy rolls a `12', all of them will probably fight until they die.
If they roll a `7', they may start to flee once wounded, and then more will flee once only half remain (but they continue to recheck only at the start of a round).

Most combats will end with one side or the other running away -- few troops want to fight to the last man when they could potentially be safe at home by the end of the day.

When people fail a morale check, they often don't show any fear -- instead they become friendly.
Nobody wants to get on the bad side of someone who looks like they mean business and a bloody nose.
Of course, those people can reassess the situation.
If they \glspl{pc} take no care about their travelling companions, they may find that those `penniless traders', meant them harm all along, and just wanted to wait until the right moment to slit their throats.

The players do not take Morale checks -- they decide when it's time to run away by the look of the situation.
Usually a good time is when all the \gls{fp} have run out.
\footnote{The \glsentrytext{gm} may also wish to cut all Morale checks for any \glspl{npc} with remaining \glsentrytext{fp}.}

\moralechart

\end{multicols}

\section{Happenings \& Situations}

\begin{multicols}{2}

\subsubsection{Village Events}
\index{Villages!Events}

Roll $2D6$ for a random village event.

\encVillageEvent

\end{multicols}

\section{Missions}
\index{Adventure Generator}
\index{Missions Generator}

\begin{multicols}{2}

\noindent
\Glspl{guard} recruits can expect harsh duties.
The character with the highest rank will be asked to lead the party (ties are broken by Charisma + Deceit), and give a bonus to the first die-roll.

\begin{itemize}
  \item
  Roll $1D6$ for Fodder
  \item
  Roll $1D6+1$ for Archers
  \item
  Roll $1D6+2$ for Cutters
  \item
  Roll $1D6+3$ for Rangers
\end{itemize}

Then add a complication with the same bonus as before (\autopageref{missionComplications}).

\subsubsection{Missions}
\index{Missions}

\ngMissions

\subsubsection{Complications}
\label{missionComplications}

Add a bonus for rank, as before.

\missionComplications

\end{multicols}

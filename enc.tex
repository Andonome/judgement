\chapter[Encounters at the Crossroads]{Encounters}
\index{Random Encounters}
\index{Encounters}
\label{encounters}

\section{Things on the Road}

When it rains, it pours; and when creatures attack, the \glspl{pc} will often find them among interesting plants, or else bandits might attack by weakening a bridge while the \glspl{pc} cross it.


\begin{multicols}{2}

All encounters combine multiple elements together.

\begin{enumerate}
  \item
  Start with the current Season

  \begin{exampletext}
    Let's assume a hot Season.
  \end{exampletext}
  \item
  Select the location: villages, edge, or forest.

  \begin{exampletext}
    And assume the forest for now.
  \end{exampletext}
  \item
  Roll $1D3$ to find the row, and add those three encounters.

  \begin{exampletext}
    So `2' would mean
    \setcounter{enc}{1}%
    \setcounter{track}{2}%
    `\encBigList'\stepcounter{track},
    `\encBigList'\stepcounter{track}, and
    `\encBigList'.
  \end{exampletext}

  \item
  Roll $1D3$ and add the remaining two encounters on that column.

  \begin{exampletext}
    Rolling `2' means adding
    \setcounter{enc}{0}%
    \setcounter{track}{3}%
    that `\encBigList'\stepcounter{enc}\stepcounter{enc}, and
    `\encBigList'
    (we already have the result between them).
  \end{exampletext}

  \item
  Roll to see what monsters and sightings the \glspl{pc} come across, and combine all five encounters into a single encounter.

\end{enumerate}

  \begin{speechtext}
    The rain drizzles from every leaf in the forest, making a constant racket.
    Pulling back your hood to look around, you see a plant with gigantic leaves -- the perfect spot to shield from the rain.
    Unfortunately a boar seems to have taken the place already.
    She rests, unaware that you stand only ten steps away.

    What do you do?
  \end{speechtext}

  \ldots\ldots

  \begin{speechtext}
    You have your bow out, but before loosing it, long, dark-brown legs wrap around the tree above the boar.
    Each leg seems longer than horse, but grips the tree silently.
  \end{speechtext}


\end{multicols}

\vfill\null

\clearpage

\begin{multicols}{2}

\subsection{Hot Seasons}

\noindent
Heat transforms the land around the villages.
Without the great forest canopies, the Sun scorches the earth, providing a protective ring of open land around villages.
Once people can see creatures coming from a long way off, archers have a much easier time.

The hot seasons melt mountaintops, which floods plains and forests below.
Swelling rivers destroy bridges and any houses built too close to the banks.
Stone-walled cities often create miniature lakes or run-off areas for the river to expand into.

\setcounter{enc}{0}
\setcounter{track}{0}

\encWarmVillages

\encWarmEdge

\encWarmForest

\begin{multicols}{2}
\subsubsection*{Monsters}

\setcounter{enc}{4}
\begin{dlist}
  \Repeat{6}{
    \item
    \encMons
  }
\end{dlist}

\subsubsection*{Sightings}

\setcounter{track}{3}
\begin{dlist}
  \Repeat{6}{
    \item
    \encSight
  }
\end{dlist}

\end{multicols}

\subsubsection{Floods}

Heat often brings flooding, especially around rivers.
Anyone doing anything (or doing nothing) by a river must make some kind of check at \tn[8] to ensure the flood does not sweep away their plans.
Floods can destroy bridges, roads, and even houses, although most people who live in areas prone to flooding will build their houses somewhere high, or put them on stilts.

\subsubsection{Hot Spell}

The forests present many dangers, but they at least provide protection against one of the most insidious.
Heat begins by stripping people of their senses, and soon burns their skin.

Mechanically, heatwaves inflict $1D6$ \glspl{fatigue} on anyone in the area, although the right clothing and preparation (\roll{Intelligence}{Wyldcrafting}) will half the amount taken.

Anyone foolish enough to wear heavy armour in a heatwave will always take the full brunt of the \glspl{fatigue}, plus one per \gls{weight} of the armour.

\end{multicols}

\bigLine

\begin{multicols}{2}

\subsection{Mild Seasons}

Mild Seasons outnumber the others, but also bring heavy storms.
During peaceful months, people wander and trade, farm and plan.

When storms come, strong winds and earthquakes can tear apart bridges and houses.
Stone castles, if properly made, can withstand earthquakes, and anyone living far enough underground often won't notice as the heavy earth around them protects.
But anyone on the land should leave their house quickly.

\encMildVillages

\encMildEdge

\encMildForest

\begin{multicols}{2}
\subsubsection*{Monsters}
\label{monsterEncounters}

\setcounter{enc}{3}
\begin{dlist}
  \Repeat{6}{
    \item
    \encMons
  }
\end{dlist}

\subsubsection*{Sightings}

\setcounter{track}{2}
\begin{dlist}
  \Repeat{6}{
    \item
    \encSight
  }
\end{dlist}

\end{multicols}

\end{multicols}

\bigLine
\begin{multicols}{2}

\subsection{Cold Seasons}

Snow covers the land so deeply that people only know where the road lies by the lack of trees.
Protective clothing becomes a necessity.

Despite this, people can travel the lonely roads safer than before, as two of Fenestra's most dangerous creatures -- the chitincrawler and basilisk -- must hibernate when cold.
Wolves and bears 

\encColdVillages

\encColdEdge

\encColdForest

\subsubsection*{Monsters}

\begin{multicols}{2}
\setcounter{enc}{1}
\begin{dlist}
  \Repeat{6}{
    \item
    \encMons
  }
\end{dlist}

\subsubsection*{Sightings}

\setcounter{track}{1}
\begin{dlist}
  \Repeat{6}{
    \item
    \encSight
  }
\end{dlist}

\end{multicols}

\end{multicols}

\bigLine

\begin{multicols}{2}

\subsection{Bandits \& Brigands}

Once farmers cannot sustain themselves any more, many turn to banditry.

When soldiers cannot or will not sustain themselves on their duties, they become brigands.
Brigands plan better, have excellent training, and the best weapons money can buy.

The number appearing is equal to $1D6\times 2 + 4$.
Rolling an odd number means brigands, rather than bandits.


\encBandits

\subsection{Traders \& Nomads}

Local traders wander back and forth, pedalling what they have to places that do not have that thing.
They form the blood of civilization.
Unfortunately, they have a very short life expectancy, given all the beasts and bandits on the road.
As a result, most wander the road armed, wearing armour outside of the warmer seasons, and often with a crossbow at their side.

Traders often have important knowledge of the way ahead, since they will almost invariably come towards the troupe, rather than from behind them.%
\footnote{People only come behind you on the road if you walk very slowly, or if they're going very fast.}
If a path ahead if blocked by a tree, or a bridge looks like it's about to give out, they'll know it.
They may also warn of bandits ahead, especially if they've just been robbed!

Traders carry all manner of goods.
Roll $3D6$ and add all results.

\encTraders

\end{multicols}

\section{Happenings \& Situations}

\begin{multicols}{2}

\subsubsection{Village Events}
\index{Villages!Events}

Roll $2D6$ for a random village event.

\encVillageEvent

\end{multicols}

\section{Missions}

\begin{multicols}{2}

\Glspl{guard} recruits can expect harsh duties.
Roll $1D6 +$ rank.

\subsubsection{Missions}
\index{Missions}

\ngMissions

\subsubsection{Complications}

\missionComplications

\end{multicols}

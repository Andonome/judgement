\chapter{Encounters at the Crossroads}
\index{Seasons}
\label{encounters}

\Gls{fenestra}'s core loop starts with a mission, interrupts the mission with an encounter, then pulls the \glspl{pc} into the context of \gls{fenestra}'s daily life.

If a mission has gone well, the troupe may return the way they came, and wrap things up nicely.
Otherwise, they will have to prepare an excuse, and hope that one of the locals will attest to the troupe's honest attempts.

You can use these little scenes like a chaos-flavoured sauce on top of the regular \gls{sq}~\glspl{segment}.
However, if you're just getting to know \gls{fenestra}, you may want to spend a session or so using them as the main stew; note a few encounters ahead-of-time and construct some foreshadowing, then compile a list of potential repurcussions for everything the troupe do and use it to build to a climax.

\section{Missions}
\label{NGmissions}
\index{Adventure Generator}
\index{Missions}

\begin{multicols}{2}

\noindent
\Gls{guard} recruits can expect harsh duties.
The character with the highest rank will be asked to lead the party (ties are broken by \roll{Charisma}{Deceit}), and give a bonus to the first die-roll.

\begin{itemize}
  \item
  Roll $1D6$ for Fodder or Diggers
  \index{Diggers (rank)}%
  \item
  Roll $1D6+1$ for Archers
  \index{Archers (rank)}%
  \item
  Roll $1D6+2$ for Cutters and Rangers
  \index{Cutters (rank)}%
  \index{Rangers (rank)}%
  \item
  Roll $1D6+3$ for Thanes
  \index{Thanes (rank)}%
\end{itemize}

Then add a complication with the same bonus as before (\autopageref{missionComplications}).

\subsubsection{Goal}
\index{Missions}

\ngMissions

\subsubsection{Complications}

Add a bonus for rank, as before.

\missionComplications

\end{multicols}

\section{Things on the Road}
\label{randomEncounters}

\begin{multicols}{2}

\subsection{Daily Encounters}

\begin{speechtext}
  Everyone travels together, in heavy wagons, and bows ready for action.
  But once you arrive back home, the traders finish and want to move on by morning, which doesn't leave much time to say `hello' to everyone.
  So you decide to leave with the next trading caravan, but a days later and nobody's arrived.
  Nobody tells you to go, but when food's running low you just kinda \emph{feel} it.

  The little places by the towns don't see much danger, so if you're not too far from a big place, you might just walk to the nearest road and see if you can join a caravan heading back to town.
  Of course the bow's at the ready, and a spear if you can stand the extra weight.

  When you spot a creature, you can try to stare it down.
  If that doesn't work, give it a nasty sting and most go away.
  Gotta be careful close to the cold though -- once the beasts feel the frost coming on they get \emph{fearless}.

  Once the cold settles in, the forest calms down.
  Most of the big nasty ones sleep under the snow until it melts.
\end{speechtext}

\boxPair[t]{
  \begin{encChart}{Weather}
    \Repeat{16}{
      \encLine \bigWeatherList \\
    }
  \end{encChart}
}{
  \begin{encChart}{Forest Encounter}
    \Repeat{16}{
      \encLine \bigBeastList \\
    }
  \end{encChart}
}

\begin{figure*}[b!]

\begin{multicols}{3}
\small
%%% Ex table 1
\setcounter{enc}{15}
\setcounter{diceNo}{13}
\vspace{2em}
\rowcolors{2}{}{gray!10}
\noindent
\begin{boxtable}[c|L]
  \hline
  \hline
  \textbf{Roll} & \textbf{Inner Hamlets} \\
  \hline
  \Repeat{2}{
    \addtocounter{diceNo}{-1}
    \addtocounter{enc}{-1}
    \arabic{diceNo} & \bigBeastList \\
  }
  \hline
  \addtocounter{diceNo}{-1}
  \addtocounter{enc}{-1}
  \arabic{diceNo} & Bandits! \\
  \hline
  \Repeat{8}{
    \addtocounter{diceNo}{-1}
    \addtocounter{enc}{-1}
    \arabic{diceNo} & \arabic{diceNo} wagons \\
  }
  \hline
\end{boxtable}

%%% Ex table 2
\setcounter{enc}{15}
\setcounter{diceNo}{13}
\vspace{2em}
\rowcolors{2}{}{gray!10}
\noindent
\begin{boxtable}[c|L]
  \hline
  \hline
  \textbf{Roll} & \textbf{The Lonely Road} \\
  \hline
  \Repeat{6}{
    \addtocounter{diceNo}{-1}
    \addtocounter{enc}{-1}
    \arabic{diceNo} & \bigBeastList \\
  }
  \hline
  \addtocounter{diceNo}{-1}
  \addtocounter{enc}{-1}
  \arabic{diceNo} & Bandits! \\
  \hline
  \Repeat{4}{
    \addtocounter{diceNo}{-1}
    \addtocounter{enc}{-1}
    \arabic{diceNo} & \arabic{diceNo} wagons \\
  }
  \hline
\end{boxtable}

%%% Ex table 3
\setcounter{enc}{15}
\setcounter{diceNo}{13}
\vspace{2em}
\rowcolors{2}{}{gray!10}
\noindent
\begin{boxtable}[c|L]
  \hline
  \hline
  \textbf{Roll} & \textbf{Outer \Glspl{village}} \\
  \hline
  \Repeat{9}{
    \addtocounter{diceNo}{-1}
    \addtocounter{enc}{-1}
    \arabic{diceNo} & \bigBeastList \\
  }
  \hline
  \addtocounter{diceNo}{-1}
  \addtocounter{enc}{-1}
  \arabic{diceNo} & Bandits! \\
  \hline
  \Repeat{1}{
    \addtocounter{diceNo}{-1}
    \addtocounter{enc}{-1}
    \arabic{diceNo} & \arabic{diceNo} wagons \\
  }
  \hline
\end{boxtable}

\end{multicols}

\end{figure*}

\noindent
Encounters adjust, depending on the season, time of day, and how far the troupe have travelled from civilization.

\begin{description}
  \item
  Roll 3 dice read the next encounter.
  \begin{description}
    \item[Die 1]
    shows how many \glspl{interval} until the encounter.
    \item[Die 2 + die 3]
    tell you the weather.
    \item[Die 1 + die 2]
    tell you what creature emerges.
    \item[Die 3]
    may tell you the number of creatures appearing.
  \end{description}
  \item[During the day]
  staying close to settlements means a low encounter roll replaces the normal predators with traders.

  However, it can also mean a greater chance of bandits.

  \item[At night]
  roll as normal, with -1 to the encounter roll and weather rolls.
  The troupe will never find traders -- they do not travel at night.
  \item[In the deep forest]
  Every noise prompts a chance for an encounter.
\end{description}

\end{multicols}

\pagebreak

\begin{multicols}{2}

\begin{exampletext}
  For example, the \gls{gm} rolls `\dicef{3} \dicef{4} \dicef{1}', during a `mild' season.

  On the first reading, this says `(\dicef{3}) after 3~\glspl{interval}, (\underline{\dicef{4} + \dicef{1}}) mist covers the area'.

  On the second reading, (\underline{\dicef{3} + \dicef{4}}) wolves arrive, (\dicef{1}) numbering 7.

\diceTrio{3}{4}{1}{}{Mist}{\dicef{1} +6 Wolves}

\end{exampletext}

\subsubsection{Morale Checks}
\label{morale}
\index{Morale}

Give the \gls{gm} an impartial way to rule when antagonists flee from combat.
At the start of an encounter, roll $2D6$ and leave the result on the table.
If the creature has more \glspl{hp} than the roll, it attacks; otherwise, watches the troupe.

\sidebox{
  \begin{boxtable}
    \textbf{Roll} & \textbf{\glsfmtlongpl{hp}} \\
    \hline
    \twoDice{5} & 10 (\textit{ATTACK!}) \\
    \twoDice{5} &  8 (\textit{ATTACK!}) \\
    \twoDice{5} &  5 (\textit{wait\ldots}) \\
    \twoDice{8} &  4 (\textit{RUN!}) \\
    \twoDice{8} &  2 (\textit{RUN!}) \\
  \end{boxtable}
}

\Glspl{crawler} often have 10~\glspl{hp}, so they often attack people the moment they spot them.
Griffins have fewer \glspl{hp} than \glspl{crawler}, so they are more likely to stalk the party, or simply leave.
And goblins might have only 5 or 6~\glspl{hp} -- they know how small they are, and prefer to spend time laying traps, or stealing equipment, rather than engaging directly.
Every \gls{hp} a creature loses could prompt it to turn and flee, so sometimes only wounded creatures will flee, while others keep fighting.

On a tie, the creature pauses to reassess the situation.
A single \gls{crawler} eyeing up a party might display its claws, then observe their reactions; if the troupe do nothing to change the situation (such as casting a spell, or making a loud noise) then the creature rolls again.

\Glspl{pc} do not take Morale checks -- the players decide when it's time to run away by the look of the situation.
Usually a good time is when all the \glspl{fp} have run out.
\index{Henchmen}%
However, \gls{npc} travelling with the \glspl{pc} still take morale checks as usual.%
\footnote{The \glsentrytext{gm} may also wish to cut all Morale checks for any \glspl{npc} with remaining \glsentrytext{fp}.}%


\subsection{The Civilization Rating}
\label{civilizationRating}
\index{Civilization Rating}

Roads become safer near \pgls{broch}, and safer still closer to a town.
At the outer \glspl{village}, the only safety comes from \pgls{broch} which blasts noise to summon beasts, so archers can sting them.
\Glspl{broch} can summon a few beasts every week like this, and either kill them, or give them enough of a nasty shock that they leave the area.
Of course, their children won't remember the lesson, so the remedy must be readministered endlessly.

By the time one sees a town's walls, everything is safe during the day.
Along the road between settlements, a few beasts wander through at night, making these areas always a little dangerous.

Almost nothing gets through the protective layers of \glspl{broch} and outer \glspl{village} which surround towns.
Wild predators could move between the \glspl{village} if they wanted, but they don't have the brains -- most just move towards the first noises they hear, and then receive a nasty sting from archers.
After enough predators die or leave an area, it becomes more reliably safe, at least during the daylight.

You can think of the Civilization Rating as a line which moves down, slowly altering the encounter tables.
Past the end of the last road, the forest replaces the all traders with beasts.

\subsubsection{Travelling Traders}
\label{traders}

While moving, everyone meets other travellers head-on, moving in the opposite direction from them.
They briefly swap news of the road behind, occasionally make a trade, then move on.

\encTraders

Each of your encounter dice also represent an interesting item the traders have.
For example, rolling \dicef{8}\dicef{3} would indicate that the \glspl{pc} can buy some rope and pitch.

\subsection{Noises in the Forest}

In the dark forests, where towering trees dampen the Sunlight, any out-of-place noise, light, or smell, can excite and draw in the predators which live there.
Whenever the troupe draw attention to themselves, roll $1D6$.
On the roll of a `1', roll for a random encounter.

You can pull this off quickly with $3D6$, where only the first die indicates the encounter, where the others indicate \emph{which} (if any).

\subsection{Weather}
\label{weather}
Cosmological and meteorological events can boost people's abilities to cast spells -- even those who had none before.
These bonuses stack with \glspl{boon}, so someone with the right know-how can (and do) stand under the unnatural darkness of an eclipse, holding the ground-up remains of a stirge queen, with auroch hooves boiled to melting point, at the top of a mountain,%
\footnote{High level spells often have such a long casting range that casters must climb high to see their target.}
screeching about opening a portal from a distant \gls{village} to an underground realm.

\subsubsection{Cold Snaps}
\index{Cold Snap \glsentrytext{R}}
\index[mana]{Fate!Cold Snaps \glsentrytext{R}}
drain the life from characters and wildlife in a blink.
Until the weather warms up again, all actions outdoors inflict an additional \gls{ep} or more, depending on clothing.
Someone journeying outdoors effectively naked will receive 3 \glspl{ep}, per \gls{interval}.

The mass disturbances in the air, mixed with the darkness, creates a kind of magic in the air, which grants +1 to everyone's Fate Sphere, including those who cannot normally cast spells.

\subsubsection{Earthquakes}
\index{Earthquakes \glsentrytext{R}}
\index{Quakes}
\index[mana]{Earth!Quakes \glsentrytext{R}}
can topple \gls{village} walls, houses collapse inward, and gnomish warrens have their planning tested from the foundations to the rooves.
Despite the shaking, most structures remain standing.
Architects in \gls{fenestra} design castles for quakes, and dwarvish settlements, down at the edge of the \gls{deep}, often feel nothing, as underground caverns don't shake during quakes as badly as the surface does.

When quakes alter rock, everyone with the tiniest understanding of magic finds themselves able to speak to stone, at least a little, as rocks and ice start to wake up.
This time grants a +1 bonus to everyone's Earth Skill, for the short duration of the quake.

\subsubsection{Eclipses}
\index{Eclipses \glsentrytext{R}}
\index[mana]{Air!Eclipses \glsentrytext{R}}
plunge \gls{fenestra} into sudden daytime darkness, as the \gls{ainumar} blocks the Sun.
For this entire \gls{interval}, the winds becomes more pliable, and anyone with a voice to speak gains +1 to their Air Skill.

The average person makes little use of this time -- they have more pressing concerns given the sudden lack of Sunlight.
But casters often plan grand rituals to coincide with an eclipse, as the extra powers allow them to perform wide-ranging, potent spells.

\subsubsection{Floods}
\index{Floods \glsentrytext{R}}
\index[mana]{Water!Floods \glsentrytext{R}}
damage infrastructure worse than earthquakes.
They rot food, and degrade the foundations of houses in subtle ways, which only become apparent years later.
High fortifications remain untouched, but underground dwelling runs the risk of water pouring in from above, and driving everything inside up into the Sunlight.

The predators of \gls{fenestra} seem to have an instinct for floods, and will camp outside any homely holes in their territories.

Travel during a flood poses serious problems -- any affected area will half the travellers' rate of movement.

While floods occur, everyone gains a +1 Bonus to their Water Skill.

\subsubsection{Heatwaves}
\index{Heatwaves}
\index[mana]{Fire!Heatwaves \glsentrytext{R}}
pose problems for anyone who wants to wear armour.
All travel in direct Sunlight will have the travellers endure an additional +1 \glspl{ep} when wearing loose-fitting clothing, or +2 when wearing inappropriate clothes.
Armour of any kind does not count as `appropriate', so a heatwave can force any warrior to remove their armour.

During a heatwave, everyone gains a +1 bonus to their Fire Skill.

\subsubsection{Hurricans}
\index{Hurricans}
\index{Marching!Hurricans}
Make travel challenging, and reduce the standard miles travelled by 1.
All large items -- rooves, carts, et c. -- have a 1 in 6 chance of getting pulled off and possibly thrown by the wind.

\subsubsection{Snow}
\index{Snow}
\index{Marching!Snow}
slows travel.
Travelling over snow-covered lands reduces the standard travel to half.
In the deep forests, this makes little difference, but an open road which once granted five miles of easy travel per \gls{interval} would only allow three miles once covered in frost.

With enough foot-fall, the penalty reduces to nothing, as others stamp the snow down, reducing it to frosty mud.

\end{multicols}

\setcounter{diceNo}{13}
\setcounter{diceNo2}{15}
\setcounter{enc}{17}

\begin{wideTable}[c|c|c|LLLL]{Encounters Across Biomes}
  \hline
  \hline
  \textbf{Warm} & \textbf{Mild} & \textbf{Cold} & \textbf{Forest} & \textbf{Lakeside} & \textbf{Mountains} & \textbf{Swamp} \\
  \hline
  \encLine \bigBeastList & \encLakeside & \encMountains & \encSwamp \\
  \encLine \bigBeastList & \encLakeside & \encMountains & \encSwamp \\
  \encLine \bigBeastList & \encLakeside & \encMountains & \encSwamp \\
  \encLine \bigBeastList & \encLakeside & \encMountains & \encSwamp \\
  \encLine \bigBeastList & \encLakeside & \encMountains & \encSwamp \\
  \encLine \bigBeastList & \encLakeside & \encMountains & \encSwamp \\
  \encLine \bigBeastList & \encLakeside & \encMountains & \encSwamp \\
  \encLine \bigBeastList & \encLakeside & \encMountains & \encSwamp \\
  \encLine \bigBeastList & \encLakeside & \encMountains & \encSwamp \\
  \encLine \bigBeastList & \encLakeside & \encMountains & \encSwamp \\
  \encLine \bigBeastList & \encLakeside & \encMountains & \encSwamp \\
  \encLine \bigBeastList & \encLakeside & \encMountains & \encSwamp \\
  \encLine \bigBeastList & \encLakeside & \encMountains & \encSwamp \\
  \encLine \bigBeastList & \encLakeside & \encMountains & \encSwamp \\
  \encLine \bigBeastList & \encLakeside & \encMountains & \encSwamp \\
  \encLine \bigBeastList & \encLakeside & \encMountains & \encSwamp \\
  \hline
\end{wideTable}


\section{Emergent Situations}
\label{roadEncounters}

The \glspl{pc} will find more trouble than monsters as they travel.
Whenever a journey ends, a situation begins.

\begin{multicols}{2}

\subsection{On the Road}

\subsubsection{\Glsfmtplural{bothy}}
\glsdesc{bothy}.

Each \gls{bothy} begins when someone dies on the road, and others leave rocks there to pay their respect.
If enough people gather rocks, it becomes a cenotaph, and takes the name of the fallen.
If more rocks follow, the cenotaph becomes \pgls{bothy}.

Anyone using \pgls{bothy} must gather and leave as much firewood as they found when they entered, except the \gls{guard}, who must leave even more, fix anything which needs fixing.
Nobody will trust someone known to leave \pgls{bothy} in a bad state, but if \glspl{guard} leave them in a bad way, they will face questions in the \gls{court}.

Roll to check \pgls{bothy}'s circumstances upon arrival:

\bothyEvents

%!
\needspace{18em}
\subsubsection{\Glsfmtplural{village}}
\label{villageEvents}

\encVillageEvent

\subsection{Lonely Taverns}
\label{lonelyTaverns}
\index{Lonely Taverns}

These taverns exist on long stretches of road, far from any town, and
charge high prices for a drink. They must live off traders passing
through, and survive whatever the forest brings out.

Normal people don't stay for long.
Those who stay a while often have problems with the local
law, as these places often make their own laws.
Barkeeps punish any robbery close to the tavern harshly, but don't often care about what people do around the towns.
This makes these taverns a safe intermediate location where anyone can talk in peace.

Of course, bandits won't announce themselves as such when speaking with \glspl{guard}, but then the \glspl{guard} often won't announce their employment either, no matter how obvious that sword on their back makes them.

\subsubsection{The Barkeep}

Roll $1D6$ to find this season's barkeep (they change all the time):

\begin{dlist}
  \item
  A veteran of the \gls{guard}, with a hundred war-stories. Of course
  when he tells them, nobody can get a drink, so don't ask!
  \item
  Someone from point 4 on the map, hiding here with a bounty on their head for thieving from \pgls{warden}.
  \item
  An outlander from a land so far away, nobody has ever heard of it.
  Every story she tells sounds made-up, but the strange accent shows she really does come from somewhere distant.
  \item
  A powerful \gls{witch} who swore an oath never to use magic again.
  He won't say why.
  \item
  A collective -- you stay as long as you like, earn your keep, then go
  when you please. Sometimes in the colder Seasons, the place just lies
  barren.
  \item
  A dwarf who records all he can.
  The patrons say he works as a spy for someone, but they disagree about whom.
\end{dlist}

\subsubsection{The Menu}
\index{Menus|see {Lonely Tavern}}

Roll $3D6$ -- the \glspl{pc} can order any of these meals.

\newcommand\menuItem[3][(\arabic{r12} \glspl{cp})]{%
  \randomdozen%
  \randomthree%
  \randomfourB%
  \ifodd\value{enumi}
    \randomthreeC%
    \randomfour%
  \fi
  \item
  \textbf{#2:}
  #3
  #1
}

\begin{dlist}
  \menuItem{Griffin-wing}{freshly killed this morning, after the griffin tried to fly away with a gnomish patron.}
  \menuItem{Mystery-stew}{why are you hesitating?
  It goes rotten quick, so get eating!
  \footnote{\Glsfmttext{crawler} `meat' (webbing as sauce!)}
  }
    \ifnum\value{r4}<2
      \newcommand\morningSoup{uproot}
    \else
      \newcommand\morningSoup{marching_mushroom}
    \fi
    \menuItem{Sunrise Soup}{the chef found a new plant this morning, and he's already learning how to cook it!
    \footnote{In fact this is \nameref{\morningSoup}, see \autopageref{\morningSoup}, for the effects.}
    }
  \menuItem{Deer}{thank the man in green, sitting in the corner -- he caught it this morning.}
  \ifodd\value{r4}
    \menuItem{Dwarf-beard}{actually just a type of seaweed, left as payment by a local trader; but it tastes just like the real thing!}
  \else
    \menuItem{Eye-Spy}{made with actual \gls{woodspy}.}
  \fi
  \ifodd\value{r3}
    \menuItem[(0 \glspl{cp})]{Get bent}{the barkeep's in a foul mood, because they need a day off.}
  \else
    \menuItem[]{\ldots and bugger-all-else}{a few barrels turned out to be rotten, and now someone's stolen an entire pot of soup.
    The menu will be limited for the day.}
  \fi
\end{dlist}

\subsubsection{The Patrons}
start with $2D6$ \gls{guard} \gls{fodder}.
Roll $3D6$ for the rest, and accept every unique result.

\begin{dlist}
  \item
  An elf who doesn't speak the \gls{tradeTongue}.
  \item
  \Pgls{mixer} from the \gls{templeOfSickness}, on a mission to recruit from the \glspl{village}.
  \item
  \Pgls{doula}, carrying a map of one point (roll $1D6+4$), and preparing to make another
  \item
  A piper who really wants to practice, and keeps justifying why they should be allowed to make loud noises.
  \item
  A caravan of $1D6$ traders (see \vpageref{traders} for their wares).
  \item
  \Pgls{seeker}, here to collect information.
\end{dlist}


\end{multicols}

\encTownEvents



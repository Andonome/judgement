\section{History}
\index{History}

\begin{multicols}{2}

\subsection{A Creation Tale}
\index{Creation Tale}

\begin{exampletext}

\ifcase\value{r4}\relax\or
  \noindent
  In the beginning, we traipsed through mud.
  The world felt weak, and filth clung to our beards.
  So we pulled away the muck and filth, creating a great chasm as we searched for something real.
  And at the bottom, we found \emph{stone}.

  We pulled it up, learned to work with it, and shape it.
  We built dry land, then piled it up to create houses, then castles, and structures even greater.
  Weapons came soon after, and the stone let us best all the unarmed creatures of the world.

  We felt ourselves lords.
  Nobody could stand against the might of stone.

  But we boasted too much, and felt too much pride.
  And soon the gods laughed at us.

  The gods shook the earth, and the shaking pulled stone from beneath the muck, more and more, until mountains formed.
  Soon every man and maggot could see great bodies of stone, larger than any which we could have imagined.
  We had lost our edge.

  We continue to live around stone, and work with it well.
  And now we dig up iron, silver, platinum, and gold.
  But we must take care never to boast about them too much, because the gods will laugh at us again, and send down the golden monster, which will pour gold over all of the lands, and turn the hard-earned savings of every clan to nothing.
\or
  Humans first arrived from across the ocean.
  We can't say where from, but somewhere \emph{far}, certainly.

  We don't fit in here.
  The gnomes burrow safely underground.
  The gnolls seem to live freely, and unafraid of the beasts.
  The elves wander like masters of the realm; like children playing in the forest.
  Only humanity struggles so much to merely survive.

  Once we settle properly, and find a true Rex to lead us, we will build massive boats, and sail again.
  Our ancestors must think us dead, but when they see us again, they can use the plentiful supplies of their paradise to give us more supplies -- food and weapons.

  Once we have established ourselves, we can create an efficient army, and burn all the forests in the land to the ground, from the sea to the centre.
  We will slay the gods, and turn the forest into farmland.
\else
  \noindent
  Deep beneath the earth, worms the size of castles roam, occasionally rising to push up the earth into great mountain-like mole-hills.
  These worms were created by the gods.

  Making such big creatures was difficult, so they all had to work together.
  However, the creature proved too complex, so most of the parts were removed before the gods woke it up.
  At first, it had too many eyes, so the eyes were removed, but the gods did not notice that they were already a little bit awake, so they walked away, and mated with each other, and eventually those eyes turned into elves.
  The worm's face looked like a dragon, so of course it had a beard, but the gods decided this was unnecessary and cut it off.
  Where the beard fell, a little dwarf was formed, and quickly escaped by burrowing underground.
  When the creature was seen to be too smart, they removed some of its heart to make it less able to think (because our thoughts and souls reside in our hearts).
  The heart burrowed underground to become warmer, and turned into a little gnome.
  Finally, they decided that its arms were only holding it back in the narrow tunnels it made, so they took its muscular arms off and threw them away.
  They scuttled off and later became strong humans.

  The gods later noticed all the creatures running around the world.
  They wanted one to rule the world, governing it properly, and put fire in their blood so that they could command every part of the world.
  At first the job was given to the elves, but they did nothing but sit about in their forests reading poetry.
  The gods declared that they would no longer rule and at that moment all the elves in the world swore never to recognise them again.
  Later, the gods requests that the dwarves would rule the world, and gave them the gift of runic magic to help rule.
  But the dwarves just mined and mined, ignoring everything which was happening on the surface, favouring exploration of the great worm tunnels.
  Next the gods gave dominion to the gnomes, and to help them govern wisely told them secrets about how the universe works and how to bend it to their will.
  The gnomes tried in earnest to figure out how best to govern, and the discussion grew and grew but they could not reach a conclusion.
  While the gnomes continue to argue to this day, the gods did not have all day, so they decided it was time to speak directly.
  They selected humans to commune with, speaking with them directly through divine prayer.

  And that, little one, is why we live here, instead of them, and why your dad needs to go and get more land for all the little brothers and sisters you might have later; because if we want to govern the land we have to take it first.
\fi

\end{exampletext}

\subsection{The War of Lies}
\label{warOfLies}
\index{War of Lies}

\begin{exampletext}
  \ifcase\value{r4}\relax\or
    The war of lies was not a real war, but a simple legal error, compounded until records became a mess.
    \Glspl{scribe} called out for a distinction between fact and fable, which implied a consistent view of history.
    This meant that once a `fact' had been established, scholars changed all other records and accounts of history to become consistent.
    Every new `corrected' book had to make the long journey from city to city, where others would rewrite it, and pass copies along.

    Scholars guessed at what they did not know -- consistency eventually became more important than truth, and every spelling mistake and inkblot which made its way past the rushed editorial process dragged history in a new direction.

    The end of this era came with stories about how the initial confusion arose, which themselves had to create a consistent structure, so scholars began to explain the fake books entirely by reference to the adventurers who would pay incredible sums for books about long-lost treasures, without knowing if the book was true or false.
    As a result, early history contains more guesswork than real records.
  \or
    After the great elven crime, they began to cover up their lies with myriad false histories.
    Of course, writing simple falsehoods would fool no-one, so instead the elves produced thousands of copies of nearly-correct works, obscuring their own misdeeds in a mountain of conflicting histories, with names changing here, a king swapping his generals, or warriors slotting into the wrong century.

    For their final trick, the elves published books on how the confusion and lies all began with misguided rules for consistency, which began a fashion among scholars of laughing at the notion that any of their history has to contain `facts', or that it should `make sense when taken with the other stories'.
    The lie has worked, and to this day, nobody can say for sure exactly what the elves did that required so much work to cover up.
  \or
  Copying out complete books is a lot of work.
  A scribe must dip a pen into an inkwell around fifty-thousand times for a book of three hundred pages.
  If an inkwell takes around two hundred dips, then a book demands two hundred and fifty vials of ink.

  Many gnomes seek to optimize everything in their path, or anyone else's path.
  One of these gnomes, horrified with the work of writing, and too lazy to perform it, decided to create a complete solution to the problem, using alchemy.

  Life magic grew reeds for paper, and death magic helped to dry them once done.
  He created an apparition of a scribe with Light magic, then gave it the ability to touch and move objects with Force magic.
  More Life magic provided ink, and Force magic provided the inkwells (or a practical, if unnatural facsimile).

  Once his production-line stood complete, Mind magic gave the scribes the single wish of copying out books on history.
  They identified the books he owned which spoke about history, and began to copy them, day and night, seeking always to have \emph{more and more history}.

  These fake scribes, created through a web of alchemical theories, quickly found a hard limit to copying out history: they only had so much history to copy, so they could only create more copies, but could not perform their only desire -- to copy out \emph{more history}.

  The first day reached a fine conclusion as the `scribes' produced a copy of a pile of history books.
  On the second day, the gnome did not check the new books, so he had not noticed that they had \emph{additional} history.

  The study, practice, and essential meaning of history does not mean `truth'.
  Books on history do not need to be accurate.
  Books on history contain reasonable attempts at guessing our past.
  So in order to maximize the quantity of history, the `scribes' created more, with their best guesses.
  These scribes then created explanations for their own malformed views on history, and began to describe a period known as `the War of Lies`, where history became muddled due to a lot of sell-swords purchasing faked books on history.

  The books sold well, and for a short time, nobody noticed the inconsistencies between them.
  And by the time they did notice, children had been raised with false ideas, and nobody wanted to admit their families had learned a lot of nonsense.

  People learn much easier than they unlearn.

    One silver lining has come from this calamity -- the only thing the nonsensical history books kept constant -- a universal measurement of time, used by gnomish machines.
    Humans still use \gls{gmt} as the standard for measuring time to this day.
  \else
    Many a travelling adventurer crew made their money from researching fallen cities, then raiding locations which had the best loot.
    This produced a market for antiquated books which mentioned golden thrones, magical wonders, lochs of eternal strength, and trapped scrolls which reveal the answer to any question posed.

    Unethical scholars soon began creating a horde of `hidden histories' books, which looked authentic, but contained subtle clues to glorious treasures.
    Of course, claiming the `lost city of gold' was over the hill meant angry adventurers returning soon after.
    The smartest of the con-artists always hinted at treasures lying in the most dangerous areas of the deep forest.

    Scholars `deduced' any historical details they did not know about, but since adventurers never read much about history, the books' accuracy hardly mattered.
    Sensationalism trumps truth when it comes to sales, so the writers relied on creating tales to cater to their readers' prejudices and hopes.
    And when angry adventurers returned from some desolate swamp, empty-handed, the charlatans could always blame their book's errors on `an elvish conspiracy'.
  \fi

\end{exampletext}

\end{multicols}

\printindex[talismans]
\label{talismanIndex}

\indexprologue{
  \begin{multicols}{2}
  \rotRates
  \noindent
  Various events and items can boost spells, whether cast through \pgls{skill}, or \gls{alchemy}.
  Each boon has a symbol showing whether it comes from an animal (`\A'), or a plant (`\Pl').
  Others (`\glsentrytext{R}') indicate meteorological events which give a boost to \Pgls{sphere}.
  These meteorological events, and \glspl{boon} made from \pgls{ingredient} each add +1 to \pgls{skill} \gls{sphere}, but only one of each can apply; i.e. to get +2 to the Fire \gls{sphere}, one needs to use a bear's heart during a heatwave, but many bear-hearts will not help.

  \Glspl{ingredient} taken from plants or the bodies of beasts can spoil quickly, depending on the temperature of the season.

  The \glsentrytext{deep} remains cold all year round, so it always counts as a cold season down there.
  \end{multicols}
}

\printindex[mana]

\label{manaIndex}

\printindex


\chapter{Guilded Temples}

Nothing in \gls{fenestra} exists to serve.
As any idiot can plainly see, everything in the world wants to strangle civilization, and kill everything which walks on the surface.
So when people look to the divine, they do not look for salvation, but for danger.

Each of the gods in \gls{fenestra} takes the souls of those who die in their domain back to their own realm.
The god of the cold takes those who freeze to death, and the god of the forests takes the souls of those eaten by the forest's creatures.

Temples exists to protect civilization from the wrath of their god.
Each one also has a divine monopoly within their domain, which includes anything that could plausibly defend against the god.

The temples are guilds, and each guild hall also serves as a temple.
Even a `guild hall' which is nothing more than an old lady's house still offers divine guidance.
Citizens of \gls{fenestra} make no distinction between asking for salvation from violence and purchasing armour.
The Temple of Hate offers respite from Hate, and the Temple of Frost offers warmth.

The captains of the \gls{guard} may as well be bishops, and priests of the Temple of Poison shepherd the civilized world with preservatives.

\section{Temples of Death}

\begin{multicols}{2}

\newcommand\guild[8][\textxswup]{
  \subsection[The Temple of #3]{#1~#2~#1 \\ The Temple of #3}
  \ifdefmacro{#2}{}{\index{#2}}
  \index{#3 (God)}
  \ifdefmacro{#7}{}{\index{#7}}
  \index{Gods}


  \begin{exampletext}
  \noindent
  #4
  \end{exampletext}

  \noindent
  \begin{minipage}{\linewidth}
  \begin{description}
  \item[Domain:] #5

  \item[Defence:] #6

  \item[Watchers:] #7

  \item[Activities:] #8

  \end{description}
  \end{minipage}

  \subsubsection{#7}
}


\guild{\hphantom{Nemo}}% Name
  {Misgenesis}% Death
  {The most terrifying of the gods has an empty realm.
  Not a single soul resides in her realm, as she kills so thoroughly that her victims never were.

  Speaking her name attracts her attention, so it soon became a curse, then a rarity, and eventually became forgotten.
  The doula claim pass her name from one to another, by entering each others' dreams, then writing the name on sand, one glyph at a time.
  }% God
  {Failure to thrive, regret, the road untaken}% Domain
  {Hope}% Defence
  {Doula}% Watchers
  {
    Birthing, lambing, planting, marriages, business mergers, novel journeys, new settlements and death.
  }% Guild activities

  Almost every village has one, and few have more (except when they take on an apprentice.
  Everyone who wants to start something new goes to the doula -- whether starting a business venture, asking someone to marry them, or just travelling between settlements, people request the doula's blessings.
  They frequent towns less often, as the demand on their attention rises rapidly once people know a doula is available.
  Despite the high demand for their attention, they never charge much -- everyone knows they should, by rights, given their blessings in return for a scarf, or a little meat.
  If people see them charging much more, they generally accuse the doula in question of going sour, and begin to refer to her as a `witch'.

  \paragraph{Rank \& Structure}
  stay carefully hidden from public view.
  A few doula know real witchcraft, but those who do won't display it, so those who don't can hint that they have it.
  When one casts a soothsaying spell, she tells others.
  And almost nobody can really track how effective a blessing has been, but eventually, everyone needs some safety from \hphantom{Nemo}, in one way or another.

  \begin{description}
    \item[Apprentices]
    serve for as long as it takes to pick up the basics -- often until their teacher has died (which means they get to keep the house).
    \item[Doula]
    care for the young, assist with births, and wish people well on new ventures; and whether or not their blessings truly change people's fate; or their predictions really see the future, nobody can tell.
    \item[Wayfinders]
    have sufficient magical abilities that they walk fearlessly through even the deepest parts of the forest.
    Wardens pay them handsomely (but rarely in direct coinage) to seek out areas in the forest fit to built a walled village.

    While the doula never like to reveal their rank, seeing an old woman walk out of the green darkness, where even the \glspl{guard} fear to enter, with twigs clogging up her hair, quickly gives the game away.
  \end{description}

\subsubsection{Temples with Tea}
The doula tend to live away from others, in case of mishaps with their witchcraft.
They have no grand halls, just shoddy old houses, or occasionally a potion shop in a large city.

\guild{Wrecan}% Name
  {Hate}% Death
  {When men fight, when hunger threatens and temper flairs, she comes for souls.
  Within her realm, everyone must pick a side, and every side gains new enemies.
  She watches over wars with contempt for both sides, then claims the souls of the dead.}% God
  {Violence, banditry, bigotry, and blood feuds}% Domain
  {Meditation, camaraderie, reconciliation, diplomacy, and quality armour}% Defence
  {Armourers}% Watchers
  {Creating armour, treaties, and reconciling local disagreements}% Guild activities

Anyone who works with leather or metal might join the guild, but that's not usually how they start.
Families with bad-tempered, narcissistic, and self-righteous children send them to the Temple of Wrecan to learn self-control, and possibly inner-peace.
Within the temple, they learn that aggressively striking iron only makes mangled iron -- to really forge something, a metallurgist must strike solidly, reliably, and with precision.

Apprentices must negotiate their own departure by showing that lessons have been learnt, and by selling at least a few pieces (while giving the guild its cut).
Anyone who becomes cynical and contemptuous of the long process finds themself stuck crafting for years.

Those few with the grit to stay become master craftsmen.

\paragraph{Rank \& Structure}
displays itself in two distinct ways.

\begin{description}
  \item[Crafters]
  form the entire guild.
  Masters produce good quality armour, and sell it for a high price, but even the least skilled among the armourers earn well enough.
  \item[Negotiators]
  usually serve for a season or two, before returning to proper work.
  It doesn't pay well (or at all, sometimes), but at least the guild can continue selling while both sides live.

  While nobody enjoys this job, many gain a reputation for their excellent mediation skills, and find themselves forced into fulfilling the demand, either by various plaintiffs, or the rest of the Temple of Wrecan.
\end{description}

\noindent
People value the mediation offered by the Temple of Wrecan for two reasons.
Firstly, they tend to make impartial rulings.
Secondly, nobody wants to visit the Temple of Wardens to receive an official ruling.

Common people often make an agreement to accept whatever the local armourer says, and give him a small payment as thanks for listening (both sides pay equally at all times).
Wardens pay substantially more, as they often require the armourers to travel to distant towns in order to help them calm a rival (or lull them into a false sense of security).

Despite their calling, they never profit from war.
People need enough armour just to survive the forests -- if they start killing each other as well, entire villages could disappear, leaving them with less demand overall.

\subsubsection{Clanging Halls}
The armourers produce and sell in their temples.
The constant noise of hammers, metal, and shrieks from accidents forces everyone to yell instead of talk.

The temple of hate allows people to practice inner calm, but never hands out peace freely.

\guild{Sylf}% Name
  {Beasts}% Death
  {%
    A painfully pregnant belly, and a thorax bloated with eggs; the mother of \gls{fenestra}'s predators overflows with life.
    She meanders around the world, laying monstrosities and feeding them with anything she finds on the world's surface.

    She holds the forms, abilities, and hunger of every predator in the world.
    And when full villages disappear in a night, many wonder if Sylf herself came to eat them.
  }% God
  {The deep forest, darkness, barbarism, and anything which feeds on humanity}% Domain
  {Dogs, bows, and strong humans staying up all night}% Defence
  {\Glsentrytext{guard}}% Watchers
  {
    Protecting traders, manufacturing bows, and burning the forest.
  }% Guild activities

Informally, anyone who stays watch for beasts at night guards against Sylf.
However, the \gls{guard} proper take in anyone they can, i.e. anyone society has deemed surplus.
The \gls{guard} answer the call of villages struggling to survive, build new settlements, and perform missions beyond the \gls{edge}.

Those who survive a few years fall into a few categories.
Firstly, anyone from a rich family can often buy their child into a promotion good enough to keep them safe.
Secondly, the fastest of the \gls{guard} can survive almost any danger, as long as they don't take the lead often, and stay alert when someone makes them.
Almost every beast in the forests will stop attacking once it has a body to feed on, so above all, survival depends on sprinting faster than one's companions.

\paragraph{Rank \& Structure}

\begin{description}
  \item[Fodder]
  do as anyone tells them, until the forest eats them, or they gain proper rank.
  None may enter a settlement of any kind.
  They sleep outside, at the foot of remote towers.

  \textbf{Pay:} no.

  \item[Archers]
  have joined a group past the \gls{edge} and survived.
  This encourages others to treat them like real people, but does not earn them the right to return to civilization without special permission.

  Despite they name, they do not need to use a bow, but most do.

  \textbf{Pay:} 10 \glspl{cp} per week, which they may donate to family or a temple, but may not keep.

  \item[Cutters]
  have at least one kill witnessed by a Horse Rider or better.
  They may enter any settlement if they have a solid reason.

  They build new settlements, protect caravans, and occasionally try to sneak off to relax in a town somewhere, without anyone noticing.

  \textbf{Pay:} 30 \glspl{cp} per week, which they may donate, save, or use to purchase weapons.%
  \footnote{Paying the \gls{guard} leads to terrible stagflation problems, as every death means some more coinage goes missing.
  Of course, people frown upon the practice, but still understand that it's a necessary evil.}

  \item[Riders]
  have at least three heads from the forest -- beasts preferred, but bandits will do.
  They protect fast caravans, deliver urgent messages between settlements, and provide fast reinforcements to villages around the \gls{edge} dealing with basilisks.

  \textbf{Pay:} 1 \gls{sp} per week, which allows them to save up for weapons and armour.

  \item[Jotters]
  need only show their literacy, and some basic organizational skills.
  They can go straight from Cutters, as long as the local \gls{guard} have need of a writer at the time.

  \textbf{Pay:} 2 \gls{sp} per week.

  \item[Rangers]
  must return with captured beasts (although their young, or even eggs, will suffice) in order to gain this rank.
  They track down deserters, burn nests, shoot sleeping creatures in the dark, and often travel through the forest with such stealth and foresight that can move alone.
  
  \textbf{Pay:} 3~\gls{sp} per week, which nobody wants to give them, but they go through armour and arrows so quickly that nobody can argue.

  \item[Builders]
  organize men to build walls in new settlements, or strengthen existing walls.

  \textbf{Pay:} 4 \gls{sp} per week.

  \item[Overseers]
  plan new settlements, organize funds, ensure new recruits have weapons (or don't, if weapons become too expensive), and otherwise do as they please.

  \textbf{Pay:} Nothing.
  They take what they need, and keep all the books.

\end{description}

\noindent
Would-be \gls{guard} begin their journey far from any civilization, in one of the \gls{guard} outposts.
This leaves them with no possibility to protest their position.

Anyone can give the Fodder any order, at any time.
Eventually, they become miserable enough to request a weapon, and join a band of others, in the hopes of gaining some rank by slaughtering a beast.

\paragraph{Funding}
comes from wardens, who pay the guard to keep settlements safe.

The wardens generally think of the \gls{guard} as a protection racket -- stop payment, and you could lose a village.
The \gls{guard} think of the wardens as middle-men, since they receive their funds from the surrounding villages.

\subsubsection{Towers of Light}
\index{Light Towers}
Half way along the long roads, or at the end of roads which go nowhere, tall towards stand watch.
At the bottom, new fodder build fires where they can.
A little back, archers try to sleep through the constant false-alarm cries.
And above, the rest of the guard look out their narrow windows, and occasionally tell those below to investigate this way or that.

Roads which go somewhere often receive passing caravans, and the fodder may get some food, or even find equipment donated to them.
Other light towers exist just to stop the forest getting too close to a nearby town.
These generally defend against the side of a town without water, which makes them ten times as dangerous as any other \gls{guard} tower, since someone must fetch water daily.

\guild[\decotwo]{Eldren}% Name
  {Sickness}% Death
  {
    Those who reach the end unharmed can let their soul fall out peacefully.
    Death by old age guarantees a place with the peaceful dead, in Eldren's realm.
  }% God
  {Pensions, disability, family, and ancestry}% Domain
  {None}% Defence
  {Helpers}% Watchers
  {
    Caring for the old, sick and infirm.
  }% Guild activities

Parents with disables children try to have their children work as a helper, rather than joining the \gls{guard}.

Most live their lives inside a city, not far from their temple.
A rare few travel around neighbouring villages, checking for anyone who's become a burden to their community.

\paragraph{Funding}
comes from people who give a little to the temple throughout their lives, in the hopes of one day dying in it.

\subsubsection{Temple Hospitality}
Approaching the doors, the sick may find a woman with her left-hand limbs missing, and an older man with downs syndrome, ushering them into large, open halls, full of beds where people chat.

Further down, a long hallways presents doors, where less comfortable residents try to die peacefully.
Unfortunately the temple forbids alcohol, as they do not want to risk a single soul to death by poison.
Even the gods have their own laws and technicalities.

\end{multicols}

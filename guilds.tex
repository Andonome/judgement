\chapter{Guilded Temples}

Nothing in \gls{fenestra} exists to serve.
As any idiot can plainly see, everything in the world wants to strangle civilization, and kill everything which walks on the surface.
So when people look to the divine, they do not look for salvation, but for danger.

Each of the gods in \gls{fenestra} takes the souls of those who die in their domain back to their own realm.
The god of the cold takes those who freeze to death, and the god of the forests takes the souls of those eaten by the forest's creatures.

Temples exists to protect civilization from the wrath of their god.
Each one also has a divine monopoly within their domain, which includes anything that could plausibly defend against the god.

The temples are guilds, and each guild hall also serves as a temple.
Citizens of \gls{fenestra} make no distinction between asking for salvation from violence and purchasing armour.
The captains of the \gls{guard} may as well be bishops, and priests of the Temple of Poison shepherd the civilized world with preservatives.

\section{Temples of Death}

\begin{multicols}{2}

\newcommand\guild[7]{
  \subsection[The Temple of #1]{The Temple of #1 \\ \textxswup~Death by #2~\textxswup}
  \ifdefmacro{#1}{}{\index{#1}}
  \index{#2 (God)}
  \ifdefmacro{#6}{}{\index{#6}}
  \index{Gods}


  \begin{exampletext}
  \noindent
  #3
  \end{exampletext}

  \noindent
  \begin{minipage}{\linewidth}
  \begin{description}
  \item[Domain:] #4

  \item[Defence:] #5

  \item[Watchers:] #6

  \item[Activities:] #7

  \end{description}
  \end{minipage}

  \subsubsection{#6}
}


\guild{\hphantom{Nemo}}% Name
  {Misgenesis}% Death
  {The most terrifying of the gods has an empty realm.
  Not a single soul resides in her realm, as she kills so thoroughly that her victims never were.

  Speaking her name attracts her attention, so it soon became a curse, then a rarity, and eventually became forgotten.
  The doula claim pass her name from one to another, by entering each others' dreams, then writing the name on sand, one glyph at a time.
  }% God
  {Failure to thrive, regret, the road untaken}% Domain
  {Hope}% Defence
  {Doula}% Watchers
  {
    Birthing, lambing, business deals, new journeys, new settlements and death.
  }% Guild activities

  Almost every village has one, and few have more (except when they take on an apprentice.
  Everyone who wants to start something new goes to the doula -- whether starting a business venture, asking someone to marry them, or just travelling between settlements, people request the doula's blessings.
  They frequent towns less often, as the demand on their attention rises rapidly once people know a doula is available.
  Despite the high demand for their attention, they never charge much -- everyone knows they should, by rights, given their blessings in return for a scarf, or a little meat.
  If people see them charging much more, they generally accuse the doula in question of going sour, and begin to refer to her as a `witch'.

  \paragraph{Rank \& Structure}
  stay carefully hidden from public view.
  A few doula know real witchcraft, but those who do won't display it, so those who don't can hint that they have it.
  When one casts a soothsaying spell, she tells others.
  And almost nobody can really track how effective a blessing has been.

  \begin{description}
    \item[Apprentices]
    serve for as long as it takes to pick up the basics -- often until their teacher has died (which means they get to keep the house).
    \item[Doula]
    care for the young, assist with births, and wish people well on new ventures; and whether or not their blessings truly change people's fate; or their predictions really see the future, nobody can tell.
    \item[Wayfinders]
    have sufficient magical abilities that they walk fearlessly through even the deepest parts of the forest.
    Wardens pay them handsomely (but rarely in direct coinage) to seek out areas in the forest fit to built a walled village.

    While the doula never like to reveal their rank, seeing an old woman walk out of the green darkness, where even the \glspl{guard} fear to enter, with twigs clogging up her hair, quickly gives the game away.
  \end{description}

\guild{Wrecan}% Name
  {Hatred}% Death
  {When men fight, when hunger threatens and temper flairs, she comes for souls.
  Within her realm, everyone must pick a side, and every side gains new enemies.
  She watches over wars with contempt for both sides, then claims the souls of the dead.}% God
  {Violence, banditry, bigotry, and blood feuds}% Domain
  {Meditation, camaraderie, reconciliation, diplomacy, and quality armour}% Defence
  {Armourers}% Watchers
  {Creating armour, treaties, and reconciling local disagreements}% Guild activities

Anyone who works with leather or metal might join the guild, but that's not usually how they start.
Families with bad-tempered, narcissistic, and self-righteous children send them to the Temple of Wrecan to learn self-control, and possibly inner-peace.
Within the temple, they learn that aggressively striking iron only makes mangled iron -- to really forge something, a metallurgist must strike solidly, reliably, and with precision.

Apprentices must negotiate their own departure by showing that lessons have been learnt, and by selling at least a few pieces (while giving the guild its cut).
Anyone who becomes cynical and contemptuous of the long process finds themself stuck crafting for years.

Those few with the grit to stay become master craftsmen.

\paragraph{Rand \& Structure}
displays itself in two distinct ways.

\begin{description}
  \item[Crafters]
  form the entire guild.
  Masters produce good quality armour, and sell it for a high price, but even the least skilled among the armourers earn well enough.
  \item[Negotiators]
  usually serve for a season or two, before returning to proper work.
  It doesn't pay well (or at all, sometimes), but at least the guild can continue selling while both sides live.

  While nobody enjoys this job, many gain a reputation for their excellent mediation skills, and find themselves forced into fulfilling the demand, either by various plaintiffs, or the rest of the Temple of Wrecan.
\end{description}

People value the mediation offered by the Temple of Wrecan for two reasons.
Firstly, they tend to make impartial rulings.
Secondly, nobody wants to visit the Temple of Wardens to receive an official ruling.

Common people often make an agreement to accept whatever the local armourer says, and give him a small payment as thanks for listening (both sides pay equally at all times).
Wardens pay substantially more, as they often require the armourers to travel to distant towns in order to help them calm a rival (or lull them into a false sense of security).

Despite their calling, they never profit from war.
People need enough armour just to survive the forests -- if they start killing each other as well, entire villages could disappear, leaving them with less demand overall.

\end{multicols}

\chapter{Strange Growths}
\label{growths}

\section[Vegetation]{Vegetation~\glsentrysymbol{plant}}
\label{vegetation}

\begin{multicols}{2}
\renewcommand\npcsymbol{\glsentrysymbol{plant}}

\subsection{Fungi}
\label{fungi}
\setcounter{encnum}{7}

\funkyPlant[Air]{bedshroom}%
  {}%
  {}%

\funkyPlant{dryadsKiss}%
  {}%
  {}%

\funkyPlant{Fly Flowers}%
  {fly_flowers}%
  {take their name from the swarms of flies that surround them at all times.
  The stench doesn't wash off, and anyone who comes in contact with this mucus will attract enough insects to make hearing speech difficult.}%

\funkyPlant[Fire]{glowshroom}%
  {}%
  {}

\funkyPlant[Earth]{marchingMushroom}%
  {}%
  {}


\subsection{Forest Plants}
\index{Plants}
\label{plants}
\setcounter{encnum}{0}

\funkyPlant{Bedleaves}%
  {bedleaves}%
  {are massive plants appear a little like gunnera, but larger.
  Anyone sufficiently small (\gls{weight} 5 or less) can rest on them overnight and heal an additional~\gls{ep}.}%

\funkyPlant{Bloodwood}%
  {bloodwood}%
  {sap gives a scent that makes blood rush.
  Refined, this sap is considered a potent aphrodisiac, so \glspl{warden} past their prime pay well for it.
  Unfortunately, the sap is secreted only during the harshest cold snaps, 
  and tends to attract and irritate local wildlife, making procuring it a risky endeavour.
  }

\funkyPlant{disgnome}%
  {}%
  {}

\funkyPlant{horseDrops}%
  {}%
  {}

\funkyPlant{Mage Oak}%
  {mage_oak}%
  {roots tap into some mana source deep underground.
    It restores 3 \glsentrylongpl{mp} for every \gls{interval} spent near it.
    The tree will also repeat every spell cast near it, with random targets.

    Proper preparation of a single branch makes a dust which provides 3~\glspl{mp} to \emph{everyone} across \pgls{area}.
    However, the tree can die easily.
    The roll is \roll{Dexterity}{Survival} at \tn[10].
    Failure kills the tree, at which point it leeches all mana in the \gls{area}.
  }%

\funkyPlant{thornyThicket}%
  {}%
  {}%

\subsection{Wetland Plants}
\setcounter{encnum}{0}

\funkyPlant{Dirge Fruit}%
  {dirgeFruit}%
  {produces tasty tomato-like plants on a vine.
    After $1D6$~\glspl{interval}, the eater starts to hear a screeching sound, similar to tinnitus.
    The sound grows louder and louder (though nobody else can hear it), and adds a -1 Penalty to rolls to hear things.

    Every \gls{interval} the `noise' becomes louder and the Penalty increases by 1 step.
    At -3, the target cannot sleep any more.
    At -6, the target becomes effectively deaf, as they can hear nothing else.

    A cure requires an Air \gls{ingredient}, mixed with hard food and crushed violently (\roll{Strength}{Medicine}, \tn[10]).}%


\funkyPlant{Dreameater Moss}%
  {dreameater_moss}%
  {joins the dreams of anyone who sleeps on it.
    The \gls{witch} who created it still lives in those dreams, and wanders through them, eating the minds of anyone they find.
    It provides a powerful but dangerous way to communicate across long distances.

    The sleeper can roll \roll{Wits}{Empathy} to find someone they know, or roll \roll{Wits}{Deceit} to wake up.
    The \gls{tn} starts at $7 + 1D6$, and every \gls{interval} adjusts by $1D6 - 3$.

    Every failure means a roll on a random `random table' -- flip through the first book to hand, select a random table to roll with, then add that element to the dream.
    If the sleeper has already encountered this result in their waking life, memories of the encounter invade the dream.
    If these are good memories, the \gls{tn} decreases by 1.
    If bad, the \gls{tn} increases by 1.
  }

\funkyPlant[Air]{Seekmist}%
  {seekmist}%
  {sits passively, filling up with dust, waiting for a seed.
  The seeds are sticky, and float on the wind until they latch onto an animal (or traveller).
  One the carrier comes within range -- poof!
  The dust spreads everywhere, and the seeds have been properly joined.
  The dust causes nausea, and sometimes hallucinations.

  If properly prepared, they become a wicked poison, which inflicts the disease Guardbane (see \autopageref{Guardbane}).
  Preparation requires an \roll{Intelligence}{Medicine} roll (\tn[12]).
    }

\funkyPlant{Slowburn Ivy}%
  {slowburnIvy}%
  {does nothing for most of the year, but over \gls{cFive}, it spreads its seed by dumping large amounts in rivers.
  Anyone eating water contaminated with this ivy's seeds gains $1D6$ \glspl{ep}, but only after $1D6$~\glspl{interval}.
  The slow release always makes the cause difficult to track down.}

\funkyPlant[Air]{Screechmoss}%
  {screechMoss}%
  {screeches like a starving cat when stepped on.
  They tend to form a symbiotic relationship with predators, who always respond to the squeaks by investigating the source, or laying traps around the thickets.
  The moss then feeds off the blood of any victims.
  Nobles love to encourage screeching moss to grow around their estates, to protect against unwanted intruders.
  }

\funkyPlant{Whistling Cane}%
  {whistlingCane}%
  {grows in swamps and play an eerie tune on windy days.
  Inexperienced travellers might stray from their path, into the dangerous bog, looking for the source of the sound.

  They can be crafted into musical instruments, but they're never popular, since their sound is always sorrowful.}

\subsection{Other Plants}
\setcounter{encnum}{8}

\funkyPlant{Downroot}%
  {downroot}%
  {gives +2 \glspl{hp} after ingestion and a night's rest, and -1 to all \glspl{attribute} for the next three days.
  }

\funkyPlant[Earth]{Liskshine}%
  {liskShine}%
  {grows on the backs of \glspl{basilisk}, and absorbs the beast's overwhelming stench.
  When it grows large enough, small clumps of it fall off.
  A rare find, and also a warning that \pgls{basilisk} is nearby.
    }

\funkyPlant{Uproot}%
  {uproot}%
  {gives you +1 Strength for \pgls{interval} after cooking to the perfect, golden-green.
  Each \gls{interval} thereafter, the player rolls $1D3$.
  On a `1', the effect stops, on any other number the character loses 1~\gls{hp}, and the effect continues.
  }

\end{multicols}

\needspace{8\baselineskip}
\section[Diseases]{Diseases~\glssymbol{eldren}}
\label{diseases}
\index{Diseases}
\setcounter{encnum}{0}

\begin{multicols}{2}
\renewcommand\npcsymbol{\glssymbol{eldren}}

\noindent
\Gls{fenestra} does not have many illnesses.
Those which exist have a little magic to them, and so the cure often demands magical \glspl{ingredient}.

Diseases can be cured with an \roll{Intelligence}{Medicine} roll, at the \gls{tn} stated, along with one \gls{ingredient} of the correct elemental Sphere.
If the preparation goes awry, new \glspl{ingredient} must be procured for a cure.

\disease{Breathrot}%
  {makes breath foul, as the lungs slowly fill with stagnant air.
    One affected can never have less than 1~\gls{ep}.
    Every day this effect increases by~1.
  }%
  {Air}%
  {10}

\disease{Corpse Hands}%
  {starts at the fingertips, then hands, and finally a whole arms dries up and become stiff.

    The character incurs a -1 Penalty to Dexterity.
    Every 3~\glspl{interval}, the Penalty increases by~1 until -6, where they stand, paralysed for \pgls{interval}, before the effect begins to reverse, one \gls{interval} at a time.}%
  {Water}%
  {7}

\disease{Guardbane}%
  {makes legs and arms feel too heavy to move, as the character suffers a -1~Penalty to Strength and Speed.
  Every day, the higher of the two gains an additional -1~Penalty, and if they are equal, the Strength~Penalty increases by~1.
  If either Penalty reaches~-5, the character dies.}%
  {Earth}%
  {5}

\disease{Mindflash Syndrome}%
  {makes the sufferer's mind becomes restless, keeping constant watch but with clouding judgement.

    -1 Intelligence, but +1 Wits; no sleep is possible.
    Every \gls{interval} Intelligence worsens by~1, and every 2~\glspl{interval} Wits increases by~1, up to their racial maximums.%
    \exRef{stories}{Stories}{racial_limits}

    Paranoia and hallucinations are not uncommon after a while.
    Each \gls{interval} the player should roll \roll{Wits}{Vigilance} to notice danger.
    If some danger is present, the character will likely spot it, but if there is nothing dangerous to see, the \gls{gm} should concoct a plausible danger for the character to hallucinate.}
  {Water and Fate}%
  {8}

\disease{Spychoke}%
  {starts and the throat feels so tight that the character can only whisper.
  Talking loudly inflicts so much pain, that they suffer \pgls{ep}.}%
  {Fate}%
  {9}

\disease{Torpid Flesh}%
  {turns the flesh turns pinkish-pale, then flaky.
  Whenever the character spends more than 3~\glspl{ap} in a single round, they gain \pgls{ep} as the skin splits and hair begins to fall out.}%
  {Fire}%
  {9}

\disease{Thinblood}%
  {results in blood-loss every time the character pushes themselves too far, from sneezing, pissing, or crying blood.
  Every time the character would lose \pgls{ep}, they lose \pgls{hp} instead.
  Anyone standing next to them has a $\frac{1}{\epsdice{6}}$ chance of contracting thinblood as the thin, misty, blood permeates the air.

  Every morning, the character rolls $2D6 +$ \glspl{hp}.
  Rolling 12 or more means they will be cured by the next morning.}%
  {Water}%
  {10}

\end{multicols}

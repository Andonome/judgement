\chapter{Strange Growths}
\label{growths}

\section{Plants}
\label{plants}

\begin{multicols}{2}

\noindent
\Gls{fenestra}'s plants hold a thousand uses and dangers.

\newcommand\funkyPlant[6]{
  \renewcommand\npcsymbol{\glsentrytext{plant}}
  \subsubsection{#1}
  \label{#2}
  \index{#1 \glsentrytext{plant}}

  #3

  \begin{description}
    \item[\flwr~Preparation:]
    #4 (\tn[#5])
    \item[\failStar~On failure:] #6
  \end{description}
}

\subsubsection{Bedleaves}
\label{bedleaves}
These massive plants appear a little like gunnera, but larger.
Anyone sufficiently small (\gls{weight} 5 or less) can rest on them overnight and heal an additional \gls{fatigue}.

\funkyPlant{Bloodwood}%
  {bloodwood}%
  {Just the scent of sap extracted from the Bloodwood tree makes your blood rush.
  Refined, this sap is considered a potent aphrodisiac, and many a noble past their prime is willing to pay well for it.
  Unfortunately, the sap is secreted only during the harshest winters, 
  and tends to attract and irritate local wildlife, making procuring it a risky endeavour.
  }
  {\roll{Intelligence}{Medicine}}%
  {10}%
  {If allowed to rot, it makes the imbiber blind for $1D6^2$ days.}

\funkyPlant{Downroot}%
  {downroot}%
  {Eating the root gives +2 \glspl{hp} after a night's rest, and -1 to all Attributes for the next three days.
  }
  %
  {\roll{Intelligence}{Medicine}}%
  {12}%
  {only the penalty applies.}%

\funkyPlant{Dreameater Moss}%
  {dreameater_moss}%
  {
    Created by some forgotten witch, this moss joins the dreams of anyone who sleeps on it.
    The sorcerer still lives in those dreams, and wanders through them, eating the minds of anyone they find.
    It provides a powerful but dangerous way to communicate across long distances.

    The sleeper can roll \roll{Wits}{Empathy} to find someone they know, or roll \roll{Wits}{Deceit} to wake up.
    The \gls{tn} starts at $7 + 1D6$, and every \gls{interval} adjusts by $1D6 - 3$.

    Every failure means a roll on a random `random table' -- flip through the first book to hand, select a random table to roll with, then add that element to the dream.
    If the sleeper has already encountered this result, the memories invade the dream.
    If these are good memories, the \gls{tn} decreases by 1.
    If bad, the \gls{tn} increases.
  }
  %
  {None}%
  {varies}%
  {sleep until a slow death}%

\subsubsection{Dryad's Kiss}
\label{dryads_kiss}

Although edible, this mushroom makes you very gullible.
Take a -2 penalty to all Deceit and Empathy checks for a day.

\subsubsection{Fly Flowers}
\label{fly_flowers}

Named after swarms of flies that surround them at all times, these vibrant flowers secrete sticky, pungent mucus that attracts swarms of insects.
The stench is really hard to wash off, and anyone who comes in contact with this mucus will attract large numbers of insects at least for the rest of the day.

\funkyPlant{Glowshroom}%
  {glowshroom}%
  {
   These subterranean fungi give off a soft, faint light, but only in complete darkness.
   Dwarves sometimes use them instead of torches, even though the light is dimmer.

   Ingesting these plants can be deadly.
   While completely healthy to eat, once they mix with stomach acids, they begin to glow.
   This can turn someone into a sudden target in the dark, as their stomach begins to shine.

   Proper preparation of a full patch of these mushrooms will yield \pgls{boon} of the Fire Sphere.
  \index[mana]{Fire!Glowshroom}
    }
  {\roll{Intelligence}{Medicine}}%
  {12}%
  {all fires in the \gls{area} snuff out.}%

\funkyPlant{Liskshine}%
  {molted_basilisk}%
  {
  Type of lichen that grows on the backs of basilisks, it also absorbs and retains the beast's overwhelming stench.
  When it grows large enough, small clumps of it fall off.
  A rare find, and also a warning that a basilisk is nearby.

  Once spread thin, and left in the snow for a week, the resulting dust becomes a powerful Earth \gls{boon}.
  \index[mana]{Earth!Liskshine}
    }
{\roll{Dexterity}{Medicine}}%
{14}%
{No effect.}%

\funkyPlant{Mage Oak}%
  {mage_oak}%
  {
    Roots of this tree tap into a mana lake deep underground.
    It restores 3 \glsentrylongpl{mp} for every \gls{interval} spent near it.
    The tree will also repeat every spell cast near it, with random targets.

    Proper preparation of a single branch makes a dust which provides 3 \glspl{mp} to \emph{everyone} across \pgls{area}.
    However, the tree can die easily.
    Each harvesting has a 1 in 6 chance of killing the tree, at which point it leeches all mana in the \gls{area}.
  }%
  {\roll{Charisma}{Medicine}}%
  {13}%
  {the dust produced halts regeneration of all Mana.
  Afflicted targets can roll \roll{Charisma}{Empathy} (\tn[10]) each day to request the world listen to them again.}


\funkyPlant{Marching Mushroom}%
  {marching_mushroom}%
  {
  Chewing on this tough mushroom relieves one of fatigue, but slows both their body and mind.
    Ignore \gls{fatigue} penalties for a day, but with -1 penalty to Dexterity, Speed, Intelligence and Wits.

    Those with the right workshop, ingredients, patience and timing, can turn this into \pgls{boon} for the Fate Sphere.
  \index[mana]{Fate!Marching Mushroom}
    }
{\roll{Intelligence}{Medicine}}%
{16}%
{Ingesting the fungus inflicts the penalties, but no benefits.}%

\funkyPlant{Screeching Moss}%
  {screeching_moss}%
  {
  It squeaks when stepped on.
  They tend to form a symbiotic relationship with predators, who always respond to the squeaks by investigating the source, or laying traps around the thickets.
  The moss then feeds off the blood of any victims.
  Nobles love to encourage screeching moss to grow around their estates, to protect against unwanted intruders.

  Properly prepared, the moss provides \pgls{boon} for Air spells.
  \index[mana]{Air!Screeching Moss}
    }
  {\roll{Intelligence}{Medicine}}%
  {12}%
  {The air turns foul, everyone in the \gls{area} loses \pgls{fatigue}.}%

\funkyPlant{Seekmist}%
  {seekmist}%
  {
  Seekmists sit passively, filling up with dust, waiting for a seed.
  The seeds are sticky, and float on the wind until they latch onto an animal (or traveller).
  One the carrier comes within range -- poof!
  The dust spreads everywhere, and the seeds have been properly joined.
  The dust causes nausea, and sometimes hallucinations.

  If properly prepared, they become a wicked poison, which inflicts the disease: Guardbane (see \autopageref{Guardbane}).
    }
  {\roll{Intelligence}{Medicine}}%
  {12}%
  {the nasty potion explodes, inflicting Guardbane on all in the \gls{area}.}%

\subsubsection{Thorny Thickets}
\label{thorny_thickets}

Regular thorny bushes create serious complications for anyone moving through them.
They reduce running speed by 5 \glspl{step} per round, and moving through them requires a \roll{Dexterity}{Wyldcrafting} \gls{quickaction} (\tn[10]), or the would-be runner receives \pgls{fatigue} and loses all momentum, as dozens of tiny spikes scratch and pierce them.

Wearing sufficiently full armour makes one immune to the cuts, but will not stop the movement problems.

\funkyPlant{Uproot}%
  {uproot}%
  {
  Eating the root (after preparation) gives you +1 Strength for \pgls{interval}.
  Each \gls{interval} thereafter, the player rolls $1D3$.
  On a `1', the effect stops, on any other number the character loses 1 \gls{hp}, and the effect continues.
    }
{\roll{Intelligence}{Medicine}}%
{14}%
{No Strength Bonus.}%

\funkyPlant{Whistling Cane}%
  {whistlingCane}%
  {
  These swamp plants play an eerie tune on windy days.
  Inexperienced travellers might stray from their path, into the dangerous bog, looking for the source of the sound.

  They can be crafted into musical instruments, but they're never popular, since their sound is always sorrowful.
  }
{\roll{Intelligence}{Crafts}}%
{14}%
{Screechy music}%

\end{multicols}

\section{Diseases}
\label{diseases}
\index{Diseases}

\newcommand\disease[5][\roll{Intelligence}{Medicine}]{
  \subsubsection{#2}
  \index{#2 \glsentrytext{sickness}}
  \label{#2}
  #3
  \needspace{2em}
  \begin{description}
    \item[\flwr~\Glspl{ingredient}] #4
    \item[Roll] #1 (\tn[#5])
  \end{description}
}

\begin{multicols}{2}

\noindent
\Gls{fenestra} does not have many illnesses.
Those few which exist have a little magic to them, and so the cure often demands magical \glspl{ingredient}, such as those used to make \pgls{boon}.

For example, `Breathrot' demands an Air \gls{ingredient} for a cure, properly prepared.
If the preparation goes awry, new \glspl{ingredient} must be procured for a cure.

\disease{Breathrot}%
  {
    The breath becomes foul, as their lungs slowly fill with stagnant air.
    One affected can never have less than 1~\gls{fatigue}.
    Every day this effect increases by~1.
  }%
  {Air}%
  {10}

\disease{Corpse Hands}%
  {
    Those unfortunate who ail from Corpse Hands this will see first their fingertips, then their hands, and finally whole arms, dry up and become stiff.

    The character incurs a -1 penalty to Dexterity.
    Every 3~\glspl{interval}, the effect increases by~1 until reaching -6, where they stand, paralysed for \pgls{interval}, before the effect begins to reverse, one \gls{interval} at a time.
  }%
  {Water}%
  {7}

\disease{Mindflash Syndrome}%
  {The sufferer's mind becomes restless, keeping constant watch but with clouding judgement.

    -1 Intelligence, but +1 Wits; no sleep is possible.
    Every \gls{interval} Intelligence worsens by~1, and every 2~\glspl{interval} Wits increases by~1, up to their racial maximums.%
    \exRef{stories}{Stories}{racial_limits}
    Paranoia and hallucinations are not uncommon after a while.
    If the character ever fails a check to notice something, the \gls{gm} will report the failure as success, concocting a plausible danger which the character imagines.}%
  {Water and Fate}%
  {8}

\disease{Guardbane}%
  {Legs and arms feel too heavy to move, as the character suffers a -1~Penalty to Strength and Speed.
  Every \gls{interval}, the higher of the two gains an additional -1 penalty, and if they are equal, the Strength penalty increases by~1.
  If either penalty reaches -5, the character dies.}%
  {Earth}%
  {5}

\disease{Spychoke}%
  {The throat feels so tight that the character can only whisper.
  Talking loudly inflicts so much pain, that they suffer \pgls{fatigue}.}%
  {Fate}%
  {9}

\disease{Torpid Flesh}%
  {The flesh turns pinkish-pale, then flaky.
  Whenever the character spends more than 3~\glspl{ap} in a single round, they gain \pgls{fatigue} as the skin splits.}%
  {Fire}%
  {9}

\end{multicols}

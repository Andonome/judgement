\begin{multicols}{2}

\subsection{Hey there,}

Just checking you haven't got lost\ldots

This book helps \glspl{gm} set up a game of BIND, set in \gls{fenestra}.
Well, actually it's never done that before, because I'm just writing it now.
So maybe you're the first to use it raw?
Each chapter came from something I needed, or just from ideas which seemed fun:

\begin{itemize}
  \item
  In \autoref{cosmology}, you can briefly read how humans see their place in \glsentrytext{fenestra}, as small creatures with delicate lives, and how the guild-temples fight against the gods.
  \item
  In many ways, a GM's character sheet is the map, so \autoref{yourMap} has a process to create your map.
  Whether or not you make that map, the lists of potential circumstances and people make an easy way to see what people here consider normal.
  \item
  The random encounters in \autoref{encounters} also describe the world in bullet-point formats, with a focus on what the \glspl{pc} can get up to.
  \item
  Over \autoref{fiends} to \autoref{growths}, the book lists out the various dangers of the great, wild, world, and provides answers to age-old questions, such as `where do basilisks hibernate over the cold seasons?', and `what do griffins eat aside from people?'.
  \item
  Finally, \autoref{judgement} goes over the best advice I can give to organize a fun BIND session.

  You won't find much on `how to be a GM' here.
  Others have written \href{https://thealexandrian.net/so-you-want-to-be-a-game-master}{better books} on the subject, but I can at least claim to be the sole authority on being a \gls{gm} in BIND.

\end{itemize}

You'll find some references to the core rules, but you can use this book with nothing but the basic rules in an adventure module, if you prefer.

\iftoggle{stories}%
  {If your players are using the \textit{Book of Stories}, you should probably have a quick read-through of the \glspl{storypoint} system (\autopageref{stories} of that book), just so you know what kind of random allies they might bring into play.}%
  {}

\columnbreak

\subsection{Massive Thanks}

\paragraph{To Matija,}
for help proof-reading, and for many of the strange plants of \gls{fenestra}.

\subsubsection{To the artists,}
\paragraph{Irina,}
who drew the gods on pages
\pageref{Irina/forest}, 
and
\pageref{Irina/fury}.

\paragraph{Studio DA,}
for myriad monsters,
pages
\pageref{Studio_DA/chitincrawler}, 
\pageref{Studio_DA/woodspy},
and
\pageref{Studio_DA/jelly}.

\paragraph{The Lady of Hats,}
for her beasts,
pages
%\iftoggle{stories}{}{%
%  \pageref{loh/gnoll}, 
%}%
\pageref{loh/dryad}, 
and
\pageref{loh/dragon}.

\paragraph{Nelness,}
for the abandoned city,
page
\pageref{Nelness/city}
(find her on \href{https://www.fiverr.com/nelnes}{fiverr.com}).

\subsection*{Licence}

BIND is open source, and available under the {\tt GNU General Public License 3} or (at your option) any later version.

You have full access to all the source files, including art, and the right to change anything and share those changes with others.
BIND will never have any `house rules', because anyone can place their alterations directly into the book and make their rules official.

\end{multicols}

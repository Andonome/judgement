\begin{multicols}{2}

\subsection{Hey there,}

Just checking you haven't got lost\ldots

This book helps \glspl{gm} set up a game of BIND, set in \gls{fenestra}.
Well, actually it's never done that before, because I'm just writing it now.
So maybe you're the first?
But anyway, it has everything I wish `GMing' books had.

\begin{itemize}
  \item
  The `Guilded Temples', introduces the strange institutions which each claim a divine monopoly, based on their mission to protect humanity from a wrathful god.
  The Weavers Guild make clothes to ward against the cold, so people call them the `Temple of Frost'.
  The \gls{guard} protect people from the forest, so they work for the `Temple of Beasts'.
  \item
  It once had a map, but that felt too restrictive, so I replaced the map with a procedure to generate maps.
  This way, I can show you what seems normal in \gls{fenestra}, without nailing anything down.
  \item
  The random encounters work the same way.
  It's basically a way to describe a world with bullet-point lists.
  \item
  The next few chapters have creatures and challenges of all sizes, with notes on their ecology.

  `Where do basilisks hibernate over the cold seasons?'

  `What do griffins eat aside from people?'
  \item
  This last chapter has a few tips that have really sped up my games, or helped organize an open world with minimal prep.

  You won't find much on `how to be a GM' here.
  Others have written \href{https://thealexandrian.net/so-you-want-to-be-a-game-master}{better books} on the subject, but I can at least claim to be the sole authority on being a \gls{gm} in BIND.

\end{itemize}

You'll find some references to the core rules, but you can use this book with nothing but the basic rules in an adventure module, if you prefer.

\iftoggle{stories}%
  {If your players are using the \textit{Book of Stories}, you should probably have a quick read-through of the \glspl{storypoint} system (\autopageref{stories} of that book), just so you know what kind of random allies they might bring into play.}%
  {}

\columnbreak

\subsection{Massive Thanks}

\paragraph{To Matija,}
for help proof-reading, and for many of the strange plants of \gls{fenestra}.

\subsubsection{To the artists,}
\paragraph{Irina,}
who drew the gods on pages
\pageref{Irina/forest}, 
and
\pageref{Irina/fury}.

\paragraph{Studio DA,}
for myriad monsters,
pages
\pageref{Studio_DA/chitincrawler}, 
\pageref{Studio_DA/woodspy},
and
\pageref{Studio_DA/jelly}.

\paragraph{The Lady of Hats,}
for her beasts,
pages
\iftoggle{stories}{}{%
  \pageref{loh/gnoll}, 
}%
\pageref{loh/dryad}, 
and
\pageref{loh/dragon}.

\paragraph{Nelness,}
for the abandoned city,
page
\pageref{Nelness/city}
(find her on \href{https://www.fiverr.com/nelnes}{fiverr.com}).

\subsection*{Licence}

BIND is open source, and available under the {\tt GNU General Public License 3} or (at your option) any later version.

You have full access to all the source files, including art, and the right to change anything and share those changes with others.
BIND will never have any `house rules', because anyone can place their alterations directly into the book and make their rules official.

\end{multicols}

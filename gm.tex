\chapter[Bastion of Judgement]{Judgement}
\label{judgement}

\section{Basic Prep \& Play}

\begin{multicols}{2}

\subsection{Pre-Game Prep}

Remember the basics.

\begin{itemize}
  \item
  Pencils
  \item
  Rubber (or `eraser' for Americans)
  \item
  $4D6$ per player, with multiple colours so players can distinguish Damage dice.
  \iftoggle{stories}{
    \item
    A boat-load of character sheets
  }{
    Some pre-made characters, including spares
  } (the game can be lethal).
  \item
  Some scenes for the session, in one form or another.
  \item
  Lots of coins
  \item
  The \gls{gm} sheet, to track whatever comes up.
\end{itemize}

As a \gls{gm}, it's always good to have at least 3 different types of coins to keep track of \glspl{ap}.
Let's say you're orchestrating a battle with a hobgoblin leader, some hobgoblin troops and a goblin spellcaster.
Assign each one a coin and make a little mnemonic -- the spellcaster has dark magic so it gets the little copper penny.
The hobgoblins get the silver coin to represent their use of weapons, and the largest coin goes to the hobgoblin leader.
Don't worry about the players' \glsentrylongpl{ap} -- they'll keep track of their own characters.

Coins can also be used to keep track of \gls{fp} and \glspl{fatigue} as they change so often.
It'll help cut down on wear to the character sheet, and lets you see how much Damage and \glspl{fatigue} the \glspl{pc} have at a glance.

\subsubsection{The \Glsfmttext{gm} Sheet}

You should find a \gls{gm} `character sheet' along with this book, to help track information.%
\footnote{If you don't have a copy, scan the QR code at the back of the book.}

You can put a name to your campaign at the top box, or simply use it to designate a `theme', like `fire', `karma', or `ancestry', as a reminder to look out for opportunities to hark on the theme often.

`Area' and `season' should be updated at the end of each session, especially as the troupe may move about.
And the `encounters' box lets you note down a couple of encounters, with ideas on the specifics.

Long-standing \glspl{npc} should also have their \glspl{fp} listed next to the character, as \glspl{npc} gain \glspl{fp} at the end of each \gls{interval}.
This helps beloved \glspl{npc} stay alive, as well as adding a little extra gravitas to any antagonists who encounter the \glspl{pc} multiple times.

Notice the little circles next to the \gls{ap} counter.
This lets you use the counter for two purposes -- \glspl{ap}, and tracking future \glspl{interval}.
You can note down when the next encounter check is, when a \gls{pc}'s disease will flair up, when a spell will wear off, and anything else you can make a note of in a single word or symbol.

\subsubsection{Plan, or Don't}

You might have noticed that the system pushes towards random narrative elements quite a lot.
This push also comes from smaller elements, like the fact that an encounter roll just needs $3D6$, and shows weather, a monster, how many, when the encounter occurs, \textit{et c}., so you can determine the encounter mid-game, on the fly, and just roll with it.

You don't have to plan much to have an emergent narrative, though if you prefer starting out with a lot written down, download a copy of
\href{https://gitlab.com/bindrpg/oneshot/-/jobs/artifacts/master/raw/hardcore_horde_escape.pdf?job=build}{Escape from the Horde}, and the
\href{https://gitlab.com/bindrpg/oneshot/-/jobs/artifacts/master/raw/hardcore_handouts.pdf?job=build}{handouts} to go with it. 

\subsubsection{The Opening Line}

The opening line of any story can colour so much of the rest.
I never really know how to start, but recently I've focussed on the question of encounters.

\begin{speechtext}
  Have you ever wondered how people would survive in a world of random encounters, with farmers living where monsters just wander about?
\end{speechtext}

People have opinions about this question.
They have ideas about whether we should be asking about realism, and ideas on how such a world might work.
And before you know it, they're talking about \gls{fenestra}.

This gives you a chance to explain the \gls{guard}, and hand out some character sheets.

\subsection{General Notes}

\subsubsection{Describe Back-to-Front}

If you start a description with a bandit leader, at least one player will respond and interrupt the rest of your description.
You won't be able to tell the players about the armour with that guild's insignia, or the potions on the table.
The players are not the problem here -- the problem lies with the order of the information.

\begin{enumerate}
  \item
  Start every description with the background scenery.
  \item
  Mention some details.
  \item
  Give the \glspl{pc} something to respond to.
\end{enumerate}

Personally, I like to imagine the scene behind me, as if the players are looking at it.
That way, I can point to a door `\emph{over there}', rather than saying `you see a door in the North-Eastern side of the room'.
If I ever see a player point towards something in the game, it tells me they have engaged with the world.

Putting the background first also makes full descriptions easier.
We all want to put in details like how things smell, and the colour of the flowers, then every \gls{gm} forgets in the moment.
But once the background comes first, any time you draw a blank for a second, those details are what comes up.

\paragraph{In the forest\ldots}

Having these background details also helps focus on what the players see, rather than the bare events.

\begin{boxtext}
  You see four guards in the forest, walking towards you.
\end{boxtext}

\textit{\Large\raggedright `Branches crack all around the forest, always somewhere far away, but all around, as if the forest were snoring.'}

\textit{\large\raggedright `Every second step, your torch helps you navigate, but also alerts everything around you with its light.'}

\textit{`In the distance, another light shines alongside the crunch of feet, then that light goes out, but the feet continue towards you, slower now\ldots.'}

\paragraph{In town\ldots}

In social settings, the surrounding mood often forms the `background', as much as the actual events.
Taking an extra moment to give details about the surroundings also helps give players the kinds of realistic expectations that their characters would have, without a lot of conversations later.

\begin{boxtext}
  A drunk member of the \gls{sunGuard} approaches to ask why you're in town.
\end{boxtext}

\textit{\Large\raggedright `Once the rain fades, the town stinks of rot, misery, and dry throats as the taverns still have no ale.'}

\textit{\large\raggedright `Open window-shutters close a little as you pass, and the closed shutters open a crack.  The town wants to watch.'}

\textit{`A member of the \gls{sunGuard} approaches, somehow drunk, and yells ``long-ears! You got permission to walk my streets?'''}

\subsubsection{Literal Interpretations}

If it's ever unclear how to resolve a situation, the first attempt should always be a strict interpretation of the rules.
For example, if a player says `If I charge round a corner, rather than a straight line, can I still use the Fast Charge knack?', the answer is `yes', because the rules as they stand don't prohibit going round a corner.

No rules will work all of the time, but by following a literal interpretation of the rules whenever possible, players feel better able to predict and navigate the world, and \glspl{gm} do not have to waste so much energy on making on-the-fly rulings.

Broadly, the \gls{gm} should consider themself bound by the rules as much as the players.
A good rule of thumb is to make as few decisions as possible, and let yourself focus on description and planning.

\subsubsection{Handling Money}
I have no idea how much a suit of chainmail, plus helmet, would cost in Middle Ages Europe.
I doubt historians have a solid grasp on all the prices of the time.
The prices here usually assume 1~\gls{cp} = \pounds1, so if you guesstimate that a hand-made hate costs \pounds150, then that's 1~\gls{sp}, 50~\glspl{cp}.

Having \pgls{cp} valued at 100 to \pgls{sp} should let your group `ignore the pennies', if nobody wants to deal with fiscal details.

\subsubsection{Let Players `Ruin' the Mission}

Encounters don't have to play through like you think they will.
If the players flood a labyrinth, cast a fireball at the king, or raise their Life and Mind Spheres so high that every wild animal encounter turns into a pet in a growing army, take a breath, re-examine the situation, and go from there.

Perhaps the labyrinth has a high-point inside which isn't flooded, which at least saves that part of the labyrinth; or perhaps it's flooded forever, and nobody will see that treasure again.
Perhaps the troupe have to become outlaws, and every future adventure has to take this into account.
And even if all those pets feel enamoured with the caster, they don't need to like each other -- maybe they start to fight, or try to kill the other troupe members, but only when they fall down, wounded and weak!

\subsubsection{Recap \& Report}

Having players recap of the previous session can help them remember where their characters left off, catch up any players who could not attend the last session, and help you understand your players' perspective on the game.

If you want an easy excuse, tell the players that the local \gls{jotter}%
\footnote{See \autopageref{jotter}.}
has fallen behind on their paperwork, and wants the troupe to give them a complete report of their last mission.

\subsubsection{Torture}
\index{Torture}

If players torture an \glspl{npc}, have the \gls{npc} give out a false narrative.
If they ask for a location, it sends them somewhere dangerous.
If they ask who's in charge of a conspiracy, they finger a well-known official, or priest.

Even stupid \glspl{npc} can create a basic narrative, so start pulling up enough nonsense that the \glspl{pc} become completely confused!

Any attempt to notice the lies receives a -4 penalty -- it's hard to tell odd behaviour when someone's under perpetual stress!

\subsubsection{You Don't Always Need to Roll}

Get used to saying `the \gls{tn} is 10, so you succeed'.
If the \gls{tn} to open a wedged door is 10, and the \gls{pc} has plenty of time, they can just take a resting action, meaning their minimum roll is `7' -- if the Strength + Crafts Bonus totals +3, that means they succeed.

Similarly, if everyone wants to help spreading a nasty rumour around town, and you notice \pgls{pc} has a \roll{Charisma}{Deceit} score of $+4$, while another has $+3$, the total for the Team Roll would be $+6$.
If the \gls{tn} is `8', tell them they succeed after \pgls{interval}.
If the \gls{tn} is `13', tell them they succeed after spending a few days repeatedly spreading the rumour.
They should only roll when they might fail.

Once players feel emboldened to just say `okay, well at \gls{tn} 9 I succeed', the game becomes just a little faster.

\subsubsection{Everything has a \glsfmtlong{tn}}

Easy actions still have a \glsentrylong{tn}. 
Getting over a \gls{village} wall might have a \gls{tn} of 3, which means that you can normally ignore the roll, but may still find times characters will struggle.
Let's say a particularly rotund and clumsy \gls{pc} has been drinking all day, and now wants to join the troupe in sneaking over \pgls{village} wall.

\begin{speechtext}
  Give me a \roll{Speed}{Athletics} roll, at \tn[3] to get over the wall.
  Anyone who took part in the drinking contest gets a -2 penalty.
\end{speechtext}

Normally, rolling for such a low \gls{tn} would be a waste of time.
But if one member of the troupe has a basic Speed Penalty of -2, then the player needs a \gls{natural} of 7 to \emph{tie} with the action.

\subsubsection{Eyeballs Always Work}

Most of your job involves telling people what their \gls{pc} can see, so avoid asking for any kind of roll to notice something.
These rolls cut the flow of any narrative.
Use rolls to notice something only when someone actively wants to hide something from the \glspl{pc}, such as ambushes, or encrypted writing.

If you don't want to draw attention to something the \glspl{pc} can sense, then mention it briefly, or speak around it, then tell the players only if one asks for more detail.

\subsubsection{Another Look at Dice}

The dice system can be expressed any number of ways.

\begin{itemize}

  \item
  The player and \gls{gm} each roll 1D6 and add their bonuses.
  If one rolls higher, they win.
  \item
  The player rolls $1D6-1D6$, then adds their own bonus, and subtracts their opponent's bonus.
  Rolling a total of `0' means a tie, a positive score means success, and a negative score means failure.
  \item
  The player rolls 2D6 at a \gls{tn} equal to 7.
  They add their bonus, and subtract the opponent's bonus.

\end{itemize}

These three basic mechanics function exactly the same as each other, and the same as the actual rules.
But looking at the rules in these different ways can help clarify the underlying structure.
We use the rules as actually written, only because players find subtraction slightly more irritating than addition.

\subsubsection{Purging the Poison of the Computers}

We should always feel impressed when computer game RPGs manage to display a small fraction of the freedom, and emergent complexity that original RPGs do.
However, over the last decade I've seen an alarming trend of players who only know RPGs from computer games, and pull very strange habits into the game.
These strange habits always come from inappropriate expectations, and the best medicine for these expectations is usually clearer descriptions from their \gls{gm}.

\begin{boxtext}
  I light my torch, and throw it at the ogre.
\end{boxtext}

This suggestion is laughable, but don't laugh, just clarify.

\begin{speechtext}
  So while the other characters attack for a few rounds, you want your character to go through their backpack to get their tinder box and a torch, then use the tinder box to light the torch.

  Is that right?
\end{speechtext}

That shows the player where their error lies, but also alerts you to the fact that they could not properly see the world their character lives in.

\begin{boxtext}
  If I have properly enchanted the griffin, I want to ride around on it.
  I got a new mount, guys!

  Okay, now I need to go to a town for supplies.
  Can we `fast travel'?
\end{boxtext}

Once again the player cannot fully \emph{feel} the world.

\begin{speechtext}
  You clamber onto the griffin's back, and it flaps anxiously, confused by the feeling of weight on its back.
  Eventually it runs, takes flight, and you both fly towards the setting Sun through heavy gusts of wind.
  It cannot indicate where it wants to take you, any more than you can tell it where you want to go, but en route, you pass over a \gls{village}.
  The archers in the \gls{village} immediately draw back their bows.
  You find gripping onto the flapping creature very difficult.
  Roll \roll{Dexterity}{Flight} to avoid falling.
  
\end{speechtext}

\ldots perhaps that's a little cruel, but \gls{fenestra} is a cruel world.

\subsection{How to Avoid Scheduling Conflicts}

Start in town, end in town.

\subsubsection{Tackling the Root Cause}

Scheduling conflicts plague every RPG session held outside of the Antarctic circle.
This has nothing to do with RPG players, and everything to do with the hobby's unique demands.
People who play football play with whoever turns up, even if it means `five a side'; chess clubs never demand rigid, weekly attendance.
The problem with getting everyone together comes from the demand that everyone must treat the game like work.

Instead of making unrealistic expectations, we can make a very small adjustment to the gaming world -- simply end sessions at a point where characters have reached safety.
This might be in a town, \gls{village}, or \gls{bothy}.
On the next session -- whether the characters all work for the \glspl{guard} or some other organization -- the players who arrived will find their characters sit together, in a bar, \gls{bothy}, or on the road, and can all begin a new mission together.

If you tell everyone from the outset that their session must start and end in town (or some civilized area, where the \glspl{pc} can rest without random encounters), then the players will generally do their best.
Any `dungeons` must become an area they enter quickly, and jump out of once something -- anything -- has been achieved.

Once your game adopts this standard, the headache of getting five adults in a room on one of their free days vanishes.
And every time someone messages to say they can't make the game, you can mark that off as a victory for `parallel time' because the game goes on regardless.

\subsubsection{Rotating Tables}

This also opens the game up to more people.
The gaming table no longer needs to feel `full', because yet another person wants to join the campaign.
Games with revolving players can easily handle a dozen.

And if you -- the \gls{gm} -- find you can't make one week, just ask if anyone else wants to run a one-shot, or play a board-game.
Nobody's irreplaceable, not even \pgls{gm}.

\subsubsection{Proper Time-Keeping}

Keeping a campaign going, where anyone can jump in and out, requires proper time-keeping.%
\footnote{Gygax was right.}
To make this easier, \gls{fenestra}'s time runs in parallel to our own.
Each real-month, \gls{fenestra} begins another season.

Proper time-keeping doesn't just help the campaign -- it helps the night.
Having a proper ending-time means you can signal ahead that the session should come to an end within the hour.

And of course, if a session ends early, this just leaves more time to discuss how the \glspl{pc} will spend their time in town, and what they want to spend their \glspl{xp} on.

\subsubsection{Downtime Flies}

During the game, players have three weeks for a weekly session, so time should have enough elasticity to let them travel, fight, rest, journey onwards, fall victim to a gory trap and return to heal again.

Of course, if the players can't wrap up in time, their characters will have problems.
\Glspl{pc} who don't make it back in time should start their next session with 0 \glspl{fp}, and perhaps an encounter roll from their last location to explain what exactly went wrong.
A reasonable ruling always depends on the exact circumstances.

\end{multicols}

\section{\Glsfmtplural{sq} \& Story Weaving}

\label{sidequests}
\index{Weaving Stories}

\begin{multicols}{2}

\noindent
I have a strategy for writing stories, which I call `story weaving'.
You can think of it as a set of \glspl{sq}, woven together to make a tapestry.

You make a full thread from a series of \glspl{segment}, as long as each \gls{segment} follows the rules:

\begin{enumerate}
  \item
  The \gls{segment} must make sense anywhere within a broad region.
  \item
  The \gls{segment} must make sense within almost any time frame, including weeks after the previous \gls{segment}.
  \item
  The \gls{segment} must not assume any outcome from a previous \gls{segment}.
  \item
  The \gls{segment} must not assume any reactions from the \glspl{pc}.
\end{enumerate}

This example uses the broad areas `Town', `Roads', and `Forest', so the \glspl{segment} starting with `(Town)' should work almost anywhere in a local town.

\end{multicols}

\begin{nametable}[|c|L|L|]{Example 1: The Beast}
  & \textbf{\gls{segment}} & \textbf{Note} \\
  \hline
  \sqr &
  (Town) Villagers approach the troupe, asking them to help slay a beast.
  &
  The `\sqr' symbol means that the \gls{segment} is ready.
  You can put a full `X' through it, once it's done, and mark the next box as ready.
  \\
  \sqn &
  (Roads)
  Scared villagers tell the \glspl{pc} about the strange beast they've seen, and where it went.
  &
  The players should not feel `railroaded' by this.
  The peasants simply gripe because they feel scared, not because the players `have to do the quest'.
  \\
  \sqn &
  (Roads) Another troupe arrive, looking to slay the beast.
  If the \glspl{pc} have killed the beast, this troupe take the credit; otherwise they journey out to kill the beast themselves.
  &
  \setlength{\parindent}{1em}
  If the \glspl{pc} have indeed killed the beast, then we know this \gls{segment} can play out before they reach town, because they must pass through the \glspl{village} in order to reach the town.

  If the \glspl{pc} have not killed the beast, then we can play out this \gls{segment} any time, even weeks later.
  \\
  \sqn &
  (Town)
  A powerful witch arrives and asks about her missing pet (the beast).
  She explains the beast is a peaceful herbivore, who only becomes aggressive when cornered.
  She swears vengeance upon anyone who killed the beast.
  &
  Will the witch swear vengeance upon the \glspl{pc}, or that other troupe?
  \\
  \sqn \squash &
  (Forest) 
  The witch hired a tracker and some sellswords to follow her, and help her kill whoever she thinks killed the beast.
  Everyone stays behind the tracker, waiting for the right time to strike.
  &
  The `\squash' symbol indicates that this \gls{segment} should play out at the same time as another \gls{segment}, or with a random encounter, or both.
  \\
\end{nametable}

\begin{multicols}{2}

\noindent
This whole thread could play out in any number of ways.

\begin{itemize}
  \item
  The \glspl{pc} may see the other troupe promising to help the villagers, and decide they must be good people.
  When the witch attacks them in the last \gls{segment}, they will feel compelled to help.
  \item
  The \glspl{pc} may spot the tracker from the last \gls{segment} while they fight a horde of goblins, specified in another \gls{segment}, from another \gls{sq}.
  Just as they win the battle, the witch arrives, with her sellswords, to finish them off.
  \item
  Perhaps the \glspl{pc} kill the beast themselves, then just shrug their shoulders at the other troupe's claim to killing the beast, but mention their boast in passing to the witch.

  When they find the other troupe in the forest once more, they could spot a tracker following them, say nothing, and let the witch have her vengeance.
  \item
  If the \glspl{pc} go through each \gls{segment} rapidly (going through $Town \rightarrow Roads \times 2 \rightarrow Town \rightarrow Forest$) this may feel like a single, fast-paced adventure.
  \item
  But if they focus on other \glspl{sq}, they may experience every \gls{segment} here as a small diversion to their `main quest'.
\end{itemize}

\paragraph{Example 2: The Paranoid \Glsfmttext{seeker}}

\begin{exampletext}

\Pgls{seeker} (i.e. a `priest of Curiosity') is using his ability to divine the future to capture criminals \emph{before} they commit crimes.

\begin{list}{\sqn}{}

  \item[\sqr\squash]
  (Roads) A local \gls{seeker} offers to tell the troupe their fortunes.

  \item
  (Town) The characters pass by men in stocks who keep shouting that they are all innocent, and were suddenly taken away by various guards after the \gls{seeker} fingered them for a crime.
  Move this encounter back to the Roads.

  \item
  (Roads) A dozen guards are tracking the characters. Repeat.

\end{list}

The characters end up wanted by the \gls{guard}, for crimes which they might commit.

\end{exampletext}

This \glspl{sq} violates a basic rule -- it assumes the troupe won't simply kill the \gls{seeker}.
But that's okay  -- if this happens, we can simply discard the rest of the \gls{sq}.

\paragraph{Random \Glsfmtplural{sq}}

In addition to story-based \glspl{sq}, it's good to give each area a bunch of entirely random encounters.

\begin{list}{\sqn}{}

  \item{(Forest) The troupe find a gnome attempting to sell them gemstones for his trip. Some are real and others are fake.}

  \item{(Forest) A dragon flies overhead.}

  \item
  (Forest) A dead \gls{seeker} lies on the road.
  His books are valuable but should by law be returned to the Temple of Curiosity.

\end{list}

This collection of non-quests serves two functions.
The first is to provide some short encounter when the time calls for it, but without getting the troupe wrapped up in yet another mission.
If you already have five \glspl{sq} happening at the same time, that's probably as much as the players can handle.

The second use is in wrapping up a campaign.
If you have only two more plot-threads you want to wrap up, the rest of the world doesn't need to feel empty -- encounters can continue, but they needn't start more plot-threads.

Putting all the \glspl{sq} together, the events could play out as follows:

\begin{enumerate}
  \item
  \sqr~(Town)
  Villagers ask the troupe to slay a beast.
  \item
  \sqr~(Roads)
  A priest is reading villagers' fortunes.
  One villager is fated to die a horrible death soon, and the other villagers say that the beast will get him, because they saw it the other day.
  The priest asks to read the \glspl{pc}' fortunes.
  \item
  \sqr~(Roads)
  After tracking and injuring the beast, but ultimately fleeing, the \glspl{pc} rest overnight, while the rival troupe moves out to finish the beast off, and take all the glory for themselves.
  \item
  \sqr~(Town)
  An alchemist, staying in the same tavern as the \glspl{pc} asks if they've seen his pet, claiming it would never hurt anyone.
  The \glspl{pc} point him towards the rival troupe, who have already taken all the credit for killing the beast.
  \item
  \sqr~(Town)
  The next day, they see a line of criminals proclaiming they are all innocent.
  \item
  \sqr~(Roads)
  Guards try to arrest them, so they fight, and eventually have to flee into the forest.
  \item
  \sqn~(Forest)
  The old rival troupe attempt to claim the bounty on the \glspl{pc} head, and hunt them down -- but the alchemist appears at the last moment to save them.

\end{enumerate}

\paragraph{Combining \glspl{segment}}
leads to all sorts of shenanigans.
It naturally ties unrelated \glspl{sq} together, and keeps things moving quickly.
It also lets you slip that rumour about X into \pgls{segment} focusses on Y.

\paragraph{Particular Locations}
have their own space -- just note them separately.
Just make sure the various \glspl{segment} of a \gls{sq} only refer to them, but never demand the \glspl{pc} interact with them.

\paragraph{The Story, in Hindsight,}
will look very different from any plan.
\Glspl{pc} often pick up a particular thread, and follow it as `the main plot', so the plot emerges in retrospect, as the players engage with something.

The players will probably think they they have found `the main plot', while the rest are \glspl{sq}.
They are of course, correct, in every meaningful sense.

\subsubsection{Subtle Emphases}

The locations you choose for your \glspl{segment} will decide where a campaign's emphasis lies.
Players will gravitate towards these places, since they contain all the activity, but the players will also remain free to come up with any plans they want.

\begin{itemize}
  \item
  For a town-focussed game of double-dealings, and catching thieves, you could select `Slums', `Guild Halls', and `Citadel'.
  \item
  For a mountainous campaign, based on the underground labyrinth, you could pick `Town', `\glsentrytext{deep}', and `Goblin Realms'.
  \item
  For a free-roaming, wide-ranged campaign with a lot of travel, \glspl{segment} could take place in `Trollgate' (a town), `Ethervale' (the next town), `the Roads', and `Elfwood' (an area of forest, inhabited by elves).
\end{itemize}

Emphasis also comes with length.
To make \pgls{sq} longer, you can add more \glspl{segment}, or simply have it jump between locations for a while.
The more \pgls{sq} jumps between locations, the longer players will take to come across the various \glspl{segment}.

\subsubsection{Writing \Glsfmtplural{sq}}

Consider this standard fantasy plot-hook:

\begin{exampletext}
  Bandits stole an doula's magical item, and she wants the troupe to retrieve it.
\end{exampletext}

Now let's expand with some foreshadowing:

\begin{enumerate}
  \item
  Merchants enter town without their wares, having been robbed.
  \item
  An witch asks the characters to find the items bandits stole from her.
\end{enumerate}

Now let's add a decoy -- the bandits must know \glspl{guard} often come after them, so they can send someone to lead the troupe into a trap.
And finally, we might add another lead to make things interesting.
Perhaps the bandit leader's mother was a doula, and he knows a few tricks.

\begin{list}{\sqn}{}
  \item
  (Town)
  Posters go up for a local sorcerer, wanted `DEAD on sight' (in fact the bandit leader).
  \item
  (Town) 
  Merchants inter town without their wares, having been robbed.
  \item
  (Town)
  The doula asks the characters to find the item bandits stole from her.
  \item
  (Roads)
  A man offers to help the troupe find the bandits (but in fact wants to lead them into a trap).
\end{list}

The trap should be detailed as a separate location.
Or perhaps the `trap' is simply the bandits' lair, and the fake guide intends to warn the bandits of the troupe's approach, which leaves them with little chance of sneaking around to gauge the bandits' numbers.

Any of these \glspl{segment} could combine with others.
In general, I find it's best to put as many elements into a single \gls{interval} as you can fit.

\end{multicols}

\section{Combat}

\begin{multicols}{2}

\subsection{Ordered Initiative \& Hollering}

If your players insist on acting in `order of Initiative', you can help them along by hollering the \gls{ap} count, from highest to lowest.

\begin{speechtext}

``Eight! The gnolls raise their weapons''

``Seven, six! They move forward, bearing their yellowed teeth.''

``Five! Snarls abound as they speed up to a rush.''

\end{speechtext}

Nothing has actually happened by this point, but it sets the scene nicely.

\begin{speechtext}

  ``Five'', one of the players shout.
  ``I'm going at five.
  I move to protect Grotmax.''

  ``Two gnolls go for you, another two go for Grotmax.
  Roll to engage at \tn[11].''

\end{speechtext}

The count continues down quickly at all times, and the count always provides a sense of urgency.
If players don't notice it's their turn when you're shouting, they lose \pgls{ap}.
Do it once, and they'll never make the same mistake again.

\subsection{Speaking}

You may have noticed that speaking comes with \pgls{ap} cost.
It's important that this doesn't become a `gotcha' for any kind of speech.
Players shouting `charge', do not merit a penalty.

The cost for speaking exists to add a tactical decision.
When in battle, players may want to turn to a spell-caster and ask them to cast a \textit{Curse} upon an enemy, or request that someone guard them.
This speech helps the troupe tactically, so it has a cost.
When a player has a great idea about the whole group moving backwards to avoid enemy attack, they should think of the simple proposal as a tactical manoeuvre, and consider the cost of proposing it.

\subsection{\glsentrytext{npc} Fights}
\label{npcfights}

Add a few too many \glspl{npc} to a fight and you can end up feeling stumped or having players wait for you to roll an awful lot of dice on your own.
That's no fun for anyone!

If you need a quick approximation of a massive battle between \glspl{npc}, just have each \gls{npc} deal its own \gls{cr} value in Damage each round (ignoring \gls{dr}).
A guard \pgls{cr} of 10 who fights alongside the characters deals 10 Damage, which could mean killing a single creature with 10 Damage, or could mean finishing off 2 creatures the characters have already wounded, by dealing each one 5 Damage.

\begin{exampletext}

  ``The goblin platoon start throwing more spears at you, but then from the side, the garrison of guards burst into the cavern's entrance to join you.''

\end{exampletext}

If two \glspl{npc} fight, whichever has the highest \gls{cr} deals Damage first (equally matched opponents deal Damage simultaneously).
So if ten soldiers, each with \pgls{cr} 10, fight a basilisk with \gls{cr} 24, the basilisk would deal 24 Damage, killing 2 soldiers.
On the next turn, the 8 remaining soldiers would deal 80 Damage, killing the basilisk.

\begin{exampletext}

  The \gls{gm} thinks for a moment.
  That's 30 goblins and 12 guards.
  The twelve guards have \pgls{cr} of 10 each, so they deal 10 Damage each, killing 10 goblins.
  Then the 20 remaining goblins, each with \gls{cr} 10, deal 40 Damage, killing 4 guards.

  ``The guards spill in, massacring the goblin horde.
  You see some surrounded, and spears driven into them, but the rest keep fighting.''

\end{exampletext}

Obviously, this system is not going to represent anything with much accuracy, but it's better than halting a game so you can roll dice for twenty minutes alone.

\subsection{Illusions}
\index{Illusions}

Whether players are attempting to use illusions in combat, or trying to attack your \glspl{npc}'s illusions, the same rules apply; everyone must spend \glspl{ap} to attack.
Three illusory armoured warriors might not seem like much of a worry, but they present a lethal choice:

\begin{description}
  \item[Defend]
  and spend all your \glspl{ap}.
  This may leave the character on negative \glspl{ap}, which will leave them open to real attacks.
  \item[Don't defend]
  and run the risk of a real opponent striking real flesh with a real knife. 
\end{description}

Not everyone can wield illusions well, but when used correctly, they can devastate an army.

\subsection{Tactics}
\index{Tactics for the \glsentrytext{gm}}

Nobody like an opponent who's always letting them win.
A \gls{gm} pulling out three basilisks on new \glspl{pc} is bad form, but it's even worse when the players are allowed to win by poor tactics.
At least dying to a basilisk means dying with honour!

\subsubsection{Focus}

Basic tactics include two things: it's best to focus all attacks on single targets, and it's good to flank opponents whenever possible.

Don't parcel up opponents in a fair and even-handed way -- they're there to destroy the \glspl{pc}, so set all of them against one \gls{pc}, and if that player wants their character to survive, they'd best move back, or the other \glspl{pc} had better guard them.%
\exRef{core}{Core rules}{guarding}
If the \glspl{pc} have left their anterior side exposed, enemies should spend 1 \gls{ap} to move to their side and allow half the group to flank the \glspl{pc}.

\subsubsection{Throw}

Next up, remember the use of ranged weapons.
Everyone from thieves to goblins can throw spears, and if no spears are available, they can throw rocks.
$1D6-1$ Damage may not seem like much, but try it with 10 goblins, and see how many achieve a debilitating \glsentrytext{vitalShot}.

\subsubsection{Fight Dirty}

So you have twenty goblins facing off against four \glspl{pc}, but the \glspl{pc} have plate armour, a round shield, and a bad attitude.
They're invincible.
The battle looks hopeless, despite the goblins' tenacity, hunger, and greater numbers.
Now is the time to think tactically.

Perhaps goblins don't have the strength to wield pole-arms properly, but they can still wield them \emph{improperly}, and for plenty of Damage!
Once a \gls{pc} has run out of \glspl{ap}, keep attacking so that the \gls{pc} gets a penalty for defending while below 0 \glspl{ap}.

When your antagonists with big weapons have only 1 \gls{ap} left, they don't always have to swing those big swords -- make them kick instead.
It only inflicts \glspl{fatigue}, but it stops them taking penalties from having negative \glspl{ap}.

Turn the lights out, place a hiding enemy in behind a curtain \glspl{pc} might run past, and them grapple someone.
Freak the players out.

\subsubsection{Magical Chicanery}

At some point, the troupe's spellcaster may put together a \textit{duplicated, distant, divergent, Murder-Spell}, and destroy anything in their path.
This isn't an abuse of the rules -- it's just intelligent use.
It's time to up the ante and miss out encounters.
If the troupe's witch can handle 30 goblins with a single spell, just gloss over goblin encounters with a brief roll of the dice.
If the troupe's witch somehow fails the spell-casting roll, run the encounter; otherwise, move on.
Focus on the encounters that still matter.
There are always creatures big enough to challenge a caster of any magnitude.

\subsection{Fine-Grained Armour}

BIND's armour basically reduces everything to `good armour covers 3 points, bad armour covers 5 points', so it won't represent the finer points of brigandine vs chainmail.
However, if your players insist on constructing armour, and you really need \emph{something} to differentiate the various pieces they have attached, the system does admit of two potential rule-expansions.

Firstly, you don't need to limit armour to just `complete' and `partial'.
You might have `minimal' (which requires hitting 1 over the \gls{tn} to hit), `basic', and basically any other number from 1-5.

Secondly, you can divide the armour types.
Someone could have Partial armour which provides \gls{dr} 5 (meaning just a helmet and chest-plate), while also having Complete leather armour (meaning a roll 3 over the \gls{tn} bypasses the plate, but receives 3 \gls{dr} from the leather).

I wouldn't recommend doing any of that for \glspl{npc}, as it will inevitably turn into a headache.

\subsection{How to Balance Encounters}

Bring extra character sheets, and make sure new \glspl{pc} have somewhere they can enter the plot once \pgls{pc} dies.
Let the dice dictate encounters quickly, and rush through every scene that looks dull with a brief description, and another roll to check for monsters the next day.

After a couple of deaths, the players will understand that survival demands planning, intelligent, and brutality.
Let them know that the world is as it is, and will not warp around their needs unless they find a way to \emph{make} the world change.

Wait, did I say `balance'?
I meant `run'.
We should run.
The trolls will arrive soon, and we know what happens if they find us\ldots

\subsection{Damage, Death \& Dismemberment}

\subsubsection{Damage}

Losing \gls{hp} is a massive, screaming deal in BIND.
It's easy to take habits over from other games where losing one's liver is all part of a normal Tuesday afternoon but here \glspl{pc} should lose \glspl{fp}, then attempt to flee and only in the most dire situations should they start to bleed.
Damage which doesn't hit home can be brushed over with a brief note about `avoiding the swing' but if anyone loses a single \glsentrylong{hp} the \gls{gm} should grind the description and combat to a halt to emphasise exactly how eyeball poppingly, knee-cap shatteringly painful and side-splittingly debilitating a knife can be.
Take your time.
Make the words secrete congealed blood.
If the \glspl{pc} start to lose \glspl{hp} and don't realise how serious this situation is they might perish where they otherwise would have run away to fight another day.

\subsubsection{Death}
\label{pcdeath}
\index{Death}

Players who want their characters to survive should retire them.
After all, few active \glspl{guard} survive for long.
Character creation should be relatively easy, and no main plot-line should rely on a particular character.

Once death has come, the player should select a character from the existing pool of \glspl{npc}\iftoggle{stories}{brought into the world with \glspl{storypoint}}{}.
Any \gls{npc} should be allowed, just as long as they might plausibly arrive in the current area within \pgls{interval} or two, and have some plausible motivation to join the troupe.

If no \glspl{npc} have been established, anyone in the troupe can establish one immediately.
If none of the \glspl{pc} have any \glspl{storypoint} left, the new character must spend 2 \glspl{storypoint} immediately, and explain their ties to an existing \gls{pc}.

Players, rather than characters, keep their unspent \glspl{xp}, so any time a character dies, any unspent \glspl{xp} should be immediately given to the new character.

If the \gls{pc} died on the road, the local \gls{jotter} may well commission a \gls{bothy} where they fell, and name it after them.
This can be an excellent way to commemorate a fallen character over a long period.
At the start, \pgls{bothy} may start as just a pile of rocks which people leave, then the \glspl{pc} might start a session arranging those stones before they hear their main mission, and finally, the \gls{bothy} will receive the fallen \gls{pc}'s name on a plaque once completed.

\subsubsection{Dismemberment}

If a \gls{pc} is totally out of commission, with 1 \gls{hp} left, 4 \glspl{fatigue} from being bled dry, and a disease nobody can cure, consider letting them play \pgls{npc} and letting them keep all \glspl{xp} gained during this time.

\end{multicols}

\section{Parting Notes}

\begin{multicols}{2}

\subsection{Follow the Rules}
\index{Blorb}

If you've ever played a computer game, then used cheat-codes, you'll notice a shift in tone descends the moment the cheat-codes come in.
Once you have no restrictions, you lose some respect for the game, as your relationship to the world changes.

The opposite shift occurs when people try a rogue-like game, where you can't save the game and try again from the same spot.
Where they once jumped towards anything that moved, sword-in-hand, ready to hit `X' as fast as they could, a couple of deaths tell them to approach things differently.

\begin{speechtext}
  `How far can that acid spray reach?
  Can it cross the river?
  Maybe I could swim across the river, and throw rocks at it until it dies.'
\end{speechtext}

These problem-solving habits don't emerge so easily, or in the same manner, when people don't fear death (or at least serious consequences).

Of course, once \pgls{pc} dies, you should be `on their side'.
You can say `I'm sorry your character died', and you should never appear happy about a character's death (unless, for some reason, the player is happy about it).
But the dice should be allowed to land where they land.

The rules have been designed around this principle; when \glspl{pc} attack something, you can simply state the \gls{tn} to attack it, and let the player roll the dice.
Mechanically, it makes no difference if you roll, or they do, but it lets the players feel responsible for the Damage their characters take when they fail a roll.

On a larger scale, letting the dice fall where they may means trusting the players.
If the troupe start out, on their first path, on game one of your campaign, you might roll a basilisk as their first encounter, and then you might feel the temptation to re-roll the result, because a basilisk will be too much for any starting characters to handle.
But if you stick to the rules then the players might surprise you with something clever, and they'll feel all the smarter for having done so.

It's not your job to help them win the fight.
It's your job to tell them that a basilisk is coming, and see what they do.

\subsection{Going Off the Rails}

BIND only focusses on the \gls{guard} as a convenient plot hook, but your nights don't \emph{need} to revolve around them.

\begin{itemize}
  \item
  Play a group of reckless doula, who defy the orders of their guild, and spend their nights focussed intently in maximizing their magical abilities, and collecting potent \glspl{boon}.
  \item
  Make a troupe of bog-standard adventurers, including a bard and a monk.
  \item
  Craft a story all about the Temple of Poison, where each major \gls{npc} has three desires.
  Lowly servants of the temple receive conflicting orders, where each must realise two of these desires, each from different \glspl{npc}.
  Craft a tangled web from these simple threads, and throw plot to the wind.
  \item
  Restrict the races to only gnomes.
  Tell the story of their warren, as they try to protect themselves from goblins invading the surrounding warrens.
  The other gnomish villages fall, one by one, and the elders argue about solutions all day, but do nothing.
\end{itemize}

\subsection{Using Types as Cultures}

\Gls{fenestra} might seem a little bare-bones in terms of people.
Other worlds have a complete rainbow of different gnomish `subraces', where \gls{fenestra} just has `gnomes'.
However, I'd like to suggest a perspective-shift.

I think of \gls{fenestra} as having myriad different types of elves and gnomes, but I also think that all the `subraces' common to fantasy lore were really just different cultures in disguise.
Snow elves and wood elves do different things, and have different skills, but their bodies didn't show any differences in most fantasy worlds.

So while this book only shows `dwarves', and those dwarves will (as we all do) consider themselves just `the normal ones', living in `the middle place' in their world, where they speak `the language', they will likely see distant dwarves as `hill dwarves', or `deep dwarves', and those dwarves will have a particular name for the local ones.
People may call goblins `svarts', `nockers', `kobolds', or `gremlins', and perhaps gremlins live in forests and have green skin instead of white, but at the end of the day, they're all goblins.

So when making some new part of \gls{fenestra}, you might give some islands a group of `boat humans', who consider Seafaring to be a standard Skill, which every adult should have; or make `stone elves', who live across a particular mountain range, partially underground; or feature a war between the gremlins and nockers, who believe themselves to be very different people.
These different people could even share a cultural background.
Mountain elves living beside dwarves may end up speaking a similar language, and sharing similar skills.

We don't need any new rules for these distinctions -- shared languages, and Skills give enough bare-bones mechanics to hang a story on top of.

\subsection{The Metaphysics of Becoming a Tree}

All magic systems imply some ontology, and most of the European ontologies come from Aristotle's view of the world.
This is why someone turned into stone might retain their vision, motion, and thoughts.

In Aristotle's view, everything has:

\begin{itemize}
  \item
  form (the shape)
  \item
  substance (what it's made out of)
  \item
  telos (what it's \emph{for})
  \item
  first-cause (where it came from)
\end{itemize}

For example:

\begin{itemize}
  \item
  Rain forms drops, is made out of water, it's for watering plants, and it comes from the sky.
  \item
  People have people-shapes; flesh, sinew and bone constitute their bodies; people exist to praise the gods and achieve glory; and mothers provide the first cause.
  \item
  Fireballs have rain-drop shape but fire for substance, they exist to burn, and come from angry wizards.
\end{itemize}

Anyone transformed to stone would retain their organs with the same telos, meaning that a liver would continue to process toxins, and the eyes continue to see, because that's what eyes are for (regardless of their `substantive cause' (meaning, the substance they are made from).

Similarly, when a spell makes someone big, they wouldn't have problems with vision, their lungs won't collapse, and they won't suffer any back-pain, because they are the same as they were, except for the size, which is simply a larger version of their original form, with more substance added via the magic of the growth spell.

If none of that makes sense, blame Aristotle.

\subsection{Don't Have a Fun Game}

Tell a strange tale, and make your players uncomfortable.
Or tell an even stranger tale, and invite them to think about the importance of rabbits to Middle Ages economics.
Make \pgls{segment} casual, then tense.
Zoom into the death of \pgls{npc} so hard that the players change how they approach the world.
Give them a puzzle without an answer, simply because stories should be true to life, even when they contain goblins with wings.

You can't approach `fun' head-on.
The moment you ask someone `are you having fun?', you cut their focus, and move them to introspection.
Great stories never ask the reader how they feel.
They engage the reader in something else.

Any time all the players feel engaged, the game and the \gls{gm} serve their purpose.
Players can ponder, laugh, or cry.
As long as they all feel like they want to engage with the narrative, the flavour of engagement doesn't matter.
But then again, if it doesn't feel like it matters to you, it won't matter to them; once again, engaging with the goal head-on, eyes-open, doesn't work.

So leave fun to one side, find an engaging problem, and ask them all `what do you do?'.

\end{multicols}

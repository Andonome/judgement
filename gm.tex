\chapter[Bastion of Judgement]{Judgement}

\section{The Unstoppable Stars}
\label{astronomy}
\label{seasons}
\index{Seasons}
\index{Cycles}
\index{Astronomy}
\index{Time Parallel}

\begin{multicols}{2}

\setcounter{season}{\month}

\noindent
\Gls{fenestra}'s time keeps moving, just as ours does, but slightly faster.
For every week which passes in our world, three weeks pass over there, so after a single month here, \gls{fenestra} experiences a full season.

While we have \trackMonth, \gls{fenestra} has `\showSeason' -- a \showTemperature\ season.
Just as we have twelve months, \emph{they} have twelve seasons.

After three years, all twelve seasons have passed and a new cycle begins.
\Gls{fenestra}'s inhabitants track `cycles' more often than years, unsurprisingly.

\begin{nametable}[l|Y|c|c]{Seasons}

  \textbf{Real Month} & \textbf{Far Season} & \textbf{Weather} & \textbf{Event} \\
  \hline
  \hline
  \showSeasonLine{1}
  \showSeasonLine{2}
  \showSeasonLine{3}
  \showSeasonLine{4}
  \hline
  \showSeasonLine{5}
  \showSeasonLine{6}
  \showSeasonLine{7}
  \showSeasonLine{8}
  \hline
  \showSeasonLine{9}
  \showSeasonLine{10}
  \showSeasonLine{11}
  \showSeasonLine{12}

\end{nametable}

\subsubsection{Time Flies}
During the game, the players can let downtimes come and go as they wish.
Players can rest for days, or up to three weeks.
The only restriction on time is the three-week limit, and returning to safety once everything has ended.

Of course, if a troupe can't make it back in time, they will have problems.
Characters who don't make it back in time should start their next session with 0 \glspl{fp}, and perhaps an encounter roll from their last location to explain what exactly went wrong.
A reasonable ruling always depends on the exact circumstances.

\subsection{How to Avoid Scheduling Conflicts}

Start in town, end in town.

Scheduling conflicts plague every RPG session held outside of the Antarctic circle.
Reality will not adjust to a game's plot, which means the fantasy must adjust to reality,\ldots but not too much.

If you tell everyone from the outset that their session must start and end in town (or some civilized area, where the \glspl{pc} can rest without random encounters), then the players will generally do their best.
Any `dungeons` must become an area they enter quickly, and jump out of once something -- anything -- has been achieved.

Once your game adopts this standard, the headache of getting five adults in a room on one of their free days vanishes.
And every time someone messages to say they can't make the game, you can mark that off as a victory for parallel time because the game goes on regardless.

This also opens the game up to more people.
The gaming table no longer needs to feel `full', because yet another person wants to join the campaign.
It will always have a seat open, even if seats fill up for a particular week-night.

Some \glspl{pc} missing some plot-points must be taken in their stride.
But players always forget details from week-to-week, so this shouldn't dampen any campaign more than the usual memory-loss, and many players will fill each other in on what happened during the previous mission.

\subsubsection{Proper Time-Keeping}

Keeping a campaign going, where anyone can jump in and out, requires proper time-keeping.%
\footnote{Gygax was right.}
The \gls{gm} sheet has a spot open at the top for the date, where you can note the season for the month.

Proper time-keeping doesn't just help the campaign -- it helps the night.
Having a proper ending-time means you can signal ahead that the session should come to an end within the hour.

And of course, if a session ends early, this just leaves more time to discuss how the \glspl{pc} will spend their time in town, and what they want to spend their \glspl{xp} on.

\end{multicols}

\section{Basic Prep \& Play}

\begin{multicols}{2}

\subsection{Pre-Game Prep}

The basic tools of the \glsentrylong{gm} must begin with with the obvious -- $4D6$ per player with multiple $D6$ colours so players can differentiate their Damage dice from their Action dice.
Remember pencils and a rubber, as players never bring their own.
Lastly, print out a load of character sheets.
This can be a lethal game, so players may need more than one.
They'll also need character sheets for any \glspl{npc} they bring into the game.

\subsection{Tracking Information}

Print out the \gls{gm} sheet at the end of the book for a little help handling all the information you'll need to keep track of during a campaign.
In particular, this is a good place to keep track of stats for all those \glspl{npc} that you need to make up on the fly.
Remember that it doesn't matter what you put for \gls{npc} stats, so long as those stats are consistent.

Long-standing \glspl{npc} should also have their \glspl{fp} listed next to the character, as \glspl{npc} gain \glspl{fp} at the end of each \gls{interval}.
This helps beloved \glspl{npc} stay alive, as well as adding a little extra gravitas to any antagonists who encounter the \glspl{pc} multiple times.

\subsubsection{Coins}

As a \gls{gm}, it's always good to have at least 3 different types of coins to keep track of \glspl{ap}.
Let's say you're orchestrating a battle with a hobgoblin leader, some hobgoblin troops and a goblin spellcaster.
Assign each one a coin and make a little mnemonic -- the spellcaster has dark magic so it gets the little copper penny.
The hobgoblins get the silver coin to represent their use of weapons, and the largest coin goes to the hobgoblin leader.
Don't worry about the players' \glsentrylongpl{ap} -- they'll keep track of their own characters.

Coins can also be used to keep track of \gls{fp} and Fatigue Points as they change so often.
It'll help cut down on wear to the character sheet.

\subsection{How to Make Rulings}

\subsubsection{Literal Interpretations}

If it's ever unclear how to resolve a situation, the first attempt should always be a strict interpretation of the rules.
For example, if a player says `If I charge round a corner, rather than a straight line, can I still use the Fast Charge knack?', the answer is `yes', because the rules as they stand don't prohibit going round a corner.

No rules will work all of the time, but by following a literal interpretation of the rules whenever possible, players feel better able to predict and navigate the world, and \glspl{gm} do not have to waste so much energy on making on-the-fly rulings.

Broadly, the \gls{gm} should consider themself bound by the rules as much as the players.
A good rule of thumb is to make as few decisions as possible, and let yourself focus on description and planning.

\subsubsection{Let Players `Ruin' the Mission}

Encounters don't have to play through like you think they will.
If the players flood a dungeon, cast a fireball at the king, or raise their Aldaron and Wyldcrafting so high that every wild animal encounter turns into a pet in a growing army, take a breath, re-examine the situation, and go from there.

Perhaps the dungeon has a high-point inside which isn't flooded, which at least saves that part of the dungeon; or perhaps it's flooded forever, and nobody will see that treasure again.
Perhaps the party have to become outlaws, and every future adventure has to take this into account.
And even if all those pets feel enamoured with the caster, they don't need to like each other -- maybe they start to fight, or try to kill the other party members, but only when they fall down, wounded and weak!

\subsubsection{Torture}
\index{Torture}

If players torture an \glspl{npc}, have the NPC give out a false narrative.
If they ask for a location, it sends them somewhere dangerous.
If they ask who's in charge of a conspiracy, they finger a well-known official, or priest.

Even stupid \glspl{npc} can create a basic narrative, so start pulling up enough nonsense that the \glspl{pc} become completely confused!

Any attempt to notice the lies receives a -4 penalty -- it's hard to tell odd behaviour when someone's under perpetual stress!

\subsubsection{You Don't Always Need to Roll}

Get used to saying `the \gls{tn} is 10, so you succeed'.
If the \gls{tn} to open a wedged door is 10, and the \gls{pc} has plenty of time, they can just take a resting action, meaning their minimum roll is `7' -- if the Strength + Crafts Bonus totals +3, that means they succeed.

Similarly, if everyone wants to help spreading a nasty rumour around town, and you notice one \gls{pc} has a Charisma + Deceit score of +4, while another has +3, the total for the Team Roll would be +6.

If the \gls{tn} is `8', tell them they succeed after an interval.
If the \gls{tn} is `13', tell them they succeed after spending a few days repeatedly spreading the rumour.
They should only roll when they might fail.

Once players feel emboldened to just say `okay, well at \gls{tn} 9 I succeed', the game becomes just a little faster.

\subsubsection{Another Look at Dice}

The dice system can be expressed any number of ways.

\begin{itemize}

  \item
  The player and \gls{gm} each roll 1D6 and add their bonuses.
  If one rolls higher, they win.
  \item
  The player rolls $1D6-1D6$, then adds their own bonus, and subtracts their opponent's bonus.
  Rolling a total of `0' means a tie, a positive score means success, and a negative score means failure.
  \item
  The player rolls 2D6 at a \gls{tn} equal to 7.
  They add their bonus, and subtract the opponent's bonus.

\end{itemize}

These three are the same as the primary rules.
The main rules only work as they do so that almost every roll uses addition, rather than subtraction (which should help the flow).
But looking at the rules in these different ways can help clarify the underlying structure.

\subsection{Downtime}

\paragraph{Healing}
makes downtimes particularly important when characters cannot heal without it.

\paragraph{The cost of living}
comes right after, especially if players have a particularly long downtime.

The player should always agree that the cost makes sense, but 10\% makes a good rough benchmark (people who have more money, spend more money).

\paragraph{Buying traits}
should only occur at the start or end of a session (unless a player wants to spend a Story Point to explain why they have this ability).

\sidebox{
  \begin{rollchart}
    Years & Story Points \\\hline
    1-2 & 1 \\
    3-6 & 2 \\
    7+ & 3 \\
  \end{rollchart}
}

\paragraph{Longer downtimes}
may call for Story Points.
Don't ask the players what their characters want to do, just jump straight to the next scene, years later, and let them explain their actions in-game, with the new Story Points.
Perhaps they travelled and learnt a new language or made a new ally.

Each year of downtime should cost 10\% of the character's wealth, or 10\gls{gp} (whichever is higher), to represent the money they've spent during this time.
Characters without any other means of sustaining themselves should default to spending 10\glspl{gp} per year.

\end{multicols}

\section{Side Quests}\label{sidequests}
\index{Side Quests}

\begin{multicols}{2}

\noindent
Another way to add impromptu elements into your game is Side Quests.
These are short encounters which slowly feed elements into the background of your game.
They're good for foreshadowing without too much planning, and good for adding things to the path of players who simply want to run around in a sandbox, without the constraint of a full-on plot-arc.

\subsubsection{Example 1: The Beast}

\begin{list}{\sqn}{}

  \item[\sqr]
  (Town) Villagers approach the party, asking them to help slay a beast.
  \item
  (Villages)
  Scared villagers tell the \glspl{pc} about the strange beast they've seen, and where it went.
  \item
  (Villages) Another troupe, also looking to slay the best arrive. If they characters have killed the beast, they take credit; otherwise they journey out to kill the beast themselves.
  \item
  (Villages)
  A powerful alchemist arrives and asks about his missing pet (the beast).
  He explains the beast a peaceful herbivore, who only becomes aggressive when cornered.
  He will plan vengeance upon anyone who killed the beast.
  \item
  (Forest) \squash
  The alchemist hired a tracker to follow whoever killed the beast, along with a soldier.
  He assaults them at the worst possible time, during another encounter (the \glspl{pc} may only hear about the results of this encounter afterwards if they are not present).

\end{list}

Note they key tenants of Side Quests:

\begin{enumerate}

  \item
  Every part has an area, and this part of the side quest can happen almost anywhere within that area.
  \item
  Specific locations (like a dungeon) may exist, but they are not part of the Side Quest.
  \item
  No part of the Side Quests will rely on a particular outcome from a previous part, or a specific time-period.

\end{enumerate}

Villagers might approach the player-troupe, wherever they are in town.
And when the troupe return, the rival troupe could be in any village, no matter which route home the troupe takes.
This is not meant to `rail-road' players -- if they want to avoid a known location or mission, they can.
Accepting the villager's plea does not feature in the quest.

Pre-planned locations which must be approached can still exist, along with details, but Side Quest parts cannot assume the players choose to engage with them in a particular way (or at all).

That said, you can place some locations as random encounters.
If a tavern could be anywhere in town, you can put the tavern on the map once the \glspl{pc} encounter it, or have them find a mad hermit's house on the outskirts of a village.
Just mark his location (once the \glspl{pc} find him), and the story remains consistent.

Each part's independence means that most Side Quests can still play out, without worrying about how the last one resolved.
However, previous actions can still affect how a given Side Quest plays out -- the alchemist may be friendly or hostile to the \glspl{pc}, depending on what happened with his beast.
Even if the \glspl{pc} did nothing, every part would still begin -- previous actions affect the new encounter's actions, but should not stop it occurring.

If any Side Quest contains a repeating \gls{npc}, who appears in more than one encounter, it's good to introduce multiple \glspl{npc} who can fill this role, in case one dies early one.
Alternatively, a single arch-nemesis might meet the \glspl{pc} in a tavern, and befriend them in the first encounter, before subsequent encounters can establish them as a villain without them being present.

\subsubsection{Example 2: The Suspicious Priest}

\begin{exampletext}

A priest is using his ability to divine the future to capture criminals \emph{before} they commit crimes.

\end{exampletext}

\begin{list}{\sqn}{}

  \item[\sqr\squash]
  (Villages) A local priest offers to tell the party their fortunes.  Combine this with the next encounter, then move it to Town.

  \item
  (Town) The characters pass by men in stocks who keep shouting that they are all innocent, and were suddenly taken away by various guards after the local priest fingered them for a crime.  Move this encounter back to the villages.

  \item
  (Villages) A dozen guards are tracking the characters. Repeat.

\end{list}

The characters are now wanted by the guards who wander the villages, hunting for would-be criminals.

Notice that the first part combines with the encounter below it, meaning `whatever encounter is next on the list'.%
\footnote{These Side Quests are marked with the symbol \squash.}
This new encounter must always be from some other Side Quest, so that Side Quests merge together.
Exactly how these merged encounters play out rests in the hands of the \gls{gm}, but it's generally enough to simply run both encounters in quick succession.

\subsubsection{Random Side Quests}

In addition to story-based Side Quests, it's good to give each area a bunch of entirely random encounters.

\begin{list}{\sqn}{}

  \item{(Forest) The party find a gnome attempting to sell them gemstones for his trip. Some are real and others are fake.}

  \item{(Forest) A dragon flies overhead.}

  \item
  (Forest) A dead mage lies on the road. His books are valuable but should by law be returned to the alchemists.

\end{list}

This collection of non-quests serves two functions.
The first is to provide some short encounter when the time calls for it, but without getting the party wrapped up in yet another mission.
If you already have five Side Quests happening at the same time, that's probably as much as the players can handle.

The second use is in wrapping up a campaign.
If you have only two more plot-threads you want to wrap up, the rest of the world doesn't need to feel empty -- encounters can continue, but they needn't start more plot-threads.

\subsubsection{Summary}

Think of your campaign in terms of areas; a mountainous area by the sea might have `\emph{Underground}', `\emph{Mountains}', and `\emph{Coast}', while a deep forest might have `\emph{Elfwoods}', `\emph{Villages}', and `\emph{Swampland}'.

Each encounter is tied to an area, so when the players enter that area, they get the next encounter available there.
When the players enter the `\emph{Villages}', they encounter the next available a Side Quest.

Since Side Quests can leave the `Forest' area when the next part is in `Town', players will find themselves starting on a new Side Quest in the Forest, then returning to an old one once they enter Town again.
This format will soon have them engaged with multiple plot-arcs at the same time.
The party can often engage with these quests by seeking out a particular area, or going to preset locations, but if they choose to ignore any plot hooks then that's fine -- the plot will march on and conclude one way or another without their input.

If you want to run Side Quests as a secondary part of your game, you can just run them any time the group doesn't get a random encounter.

If you want them to be the primary mover in your campaign, you can run a Side Quest every time the group enters a new area.
You can also make one plot line the \emph{primary} quest by making it longer than the others.

Putting the above Side Quests together, the events could play out as follows:

\begin{enumerate}

  \item
  \sqr~(Town)
  Villagers ask the party to slay a beast.
  \item
  \sqr~(Village)
  A priest is reading villagers fortunes.
  One villager is fated to die a horrible death soon, and the other villagers say that the beast will get him, because they saw it the other day.
  The priest asks to read the \glspl{pc}' fortunes.
  \item
  \sqr~(Village)
  After tracking and injuring the beast, but ultimately fleeing, the \glspl{pc} rest overnight, while the rival troupe moves out to finish the beast off, and take all the glory for themselves.
  \item
  \sqr~(Town)
  An alchemist, staying in the same tavern as the \glspl{pc} asks if they've seen his pet, claiming it would never hurt anyone.
  The \glspl{pc} point him towards the rival troupe, who have already taken all the credit for killing the beast.
  \item
  \sqr~(Town)
  The next day, they see a line of criminals proclaiming they are all innocent.
  \item
  \sqr~(Village)
  Guards try to arrest them, so they fight, and eventually have to flee into the forest.
  \item
  \sqn~(Forest)
  The old rival troupe attempt to claim the bounty on the \glspl{pc} head, and hunt them down -- but the alchemist appears at the last moment to save them.

\end{enumerate}
\noindent

For particularly long Side Quests, consider rewarding the troupe with a Story Point.
\exRef{stories}{the Book of Stories}{stories}

\subsubsection{Writing a Side Quest}

Consider this standard fantasy plot-hook:

\begin{exampletext}

  Bandits stole an alchemist's magical item, and he wants the party to retrieve it.

\end{exampletext}

Now let's expand with some foreshadowing:

\begin{enumerate}
  \item
  Merchants enter town without their wares, having been robbed.
  \item
  An alchemist asks the characters to find the items bandits stole from him.
\end{enumerate}

Now let's add a decoy -- the bandits must know \glsentrytext{guard} often come after them, so they can send someone to lead the party into a trap.
And finally, we might add another lead to make things interesting.
Perhaps the bandit leader began as an alchemist who left his circle because they began practising Necromancy, and they want him dead so he doesn't spill the beans.

\begin{enumerate}
  \item
  (Town)
  Posters go up for a local alchemist, wanted `DEAD on sight'.
  \item
  (Town) 
  Merchants inter town without their wares, having been robbed.
  \item
  (Town)
  An alchemist asks the characters to find the items bandits stole from him.
  \item
  (Villages)
  A man offers to help the party find the bandits, but in fact wants to lead them into a trap.
\end{enumerate}

The first two encounters can combine with whatever else the players want to do.
If they want to drink, they drink and notice the posters.
If they want to buy weapons, they find prices have risen since shipments of iron and various other components have been raided by bandits, leaving the merchants with nothing but their skins.

If the characters skip town to find the bandits, we don't need the mage with the mission - they can simply encounter the guide in the village.

Once we've finished with those encounters, you might leave the rest to the areas: the bandits' hide-out, and then a local tower where alchemists secretly learn about Necromancy.
Or perhaps another part might slot into the town, or forest, involving an alchemist, and lengthen the Side Quest into a meatier tale.

Breaking things up like this allows different smaller plots to meld into one.
Perhaps characters go looking for the bandit alchemist immediately, and find some other encounter in the forests.
And maybe when the bandit spy joins the party, he'll end up in an unrelated encounter where the party find themselves trapped in a giant web, and he takes the opportunity to finish them off.

\end{multicols}

\section{Combat}

\begin{multicols}{2}

\subsection{Ordered Initiative \& Hollering}

If your troupe insist on acting in order of Initiative, you can help them along by going from highest to lowest while saying the number out loud.

\begin{exampletext}

``Eight! The gnolls raise their weapons''

``Seven, six! They move forward, bearing their yellowed teeth.''

``Five! Snarls abound as they speed up to a rush.''

\end{exampletext}

Nothing has actually happened by this point, but it sets the scene nicely.

\begin{quote}

``Five'', one of the players shout.  ``I'm going at five.  I move to protect Max.''

``Two gnolls go for you, another two go for Amelia.  Roll to engage at \gls{tn} 11.''

\end{quote}

The initiative-count continues down quickly at all times, and the count always provides a sense of urgency.
If players don't notice it's their turn when you're shouting, that's 1 \gls{ap} lost.
Do it once, and they'll never make the same mistake again.

\subsection{Speaking}

You may have noticed that speaking comes with \pgls{ap} cost.
It's important that this doesn't become a `gotcha' for any kind of speech.
Players shouting `charge', do not merit a penalty.

The cost for speaking exists to add a tactical decision.
When in battle, players may want to turn to a spell-caster and ask them to cast a \textit{Curse} upon an enemy, or request that someone guard them.
This speech helps the party tactically, so it has a cost.
When a player has a great idea about the whole group moving backwards to avoid enemy attack, they should think of the simple proposal as a tactical manoeuvre, and consider the cost of proposing it.

\subsection{\glsentrytext{npc} Fights}
\label{npcfights}

Add a few too many \glspl{npc} to a fight and you can end up either being a stumped \gls{gm} or having players wait for you to roll an awful lot of dice on your own.
That's no fun for anyone!

\begin{exampletext}

  ``The goblin platoon start throwing more spears at you, but then from the side, the garrison of guards burst into the cavern's entrance to join you.''

\end{exampletext}

If you need a quick approximation of a massive battle between \glspl{npc}, just have each \gls{npc} deal its own \gls{xp} value in Damage each round (ignoring \gls{dr}).
A guard worth 10 \gls{xp} who fights alongside the characters deals 10 Damage, which could mean killing a single creature with 10 Damage, or could mean finishing off 2 creatures the characters have already wounded, by dealing each one 5 Damage.

\begin{exampletext}

  The \gls{gm} thinks for a moment.
  That's 30 goblins and 12 guards.
  The twelve guards are worth 10 \glspl{xp} each, so they deal 10 Damage each, killing 10 goblins.
  Then the 20 remaining goblins, worth 4 \glspl{xp} each, deal 40 Damage, killing 4 guards.

\end{exampletext}

If two \glspl{npc} fight, whichever individual is worth the most \glspl{xp} deals Damage first (or at the same time, if equally matched).
So if ten soldiers worth 10 \gls{xp} each fight a basilisk worth 24 \gls{xp}, the basilisk would deal 24 Damage, killing 2 soldiers.
On the next turn, the 8 remaining soldiers would deal 80 Damage, killing the basilisk.

\begin{exampletext}

  ``The guards spill in, massacring the goblin horde.
  You see some surrounded, and spears driven into them, but the rest keep fighting.''

\end{exampletext}

Obviously, this system is not going to represent anything with much accuracy, but it's better than halting a game so you can roll dice for twenty minutes alone.

\subsection{Illusions}

Whether players are attempting to use illusions in combat, or trying to attack your \glspl{npc}'s illusions, the same rules apply; everyone attacks on the same initiative click.
If the players are attempting to attack the illusion of an armoured knight, the (illusory) armoured knight gets a low initiative counter, and any players acting at a particular step attack him.
If they hit (and they probably will), the illusion is vanquished, and the players are left with a wasted action.

Similarly, if a player attempts to cast an illusion of a strong man, and the horde of twenty goblins are acting on initiative 5, then each of them will attach the knight, and each of them pay the 1 \gls{ap} cost for attacking.

\subsection{Tactics}

Nobody like an opponent who's always letting them win.
A \gls{gm} pulling out three basilisks on new \glspl{pc} is bad form, but it's even worse when the players are allowed to win by poor tactics.
At least dying to a basilisk means dying with honour!

\subsubsection{Focus}

Basic tactics include two things: it's best to focus all attacks on single targets, and it's good to flank opponents whenever possible.

If the \glspl{pc} have left their anterior side exposed, enemies should spend 2 \glspl{ap} to move to their side and allow half the group to flank the \glspl{pc}.
Don't parcel up opponents in a fair and even-handed way -- they're there to destroy the \glspl{pc}, so set them all against one, and if that player wants their character to survive, they'd best move back, or the other \glspl{pc} had better guard them.

If the \glspl{pc} want to survive, they'll need to take start stepping back at the right time, guarding each other, and killing faster.

\subsubsection{Throw}

Next up, remember the use of ranged weapons.
Everyone from thieves to goblins can throw spears, and if no spears are available, they can throw rocks.
If the creatures have no Projectiles Skill, they'll have a -1 penalty, but twelve goblins throwing rocks can still cause some damage.
Players will have to choose between spending 1 \gls{ap} on Keeping Edgy%
\footnote{See page \pageref{edgy}}
 or taking the hits and rushing forwards.

\subsubsection{Fight Dirty}

So you have twenty goblins facing off against four \glspl{pc}, but the \glspl{pc} have plate armour, a round shield, and a bad attitude.
They're invincible.
The battle looks hopeless, despite the goblins' tenacity, hunger, and greater numbers.
Now is the time to think tactically.

Goblins won't get on well wielding polearms, but goblins aren't smart -- give the little critters some massive weapons for a first-wave attack, and have goblins behind jump in to distract the \glspl{pc}!

Once a \gls{pc} has run out of initiative, keep attacking so that the \gls{pc} gets a penalty for defending while below 0 initiative.

And if an enemy soldier has only 1 \gls{ap} left, making a full attack with a big weapon could give them a -2 penalty.
Instead of attacking, they could just kick the \gls{pc} for 1 \gls{ap}, inflicting Fatigue Points.

Turn the lights out, place a hiding enemy in behind a curtain \glspl{pc} might run past, and them grapple someone.
Freak the players out.

\subsubsection{Magical Chicanery}

At some point, the party spellcaster may put together a \textit{Massive, Range, Sentient, Murder-Spell}, and destroy anything in their path.
This isn't an abuse of the rules -- it's just intelligent use.
It's time to up the ante and miss out encounters.
If the party mage can handle 30 goblins with a single spell, just gloss over goblin encounters with a brief roll of the dice.
If the party mage somehow fails the spell-casting roll, run the encounter; otherwise, move on.
Focus on the encounters that still matter.
There are always creatures big enough to challenge a caster of any magnitude.

\subsection{Fine-Grained Armour}

BIND's armour basically reduces everything to `good armour = 3 points, bad armour = 5 points', so it won't represent the finer points of brigandine vs chainmail.

However, if your players insist on constructing armour, and you really need \emph{something} to differentiate the various pieces they have attached, the system does admit of two potential rule-expansions.

Firstly, you don't need to limit armour to just `complete' and `partial'.
You might have `minimal' (which requires hitting 1 over the \gls{tn} to hit), `basic', and basically any other number from 1-5.

Secondly, you can divide the armour types.
Someone could have Partial armour which provides \gls{dr} 5 (meaning just a helmet and chest-plate), while also having Complete leather armour (meaning a roll 3 over the \gls{tn} bypasses the plate, but receives 3 \gls{dr} from the leather).

I wouldn't recommend doing any of that for \glspl{npc}, as it will inevitably turn into a headache.

\end{multicols}

\section{The Players}

\begin{multicols}{2}

\subsection{Damage, Death \& Dismemberment}

\subsubsection{Damage}

Losing \gls{hp} is a massive, screaming deal in BIND.
It's easy to take habits over from other games where losing one's liver is all part of a normal Tuesday afternoon but here \glspl{pc} should lose \glspl{fp}, then attempt to flee and only in the most dire situations should they start to bleed.
Damage which doesn't hit home can be brushed over with a brief note about `avoiding the swing' but if anyone loses a single \glsentrylong{hp} the \gls{gm} should grind the description and combat to a halt to emphasise exactly how eyeball poppingly, knee-cap shatteringly painful and side-splittingly debilitating a knife can be.
Take your time.
Make the words secrete congealed blood.
If the \glspl{pc} start to lose \glspl{hp} and don't realise how serious this situation is they might perish where they otherwise would have run away to fight another day.

\subsubsection{Death}
\label{pcdeath}
\index{Death}

Players who want their characters to survive should retire them.
After all, few of the active Night Guard survive for long.
Character creation should be relatively easy, and no main plot-line should rely on a particular character.

Once death has come, the player should select a character from the existing pool of \glspl{npc} brought into the world with the story, \nameref{oldnpc} (see page \pageref{oldnpc}).
Any \gls{npc} should be allowed, just as long as they might plausibly arrive in the current area within \pgls{interval} or two, and have some plausible motivation to join the party.

If no \glspl{npc} have been established, anyone in the part can establish one immediately.
If none of the party have any Story Points left, the new character begins with 2 fewer Story Points.

Players, rather than characters, keep their unspent \glspl{xp}, so any time a character dies, any unspent \glspl{xp} should be immediately given to the new character.
\Glspl{xp} received from spending Story Points do not reset, so if the old character had spent 4 Story Points, the new one would not receive any more \glspl{xp} from Story Points until they had spent 4.
In this way, the entire group should have a constant maximum number of points they can receive from Story Points.

\subsubsection{Dismemberment}

If a \gls{pc} is totally out of commission, with 1 \gls{hp} left, 4 Fatigue Points from being bled dry, and an inexplicable curse, consider letting them play an \gls{npc} and letting them keep all \glspl{xp} gained during this time.

\end{multicols}


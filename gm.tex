\chapter[Bastion of Judgement]{Judgement}
\label{judgement}

\section{Preparation}

\begin{multicols}{2}

\subsection{Inventory}

You will need:

\begin{itemize}
  \item
  Pencils (player never remember, best to accept it).
  \item
  A rubber (or `eraser' for Americans).
  \item
  $4D6$ per player, with multiple colours so players can distinguish Damage dice.
  \item
  \iftoggle{stories}{
    A boat-load of character sheets
  }{
    Some pre-made characters, including spares
  } (the game can be lethal).
  \item
  A bag of coins to track \glspl{hp}, \glspl{fp}, and any other points.
  \item
  A blade.
  \item
  The \gls{gm} sheet, to track whatever comes up.
  Take your blade, and slice it from the back of this book.

  Alternatively, scan the QR code at the back of the book for the \gls{gm}-shield \gls{pdf}.
  It changes every week, updating with the weather and creatures which are active in \gls{fenestra}.
  \item
  Some scenes for the session, in one form or another.
  \begin{itemize}
    \item
    You can either use a pre-written, self-contained module, or
    \item
    use your own map (\vpageref{makingTheMap}),
    prepare the next few encounters (\vpageref{encounters}),
    generate a mission (\vpageref{ngMissions}),
    and write some \glspl{sq}.

    For a pre-written \gls{campaign}, with its own map and \glspl{sq}, print a copy of \textit{Missions in Maitavale}.
  \end{itemize}
\end{itemize}

As \pgls{gm}, it's always good to have at least three different types of coins to keep track of \glspl{ap}.
Let's say you're orchestrating a battle with a hobgoblin leader, some hobgoblin troops and a goblin spellcaster.
Assign each one a coin and make a little mnemonic -- the spellcaster has dark magic so it gets the little copper penny.
The hobgoblins get the silver coin to represent their use of weapons, and the largest coin goes to the hobgoblin leader.

\subsubsection{The \Glsfmttext{gm} Sheet}
is on \vpageref{gmSheet}.
In the centre, you have a horrifying problem -- you can either write down the currently active \glspl{pc}, and hope to note them elsewhere after each session, so you can comfortably update this space; or you can try to write in delicate mouse-letters, in order to fit every \gls{pc} who ever joins in the section.

\vspace{\baselineskip}

\begin{tabular}{lccl}
  \hiderowcolors%
  \setcounter{enc}{0}%
  \outline{\large\arabic{enc}}%
  & \Square%
  \stepcounter{enc}%
  & \showInterval{0} &
  \gls{basilisk} + storm \\
  \outline{\large\arabic{enc}}
  & \Square%
  \stepcounter{enc}
  & \showInterval{1} &
  \\
  \outline{\large\arabic{enc}}
  & \Square%
  \stepcounter{enc}
  & \showInterval{2} &
  12 bandits/ flood \\
  \outline{\large\arabic{enc}}
  & \Square%
  \stepcounter{enc}
  & \showInterval{3} &
  \\
  \outline{\large\arabic{enc}}
  & \Square%
  \stepcounter{enc}
  & \showInterval{0} &
  Sexy goblin? \\
\end{tabular}%

\vspace{\baselineskip}

\noindent
The number-line at the far left functions as \pgls{ap}-tracker, just like the player's sheet.

Next to it, the \gls{interval}-tracker lets you keep a quick track of upcoming encounters, and other events.
The tracker only has three days, so if you need more --

\subsubsection{Plan to Improvise}
because players destroy plans, but improvising is easier when you prepare something open beforehand.
Once you make a few random encounter rolls, you'll find you can quickly generate plenty of content with that $4D6$ roll \vpageref{randomEncounters}.

\subsection{Running the First Session}

\subsubsection{The Opening Line}
\label{openingLine}
of \pgls{campaign} can colour so much of the rest.
I never really know how to start, but \gls{fenestra}'s core concept sets the right tone:

\begin{speechtext}
  Have you ever wondered how people would survive in a world of random encounters, with farmers living where monsters just wander about?
\end{speechtext}

People have opinions about this question.
They have ideas about whether we should be asking about realism, and ideas on how such a world might work.
This gives you a chance to explain \gls{fenestra}, the \gls{guard}, and hand out some character sheets.

\subsubsection{Recap \& Report}
the previous session by asking players.
It help them remember where their characters left off, catch up any players who could not attend the last session, and help you understand your players' perspective on the game.

If you want an easy excuse, tell the players that the local \gls{jotter}%
\footnote{See \autopageref{jotter}.}
has fallen behind on their paperwork, and wants the troupe to give them a complete report of their last mission.

\subsubsection{Closing Up}
should end with players gaining \glspl{xp}, and spending them on their characters.
This time-buffer helps ensure that when a session runs on too long, and the \glspl{pc} struggle to return to civilization, they have somewhere to steal a little time from.

Wrapping up the session on a high note also helps frame the game around what the \glspl{pc} have gained, or (if \pgls{pc} has died) ensure that players know their new character before their next session.

\subsection{Setting the Scene}

\subsubsection{Describe Back-to-Front}
to make sure everyone gets the context.
If you start a description with a bandit leader, at least one player will respond and interrupt the rest of your description.
You won't be able to tell the players about the armour with that guild's insignia, or the potions on the table.
The players are not the problem here -- the problem lies with the order of the information.

\begin{enumerate}
  \item
  Start every description with the background scenery.
  \item
  Mention some details.
  \item
  Give the \glspl{pc} something to respond to.
\end{enumerate}

Personally, I like to imagine the scene behind me, as if the players are looking at it.
That way, I can point to a door `\emph{over there}', rather than saying `you see a door in the North-Eastern side of the room'.
If I ever see a player point towards something in the game, it tells me they have engaged with the world.

Putting the background first also makes full descriptions easier.
We all want to put in details like how things smell, and the colour of the flowers, then every \gls{gm} forgets in the moment.
But once the background comes first, any time you draw a blank for a second, those details are what comes up.

\paragraph{In the forest\ldots}

Having these background details also helps focus on what the players see, rather than the bare events.

\begin{boxtext}
  You see four guards in the forest, walking towards you.
\end{boxtext}

\textit{\Large\raggedright `Branches crack all around the forest, always somewhere far away, but all around, as if the forest were snoring.'}

\textit{\large\raggedright `Every second step, your torch helps you navigate, but also alerts everything around you with its light.'}

\textit{`In the distance, another light shines alongside the crunch of feet, then that light goes out, but the feet continue towards you, slower now\ldots.'}

\paragraph{In town\ldots}

In social settings, the surrounding mood often forms the `background', as much as the actual events.
Taking an extra moment to give details about the surroundings also helps give players the kinds of realistic expectations that their characters would have, without a lot of conversations later.

\begin{boxtext}
  A drunk member of the \gls{sunGuard} approaches to ask why you're in town.
\end{boxtext}

\textit{\Large\raggedright `Once the rain fades, the town stinks of rot, misery, and dry throats as the taverns still have no ale.'}

\textit{\large\raggedright `Open window-shutters close a little as you pass, and the closed shutters open a crack.  The town wants to watch.'}

\textit{`A member of the \gls{sunGuard} approaches, drunk but standing fine, and yells ``long-ears! You got permission to walk my streets?''.'}

\subsubsection{Relations Over Lists}
\index{Encounters!Descriptions}
make the world feel solid.
Pair your statements up by giving them relations.
The rules can only give you lists of facts about \gls{fenestra}, but if you give the players raw facts, they have to memorize them.
Boring!
Take a moment to make one fact interact with the next, and players will can see how the world interacts.

Take these raw facts:

\begin{itemize}
  \item
  It is evening (the day's third \gls{interval}).
  \item
  The \glspl{pc} are in \pgls{broch}.
  \item
  The missions is to scrape \nameref{dreameater_moss} off the \gls{broch} walls so everyone can sleep.
  \item
  The mission's complication is: `there are two of them'.
  (perhaps more moss hides on the roof?)
  \item
  There is a storm.
  \item
  The encounter is 4 elves on a quest.
\end{itemize}

\begin{boxtext}
  The \gls{jotter} stares out the window, as if entranced by the raging storm.
  But noticing you, she turns round and looks at you with bulging bags under her eyes.

  ``I haven't slept in two nights.
  Anyone who sleeps here has something enter the dreams.
  It gets in through the moss on the walls.
  Take the brush, and anything you think you can use -- you'll need to scrub the \gls{broch}'s walls, even up high.

  You can see where the moss is from outside, but wait till the piper's finished before you\ldots''

  Then the drone of the pipes interrupt her, as the pipers warm up.
\end{boxtext}

The players make a decision at this point, and one wants to look out the window.

\begin{boxtext}
  Opening the shutter, the wind tries to blow you off the winding stairs, and onto the ground below.
  Outside, lightning illuminates four figures, standing in the dark forest and looking back at you.
\end{boxtext}

The \gls{pc} won't be able to see what the figures are, just that they look like people.
Then the players make another decision, you resolve their action, and go again:

\begin{itemize}
  \item
  It is afternoon (the day's second \gls{interval}).
  \item
  The pipers have finished the morning song, with nothing approaching.
  \item
  The missions is to help \pgls{village} by widening a road.
  \item
  The mission's complication is that rival \glspl{guard} work on a related mission.
  \item
  The sky is overcast.
  \item
  The encounter is a warthog.
\end{itemize}

\begin{boxtext}
  The \gls{jotter} wakes late, and looks much more alive than she once did, and more alive than you feel after the long night's work.

  ``I'm heading to town.
  Last time, the road was all overgrown, there's no room to see anything coming at the first bend, and two miles up there's an overhanging tree, then some bushes after that, almost clawing at the road, then\ldots
  just head along to the nearest \gls{village} and widen the road out.
  Take a few whetstones with you.''

  Open it all out before I get back tomorrow, and I'll make sure you get some extra grub and one \glsentrylongpl{sp} each.''

  But here \pgls{soldier} stops wrestling his donkey, and runs towards the \gls{jotter}, arguing that he and his crew should go first.

  ``If they start making noises on the road, while we're slowly carting these massive traps along the road, it'll attract something from the forest.
  They should wait till evening, we should go first.''

  The \gls{jotter} just shrugs and says `whatever', then leaves on her horse, along with three other \glspl{guard}.
\end{boxtext}

The \glspl{pc} can argue for a moment, but will immediately be cut off by the sound of an aggressive warthog screeching at something in the distance.
However goes out there, will definitely create a lot of noise\ldots

\bigLine

This isn't an easy thing to do during the maelstrom, but it's easy to prepare a few ideas.
Just select \pgls{interval}, grab a mission (\autopageref{ngMissions}), and roll up an encounter (\autopageref{randomEncounters}).

\subsubsection{Feels over Facts}
makes a situation into a scene.
Emphasise the emotional aspects of what the \glspl{pc} see, as often as possible.
If the situation has obvious stakes, then engagement from the players should come easily; but even other situations can become important once you make yourself emotionally invested.

\begin{description}
  \item[Fact:] You see \pgls{crawler}.
  \item[\quad Feels:] The horses stop, then try to back, scared, and the entire caravan stops abruptly.
  Following their gaze up, giants legs cling half way up a tree.
  \item[Fact:] It is misty.
  \item[\quad Feels:] a man calls out for his children, and the children call back to him, but neither one can locate the other through the mist.
  \item[Fact:] This underground cliff goes down a long way to a river.
  \item[\quad Feels:] the gentle sound of running water sounds like a river might be five minutes' walk away.
  Then a patch of darkness by the cavern's side shows a drop; the river lies at the bottom.
\end{description}

\subsection{Making Rulings}

\subsubsection{Literal Interpretations}
make rules predictable.
So if it's ever unclear how to resolve a situation, the first attempt should always be a strict interpretation of the rules.
For example, if a player says `If I charge round a corner, rather than a straight line, can I still use the Fast Charge knack?', the answer is `yes', because the rules as they stand don't prohibit going round a corner.

No rules will work all of the time, but by following a literal interpretation of the rules whenever possible, players feel better able to predict and navigate the world, and \glspl{gm} do not have to waste so much energy on making on-the-fly rulings.

Broadly, the \gls{gm} should consider themself bound by the rules as much as the players.
A good rule of thumb is to make as few decisions as possible, and let yourself focus on description and planning.

\subsubsection{Default Bonuses}
\index{Bonuses, Environmental}
\index{Environmental Bonuses}
use the default of `\gls{weight}~1 = +1~Bonus'.
If someone takes along bandages and herbs, they should gain a +1~Bonus to medicine rolls.
If they take a large bag of medical equipment (with \pgls{weight} of 2), they should gain a +2 Bonus.
Swords mostly follow this pattern.

I don't have a rule for environmental Bonuses, except to apply them liberally.
Player wants a Bonus to knock an enemy over, because it's raining, and therefore slippery?
+2 Bonus!
Do they want a Bonus for finding information from \pgls{warden}'s staff, because the staff have been drinking?
+3!

Players won't ask about this until you tell them, and the best way to tell them is to tell them while setting the \gls{tn}.
Don't just give them a `\tn[8], because you can bribe thirsty people easily' -- tell them `bribing the \gls{sunGuard} is \tn[10], but you can get a +2 Bonus since they're thirsty'.

\subsubsection{Realistic Money}
won't happen.
I have no idea how much a suit of chainmail, plus helmet, would cost in Middle Ages Europe, and even less of an idea once you factor in all the oddities of \gls{fenestra}.
The prices here usually assume 1~\gls{cp} = \pounds1, so if you guesstimate that a hand-made hat costs \pounds150, then that's 150~\glspl{cp}, or 1~\gls{sp}, 50~\glspl{cp}.

Having 100~\glspl{cp} equal 1~\gls{sp} may feel strange, but it helps stop penny-pinching and change-faff.
It helps players who feel bored of the small numbers, because you can easily hand-wave away the \glsentrylongpl{cp}, because they have such little value.
If your players don't like keeping track of the 15~\glspl{cp} their character starts with, just ignore them, and assume all of the \glspl{pc} can buy a couple of drinks and a tinder-box.

The low value also helps you avoid penny-pinching players.
If anyone ever insists on staying to loot \emph{every} corpse they find or create, even at the detriment of the game's pace, down to the last penny, remember that 100~\glspl{cp} has \pgls{weight} of~1.
So if those eight bandits have 25~\glspl{cp} each on their statblock, that's 200~\glspl{cp}, which has \pgls{weight} of~2.

\subsubsection{`Okay, and then'}
is a little like the old `yes,~and' advice from improv' games.
But instead of extending the premise the players give you, focus on the most plausible consequences.

The \glspl{pc} found the old ruin's treasures before the bandits could locate it?
Okay, they have the chest, and then all that noise definitely told the bandits where to meet them.
And then it's time to check if any wandering \glspl{monster} heard all that noise.

\Pgls{pc} killed the bandit-\gls{witch} with a single, lucky arrow?
Okay, and then the rest of the bandits would retreat into the forest, then become hungry.
And once hungry, they would have no alternative but to approach \pgls{village}, and raid the \gls{village} for food.
So roll an encounter for the bandits, and see if they find a trader before \pgls{village}, or if some wandering \gls{monster} finds them instead\ldots
and then there would be screaming either way, alerting anyone within a couple of miles.

\Pgls{pc} slashed a trader's throat open, just to steal 4~\glspl{gp} and his fancy necklace?
Okay -- and then his bloods sprays everywhere!
In the \gls{pc}'s hair, tabard, shoes, and all across their face.
How will they wash?
What happens to the corpse and wagon?
\ldots Okay, and then they have his jewellery -- the golden-chained sapphire, engraved with a tiny map of the area.
If the \glspl{pc} try to sell it in town, they would probably find a fence okay, and then the fence would probably recognize such a unique item\ldots and then\ldots

The \glspl{pc} spent the remainder of the week buying drinks for half the tavern, in the town's worst tavern?
Okay, and then they would recognize half the thieves and cutthroats in the area, and then when bandits shout `stand and deliver', the \glspl{pc} might recognize the voice, and could even threaten to have them barred from the Green Cock.

\Pgls{pc} raised their Mind \gls{sphere} so high that they treat \gls{monster} encounters as a chance to collect new, enchanted pets?
Okay, and then at the next \gls{village}, the farmers don't react well to the predators swarming up the road.
And then those pets have to eat, because \glspl{griffin} need to eat about as much as the \glspl{pc}.

Creating meaningful feedback means avoiding any unlikely consequences.
If the \glspl{pc} kill someone, and you decide that his brother is a legendary warrior, and wants revenge, \textit{and} managed to identify them from their tracks, then the players can only see the world as a guessing game.
But if some initial cause can unfurl itself, predictably, logically, and with such mundane results that the players would have to say that any other result would have been unnatural, then the players will soon consider these results too.
And once they consider their actions' most likely consequences, they have begun to relate to \gls{fenestra} in a new way.

\subsubsection{You Don't Always Need to Roll}

Get used to saying `the \gls{tn} is 10, so you succeed'.
If the \gls{tn} to open a wedged door is 10, and the \gls{pc} has plenty of time, they can just take a resting action, meaning their minimum roll is `7' -- if the Strength + Crafts Bonus totals +3, that means they succeed.
Once players feel emboldened to just say `okay, well at \gls{tn} 9 I succeed', the game becomes just a little faster.

\subsubsection{Everything has \pglsfmtlong{tn}}
in theory, so even mundane actions can present a real challenge when the circumstances really go wrong.
Getting over \pgls{village} wall might have \pgls{tn} of 3, which means that you can normally ignore the roll, but may still find times characters will struggle.
Let's say a particularly rotund and clumsy \gls{pc} has been drinking all day, and now wants to join the troupe in sneaking over \pgls{village} wall.

\begin{speechtext}
  Give me a \roll{Speed}{Athletics} roll, at \tn[3] to get over the wall.
  Anyone who took part in the drinking contest gets a -2 penalty.
\end{speechtext}

Normally, rolling for such a low \gls{tn} would be a waste of time.
But if one member of the troupe has a basic Speed Penalty of -2, then the player needs a \gls{natural} of 7 to \emph{tie} with the action.

\subsubsection{Eyeballs Always Work}
so avoid asking for any kind of roll to notice something.
These rolls cut the flow of any narrative.
Use rolls to notice something only when someone actively wants to hide something from the \glspl{pc}, such as ambushes, or encrypted writing.

If you don't want to draw attention to something the \glspl{pc} can sense, then mention it briefly, or speak around it, then tell the players only if one asks for more detail.

\subsubsection{Another Look at Dice}
can help elucidate the system.
Consider these \ref{rulesMinusDie} ways to resolve an action (assume all \glspl{trait} remain unchanged):

\begin{enumerate}
  \item
  The player rolls $2D6$ at \tn[7].
  They add their bonus, and subtract the opponent's Bonus.
  \label{rulesMinusOpponent}
  \item
  The player and \gls{gm} each roll 1D6 and add their bonuses.
  If one rolls higher, they win.
  If they both roll the same, they lose.
  \label{rulesTwoRolls}
  \item
  The player rolls $1D6-1D6$, then adds their own bonus, and subtracts their opponent's bonus.
  Rolling a total of `0' means a tie, a positive score means success, and a negative score means failure.
  \label{rulesMinusDie}
\end{enumerate}

The trick here, is that these are not different rules -- each one is the same as each other, and the same as the standard \gls{bind} rules.

Notice in example \ref{rulesMinusOpponent}, that it makes no difference if we subtract \pgls{npc}'s total score, or add that score to \pgls{tn}.
And example \ref{rulesTwoRolls} has two dice being rolled as usual, even if two different people roll the dice.

We use the rules as actually written, only because players find subtraction slightly more irritating than addition.
But looking at the rules in these different ways can help clarify the underlying structure.

\subsubsection{Torture}
\index{Torture}
makes a tiresome tale.
If players torture \pgls{npc}, have the \gls{npc} give out a false narrative.
If they ask for a location, the \gls{npc} sends them somewhere dangerous.
If they ask who's in charge of a conspiracy, the \gls{npc} fingers a well-known \gls{warden} or \gls{mixer}.

Even stupid \glspl{npc} can craft a plausible fiction, so start pulling up enough nonsense to baffle the \glspl{pc}!
Any attempt to notice the lies receives a -4 penalty -- it's hard to tell odd behaviour when someone's under perpetual stress!

\subsubsection{Purging the Poison of the Computers}
begins with noticing it.
We should always feel impressed when computer game RPGs manage to display a small fraction of the freedom, and emergent complexity that tabletop RPGs do.
However, over the last decade I've seen an alarming trend of players who only know RPGs from computer games, and pull strange habits into the game.
These strange habits always come from inappropriate expectations, and the best medicine for these expectations is usually clearer descriptions from their \gls{gm}.

\begin{boxtext}
  I light my torch, and throw it at the ogre.
\end{boxtext}

This suggestion is laughable, but don't laugh, just clarify.

\begin{speechtext}
  So while the other characters attack for a few rounds, you want your character to go through their backpack to get their tinder box and a torch, then use the tinder box to light the torch.

  Is that right?
\end{speechtext}

That shows the player where their error lies, but also alerts you to the fact that they could not properly see the world their character lives in.

\begin{boxtext}
  If I have properly enchanted the \gls{griffin}, I want to ride around on it.
  I got a new mount, guys!

  Okay, now I need to go to a town for supplies.
  Can we `fast travel'?
\end{boxtext}

Once again the player cannot fully \emph{feel} the world.

\null
\begin{quote}
  You clamber onto the \gls{griffin}'s back, and it flaps anxiously, confused by the feeling of weight on its back.
  Eventually it runs, takes flight, and you both fly towards the setting Sun through heavy gusts of wind.
  It cannot indicate where it wants to take you, any more than you can tell it where you want to go, but en route, you pass over a \gls{village}.
  The archers in the \gls{village} immediately draw back their bows.
  You find gripping onto the flapping creature difficult.
  Roll \roll{Dexterity}{Flight} to avoid falling.
\end{quote}

\ldots perhaps that's a little cruel, but \gls{fenestra} is a cruel world.

\subsection{How to Avoid Scheduling Conflicts}
\label{realtimeScheduling}

End each session with the \glspl{pc} safely back in civilization, and the game will grow a thousand times stronger.

\subsubsection{Tackling the Root Cause}
is easy.
Scheduling conflicts plague every RPG session held outside of the Antarctic circle.
This has nothing to do with RPG players, and everything to do with the hobby's bizarre demands.
People who play football play with whoever turns up, even if it means `five a side'; chess clubs never demand rigid, weekly attendance.
The problem with getting everyone together comes from the demand that everyone must treat the game like work.

Instead of making unrealistic expectations, we can make a small adjustment to the gaming world -- end sessions at a point where characters have reached safety.
This might be in a town, \gls{village}, or \gls{broch}.
On the next session, the players who arrived will find their characters sitting together, in a bar, \gls{broch}, or on the road, and can all begin a new mission together.

If you tell everyone from the outset that their session must start and end in \pgls{broch} (or some civilized area), then the players will generally do their best.
Any `dungeons' become an area they enter quickly, and jump out of once something --- anything --- has been achieved.

Once your game adopts this standard, the headache of getting five adults in a room on one of their free days vanishes.
And every time someone messages to say they can't make the game, you can mark that off as a victory for `parallel time' because the game goes on regardless.

\subsubsection{Rotating Chairs}
opens the game up to more people.
The gaming table no longer needs to feel `full', because yet another person wants to join the \gls{campaign}.

\subsubsection{Massive \Glsfmtplural{campaign}}
\index{Westmarches \Glsfmtplural{campaign}}
with a dozen players become much easier to run once the table opens up to more players.
The game also lets players keep multiple characters, as they can select which they want to use on a week-by-week basis.

And if players want to bring someone into the game, you can ask them to donate a character from their \gls{characterPool}.
This helps tie the new person to the world.

\index{Parallel Time}
\subsubsection{Proper Time-Keeping}
makes the world solid enough for a rotating table.%
\footnote{Gygax was right.}
To make this easier, \gls{fenestra}'s time runs in parallel to our own.
Each real-month, \gls{fenestra} experiences two \glspl{cycle}.

Proper time-keeping doesn't just help the \gls{campaign} -- it helps the night.
Having a proper ending-time means you can signal ahead that the session should come to an end within the hour.

And of course, if a game ends early, this just leaves more time to discuss how the \glspl{pc} will spend their time in town, and what they want to spend their \glspl{xp} on.

\subsubsection{\Glsfmttext{downtime} is Limited}
during the game, as each Mundane week only grants about thirty days in \gls{fenestra}.
This should still be enough for the \glspl{pc} to travel, fight, retreat to safety, heal, and try again.

Of course, if the players can't wrap up in time, their characters will have problems.
\Glspl{pc} who don't make it back in time might start their next session with 0~\glspl{fp}, and perhaps an encounter roll from their last location to explain what exactly went wrong.
A reasonable ruling always depends on the exact circumstances.

\subsubsection{Concluding a Session}
becomes a lot easier once players have done it once or twice.
If the players insist that their \glspl{pc} still have plans inside the \gls{deep}, twenty miles into the forest, and would not want to return to civilization, a few things can help:

\begin{itemize}
  \item
  Recheck each character's equipment.
  Do they have enough torches to actually descend?
  Do they have enough food to enter, and get back?
  Does anyone really want to push their luck in the dank abyss without any rope?
  \item
  Remind them they have jobs in the \gls{templeOfBeasts}, and if they want to keep running around abandoned shrines in the forest then they should ask \pgls{jotter} to approve the mission.
  \item
  Note the uneasy feeling the characters have, which tells them their luck is running out, and that tonight will have a bad star above.
  \item
  Maybe this dank little cave counts as `civilization' after all!
  If you've already established a nearby gnomish warren, or a dwarvish stronghold, the troupe might rest exactly where they are.
  And next week, if some players go, their characters can work on \gls{guard} duties with the civilization there, while other characters arrive from the same place.

  On the other hand, if you have not established any nearby civilization, this explanation will cheapen the \gls{campaign} instantly.
  \item
  Be fast, and end late.
  Maybe the \glspl{pc} can get whatever they wanted and return within the next half-hour real-time.
\end{itemize}

\subsection{Using Types as Cultures}

\Gls{fenestra} might seem a little bare-bones in terms of people.
Other worlds have a complete rainbow of different gnomish `subraces', where \gls{fenestra} just has `gnomes'.
However, I'd like to suggest a perspective-shift.

I think of \gls{fenestra} as having myriad different types of elves and gnomes, but I also think that all the `subraces' common to fantasy lore were really just different cultures in disguise.
Snow elves and wood elves do different things, and have different skills, but their bodies didn't show any differences in most fantasy worlds.

So while this book only shows `dwarves', and those dwarves will (as we all do) consider themselves just `the normal ones', living in `the middle place' in their world, where they speak `the language', they will likely see distant dwarves as `hill dwarves', or `deep dwarves', and those dwarves will have a particular name for the local ones.
People may call goblins `svarts', `nockers', `kobolds', or `gremlins', and perhaps gremlins live in forests and have green skin instead of white, but at the end of the day, they're all goblins.

So when making a settlement by a lake, you might make a group of `boat humans', who consider Seafaring to be a standard Skill for everyone; or make `stone elves', who live across a particular mountain range, partially underground.

We don't need any new rules for these distinctions -- shared languages, and Skills give enough bare-bones mechanics to hang a story on top of.

\end{multicols}

\pagebreak

\section{Melee}

\begin{multicols}{2}

\subsection{Pacing}

\subsubsection{Hollering `Initiative'}
works for tables where people don't like taking turns in a clock-wise direction.
Just shout numbers while describing the scene, and keep going down until someone interrupts with their action.

\begin{speechtext}

``Eight! The gnolls raise their weapons''

``Seven, six! They move forward, bearing their yellowed teeth.''

``Five! Snarls abound as they speed up to a rush.''

\end{speechtext}

Nothing has actually happened by this point, but it sets the scene nicely.

%!
\null
\begin{speechtext}
  ``Five'', one of the players shout.
  ``I'm going at five.
  I move to protect Grotmax.''

  ``Two gnolls go for you, another two go for Grotmax.
  Roll to engage at \tn[11].''

\end{speechtext}

The count continues down quickly at all times, and the count always provides a sense of urgency.
If players don't notice it's their turn when you're shouting, the next players goes immediately (they still have their \glspl{ap}).

\subsubsection{The Cost of Speech}
is \pgls{ap}, but it's important that this doesn't become a `gotcha' for any kind of speech.
Players shouting `charge', should not get a surprising penalty.

The cost for speaking exists to add a tactical decision.
When in battle, players may want to turn to a spell-caster and ask them to cast a \textit{Curse} upon an enemy, or request that someone guard them.
This speech helps the troupe tactically, so it has a cost.
When a player has a great idea about the whole group moving backwards to avoid enemy attack, they should think of the simple proposal as a tactical manoeuvre, and consider the cost of proposing it.

\subsubsection{\glsentrytext{npc} Fights}
\label{npcfights}
threaten to make you roll dice alone for five minutes, if you feel like using the regular, full system.
So if you need a quick approximation of a massive battle between \glspl{npc}, just have each \gls{npc} deal its own \gls{cr} value in Damage each round (ignoring \gls{dr}).
A guard with \gls{cr}~10 deals 10 Damage; this could mean killing a single creature with 10 \glspl{hp}, or could mean finishing off 2 creatures the \glspl{pc} have already wounded, by dealing each one 5 Damage.

\begin{exampletext}
  ``The goblin platoon start throwing more spears at you, but then from the side, the garrison of guards burst into the cavern's entrance to join you.''
\end{exampletext}

If two \glspl{npc} fight, whichever has the highest \gls{cr} deals Damage first (equally matched opponents deal Damage simultaneously).
So if ten soldiers, each with \pgls{cr} 10, fight \pgls{basilisk} with \gls{cr} 24, the \gls{basilisk} deal 24 Damage, killing 2 soldiers.
On the next turn, the 8 remaining soldiers would deal 80 Damage, killing the \gls{basilisk}.

\begin{exampletext}

  The \gls{gm} thinks for a moment.
  That's 30 goblins and 12 guards.
  The twelve guards have \pgls{cr} of 10 each, so they deal 10 Damage each, killing 10 goblins.
  Then the 20 remaining goblins, each with \gls{cr} 10, deal 40 Damage, killing 4 guards.

  ``The guards spill in, massacring the goblin horde.
  You see some surrounded, and spears driven into them, but the rest keep fighting.''

\end{exampletext}

Obviously, this system is not going to represent anything with much accuracy, but it's better than halting a game so you can roll dice for twenty minutes alone.

\subsection{Tactics}
\index{Tactics}

Nobody like an opponent who's always letting them win.
A \gls{gm} pulling out three \glspl{basilisk} on new \glspl{pc} is bad form, but it's even worse when the players are allowed to win by poor tactics.
At least dying to \pgls{basilisk} means dying with honour!

\subsubsection{Basic Tactics}
include two things: focus all attacks on one target, and flank whenever possible.
Don't parcel up opponents in a fair and even-handed way -- they're there to destroy the \glspl{pc}, so set all of them against one \gls{pc}, and if that player wants their character to survive, they'd best move back, or the other \glspl{pc} had better guard them.%
\exRef{core}{Core rules}{guarding}
If the \glspl{pc} have left their anterior side exposed, enemies should spend 1~\gls{ap} to move to their side and allow half the group to flank the \glspl{pc}.

\subsubsection{Illusion Spells}
\index{Illusions}
demand \glspl{ap} to attack, even if they can't hit anyone.
Three illusory armoured warriors might not seem like much of a worry, but they present a lethal choice:

\begin{description}
  \item[Defend]
  and spend all your \glspl{ap}.
  This may leave the character on negative \glspl{ap}, which will leave them open to real attacks.
  \item[Don't defend]
  and run the risk of a real opponent striking real flesh with a real knife. 
\end{description}

Not everyone can wield illusions well, but when used correctly, they can devastate an army.

\subsubsection{Fight Dirty!}
So you have twenty goblins facing off against four \glspl{pc}, but the \glspl{pc} all have plate armour, a round shield, a plan, and an attitude.
They're invincible.
The battle looks hopeless.
It looks \emph{boring}.
So call for a break, and really \emph{think} about what goblins can do.

\begin{itemize}
  \item
  Have goblins throw rocks, then run away, and throw again.
  Maybe they'll score \pgls{vitalShot}, or simply wear down the \glspl{pc}'~\glspl{ap}.
  \item
  Have them pick up a poleaxe; they can't wield that weight properly, but they can still wield it.
  \item
  They can jump on a player, and weigh them down.
  \item
  Turn off the lights, and pull out the rules for fighting in the dark with a grin.%
  \exRef{core}{Core rules}{darkness}
\end{itemize}

\subsubsection{Fine-Grained Armour}
works easily, if you want more detail for combat.
BIND's armour basically reduces everything to `basic armour covers 3 points, full armour covers 5 points', so it won't represent the finer points of brigandine vs chainmail.
However, if your players insist on constructing armour, and you really need \emph{something} to differentiate the mismatching pieces they have attached, the system does admit of two potential rule-expansions.

Firstly, you don't need to limit armour to just `complete' and `partial'.
You might have `minimal' (with \pgls{covering} of 1), `basic', and basically any other number from 1-5.

Secondly, you can divide the armour types.
Someone could have Partial armour which provides \gls{dr} 5 (meaning just a helmet and chest-plate), while also having Complete leather armour (meaning a roll 3 over the \gls{tn} bypasses the plate, but receives 3 \gls{dr} from the leather).

I wouldn't recommend doing any of that for \glspl{npc}, as it will inevitably turn into a headache.

\subsubsection{Impartiality}
isn't just good form, it also cuts down on brain-fatigue from making too many rapid decisions.

So if \pgls{npc} attacks the \glspl{pc}, but you have no idea \textit{which} \gls{pc} they would attack first, have them attack whoever has the most \glsentrylongpl{fp}, then keep attacking that \gls{pc} until they have a reason to change their behaviour.

\subsubsection{To Balance Encounters}
\index{Encounters!Balance}
bring extra character sheets, and make sure new \glspl{pc} have somewhere they can quickly re-enter the plot once \pgls{pc} dies.
Let the dice dictate results, without bias or mercy.
After a couple of deaths, the players will understand that survival demands planning, intelligence, and brutality.
Let them know that the world is as it is, and will not warp around their needs unless they find a way to \emph{make} the world change.

Wait, did I say `balance encounters'?
I meant `run'.
We should run.
The trolls will arrive soon, and we know what happens if they find us\ldots

\subsubsection{\Glsfmtlongpl{cr}}
come from an inelegant formula.
This formula simply multiplies the \glspl{trait} which let \glspl{npc} deal Damage, then multiplies the \glspl{trait} which keep them alive, and adds a bit for having \glsentrylongpl{mp}.

\begin{align*}
  \frac{Att \times Dam \times \glsentrytext{ap}}{50}
  -
  heft
  +
  \frac{\glsentrytext{hp} \times \glsentrytext{dr}}{10}
  +
  \frac{\glsentrytext{mp}}{2}
\end{align*}

\noindent
Here `heft' means the \gls{ap}~cost to attack, which is `1' for most \glspl{monster}.

This cheap estimation has no more authority or accuracy than a semi-educated guess, so if you make your own \glspl{npc}, and you want to give them \pgls{cr}, it's better to go with your gut than your calculator.

\subsection{Damage, Death \& Dismemberment}

\subsubsection{Damage}
is a massive, screaming deal in BIND.
It's easy to take habits over from other games where losing one's liver is all part of a normal Tuesday afternoon but here \glspl{pc} should lose \glspl{fp}, then attempt to flee and only in the most dire situations should they start to bleed.
Damage which doesn't hit home can be brushed over with a brief note about `avoiding the swing' but if anyone loses a single \glsentrylong{hp} the \gls{gm} should grind the description and combat to a halt to emphasise exactly how eyeball poppingly, knee-cap shatteringly painful and side-splittingly debilitating a knife can be.
Take your time.
Make the words secrete congealed blood.
If the \glspl{pc} start to lose \glspl{hp} and don't realise how serious this situation is they might perish where they otherwise would have run away to fight another day.

\subsubsection{Death}
\label{pcdeath}
\index{Death}
is natural.
Players who want their characters to survive should retire them.
After all, few active \glspl{guard} survive for long.
Character creation should feel fast and easy, and no main plot-line should rely on a particular character.

Once death has come, the player should select a character from their \gls{characterPool}.
The new \glspl{pc} begins with a number of \glspl{storypoint} equal to the previous character's \glspl{fp} (minimum of 5).%
\exRef{stories}{Stories}{storyPointInheritance}
They can keep the rest of their \gls{characterPool}, and even reintroduce those characters with an anecdote about how the new \gls{pc} also knows them.

If the player has an empty \gls{characterPool}, they can always roll up a random \gls{guard}, or perhaps another player will gift a character from their \gls{characterPool}.

If the \gls{pc} died on the road, \pgls{jotter} may commission \pgls{bothy} where they fell, and name it after them.
This makes an excellent way to commemorate a fallen character, as their name will live forever within that section of \gls{fenestra}.

\subsubsection{Dismembered \Glspl{pc}}
are no fun to play.
If \pgls{pc} is totally out of commission, with 1 \gls{hp} left, 4 \glspl{ep} from being bled dry, and a disease nobody can cure, consider letting them play \pgls{npc} and letting them keep all \glspl{xp} gained during this time.

\end{multicols}

\section{Parting Notes}

\begin{multicols}{2}

\subsection{Rules as Reality}
\index{Blorb}

If you've ever played a computer game, then used cheat-codes, you'll notice a shift in tone descends the moment the cheat-codes come in.
Once you can break the rules, you lose some respect for the game, as your relationship to the world changes.

The opposite shift occurs when people try a rogue-like game, where you can't save the game and try again from the same spot.
Where they once jumped towards anything that moved, sword-in-hand, ready to hit `X' as fast as they could, a couple of deaths tell them to approach things differently.

\begin{speechtext}
  `How far can that acid spray reach?
  Can it cross the river?
  Maybe I could swim across the river, and throw rocks at it until it dies.'
\end{speechtext}

These problem-solving habits don't emerge so easily, or in the same manner, when people don't fear death (or at least serious consequences).

Of course, once \pgls{pc} dies, you should be `on their side'.
You can say `I'm sorry your character died', and you should never appear happy about a character's death (unless, for some reason, the player is happy about it).
But the dice should be allowed to land where they land.

The rules have been designed around this principle; when \glspl{pc} attack something, you can simply state the \gls{tn} to attack it, and let the player roll the dice.
Mechanically, it makes no difference if you roll, or they do, but it lets the players feel responsible for the Damage their characters take when they fail a roll.

On a larger scale, letting the dice fall where they may means trusting the players.
If the troupe start out, on their first path, on game one of your \gls{campaign}, you might roll \pgls{basilisk} as their first encounter, and then you might feel the temptation to re-roll the result, because \pgls{basilisk} will be too much for any starting characters to handle.
But if you stick to the rules then the players might surprise you with something clever, and they'll feel all the smarter for having done so.

It's not your job to help them win the fight.
It's your job to tell them that \pgls{basilisk} sees them, and find out what they do.

\subsubsection{The Rules Made for Breaking}
don't have an asterisk next to them, but once players develop an intuition of what their characters can do, they should start to push back against their roles.
Starting characters in the \gls{guard} don't have the weaponry to take on monsters, so they may have to steal, beg, or flee.
\Glspl{jotter} may hand out near-impossible or suicidal missions, forcing the troupe to disobey, without bringing down the wrath of the \gls{templeOfBeasts}.

If a missions conflicts with \pgls{sq}, you don't have to cut one in favour of the other in order to make sure that the \glspl{pc} can do both.
Just make it clear that they will have to select one goal or the other, then roll a random encounter.

\subsubsection{Leave the \Glsfmttext{guard}}
and let focus the \gls{campaign} on the most interesting part of your \gls{campaign}.
BIND only focusses on the \gls{guard} as a convenient plot hook, but your nights don't \emph{need} to revolve around them.

\begin{itemize}
  \item
  Start out with \glspl{guard}, and if the troupe kills \pgls{jotter}, switch the entire \gls{campaign} to banditry, with wanted posters, questions of hygiene while on the run, and the looming threat of frost.
  \item
  Play a group of reckless \glspl{doula}, who defy the orders of their guild, and spend their nights focussed intently in maximizing their magical abilities, and collecting potent \glspl{boon}.
  \item
  Make a troupe of bog-standard fantasy adventurers, including a bard and a monk, and let them hunt for treasure.
  \item
  Craft a story all about the \gls{templeOfPoison}, where each major \gls{npc} has three desires.
  Lowly servants of the temple receive conflicting orders, where each must realise two of these desires, each from different \glspl{npc}.
  Craft a tangled web from these simple threads, and throw plot to the wind.
  \item
  Restrict the races to only gnomes.
  Tell the story of their warren, as they try to protect themselves from goblins invading the surrounding warrens.
  The other gnomish villages fall, one by one, and the elders argue about solutions all day, but do nothing.
\end{itemize}

\subsection{I Lied}
I lied about how lethal BIND is.
In truth, I've not seen many \glspl{pc} die during play.
\Glspl{pc} die rarely because the players play carefully, and they're careful because having ten spare character sheets in a pile sends a message.

Careful players throw rocks into rooms that look funny, they back off when wounded, and think of practical plans rather than trying to gloriously bonk everything that moves until gold pieces fall out.
Despite the care of careful players, characters can still die unexpectedly, but this feels less unjustified when they know ahead of time that justice is a tap-dancing jester, ready to hang them for laughs.
\glsadd{paik}
So don't spill the beans, and keep on warning players that their characters could die at any moment, because useful statements are better than true statements.%
\footnote{This statement may be false, but it is still useful.}

\subsection{The Metaphysics of Becoming a Tree}

All magic systems imply some ontology, and most of the European ontologies come from Aristotle's view of the world.
This is why someone turned into stone might retain their vision, motion, and thoughts.

In Aristotle's view, everything has:

\begin{itemize}
  \item
  form (the shape)
  \item
  substance (what it's made out of)
  \item
  telos (what it's \emph{for})
  \item
  first-cause (where it came from)
\end{itemize}

For example:

\begin{itemize}
  \item
  Rain forms drops, is made out of water, it's for watering plants, and it comes from the sky.
  \item
  People have people-shapes; flesh, sinew and bone constitute their bodies; people exist to praise the gods and achieve glory; and mothers provide the first cause.
  \item
  Fireballs have rain-drop shape but fire for substance, they exist to burn, and come from angry wizards.
\end{itemize}

Anyone transformed to stone would retain their organs with the same telos, meaning that a liver would continue to process toxins, and the eyes continue to see, because that's what eyes are for (regardless of their `substantive cause' (meaning, the substance they are made from).

Similarly, when a spell makes someone big, they wouldn't have problems with vision, their lungs won't collapse, and they won't suffer any back-pain, because they are the same as they were, except for the size, which is simply a larger version of their original form, with more substance added via the magic of the growth spell.

If none of that makes sense, blame Aristotle.

\subsection{Don't Have a Fun Game}

Tell a strange tale, and make your players uncomfortable.
Or tell an even stranger tale, and invite them to think about the importance of rabbits to Middle Ages economics.
Make a scene casual, then tense.
Zoom into the death of \pgls{npc} so hard that the players change how they approach the world.
Give them a puzzle without an answer, simply because stories should be true to life, even when they contain goblins with wings.

You can't approach `fun' head-on.
The moment you ask someone `are you having fun?', you cut their focus, and move them to introspection.
Great stories never ask the reader how they feel.
They engage the reader in something else.

Any time all the players feel engaged, the game and the \gls{gm} serve their purpose.
Players can ponder, laugh, or cry.
As long as they all feel like they want to engage with the narrative, the flavour of engagement doesn't matter.
But then again, if it doesn't feel like it matters to you, it won't matter to them; once again, engaging with the goal head-on, eyes-open, doesn't work.

So leave fun to one side, find an engaging problem, and ask them all `what do you do?'.

\end{multicols}
